\usepackage[babelshorthands]{polyglossia}
\setmainlanguage{russian}
\setotherlanguage{english}

\usepackage{amsmath,amsfonts,amssymb,amsthm,mathtools}
\usepackage{mathrsfs}

\defaultfontfeatures{
	Path = fonts/EB_Garamond/static/,
	Extension = .ttf,
	Numbers = Lowercase
}

\setmainfont{EBGaramond-Regular}[
	ItalicFont = EBGaramond-Italic,
	BoldFont = EBGaramond-Bold,
	BoldItalicFont = EBGaramond-BoldItalic
]

\usepackage{microtype}

\pagestyle{plain}

\usepackage{xcolor}

\usepackage[all]{nowidow}
\setnoclub
\setnowidow

\sloppy

\usepackage{tabu, longtable}

\usepackage{perpage}
\MakePerPage{footnote}

\usepackage{enumitem}

\usepackage{tikz}
\usetikzlibrary{shapes,arrows,shapes.geometric, shapes.multipart}
\usepackage{caption}
\captionsetup{justification=centering,font={large,bf}}

\usepackage{hyperref}
\hypersetup{
    colorlinks,
    citecolor=black,
    filecolor=black,
    linkcolor=black,
    urlcolor=gray
}

\usepackage{titlesec}
\titleformat{\section}
{\normalfont\normalsize\bfseries\centering}{}{0pt}{}

\usepackage[symbol]{footmisc}
\renewcommand{\thefootnote}{\fnsymbol{footnote}}

\newcommand{\signature}[1]{
  \begin{flushright}
    \emph{#1}
  \end{flushright}
}


\usepackage{enotez}
\setenotez{
	list-name=Примечания,
	backref=true,
	totoc=section
}

\DeclareInstance{enotez-list}{custom}{paragraph}{
	format = \normalfont
      }

%%% Local Variables:
%%% mode: latex
%%% TeX-master: "main"
%%% End:
