К этой новой редакции <<Науки логики>>, первый том
которой теперь выходит в свет, я, должен сказать, приступил
с полным сознанием как трудности предмета
самого по себе, а затем и его изложения, так и несовершенства
его редакции в первом издании. Сколько я ни старался
после дальнейших многолетних занятий этой наукой
устранить это несовершенство, я все же чувствую,
что у меня достаточно причин просить читателя быть ко
мне снисходительным. Право же на такое снисхождение
дает мне прежде всего то обстоятельство, что для содержания
я нашел в прежней метафизике и прежней логике
преимущественно только внешний материал. Хотя эти
науки разрабатывались повсеместно и часто,~-- последняя
из указанных наук разрабатывается еще и поныне,~-- все
же эта разработка мало касалась спекулятивной стороны;
в целом скорее повторялся тот же самый материал, который
попеременно то разжижался до тривиальной поверхностности,
то расширялся благодаря тому, что снова вытаскивался
старый балласт, так что от таких, часто лишь
совершенно механических, стараний философское содержание
ничего не могло выиграть. Изображение царства
мысли философски, т.\,е. в его собственной имманентной
деятельности, или, что то же самое, в его необходимом
развитии, должно было поэтому явиться новым предприятием,
и притом начинающим все с самого начала. Указанный
же приобретенный материал~-- известные уже
формы мысли~-- должен рассматриваться как в высшей
степени важный подсобный материал (Vorlage) и даже
как необходимое условие, как заслуживающая нашу признательность
предпосылка, хотя этот материал лишь кое-где
дает нам слабую нить или мертвые кости скелета, к
тому же еще перемешанные между собой в беспорядке.

Формы мысли выявляются и отлагаются прежде всего
в человеческом \emph{языке}. В наше время мы должны неустанно
напоминать, что человек отличается от животного
именно тем, что он мыслит. Во все, что для человека
становится чем-то внутренним, вообще представлением, во
все, что он делает своим, проник язык, а все то, что он
превращает в язык и выражает в языке, содержит, в скрытом
ли, спутанном или более разработанном виде, некоторую
категорию; в такой мере естественно для него логическое,
или, правильнее сказать, последнее есть сама
присущая ему \emph{природа}. Но если противопоставлять природу
вообще как физическое духовному, то следовало бы
сказать, что логическое есть, вернее, сверхприродное, проникающее
во все естественные отношения человека, в его
чувства, созерцания, вожделения, потребности, влечения
и тем только и превращающее их, хотя лишь формально,
в нечто человеческое, в представления и цели. Если язык
богат логическими выражениями, и притом специальными
и отвлеченными, для [обозначения] самих определений
мысли, то это его преимущество. Из предлогов и членов
речи многие уже выражают отношения, основывающиеся
на мышлении; китайский язык, говорят, в своем развитии
вовсе не достиг этого или достиг в незначительной степени.
Но эти грамматические частицы выполняют всецело
служебную роль, они только немногим более отделены от
соответствующих слов, чем глагольные приставки, знаки
склонения и т.\,д. Гораздо важнее, если в данном языке
определения мысли выражены в виде существительных и
глаголов и таким образом отчеканены так, что получают
предметную форму. Немецкий язык обладает в этом отношении
большими преимуществами перед другими современными
языками; к тому же многие из его слов имеют
еще ту особенность, что обладают не только различными,
но и противоположными значениями, так что нельзя не
усмотреть в этом спекулятивный дух этого языка: мышление
может только радовать, когда оно неожиданно
сталкивается с такого рода словами и обнаруживает, что
соединение противоположностей~-- результат спекуляции,
который для рассудка представляет собой бессмыслицу,~--
наивно выражено уже лексически в виде \emph{одного} слова,
имеющего противоположные значения. Поэтому философия
вообще не нуждается в особой терминологии; приходится,
правда, заимствовать некоторые слова из иностранных
языков; эти слова, однако, благодаря частому
употреблению уже получили в нашем языке право гражданства,
и аффектированный пуризм был бы менее всего
уместен здесь, где в особенности важна суть дела.~--
Успехи образования вообще, и в частности наук, даже
эмпирических наук и наук о чувственно воспринимаемом,
в общем двигаясь в рамках самых обычных категорий
(например, категорий целого и частей, вещи и ее свойств
и т.\,п.), постепенно выдвигают и более высокие отношения
мысли или по крайней мере поднимают их до большей
всеобщности и тем самым заставляют обращать на
них больше внимания. Если, например, в физике получило
преобладание такое определение мысли, как <<сила>>, то
в новейшее время самую значительную роль играет категория
\emph{полярности}\endnotemark{}, которую, впрочем, слишком à tort et à
travers [без разбора] втискивают во все, даже в учение
о свете; полярность есть определение такого различия, в
котором различаемые [моменты] \emph{неразрывно} связаны
друг с другом. То обстоятельство, что таким способом
отошли от формы абстракции, от тождества, которое
сообщает некоей определенности, например силе, самостоятельность,
и вместо этого была выделена и стала привычным
представлением другая форма определения~--
форма различия, которое в то же время сохраняется в
тождестве как нечто нераздельное,~-- это обстоятельство
бесконечно важно. Рассмотрение природы благодаря самой
реальности, в которой удерживаются ее предметы,
необходимо заставляет фиксировать категории, которые
уже нельзя более игнорировать в ней, хотя при этом имеет
место величайшая непоследовательность в отношении
других категорий, за которыми \emph{также} сохраняют их значимость,
и это рассмотрение не допускает того, чтобы,
как это легче происходит в сфере духовного, переходили
от противоположности к абстракциям и всеобщностям.

\endnotetext{
  Гегель имеет в виду <<закон полярности>>, сформулированный
  в натурфилософии Шеллинга, в частности в работе <<О мировой душе>> (1798).
}

Но хотя, таким образом, логические предметы, равно
как и их словесные выражения, суть по крайней мере
нечто всем известное в области образования, однако, как
я сказал в другом месте\endnotemark{}, то, что \emph{известно} (bekannt),
еще не есть поэтому \emph{познанное} (erkannt); между тем
требование продолжать заниматься тем, чт\'о уже известно,
может даже вывести из терпения,~-- а чт\'о более известно,
чем определения мысли, которыми мы пользуемся
повсюду, которые мы произносим в каждом предложении?
Это предисловие и имеет своей целью указать общие
моменты движения познания, исходящего из этого известного,
общие моменты отношения научного мышления
к этому естественному мышлению. Этого указания вместе
с тем, что содержится в прежнем \emph{введении}, достаточно
для того, чтобы дать общее представление о смысле логического
познания, то общее представление, которое с самого
начала желают получить о науке, еще до того представления,
которое составляет суть дела.

\endnotetext{
  В предисловии к <<Феноменологии духа>>: <<Известное
  вообще~-- оттого, что оно \emph{известно,еще не познано}>> (стр. 16).
}

Прежде всего следует рассматривать как бесконечный
прогресс то обстоятельство, что формы мышления были
высвобождены из того материала, в который они погружены
при сознающем себя созерцании, представлении,
равно как и в нашем вожделении и волении или, вернее,
также и в представляющем вожделении и волении (а ведь
нет человеческого вожделения или воления без представления),
что эти всеобщности были выделены в нечто самостоятельное
и, как мы это видим у Платона, а главным
образом у Аристотеля, были сделаны предметом самостоятельного
рассмотрения; этим , начинается их познание.
<<Лишь после того,~-- говорит Аристотель,~-- как было налицо
почти все необходимое и требующееся для жизненных
удобств и сношений, люди стали добиваться философского
познания>>\endnote{Вольный перевод из <<Метафизики>> Аристотеля А 2 982 b
  (M.~-- Л., 1934, стр.\,22).}. <<В Египте,~-- замечает он перед
тем,~-- математические науки рано развились, ибо там
жречество было рано поставлено в условия, дававшие
ему досуг>>\endnote{<<Метафизика>> А 1 981 b (стр.\,20).}.
Действительно, потребность заниматься чистыми
мыслями предполагает длинный путь, который
человеческий дух должен был пройти, она, можно сказать,
есть потребность уже удовлетворенной потребности
в необходимости отсутствия потребностей, которой человеческий
дух должен был достигнуть,~-- потребность абстрагироваться
от материала созерцания, воображения
и т.\,д., от конкретных интересов вожделения, влечения,
воли, в каковом материале закутаны определения мысли.
В тихой обители пришедшего к самому себе и лишь в себе
пребывающего мышления умолкают интересы, движущие
жизнью народов и индивидов. <<Во многих отношениях,~--
говорит Аристотель в той же связи,~-- человеческая природа
зависима, но эта наука, которой ищут не для какого-нибудь
употребления, есть единственная наука, свободная
сама по себе, и потому кажется, будто она не есть человеческое
достояние>>\endnotemark{}. Философия вообще в своих мыслях
еще имеет дело с конкретными предметами~-- богом, природой,
духом; логика же занимается этими предметами
всецело лишь в их полной абстрактности. Логика поэтому~--
обычно предмет изучения для юношества, каковое
еще не вступило в круг интересов повседневной жизни,
пользуется по отношению к этим интересам досугом и
лишь для своей субъективной цели должно заниматься
приобретением средств и возможностей для проявления
своей активности в сфере объектов указанных интересов,
причем и этими объектами оно должно заниматься теоретически.
К этим \emph{средствам}, в противоположность указанному
выше представлению Аристотеля, причисляют и науку
логики; занятия ею~-- это предварительная работа,
место для этой работы~-- школа, лишь после которой
должно следовать настоящее дело жизни и деятельность
для достижения действительных целей. В жизни уже
\emph{пользуются} категориями; они лишаются чести рассматриваться
особо и низводятся до \emph{служения} духовной выработке
живого содержания, созданию и сообщению друг
другу представлений, относящихся к этому содержанию.
С одной стороны, они ввиду своей всеобщности служат
\emph{сокращениями} (ведь какое бесконечное множество частностей
внешнего существования и деятельности объемлют
собой представления: битва, война, народ или море, животное
и т.\,д.; какое бесконечное множество представлений,
видов деятельности, состояний и т.\,д. должно быть
сведено к таким \emph{простым} представлениям, как бог или
любовь и т.\,д.). С другой стороны, они служат для более
точного определения и нахождения \emph{предметных отношений},
причем, однако, содержание и цель, правильность
и истинность вмешивающегося сюда мышления ставятся
в полную зависимость от наличествующего, и определениям
мысли самим по себе не приписывается никакой
определяющей содержание действенности. Такое применение
категорий, которое в прежнее время называлось
естественной логикой, носит бессознательный характер;
и если научная рефлексия отводит им в духе роль служебных
средств, то этим мышление превращается вообще
в нечто подчиненное другим духовным определениям.
О наших ощущениях, влечениях, интересах мы, правда,
не говорим, что они нам служат, мы считаем их самостоятельными
способностями и силами; так что мы сами
суть те, кто ощущает так-то, желает и хочет того-то,
полагает свой интерес в том-то. С другой стороны, можно
прийти к сознанию того, что мы скорее служим нашим
чувствам, влечениям, страстям, интересам и тем более
привычкам, чем обладаем ими; ввиду же нашего внутреннего
единства с ними нам еще менее может [прийти в голову],
что они нам служат средствами. Мы скоро обнаруживаем,
что такие определения души и духа суть \emph{особенные}
в противоположность \emph{всеобщности}, в качестве каковой
мы себя сознаем и в которой заложена наша свобода,
и начинаем думать, что мы скорее находимся в плену у
этих особенностей, что они приобретают власть над нами.
После этого мы тем менее можем считать, что формы
мысли, которые проходят через все наши представления,~--
будут ли последние чисто теоретическими или
содержащими материал, принадлежащий ощущениям,
влечениям, воле,~-- служат нам, что мы обладаем ими, а
не наоборот, они нами. Чт\'о остается на \emph{нашу} долю против
них, каким образом можем \emph{мы}, могу я возвышать
себя \emph{над} ними как нечто более всеобщее, когда они сами
суть всеобщее, как таковое? Когда мы предаемся какому-нибудь
ощущению, какой-нибудь цели, интересу и чувствуем
себя в них ограниченными, несвободными, то областью,
в которую мы в состоянии выбраться из них и тем
самым вновь прийти к свободе, является эта область достоверности
самого себя, область чистой абстракции,
мышления. Или, когда мы хотим говорить о \emph{вещах}, их
\emph{природу} или их \emph{сущность} мы равным образом называем
\emph{понятием}, которое существует только для мышления; о
понятиях же вещей мы имеем гораздо меньшее основание
сказать, что мы ими владеем или что определения
мысли, комплекс которых они составляют, служат нам;
напротив, наше мышление должно ограничивать себя сообразно
им и наш произвол или свобода не должны
переделывать их по-своему. Стало быть, поскольку субъективное
мышление есть наиболее характерная для нас
деятельность, а объективное понятие вещей составляет
самое суть (Sache), то мы не можем выходить за пределы
указанной деятельности, не можем стать выше ее, и столь
же мало мы можем выходить за пределы природы вещей
(Natur der Dinge). Последнее определение мы можем, однако,
оставить в стороне; оно совпадает с первым постольку,
поскольку оно есть некое отношение наших мыслей
к самой вещи (Sache), но только дало бы оно нам
нечто пустое, ибо мы этим признали бы вещь правилом
для наших понятий, а между тем вещь может быть для
нас не чем иным, как нашим понятием о ней. Если критическая
философия понимает отношение между этими
\emph{тремя} терминами так, что мы ставим \emph{мысли} между \emph{нами}
и \emph{вещами}, как средний термин, в том смысле, что этот
средний термин скорее отгораживает \emph{нас} от \emph{вещей},
вместо того чтобы соединять нас с ними, то этому взгляду
следует противопоставить простое замечание, что как раз
эти вещи, которые будто бы стоят на другом конце, по
ту сторону нас и по ту сторону соотносящихся с ними
мыслей, сами суть мысленные вещи и как совершенно
неопределенные они суть лишь \emph{одна} мысленная вещь
(так называемая вещь в себе), пустая абстракция.

\endnotetext{
  <<Метафизика>> А 2 982 b (стр.\,22). Гегель переставил предложения
  в этой цитате.
}

Все же сказанного нами будет вполне достаточно для
уяснения той точки зрения, согласно которой исчезает
отношение, выражающееся в том, что определения мысли
берутся только как нечто полезное и как средства. Более
важное значение имеет находящийся в связи с указанным
отношением взгляд, согласно которому их обычно понимают
как внешние формы.~-- Пронизывающая все наши
представления, цели, интересы и поступки деятельность
мышления происходит, как сказано, бессознательно (естественная
логика) ; то, чт\'о наше сознание имеет перед
собой, согласно этому взгляду,~-- это содержание, предметы
представлений, то, чем проникнут интерес; определения
же мысли суть \emph{формы}, которые только \emph{касаются
содержания}, а не составляют самого содержания. Но если
верно то, что мы указали выше и с чем в общем соглашаются,
а именно, если верно, что \emph{природа}, особая \emph{сущность},
истинно \emph{сохраняющееся} и \emph{субстанциальное} при
всем многообразии и случайности явлений и преходящем
проявлении есть \emph{понятие} вещи, \emph{всеобщее в самой этой
вещи} (как, например, каждый человеческий индивид,
хотя и бесконечно своеобразен, все же имеет в себе prius
[первичное] всего своего своеобразия, prius, состоящее в
том, что он в этом своеобразии есть \emph{человек}, так же как
каждое отдельное \emph{животное} имеет prius, состоящее в том,
что оно животное), то нельзя сказать, чт\'о осталось бы от
такого индивида~-- какими бы многообразными прочими
предикатами он ни был наделен,~-- если бы от него была
отнята эта основа (хотя последнюю тоже можно назвать
предикатом). Непременная основа, понятие, всеобщее, которое
и есть сама мысль, поскольку только при слове
<<мысль>> можно отвлечься от представления,~-- это всеобщее
нельзя рассматривать \emph{лишь} как безразличную форму
\emph{при} некотором содержании. Но эти мысли обо всех природных
и духовных вещах, само субстанциальное \emph{содержание},
представляют собой еще такое содержание, которое
заключает в себе многообразные определенности и
еще имеет в себе различие души и тела; понятия и соотносимой
с ним реальности; более глубокой основой служит
душа, взятая сама по себе, чистое понятие~-- сердцевина
предметов, их простой жизненный пульс, равно как
и жизненный пульс самог\'о субъективного мышления о
них. Задача и состоит в том, чтобы осознать эту \emph{логическую}
природу, которая одушевляет дух, движет и действует
в нем. Инстинктивная деятельность отличается от
руководимой интеллектом и свободной деятельности
вообще тем, что последняя осуществляется сознательно;
поскольку содержание побудительного мотива выключается
из непосредственного единства с субъектом и доведено
до предметности, возникает свобода духа, который, будучи
в инстинктивной деятельности мышления связанным своими
категориями, расщепляется на бесконечно многообразный
материал. В этой сети завязываются там и сям
более прочные узлы, служащие опорными и направляющими
пунктами жизни и сознания духа; эти узлы обязаны
своей прочностью и мощью именно тому, что они, доведенные
до сознания, суть в себе и для себя сущие понятия
его сущности. Важнейший пункт, уясняющий природу
духа,~-- это отношение не только того, чт\'о он есть \emph{в себе},
к тому, чт\'о он есть \emph{в действительности}, но и того, чем он
\emph{себя знает}; так как дух есть по своей сущности сознание,
то это знание себя есть основное определение его \emph{действительности}.
Следовательно, высшая задача логики~--
очистить категории, действующие лишь инстинктивно как
влечения и осознаваемые духом прежде всего разрозненно,
тем самым как изменчивые и путающие друг друга,
доставляющие ему таким образом разрозненную и
сомнительную действительность, и этим очищением возвысить
его в них к свободе и истине.

То, на что мы указали как на начало науки, огромная
ценность которого, взятого само по себе и в то же время
как условие истинного познания, признано было уже ранее,
а именно рассмотрение понятий и вообще моментов
понятия, определений мысли, прежде всего как формы,
отличные от содержания и лишь касающиеся его,~-- это
рассмотрение тотчас же проявляет себя в себе самом
неадекватным отношением к истине, признаваемой предметом
и целью логики. Ибо, беря их просто как формы,
как отличные от содержания, принимают, что им присуще
определение, характеризующее их как конечные и делающее
их неспособными схватить истину, которая бесконечна
в себе. Если истинное в каком-либо отношении и
сочетается вновь с ограниченностью и конечностью, то
это есть момент его отрицания, его неистинности и недействительности,
именно его конца, а не утверждения, каковое
оно есть как истинное. По отношению к убожеству
чисто формальных категорий инстинкт здравого смысла
почувствовал себя, наконец, столь окрепшим, что он презрительно
предоставляет их познание школьной логике
и школьной метафизике, пренебрегая в то же время той
ценностью, которую осознание этих нитей имеет уже
само по себе, и не сознавая, что, когда он ограничивается
инстинктивным действием естественной логики, а тем
более когда он обдуманно (reflektiert) отвергает знание и
познание самих определений мысли, он рабски служит
неочищенному и, стало быть, несвободному мышлению.
Простым основополагающим определением или общим
формальным определением совокупности таких форм служит
\emph{тождество}, которое в логике этой совокупности форм
признается законом, $A = A$, законом противоречия. Здравый
смысл в такой мере потерял свое почтительное отношение
к школе, которая обладает такими законами истины
и в которой их продолжают излагать, что он из-за
этих законов насмехается над ней и считает невыносимым
человека, который, руководясь такими законами,
умеет высказывать такого рода истины: растение есть
растение, наука есть наука и \emph{т.\,д. до бесконечности}.
Относительно формул, служащих правилами умозаключения,
которое на самом деле представляет собой одно из
главных применений рассудка, также упрочилось столь
же справедливое сознание, что они безразличные средства,
которые по меньшей мере приводят и к заблуждению
и которыми пользуется софистика; что, как бы мы ни
определяли истину, они непригодны для более высокой
истины, например религиозной; что они вообще касаются
лишь правильности познания, а не истины, хотя было бы
несправедливо отрицать, что в познании у них есть такая
область, где они должны обладать значимостью, и что
они в то же время~-- существенный материал для мышления
разума.

Недостаточность этого способа рассмотрения мышления,
оставляющего в стороне истину, может быть восполнена
лишь тем, что к мысленному рассмотрению привлекается
не только то, чт\'о обычно считается внешней формой,
но и содержание. Вскоре само собой обнаруживается,
что то, чт\'о в ближайшей самой обычной рефлексии отделяют
от формы как содержание, в самом деле не должно
быть бесформенным, лишенным определений внутри себя,
ибо в таком случае оно было бы лишь пустотой, скажем
абстракцией вещи в себе; что оно, наоборот, обладает в
самом себе формой и, более того, только благодаря ей
одушевлено и обладает содержимым (Gehalt), и что именно
она сама превращается лишь в видимость некоего содержания,
а тем самым и в видимость чего-то внешнего
по отношению к этой видимости [содержания]. С этим
введением содержания в логическое рассмотрение предметом
[логики] становятся не \emph{вещи} (Dinge), а \emph{суть}
(Sache), \emph{понятие} вещей. Однако при этом нам могут также
напомнить, что \emph{имеется} множество понятий, множество
сутей. Чем же ограничивают это множество,~-- об
этом мы отчасти уже сказали выше, а именно, что понятие
как мысль вообще, как всеобщее есть беспредельное сокращение
по сравнению с единичностью вещей, которые в своей
множественности предстают неопределенному созерцанию
и представлению. Отчасти же \emph{какое-то} понятие (ein Begriff)
есть в то же время, во-первых, \emph{само} понятие (der
Begriff) в самом себе, а последнее имеется только в единственном
числе и составляет субстанциальную основу; но,
во-вторых, оно есть некоторое \emph{определенное} понятие, каковая
определенность в нем есть то, чт\'о выступает как
содержание; определенность же понятия есть определение
формы указанного субстанциального единства, момент
формы как целокупности, момент \emph{самого понятия}, составляющего
основу определенных понятий. Это понятие чувственно
не созерцается и не представляется; оно только
предмет, продукт и содержание \emph{мышления} и в себе и для
себя сущая суть, логос, разум того, чт\'о есть, истина того,
чт\'о носит название вещей. Уж менее всего д\'олжно оставлять
вне науки логики логос. Поэтому не должно быть
делом произвола, вводить ли его в науку или оставлять его
за ее пределами. Если те определения мысли, которые суть
только внешние формы, рассматриваются истинно в них
самих, то из этого может следовать лишь то, что они конечны,
что их якобы самостоятельное бытие (Für-sich-sein-Sollen)
неистинно и что их истина~-- понятие. Поэтому,
имея дело с определениями мысли, которые вообще
пронизывают наш дух инстинктивно и бессознательно и
которые остаются беспредметными, незамеченными, даже
когда они проникают в язык, логическая наука будет также
реконструкцией тех определений мысли, которые выделены
рефлексией и фиксированы ею как субъективные,
внешние формы по отношению к материалу и содержанию.

Нет ни одного предмета, который, сам по себе взятый,
поддавался бы столь строгому, имманентно пластическому
изложению, как развитие мышления в его необходимости;
нет ни одного предмета, который в такой мере требовал
бы такого изложения; в этом отношении наука о нем
должна была бы превосходить даже математику, ибо ни
один предмет не имеет в самом себе этой свободы и независимости.
Такой способ изложения требовал бы, как это
по-своему происходит при математическом выведении,
чтобы ни на одной ступени развития не встречались определения
мысли и рефлексии, которые не возникали бы
непосредственно на этой ступени, а переходили бы в нее
из предшествующих ступеней. Но, конечно, приходится
вообще отказаться от такого абстрактного совершенства
изложения. Уже одно то обстоятельство, что наука должна
начинать с абсолютно простого и, стало быть, наиболее
всеобщего и пустого, требует, чтобы способ изложения [ее]
допускал только такие совершенно простые выражения
[для уяснения] простого без какого-либо добавления хотя
бы одного слова; единственно, чт\'о по существу дела требовалось
бы,~-- это отрицательные рефлексии, которые старались
бы не допускать и удалять то, чт\'о обычно могло
бы сюда привнести представление или неупорядоченное
мышление. Однако подобные вторжения (Einfälle) в простой,
имманентный ход развития [мысли] сами по себе
случайны и старания предотвратить их, стало быть, столь
же случайны; кроме того, было бы тщетно стремиться
предупредить \emph{все} такого рода вторжения именно потому,
что они не касаются существа дела, и было бы по крайней
мере недостаточным то, что желательно для удовлетворения
требования систематичности. Но свойственные
нашему нынешнему сознанию беспокойство и разбросанность
также не позволяют нам не принимать во внимание
более или менее доступные всем рефлексии и вторжения
[в мысль]. Пластический способ изложения требует к тому
же пластической способности восприятия и понимания\endnotemark{};
но таких пластических юношей и мужей, каких
придумывает Платон, таких слушателей, столь спокойно
следящих лишь за существом дела, сами отрекаясь от
высказывания \emph{собственных} рефлексий и взбредших на ум
соображений (Einfälle), при помощи которых доморощенному
мышлению не терпится показать себя, нельзя
было бы выставить в современном диалоге; еще в меньшей
степени можно было бы рассчитывать на таких читателей.
Наоборот, я слишком часто встречал чрезмерно
ярых противников, не способных сообразить такой простой
вещи, что взбредшие им в голову мысли и возражения
содержат категории, которые суть предпосылки и которые
сами должны быть подвергнуты критическому рассмотрению,
прежде чем пользоваться ими. Неспособность
осознавать это заходит невероятно далеко; она приводит
к основному недоразумению, к тому плохому, т.\,е. необразованному
способу рассуждения, когда при рассмотрении
какой-то категории мыслят \emph{нечто иное}, а не самое
эту категорию. Такая неспособность осознавать тем более
не может быть оправдана, что это \emph{иное} представляет собой
другие определения мысли и понятия, а в системе
логики именно эти другие категории также должны были
найти свое место и быть там самостоятельно рассмотрены.
Более всего это бросается в глаза в подавляющей части
возражений и нападок, вызванных первыми понятиями
или положениями логики,~-- \emph{бытием}, \emph{ничто} и \emph{становлением};
последнее, хотя оно и само есть простое определение,
тем не менее бесспорно~-- простейший анализ показывает
это~-- содержит указанные два определения в качестве
моментов. Основательность, по-видимому, требует,
чтобы прежде всего было вполне исследовано начало как
основа, на которой зиждется все остальное, и даже требует,
чтобы не шли дальше, прежде чем оно не окажется
прочным, и чтобы, напротив, если окажется, что это не
так, было отвергнуто все следующее за ним. Эта основательность
обладает также и тем преимуществом, что она
необычайно облегчает дело мышления: она имеет перед
собой все [дальнейшее] развитие заключенным в этом зародыше
и считает, что покончила со всем [исследованием],
если покончила с этим зародышем, а с ним легче всего
справиться потому, что он есть наипростейшее, само простое;
главным образом именно легкостью требуемой работы
подкупает эта столь самодовольная основательность.
Это ограничение [критики] простым предоставляет свободный
простор произволу мышления, которое само не
желает оставаться простым, а приводит относительно
этого простого свои соображения. Будучи вполне вправе
сначала заниматься \emph{только} принципом и, стало быть, не
вдаваться в [рассмотрение] \emph{дальнейшего}, эта основательность
сама действует обратно этому, привлекая к рассмотрению
\emph{дальнейшее}, т.\,е. другие категории, чем ту, которая
представляет собой только принцип, другие предпосылки
и предубеждения. Такие предпосылки, как то, что бесконечность
отлична от конечности, содержание не то, что
форма, внутреннее не то, что внешнее, а опосредствование
точно так же не есть непосредственность (как будто
кто-то этого не знает), приводятся в виде наставлений и
их не столько доказывают, сколько рассказывают, уверяя,
что это так. В таком наставлении как образе действий
есть~-- иначе это нельзя назвать~-- нечто глупое. По сути
же дела здесь отчасти неправомерно то, что такого рода
[положения] только служат предпосылкой и сразу принимаются;
отчасти же и в еще большей мере здесь имеется
незнание того, что потребность и дело логического мышления~--
именно исследовать, может ли быть истинным
конечное без бесконечного и, равным образом, может ли
быть \emph{чем-то истинным}, а также \emph{чем-то действительным}
такая абстрактная бесконечность, лишенное формы содержание
и лишенная содержания форма, такое внутреннее
само по себе, не имеющее никакого внешнего проявления,
и т.\,д. Но эта культура и дисциплина мышления, благодаря
которым достигается его пластическое отношение
[к предмету научного рассмотрения] и преодолевается
нетерпение вторгающейся рефлексии, приобретаются
единственно лишь движением вперед, изучением и проделыванием
всего пути развития.

\endnotetext{
  О пластичности как особенности греческого духа и искусства
  см.: \emph{Гегель}. Эстетика, т.\,3. М., 1970, стр.\,113--114.
}

Тому, кто в новейшее время работает над самостоятельным
построением философской науки, можно при
упоминании о платоновском изложении напомнить рассказ
о том, что Платон семь раз перерабатывал свои книги
о государстве. Напоминание об этом и сравнение (поскольку,
по-видимому, это напоминание заключает в себе
таковое) могли бы только еще в большей мере вызвать
желание, чтобы автору произведения, которое, как принадлежащее
нынешнему миру, имеет перед собой для
разработки более глубокий принцип, более трудный предмет
и более богатый по объему материал, был предоставлен
свободный досуг для переработки его не семь, а семьдесят
семь раз. Но, принимая во внимание, что труд
писался в условиях, диктовавшихся внешней необходимостью,
что широта и многосторонность присущих нашему
времени интересов неизбежно отрывали от работы
над ним, что автор даже испытывал сомнение, оставляют
ли еще повседневная суета и оглушающая болтливость
самомнения, довольная тем, что она ограничивается этой
суетой, возможность для участия в бесстрастной тишине
чисто мыслящего познания,~-- принимая во внимание все
это, автор, рассматривая свой труд под углом зрения
величия задачи, должен довольствоваться тем, чем этот
труд мог стать в таких условиях.

\signature{Берлин, 7 ноября 1831\,г.}


%%% Local Variables:
%%% mode: latex
%%% TeX-master: "../main"
%%% End:
