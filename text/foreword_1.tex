Полное изменение, которое претерпел у нас за последние
лет двадцать пять характер философского мышления,
более высокая точка зрения на само себя, которой
в этот период достигло самосознание духа, до сих пор
еще оказали мало влияния на облик \emph{логики}.

То, чт\'о до этого времени называлось метафизикой,
подверглось, так сказать, радикальному искоренению и
исчезло из области наук. Где теперь мы услышим или
где теперь смеют еще раздаваться голоса прежней онтологии,
рациональной психологии, космологии или даже
прежней естественной теологии? Где теперь будут интересоваться
такого рода исследованиями, как, например,
об имматериальности души, о механических и конечных
причинах? Да и прежние доказательства бытия бога излагаются
лишь исторически или в целях назидания и ради
возвышения духа. Это факт, что интерес отчасти к содержанию,
отчасти к форме прежней метафизики, а отчасти
к обоим вместе утрачен. Насколько удивительно, когда
для народа стали непригодными, например, наука о
его государственном праве, его убеждения, его нравственные
привычки и добродетели, настолько же удивительно
по меньшей мере, когда народ утрачивает свою метафизику,
когда дух, занимающийся своей чистой сущностью,
уже не имеет в нем действительного существования.

Экзотерическое учение кантовской философии, гласящее,
что \emph{рассудок не вправе перешагивать область
опыта} и что иначе познавательная способность становится
\emph{теоретическим разумом}, который сам по себе порождает
только \emph{химеры},~-- это учение оправдывало с научной
стороны отказ от спекулятивного мышления. Содействовали
этому популярному учению и вопли новейшей
педагогики (требование времени, направляющее взор
людей на непосредственные нужды) о том, что, подобно
тому как главное для познания~-- опыт, так и для преуспеяния
в общественной и частной жизни теоретическое
понимание даже вредно, а существенно, единственно
полезно~-- упражнение и вообще практическое образование.~--
Таким образом, поскольку наука и здравый человеческий
смысл способствовали крушению метафизики,
казалось, что в результате их общих усилий возникло
странное зрелище~-- \emph{образованный народ без метафизики},
нечто вроде храма, в общем-то разнообразно украшенного,
но без святыни. Теология, которая в прежние времена
была хранительницей спекулятивных таинств и
(правда, зависимой) метафизики, отказалась от этой науки,
заменив ее чувствованиями, практически общедоступными
поучениями и учено-историческими сведениями.
Этой перемене соответствует то обстоятельство, что, с
другой стороны, исчезли те \emph{одинокие}, которые приносились
в жертву своим народом и удалялись из мира, дабы
существовали созерцание вечного и жизнь, посвященная
единственно лишь этому созерцанию не ради какой-то
выгоды, а ради благодати. Это~-- исчезновение, которое
в другой связи можно рассматривать как явление, по
своему существу тождественное с вышеупомянутым. Казалось,
таким образом, что, после того как был рассеян
этот мрак, это бесцветное занятие самим собой ушедшего
в себя духа, существование превратилось в светлый, радостный
мир цветов, среди которых, как известно, нет
\emph{черных}.

\emph{Логика} испытала не столь печальную участь, как метафизика.
Предрассудок, будто логика \emph{научает мыслить},~--
в этом раньше видели ее пользу и, стало быть,
ее цель (это похоже на то, как если бы сказали, что
только благодаря изучению анатомии и физиологии мы
научаемся переваривать пищу и двигаться),~-- этот предрассудок
давно уже исчез, и дух практичности уготовлял
ей, по-видимому, не лучшую участь, чем ее сестре. Тем
не менее, вероятно ввиду приносимой ею некоторой формальной
пользы, ей было еще оставлено место среди наук,
и ее даже сохранили в качестве предмета публичного
преподавания. Но этот лучший удел касается только ее
внешней участи, ибо ее форма и содержание остались
такими же, какими они по давней традиции передавались
от поколения к поколению, причем, однако, при
этой передаче ее содержание делалось все более и более
тощим и скудным; в ней еще не чувствуется тот новый
дух, который выявился в науке не менее, чем в действительности.
Но совершенно тщетно желание сохранить
формы прежнего образования, когда изменилась субстанциальная
форма духа. Они представляют собой увядшие
листья, спадающие под напором новых почек, образовавшихся
у их основания.

\emph{Игнорирование} этой общей перемены начинает постепенно
исчезать также и в научной области. Незаметно
эти новые представления стали привычными даже
противникам, они усвоили их, и если они все еще высказывают
пренебрежение к источнику этих представлений
и лежащим в их основе принципам и оспаривают
их, то зато им приходится мириться с выводами и они
оказываются не в силах противиться влиянию последних.
Помимо того что все больше и больше слабеет их
отрицательное отношение [к указанным представлениям],
эти противники не знают иного способа придать своим работам
положительное значение, кроме как вместе с другими
начинать говорить языком новых представлений.

С другой стороны, уже прошло, по-видимому, время
брожения, с которого начинается всякое новое творчество.
Первоначально это творчество относится с фантастической
враждебностью к существующей обширной систематизации
прежнего принципа; отчасти оно опасается
также, что потеряется в пространных частностях, отчасти
же страшится труда, требуемого для научной разработки,
и, чувствуя потребность в такой разработке, хватается
сначала за пустой формализм. Ввиду этого требование,
чтобы содержание подверглось обработке и было
развито, становится еще более настоятельным. В формировании
той или иной эпохи, как и в формировании отдельного
человека, бывает период, когда речь идет главным
образом о приобретении и утверждении принципа в его
неразвитой еще напряженности. Однако более высокое
требование состоит в том, чтобы этот принцип стал
наукой.

Но, что бы ни было уже сделано в других отношениях
для сути и формы науки, логическая наука, составляющая
собственно метафизику или чистую, спекулятивную
философию, до сих пор находилась еще в большем
пренебрежении. Чт\'о я разумею более конкретно под этой
наукой и ее точкой зрения, я указал предварительно во
\emph{введении}. Необходимость вновь начать в этой науке с
самого начала, природа самого предмета и отсутствие таких
подготовительных работ, которые можно было бы
использовать для предпринятого [нами] преобразования,~--
пусть все эти обстоятельства будут приняты во внимание
справедливыми критиками, если окажется, что и многолетний
труд [автора] не смог сообщить этой попытке
большее совершенство. Важно иметь в виду, что дело
идет о том, чтобы дать новое понятие научного рассмотрения.
Философия, поскольку она должна быть наукой,
не может, как я указал в другом
месте\footnotemark{},
для этой цели
заимствовать свой метод у такой подчиненной науки, как
математика, и точно так же она не может довольствоваться
категорическими заверениями внутреннего созерцания
или пользоваться рассуждениями, основывающимися
на внешней рефлексии. Только \emph{природа содержания}
может быть тем, что \emph{развертывается} в научном познании,
причем именно лишь эта \emph{собственная рефлексия} содержания
полагает и \emph{порождает} само \emph{определение содержания}\endnotemark{}.

\endnotetext{
См. <<Феноменология духа>>: <<Наука должна организоваться
только собственной жизнью понятия\dots Содержание показывает,
что его определенность не принята от другого и не пристегнута
[к нему], но оно само сообщает ее себе и, исходя из себя, определяет
себя в качестве момента и устанавливает себе место внутри
целого>> (\emph{Гегель}. Соч., т.\,IV, М.,~1959,~стр.\,28. Все дальнейшие
цитаты из <<Феноменологии духа>> даны по этому изданию).
}

\footnotetext{
<<Феноменология духа>>. Предисловие к первому изданию.
Подлинное развитие сказанного~-- познание метода, место которого в
самой логике\endnotemark{}
}

\endnotetext{
Это примечание прибавлено Гегелем в 1831\,г. при подготовке
второго издания <<Науки логики>>.
}

\emph{Рассудок дает определения} и твердо держится их;
\emph{разум} же отрицателен и \emph{диалектичен}, ибо он обращает
определения рассудка в ничто; он положителен, ибо порождает
\emph{всеобщее} и постигает в нем особенное. Подобно
тому как рассудок обычно понимается как нечто обособленное
от разума вообще, так и диалектический разум
обычно принимается за нечто обособленное от положительного
разума. Но в своей истине разум есть \emph{дух}, который
выше их обоих; он рассудочный разум или разумный
рассудок. Он есть отрицательное (das Negative), то,
чт\'о составляет качество и диалектического разума, и рассудка.
Этот дух отрицает простое (das Einfache) и тем
самым полагает определенное различие, которым занимается
рассудок; он также разлагает это различие, тем
самым он диалектичен. Однако он не задерживается на
этом нулевом результате, а выступает в нем и как нечто
положительное, и, таким образом, восстанавливает первоначальное
простое, но как всеобщее, которое конкретно
внутри себя. Под конкретно всеобщее не подводится то
или другое данное особенное, а в указанном процессе
определения и в разлагании его уже определилось вместе
с тем и особенное. Это духовное движение, дающее
себе в своей простоте свою определенность, а в ней~-- и
равенство с самим собой, это движение, представляющее
собой, стало быть, имманентное развитие понятия, есть
абсолютный метод познания и вместе с тем имманентная
душа самого содержания.~-- Я утверждаю, что философия
способна быть объективной, доказательной наукой лишь
на этом конструирующем себя пути.~-- Таким способом
я попытался в <<Феноменологии духа>> изобразить \emph{сознание}.
Сознание есть дух, как конкретное знание, и притом
погрязшее во внешнем. Но движение форм этого
предмета, подобно развитию всякой природной и духовной
жизни, покоится только на природе \emph{чистых сущностей},
составляющих содержание логики. Сознание как
дух, который охватывает лишь явления и который освобождается
на своем пути от своей непосредственности и
сращенности с внешним, становится чистым знанием,
дающим себе в качестве предмета указанные чистые
сущности, как они суть сами по себе. Они чистые мысли,
мыслящий свою сущность дух. Их самодвижение есть
их духовная жизнь и представляет собой то, чт\'о конституирует
науку и изображением чего она является.

Этим указано [внутреннее] отношение науки, которую
я называю \emph{феноменологией духа}, к логике. Что же касается
внешнего отношения между ними, то я полагал,
что за первой частью <<Системы науки>>\footnotemark{}, содержащей
феноменологию, последует вторая часть, которая должна
была содержать логику и обе реальные дисциплины
философии~-- философию природы и философию духа,~--
так что этой частью заканчивалась бы система науки.
Но необходимость расширить объем логики, взятой сама
по себе, побудила меня выпустить ее в свет отдельно;
она, таким образом, составляет, согласно этому расширенному
плану, первое продолжение <<Феноменологии
духа>>. Позднее я разработаю обе названные выше реальные
философские науки. Этот первый том <<Логики>> содержит
первую книгу~-- \emph{учение о бытии}, вторую книгу~--
\emph{учение о сущности}, как второй раздел первого тома; второй
же том будет содержать \emph{субъективную логику}, или
\emph{учение о понятии}.

\footnotetext{
Бамберг и Вюрцбург в издательстве Гёбгарда,~1807. Во втором
издании, которое появится в свет в ближайшую пасху, это
название будет исключено. Вместо указываемой далее предполагавшейся
второй части, которая должна была содержать все другие
философские науки, я выпустил после этого в свет <<Энциклопедию
философских наук>>, вышедшую в прошлом году третьим
изданием.\endnotemark{}
}

\endnotetext{
Это примечание прибавлено Гегелем в 1831\,г. при подготовке
второго издания <<Науки логики>>. Упоминаемое здесь второе
издание <<Феноменологии духа>> он подготовить не успел, исправив
лишь 36 страниц <<Предисловия>>. С этими исправлениями <<Феноменология>>
была издана уже после его смерти, в 1832 г. <<Энциклопедия
философских наук>> впервые была издана в 1817\,г.,
третье издание~-- в 1830\,г.
}

\signature{Нюрнберг,~22~марта~1812\,г.}
