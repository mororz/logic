Ни в какой другой науке не чувствуется столь сильно
потребность начинать с самой сути дела, без предварительных
размышлений, как в науке логики. В каждой
другой науке рассматриваемый ею предмет и научный
метод различаются между собой; равным образом и содержание
[этих наук] не начинает абсолютно с самого
начала, а зависит от других понятий и связано с окружающим
его иным материалом. Вот почему за этими науками
признается право говорить лишь при помощи лемм
о почве, на которой они стоят, и о ее связи, равно как
и о методе, прямо применять предполагаемые известными
и принятыми формы дефиниций и т.\,п. и пользоваться
для установления своих всеобщих понятий и основных
определений обычным способом рассуждения.

Логика же, напротив, не может брать в качестве
предпосылки ни одной из этих форм рефлексии или
правил и законов мышления, ибо сами они составляют
часть ее содержания и сначала должны получить свое
обоснование внутри нее. Но в ее содержание входит не
только указание научного метода, но и вообще само
\emph{понятие науки}, причем это понятие составляет ее конечный
результат: она поэтому не может заранее сказать,
чт\'о она такое, лишь все ее изложение порождает это
знание о ней самой как ее итог (Letztes) и завершение.
И точно так же ее предмет, \emph{мышление} или, говоря определеннее,
\emph{мышление, постигающее в понятиях}, рассматривается
по существу внутри нее; понятие этого мышления
образуется в ходе ее развертывания и, стало быть,
не может быть предпослано. То, чт\'о мы предпосылаем
здесь в этом введении, не имеет поэтому своей целью
обосновать, скажем, понятие логики или дать наперед
научное обоснование ее содержания и метода, а имеет
своей целью с помощью некоторых разъяснений и размышлений
в рассуждающем и историческом духе растолковать
представлению ту точку зрения, с которой
следует рассматривать эту науку.

Если вообще логику признают наукой о мышлении,
то под этим понимают, что это мышление составляет
\emph{голую форму} некоторого познания, что логика абстрагируется
от всякого \emph{содержания}, и так называемая вторая
\emph{составная часть} всякого познания, \emph{материя}, должна быть
дана откуда-то извне, что, следовательно, логика, от которой
эта материя совершенно независима, может только
указать формальные условие истинного познания, но не
может содержать самое реальную истину, не может даже
быть \emph{путем} к реальной истине, так как именно суть истины,
содержание, находится вне ее.

Но, во-первых, неудачно уже утверждение, что логика
абстрагируется от всякого \emph{содержания}, что она только
учит правилам мышления, не имея возможности вдаваться
в рассмотрение мыслимого и его характера. В самом
деле, если, как утверждают, ее предмет~"--- мышление
и правила мышления, то она непосредственно в них имеет
свое, ей лишь свойственное содержание; в них она имеет
также и вторую составную часть познания, некую материю,
характер которой ее интересует.

Во-вторых, вообще представления, на которых до сих
пор основывалось понятие логики, отчасти уже сошли
со сцены, отчасти им пора полностью исчезнуть, пора,
чтобы понимание этой науки исходило из более высокой
точки зрения и чтобы она приобрела совершенно измененный
вид.

Понятие логики, которого придерживались до сих пор,
основано на раз навсегда принятом обыденным сознанием
предположении о раздельности \emph{содержания} познания
и его \emph{формы}, или, иначе сказать, \emph{истины} и \emph{достоверности}.
Предполагается, \emph{во-первых}, что материя познавания существует
сама по себе вне мышления как некий готовый
мир, что мышление, взятое само по себе, пусто, что оно
примыкает к этой материи как некая форма извне, наполняется
ею, лишь в ней обретает некоторое содержание
и благодаря этому становится реальным познанием.

\emph{Во-вторых}, эти две составные части (ибо предполагается,
что они находятся между собой в отношении составных
частей и познание составляется из них механически
или в лучшем случае химически)\endnotemark{} находятся,
согласно этому воззрению, в следующей иерархии: объект
есть нечто само по себе завершенное, готовое,
нисколько не нуждающееся для своей действительности
в мышлении, тогда как мышление есть нечто ущербное,
которому еще предстоит восполнить себя в некоторой
материи, и притом оно должно сделать себя адекватным
своей материи в качестве мягкой неопределенной формы.
Истина есть соответствие мышления предмету, и для
того чтобы создать такое соответствие~"--- ибо само по
себе оно не дано как нечто наличное;~"--- мышление должно
подчиняться предмету, сообразоваться с ним.

\endnotetext{
  В томе 3 <<Науки логики>> Гегель дает развернутое определение этим
  понятиям (раздел второй, главы первая и вторая).
}

\emph{В-третьих}, так как различие материи и формы, предмета
и мышления не оставляется в указанной туманной
неопределенности, а берется более определенно, то каждая
из них есть отделенная от другой сфера. Поэтому
мышление, воспринимая и формируя материю, не выходит
за свои пределы, воспринимание ее и сообразование
с ней остается видоизменением его самого, и от этого
оно не становится своим иным; а сознающий себя процесс
определения уж во всяком случае принадлежит
лишь исключительно мышлению. Следовательно, даже в
своем отношении к предмету оно не выходит из самого
себя, не переходит к предмету; последний остается как
вещь в себе просто чем-то потусторонним мышлению.

Эти взгляды на отношение между субъектом и объектом
выражают собой те определения, которые составляют
природу нашего обыденного сознания, охватывающего
лишь явления. Но когда эти предрассудки переносятся
в область разума, как будто и в нем имеет место то же
самое отношение, как будто это отношение истинно само
по себе, они представляют собой заблуждения, опровержением
которых, проведенным через все части духовного
и природного универсума, служит философия или, вернее,
они суть заблуждения, от которых следует освободиться
до того, как приступают к философии, так как они преграждают
вход в нее.

В этом отношении прежняя метафизика имела более
возвышенное понятие о мышлении, чем то, которое сделалось
ходячим в новейшее время. А именно она исходила
из того, что действительно истинное (das wahrhaft
Wahre) в предметах~"--- это то, чт\'о познается мышлением
о них и в них; следовательно, действительно истинны
не предметы в своей непосредственности, а лишь предметы,
возведенные в форму мышления, предметы как
мыслимые. Эта метафизика, стало быть, считала, что
мышление и определения мышления не нечто чуждое
предметам, а скорее их сущность, иначе говоря, что \emph{вещи}
и \emph{мышление} о них сами по себе соответствуют друг
другу (как и немецкий язык выражает их сродство)\endnotemark{},
что мышление в своих имманентных определениях и
истинная природа вещей составляют одно содержание.

\endnotetext{
  Dinge (вещи) и Denken (мышление) имеют некоторое
  сходство в своем звучании и начертании, на что Гегель и намекает.
}

Но философией овладел \emph{рефлектирующий} рассудок.
Мы должны точно знать, чт\'о означает это выражение,
которое часто употребляется просто как эффектное
словечко (Schlagwort). Под ним следует вообще понимать
абстрагирующий и, стало быть, разделяющий рассудок,
который упорствует в своих разделениях. Обращенный
против разума, он ведет себя как \emph{обыкновенный здравый
смысл} и отстаивает свой взгляд, согласно которому истина
покоится на чувственной реальности, мысли суть
\emph{только} мысли в том смысле, что лишь чувственное восприятие
сообщает им содержательность (Gehalt) и реальность,
а разум, поскольку он остается сам по себе,
порождает лишь химеры\endnotemark{}. В этом отречении разума от
самого себя утрачивается понятие истины, разум ограничивают
познанием только субъективной истины, только
явления, только чего-то такого, чему не соответствует
природа самой вещи; \emph{знание} низведено до уровня \emph{мнения}.

\endnotetext{
  Ср. замечание Гегеля об <<экзотерическом учении кантовской
  философии>> в начале Предисловия к первому изданию.
}

Однако это направление, принятое познанием и представляющееся
потерей и шагом назад, имеет более глубокое
основание, на котором вообще покоится возведение
разума в более высокий дух новейшей философии. А
именно основание указанного, ставшего всеобщим, представления
следует искать в понимании того, что определения
рассудка \emph{необходимо сталкиваются} с самими собой.~"---
Уже названная нами рефлексия заключается в
том, что выходят \emph{за пределы} конкретно непосредственного
и \emph{определяют} и \emph{разделяют} его. Но равным образом
она должна выходить и \emph{за пределы} этих своих \emph{разделяющих}
определений, и прежде всего \emph{соотносить} их.
В стадии (auf dem Standpunkte) этого соотнесения выступает
наружу их столкновение. Это осуществляемое
рефлексией соотнесение само по себе есть дело разума;
возвышение над указанными определениями, которое
приходит к пониманию их столкновения, есть большой
отрицательный шаг к истинному понятию разума. Но
это не доведенное до конца понимание приводит к ошибочному
взгляду, будто именно разум впадает в противоречие
с собой; оно не признает, что противоречие как
раз и есть возвышение разума над ограниченностью
рассудка и ее устранение. Вместо того чтобы сделать
отсюда последний шаг вверх, познание неудовлетворительности
рассудочных определений отступает к чувственному
существованию, ошибочно полагая, что в нем оно
найдет устойчивость и согласие. Но так как, с другой
стороны, это познание знает себя как познание только
явлений, то оно тем самым соглашается, что чувственное
существование неудовлетворительно, но вместе с тем
предполагает, что, хотя вещи в себе и не познаются,
однако внутри сферы явлений познание правильное; как
будто различны только \emph{роды предметов}, и один род предметов,
а именно вещи в себе, не познается, другой же
род предметов, а именно явления, познается. Это похоже
на то, как если бы мы приписывали кому-нибудь правильное
уразумение, но при этом прибавили бы, что он,
однако, способен уразуметь не истинное, а только ложное.
Так же как это было бы нелепо, столь же нелепо
истинное познание, не познающее предмета, как он есть
в себе.

\emph{Критика форм рассудка} привела к указанному выше
выводу, что эти формы не \emph{применимы к вещам в себе}.
Это может иметь только тот смысл, что эти формы суть
в самих себе нечто неистинное. Но так как все еще считают
их значимыми для субъективного разума и для
опыта, то критика ничего не изменила в них самих, а
оставляет их для субъекта в том же виде, в каком они
прежде имели значение для объекта. Но если они недостаточны
для познания вещи в себе, то рассудок, которому,
как утверждают, они принадлежат, еще в меньшей
степени должен был бы принимать их и довольствоваться
ими. Если они не могут быть определениями \emph{вещи в
себе}, то они еще в меньшей степени могут быть определениями
рассудка, за которым мы должны были бы
признать по крайней мере достоинство некоторой вещи
в себе. Определения конечного и бесконечного одинаково
сталкиваются между собой, будем ли мы применять их
к времени и пространству, к миру или они будут признаны
определениями внутри духа, точно так же как черное
и белое все равно образуют серое, смешаем ли мы их
на стене или только на палитре. Если наше представление
о \emph{мире} расплывается, когда мы на него переносим
определения бесконечного и конечного, то сам \emph{дух}, содержащий
в себе эти два определения, должен в еще большей
мере оказаться чем-то внутренне противоречивым,
чем-то расплывающимся. Свойство материи или предмета,
к которым мы стали бы их применять или в которых они
находятся, не может составлять [здесь] какое-либо различие,
ибо предмет внутренне противоречив только из-за
указанных определений и согласно им.

Указанная критика, стало быть, отдалила формы
объективного мышления только от вещи, но оставила их
в субъекте в том виде, в каком она их нашла. А именно,
она не рассмотрела этих форм, взятых сами по себе, со
свойственным им содержанием, а прямо заимствовала их
при помощи лемм из субъективной логики. Таким образом,
не было речи о выведении их из (an) них самих
или хотя бы о выведении их как субъективно-логических
форм, тем более не было речи о диалектическом их рассмотрении.

Более последовательно проведенный трансцендентальный
идеализм признал ничтожность сохраненного
еще критической философией призрака \emph{вещи в себе}, этой
абстрактной, оторванной от всякого содержания тени,
и он поставил себе целью окончательно его уничтожить\endnotemark{}.
Кроме того, эта философия положила начало попытке
дать разуму развернуть свои определения из самого себя.
Но субъективная позиция этой попытки не позволила
завершить ее. В дальнейшем отказались от этой позиции,
а с ней и от указанной начатой попытки и от разработки
чистой науки.

\endnotetext{
  Гегель имеет в виду философию Фихте.
}

Но совершенно не принимая во внимание метафизического
значения, рассматривают то, чт\'о обычно понимают
под логикой. Эта наука в том состоянии, в каком
она еще находится, лишена, правда, того содержания,
которое признается в обыденном сознании реальностью
и некоей истинной вещью (Sache). Однако не поэтому
она формальная наука, лишенная всякой содержательной
истины. В том материале, который в ней не находят и
отсутствием которого обычно объясняют ее неудовлетворительность,
мы, впрочем, не должны искать сферу истины.
Причина бессодержательности логических форм
скорее только в способе их рассмотрения и трактовки.
Так как они в качестве застывших определений лишены
связи друг с другом и не удерживаются в органическом
единстве, то они мертвые формы и в них не обитает дух,
составляющий их живое конкретное единство. Но тем
самым им недостает подлинного содержания (Inhalt)~"---
материи, которая была бы в самой себе содержанием
(Gehalt). Содержание, которого мы не находим в логических
формах, есть не что иное, как некоторая прочная
основа и сращение (Konkretion) этих абстрактных определений,
и обычно ищут для них такую субстанциальную
сущность вне логики. Но сам логический разум и
есть то субстанциальное или реальное, которое удерживает
в себе все абстрактные определения, и он есть
их подлинное, абсолютно конкретное единство. Нет,
следовательно, надобности далеко искать то, чт\'о обычно
называют материей. Если логика, как утверждают, лишена
содержания, то это вина не предмета логики, а только
способа его понимания.

Это размышление приводит нас к необходимости указать
ту точку зрения, с которой мы должны рассматривать
логику, поскольку эта точка зрения отличается от
прежней трактовки этой науки и есть единственно истинная
точка зрения, которой она впредь должна придерживаться
раз и навсегда.

В <<Феноменологии духа>> я представил сознание в
его поступательном движении от первой непосредственной
противоположности между ним и предметом до
абсолютного знания. Этот путь проходит через все формы
\emph{отношения сознания к объекту} и имеет своим
результатом \emph{понятие науки}. Это понятие, следовательно
(независимо от того, что оно возникает в рамках самой
логики), не нуждается здесь в оправдании, так как оно
его получило уже там; и оно не может иметь никакого
другого оправдания, кроме этого его порождения сознанием,
для которого все его собственные образы разрешаются
в это понятие, как в истину. Резонерское обоснование
или разъяснение понятия науки может самое большее
привести лишь к тому, что понятие станет объектом
представления и о нем будут получены исторические
сведения; но дефиниция науки, или, точнее, логики,
имеет свое \emph{доказательство} исключительно в указанной
необходимости ее происхождения. Та дефиниция, которой
какая-либо наука начинает абсолютно с самого начала,
не может содержать ничего другого, кроме определенного
корректного выражения того, что как \emph{известное}
и \emph{общепризнанное представляют себе} в качестве предмета
и цели этой науки. Что в качестве таковых представляют
себе именно это, [а не другое], это есть историческое
уверение, относительно которого можно сослаться
лишь на то или иное признанное или, собственно говоря,
можно только в виде просьбы предложить, чтобы считали
то или иное признанным. Вовсе не удивительно, что
один отсюда, другой оттуда приводит какой-нибудь случай
или пример, показывающий, что под таким-то выражением
нужно понимать еще нечто большее и иное и
что, стало быть, в его дефиницию следует включить еще
одно более частное или более общее определение и с
этим должна быть согласована и наука.~"--- При этом от
резонерства зависит, до какой границы и в каком объеме
те или иные определения должны быть включены или
исключены; само же резонерство имеет перед собой на
выбор самые многообразные и самые различные воззрения,
застывшее определение которых может в конце
концов давать только произвол. При этом способе начинать
науку с ее дефиниции нет и речи о потребности
показать \emph{необходимость ее предмета} и, следовательно,
также ее самой.

Итак, в настоящем произведении понятие чистой науки
и его дедукция берутся как предпосылка постольку,
поскольку феноменология духа есть не что иное, как дедукция
его. Абсолютное знание есть \emph{истина} всех способов
сознания, потому что, как показало [описанное в <<Феноменологии
духа>>] движение сознания, лишь в абсолютном
знании полностью преодолевается разрыв между
\emph{предметом} и \emph{достоверностью самого себя}, и истина стала
равной этой достоверности, так же как и эта достоверность
стала равной истине.

Чистая наука, стало быть, предполагает освобождение
от противоположности сознания [и его предмета]. Она
содержит в себе \emph{мысль, поскольку мысль есть также и вещь}
(Sache) \emph{сама по себе}, или \emph{содержит вещь самое по себе},
поскольку вещь \emph{есть также и чистая мысль}. В качестве
\emph{науки} истина есть чистое развивающееся самосознание
и имеет образ самости, [что выражается в том], что \emph{в себе
и для себя сущее есть осознанное} (gewusster) \emph{понятие},
а \emph{понятие, как таковое, есть в себе и для себя сущее}.
Это объективное мышление и есть \emph{содержание} чистой
науки. Она поэтому в такой мере не формальна, в такой
мере не лишена материи для действительного и истинного
познания, что скорее лишь ее содержание и есть абсолютно
истинное или (если еще угодно пользоваться
словом <<материя>>) подлинная материя, но такая материя,
для которой форма не есть нечто внешнее, так как
эта материя есть скорее чистая мысль и, следовательно,
есть сама абсолютная форма. Логику, стало быть, следует
понимать как систему чистого разума, как царство
чистой мысли. \emph{Это царство есть истина, какова она без
покровов, в себе и для себя самой}. Можно поэтому выразиться
так: это содержание есть \emph{изображение бога, каков
он в своей вечной сущности до сотворения природы и
какого бы то ни было конечного духа}.

Анаксагор восхваляется как тот, кто впервые высказал
ту мысль, что \emph{нус, мысль}, есть первоначало (Prinzip)
мира, что необходимо определить сущность мира как
мысль. Он этим положил основу интеллектуального
воззрения на Вселенную, чистой формой которого должна
быть \emph{логика}. В ней мы имеем дело не с мышлением \emph{о}
чем-то таком, чт\'о лежало бы в основе и существовало
бы особо, вне мышления, не с формами, которые будто
бы дают только \emph{признаки} истины; необходимые формы
и собственные определения мышления суть само содержание
и сама высшая истина.

Для того чтобы представление по крайней мере понимало,
в чем дело, следует отбросить мнение, будто истина
есть нечто осязаемое. Подобную осязаемость вносят, например,
даже еще в платоновские идеи, имеющие бытие
в мышлении бога, [толкуя их так], как будто они существующие
вещи, но существующие в некоем другом мире
или области, вне которой находится мир действительности,
обладающий отличной от этих идей субстанциальностью,
реальной только благодаря этому отличию.
Платоновская идея есть не что иное, как всеобщее, или,
говоря более определенно, понятие предмета; лишь
в своем понятии нечто обладает действительностью; поскольку
же оно отлично от своего понятия, оно перестает
быть действительным и есть нечто ничтожное; осязаемость
и чувственное вовне-себя-бытие принадлежат этой
ничтожной стороне.~"--- Но, с другой стороны, можно сослаться
на собственные представления обычной логики;
в ней ведь принимается, что, например, дефиниции содержат
не определения, относящиеся лишь к познающему
субъекту, а определения предмета, составляющие его
самую существенную, неотъемлемую природу. Или [другой
пример]: когда умозаключают от данных определений
к другим, считают, что выводы не нечто внешнее и чуждое
предмету, а скорее принадлежат самому предмету,
что этому мышлению соответствует бытие.~"--- Вообще при
употреблении форм понятия, суждения, умозаключения,
дефиниции, деления т.\,д. исходят из того, что они формы
не только сознающего себя мышления, но и предметного
смысла (Verstandes).~"--- <<Мышление>> есть выражение, которое
содержащееся в нем определение приписывает
преимущественно сознанию. Но так как говорят, что
\emph{в предметном мире есть смысл} (Verstand), \emph{разум}, что
дух и природа имеют \emph{всеобщие законы}, согласно которым
протекает их жизнь и совершаются их изменения, то
признают, что определения мысли обладают также
и объективными ценностью и существованием.

Критическая философия, правда, уже превратила
\emph{метафизику} в \emph{логику}; однако подобно позднейшему
идеализму\endnote{Имеется в виду субъективный идеализм Фихте.}
она из страха перед объектом придала, как мы
уже сказали выше, логическим определениям преимущественно
субъективное значение; в то же время они
тем самым остаются обремененными объектом, которого
они избегали, и в них оставались как нечто потустороннее
вещь в себе\endnote{В философии Канта.},
бесконечный импульс\endnote{В философии Фихте.}. Но освобождение
от противоположности сознания [и его предмета],
которое наука должна иметь возможность предположить,
возвышает определения мысли над этим робким, незавершенным
взглядом и требует, чтобы их рассматривали
такими, каковы они в себе и для себя, без такого рода
ограничения и отношения, требует, чтобы их рассматривали
как логическое, как чисто разумное.

Кант в одном месте\endnotemark{} считает счастьем для логики,
а именно для того агрегата определений и положений,
который обычно носит название логики, то, что она
сравнительно с другими науками достигла столь раннего
завершения; со времени Аристотеля она, по его словам,
не сделала ни одного шага назад, но также и ни одного
шага вперед; последнего она не сделала потому, что она,
судя по всему, казалась законченной и завершенной.~"---
Но если со времени Аристотеля логика не подверглась
никаким изменениям,~"--- и в самом деле при рассмотрении
новых учебников логики мы убеждаемся, что
изменения сводятся часто больше всего лишь к сокращениям,~"---
то мы отсюда должны сделать скорее тот вывод,
что она тем более нуждается в полной переработке; ибо
двухтысячелетняя непрерывная работа духа должна
была ему доставить более высокое сознание о своем мышлении
и о своей чистой сущности в самой себе. Сравнение
образов, до которых поднялись дух практического и религиозного
миров и дух науки во всякого рода реальном
и идеальном сознании, с образом, который носит логика
(его сознание о своей чистой сущности), являет столь
огромное различие, что даже при самом поверхностном
рассмотрении не может не бросаться тотчас же в глаза,
что это последнее сознание совершенно не соответствует
тем взлетам и недостойно их.

\endnotetext{
  <<Критика чистого разума>>, предисловие ко второму изданию,
  стр.\,VIII (\emph{И. Кант}. Сочинения в шести томах, т.\,3. М., 1964,
  стр.\,82. Все дальнейшие цитаты из сочинений Канта даны по
  этому изданию).
}

И в самом деле, потребность в преобразовании логики
чувствовалась давно. Следует сказать, что в той форме
и с тем содержанием, с каким логика излагается в учебниках,
она сделалась предметом презрения. Ее еще тащат
за собой больше из-за смутного чувства, что совершенно
без логики не обойтись, и из-за сохранившегося еще привычного,
традиционного представления о ее важности,
нежели из убеждения, что то обычное содержание и занятие
теми пустыми формами ценны и полезны.

Расширение, которое она получила в продолжение
некоторого времени благодаря [добавлению] психологического,
педагогического и даже физиологического материала,
в дальнейшем почти все признали искажением.
Большая часть этих психологических, педагогических,
физиологических наблюдений, законов и правил все равно,
даны ли они в логике или в какой-либо другой науке,
сама по себе должна представляться очень плоской и тривиальной.
А уж такие, например, правила, что следует
продумывать и подвергать критическому разбору прочитанное
в книгах или слышанное, что тот, кто плохо видит,
должен помочь своим глазам, надевая очки (правила,
дававшиеся в учебниках по так называемой прикладной
логике и притом с серьезным видом разделенные на
параграфы, дабы люди достигли истины),~"--- такие правила
должны казаться излишними всем, кроме разве автора
учебника или преподавателей, не знающих, как растянуть
слишком краткое и мертвенное содержание
логики\endnotemark{}.

\endnotetext{
  Гегель имеет в виду сочинения Христиана Вольфа (1679--1754)
  и его последователей. В первом издании <<Науки логики>>
  (Нюрнберг,~1812) к этому месту имелось следующее примечание:
  <<Одно только что появившееся исследование этой науки~"--- <<Система
  логики>> Фриза (Fries)~"--- возвращается к антропологическим
  основам. Поверхностность представления или мнения самого по
  себе, составляющего исходный пункт этой <<Системы>>, а также
  ее обоснования избавляет меня от труда уделять какое-либо
  внимание этому незначительному произведению (Erscheinung)>>.
  <<Система логики>> Я. Фриза (1773--1843) вышла в 1811\,г.
}

Что же касается этого содержания, то мы уже указали
выше, почему оно так плоско. Его застывшие
определения считаются незыблемыми и ставятся лишь
во внешнее отношение друг с другом. Оттого, что в суждениях
и умозаключениях оперируют главным образом
количественной стороной определений и исходят из нее,
все оказывается покоящимся на внешнем различии, на
голом сравнении, все становится совершенно аналитическим
способом [рассуждения] и лишенным понятия
вычислением. Дедукция так называемых правил и законов,
в особенности законов и правил умозаключения,
немногим лучше, чем перебирание палочек разной длины
для сортирования их по величине или чем детская игра,
состоящая в подборе подгоняемых друг к другу частей
различным образом разрезанных картинок.~"--- Поэтому не
без основания приравнивали это мышление к счету и в
свою очередь счет~"--- к этому мышлению. В арифметике
числа берутся как нечто лишенное понятия, как нечто такое,
что помимо своего равенства или неравенства, т.\,е.
помимо своего совершенно внешнего отношения, не имеет
никакого значения,~"--- берутся как нечто такое, что
ни само по себе, ни в своих отношениях не есть мысль.
Когда мы механически вычисляем, что три четверти,
помноженные на две трети, дают половину, то это действие
содержит примерно столь же много или столь же
мало мыслей, как и соображение о том, возможен ли в
данной фигуре тот или другой вид умозаключения.

Дабы эти мертвые кости логики оживотворились духом
и получили, таким образом, содержимое и содержание,
ее \emph{методом} должен быть тот, который единственно
только и способен сделать ее чистой наукой. В том состоянии,
в котором она находится, нет даже предчувствия
научного метода. Она имеет, можно сказать, форму опытной
науки. Опытные науки для того, чем они должны
быть, нашли свой особый метод, метод дефиниции и классификации
своего материала, насколько это возможно. Чистая
математика также имеет свой метод, который подходит
для ее абстрактных предметов и для количественного
определения, единственно в котором она их рассматривает.
Главное об этом методе и вообще о подчиненном
характере той научности, которая возможна в математике,
я высказал в предисловии к <<Феноменологии духа>>,
но он будет рассмотрен нами более подробно в рамках
самой логики. Спиноза, Вольф и другие впали в соблазн
применить этот метод также и к философии и сделать
внешнее движение лишенного понятая количества движением
понятия, что само по себе противоречиво. До сих
пор философия еще не нашла своего метода. Она смотрела
с завистью на системное построение математики и,
как мы сказали, заимствовала у нее ее метод или обходилась
методом тех наук, которые представляют собой
лишь смесь данного материала, исходящих из опыта положений
и мыслей, или выходила из затруднения тем,
что просто отбрасывала всякий метод. Но раскрытие того,
чт\'о единственно только и может быть истинным методом
философской науки, составляет предмет самой логики,
ибо метод есть осознание формы внутреннего самодвижения
ее содержания. В <<Феноменологии духа>> я дал
образчик этого метода применительно к более конкретному
предмету, к \emph{сознанию}\footnotemark{}. Там я показал формы сознания,
каждая из которых при своей реализации разрешает
(auflöst) в то же время самое себя, имеет своим
результатом свое собственное отрицание,~"--- и тем самым
перешла в некоторую более высокую форму. Единственное,
что \emph{нужно для научного прогресса} и к совершенно
\emph{простому} пониманию чего следует главным образом стремиться,~"---
это познание логического положения о том, что
отрицательное равным образом и положительно или,
иначе говоря, противоречащее себе не переходит в нуль,
в абстрактное ничто, а по существу лишь в отрицание
своего \emph{особенного} содержания, или, другими словами,
такое отрицание есть не отрицание всего, а \emph{отрицание
определенной вещи}, которая разрешает самое себя, стало
быть, такое отрицание есть определенное отрицание и,
следовательно, результат содержит по существу то, из
чего он вытекает; это есть, собственно говоря, тавтология,
ибо в противном случае он был бы чем-то непосредственным,
а не результатом. Так как то, чт\'о получается
в качестве результата, отрицание, есть \emph{определенное отрицание},
то оно имеет некоторое \emph{содержание}. Оно новое
\emph{понятие}, но более высокое, более богатое понятие, чем
предыдущее, ибо оно обогатилось его отрицанием или противоположностью;
оно, стало быть, содержит предыдущее
понятие, но содержит больше, чем только его, и есть
единство его и его противоположности.~"--- Таким путем
должна вообще образоваться система понятий,~"--- и в неудержимом,
чистом, ничего не принимающем в себя
извне движении получить свое завершение.

\footnotetext{
  Позднее же~"--- применительно и к другим конкретным предметам
  и соответственно частям философии.
}

Я, разумеется, не могу полагать, что метод которому
я следовал в этой системе логики или, вернее, которому
следовала в самой себе эта система, не допускает
еще значительного усовершенствования, многочисленных
улучшений в частностях, но в то же время я знаю, что
он единственно истинный. Это само по себе явствует уже
из того, что он не есть нечто отличное от своего предмета
и содержания, ибо именно содержание внутри себя,
\emph{диалектика, которую он имеет в самом себе}, движет вперед
это содержание. Ясно, что нельзя считать научными
какие-либо способы изложения, если они не следуют движению
этого метода и не соответствуют его простому
ритму, ибо движение этого метода есть движение самой
сути дела.

В соответствии с этим методом я напоминаю, что
подразделения и заглавия книг, разделов и глав, данные
в настоящем сочинении, равно как и связанные с ними
объяснения, делаются для предварительного обзора и что
они, собственно говоря, имеют значение лишь с \emph{исторической}
точки зрения. Они не входят в содержание и корпус
науки, а суть сопоставления, произведенные внешней
рефлексией, которая уже ознакомилась со всем изложением
в целом, заранее знает поэтому последовательность
его моментов и указывает их еще до того, как они будут
выведены из самой сути дела.

В других науках такие предварительные определения
и подразделения, взятые сами по себе, также представляют
собой не что иное, как такие внешние указания;
но даже внутри самой науки они не поднимаются выше
такого характера. Даже в логике говорится, например:
<<У логики две главные части, общая часть и методика>>.
А затем в общей части мы без дальнейших объяснений
встречаем такие, скажем, \emph{заголовки}, как <<Законы мышления>>,
и далее \emph{первая глава}: <<О понятиях>>. \emph{Первый раздел}:
<<О ясности понятий>> и т.\,д. Эти определения и подразделения,
даваемые без всякой дедукции и обоснования,
образуют остов системы и всю связь подобных наук.
Такого рода логика видит свое призвание в провозглашении
того, что понятия и истины должны быть \emph{выведены}
из принципов; но когда речь идет о том, чт\'о она
называет методом, нет и намека на мысль о выведении.
Порядок состоит здесь примерно в сопоставлении однородного,
в рассмотрении более простого до [рассмотрения]
сложного и в других внешних соображениях, В отношении
же внутренней необходимой связи дело ограничивается
перечнем определений тех или иных разделов, и
переход осуществляется лишь так, что ставят теперь:
<<Вторая глава>> или пишут: <<Мы переходим теперь
к суждениям>> и т.\,д.

Заглавия и подразделения, встречающиеся в настоящей
системе, сами по себе также не имеют никакого другого
значения, помимо указания на последующее содержание.
Но, кроме того, при рассмотрении самой сути дела
должны найти место \emph{необходимость} связи и \emph{имманентное
возникновение} различий, ибо они входят в собственное
развитие определения понятия.

То, с помощью чего понятие ведет само себя дальше,
это~"--- указанное выше \emph{отрицательное}, которое оно имеет
в самом себе; это составляет подлинно диалектическое.
\emph{Диалектика}, которая рассматривалась как некая обособленная
часть логики и относительно цели и точки зрения
которой господствовало, можно сказать, полное
непонимание, оказывается благодаря этому совсем в другом
положении. \emph{Платоновская} диалектика даже в <<Пармениде>>,
а в других произведениях еще более непосредственно,
с одной стороны, также имеет своей целью
только разбор и опровержение ограниченных утверждений
через них же самих, с другой стороны, вообще имеет
своим результатом ничто. Обычно видят в диалектике
лишь внешнее и отрицательное действие, не относящееся
к самой сути дела, вызываемое только тщеславием
как некоторой субъективной страстью колебать и разлагать
прочное и истинное, или видят в ней по меньшей
мере действие, приводящее к ничто как к тому, чт\'о составляет
тщету диалектически рассматриваемого предмета.

Кант отвел диалектике более высокое место, и это
одна из величайших его заслуг: он освободил ее от видимости
произвола, которая, согласно обычному представлению,
присуща ей, и изложил ее как \emph{необходимую деятельность
разума}. Пока ее считали только умением проделывать
фокусы и вызывать иллюзии, до тех пор просто
предполагалось, что она ведет фальшивую игру и вся
ее сила зиждется на том, что ей удается прикрыть обман,
и выводы, к которым она приходит, получаются хитростью
и представляют собой субъективную видимость.
Диалектические рассуждения Канта в разделе об антиномиях
чистого разума не заслуживают, правда, большой
похвалы, если присмотреться к ним пристальнее, как мы
в дальнейшем это сделаем в настоящем произведении
более обстоятельно; однако общая идея, из которой он
исходил и которой придавал большое значение,~"--- это
\emph{объективность видимости} и \emph{необходимость противоречия},
свойственного \emph{природе} определений мысли; прежде всего,
правда, это касалось того способа, каким разум применяет
эти определения к \emph{вещам в себе}; но ведь именно то,
чт\'о они суть в разуме и по отношению к тому, чт\'о есть
в себе, и есть их природа. Этот результат, \emph{понимаемый
с его положительной стороны}, есть не что иное, как их
внутренняя \emph{отрицательность}, их движущая сама себя
душа, вообще принцип всякой природной и духовной
жизненности. Но так как Кант не идет дальше абстрактно
-отрицательной стороны диалектического, то выводом
оказывается лишь известное утверждение, что разум неспособен
познать бесконечное~"--- странный вывод: сказать,
что так как бесконечное есть разумное, то разум
не способен познать разумное.

В этом диалектическом, как мы его берем здесь, и,
следовательно, в постижении противоположностей в их
единстве, или, иначе говоря, в постижении положительного
в отрицательном, состоит \emph{спекулятивное}. Это важнейшая,
но для еще неискушенной, несвободной способности
мышления труднейшая сторона. Если эта способность
мышления еще не избавила себя от чувственно
конкретных представлений и от резонерства, то она должна
сначала упражняться в абстрактном мышлении,
удерживать понятия в их \emph{определенности} и научиться
познавать, исходя из них. Изложение логики, имеющее
в виду эту цель, должно было бы придерживаться в своем
методе упомянутых выше подразделений, а в отношении
ближайшего содержания~"--- определений, даваемых отдельным
понятиям, не вдаваясь [пока] в диалектическое.
Внешне оно стало бы похожим на обычное изложение
этой науки, впрочем, по содержанию и отличалось бы от
него и все еще служило бы к тому, чтобы упражнять
абстрактное, хотя и не спекулятивное мышление; а ведь
[обычная] логика, которая стала популярной благодаря
психологическим и антропологическим добавлениям, не
достигает даже и этой цели. То изложение логики доставляло
бы уму образ методически упорядоченного целого,
хотя сама душа этого построения~"--- метод,~"--- имеющая
свою жизнь в диалектическом, в нем не обнаруживалась
бы.

Что касается \emph{образования и отношения индивида
к логике}, то я в заключение еще отмечу, что эта наука,
подобно грамматике, выступает в двух видах или имеет
двоякого рода ценность. Она нечто одно для тех, кто
только приступает к ней и вообще к наукам, и нечто
другое для тех, кто возвращается к ней от них. Тот, кто
только начинает знакомиться с грамматикой, находит
в ее формах и законах сухие абстракции, случайные правила
и вообще множество обособленных друг от друга
определений, показывающих лишь ценность и значение
того, чт\'о заключается в их непосредственном смысле;
сначала познание не познает в них ничего кроме них.
Напротив, кто владеет каким-нибудь языком и в то же
время знает и другие языки, которые он сопоставляет
с ним, только тот и может почувствовать дух и образованность
народа в грамматике его языка; эти же правила
и формы имеют теперь для него наполненную содержанием,
живую ценность. Он в состоянии через грамматику
познать выражение духа вообще~"--- логику. Точно
так же тот, кто только приступает к науке, находит сначала
в логике изолированную систему абстракций, ограничивающуюся
самой собой, не захватывающую других
знаний и наук. В сопоставлении с богатством представления
о мире, с реально выступающим содержанием
других наук и в сравнении с обещанием абсолютной
науки раскрыть \emph{сущность} этого богатства, \emph{внутреннюю
природу} духа и мира, \emph{истину}, эта наука в ее абстрактном
виде, в бесцветной, холодной простоте ее чистых
определений кажется скорее исполняющей все что угодно,
только не это обещание, и противостоящей этому богатству
как лишенная содержания. При первом знакомстве
с логикой ее значение ограничивают только ею самой;
ее содержание признается только изолированным
занятием определениями мысли, \emph{наряду} с которым другие
научные занятия имеют собственный самостоятельный
материал и содержание, на которые логическое оказывает
разве что формальное влияние, и притом такое
влияние, которое скорее осуществляется само собой и
в отношении которого можно, конечно, в крайнем случае
обойтись без научной формы и ее изучения. Другие
науки отбросили в целом метод, придерживающийся строгих
правил и дающий ряд дефиниций, аксиом, теорем и
их доказательств и т.\,д.; так называемая естественная
логика приобретает в них силу самостоятельно и обходится
без особого, направленного на само мышление познания.
Кроме того, материал и содержание этих наук,
взятые сами по себе, остаются независимыми от логического
и они более привлекательны и для ощущения, чувства,
представления и всякого рода практических интересов.

Таким образом, логику приходится, конечно, первоначально
изучать как нечто такое, чт\'о мы, правда, понимаем
и постигаем, но в чем мы не находим сначала широты,
глубины и более значительного смысла. Лишь
на основе более глубокого знания других наук логическое
возвышается для субъективного духа не только как
абстрактно всеобщее, но и как всеобщее, охватывающее
собой также богатство особенного, подобно тому как одно
и то же нравоучительное изречение в устах юноши, понимающего
его совершенно правильно, не имеет [для
него] той значимости и широты, которые оно имеет
для духа умудренного житейским опытом зрелого мужа;
для последнего этот опыт раскрывает всю силу заключенного
в таком изречении содержания. Таким образом,
логическое получает свою истинную оценку, когда оно
становится результатом опыта наук. Этот опыт являет
духу это логическое как всеобщую истину, являет его
не как некоторое \emph{особое} знание \emph{наряду} с другими материями
и реальностями, а как сущность всего этого прочего
содержания.

Хотя логическое в начале [его] изучения не существует
для духа в этой сознательной силе, он благодаря этому
изучению не в меньшей мере вбирает в себя ту силу,
которая ведет его ко всякой истине. Система логики~"---
это царство теней, мир простых сущностей, освобожденных
от всякой чувственной конкретности. Изучение этой
науки, длительное пребывание и работа в этом царстве
теней есть абсолютная культура и дисциплина сознания.
Сознание занимается здесь делом, далеким от чувственных
созерцаний и целей, от чувств, от мира представлений,
имеющих лишь характер мнения. Рассматриваемое
со своей отрицательной стороны, это занятие состоит
в недопущении случайности резонирующего мышления
и произвола, выражающегося в том, что задумываются
над вот этими или противоположными им основаниями
и признают их [правильными].

Но главным образом благодаря этому занятию мысль
приобретает самостоятельность и независимость. Она привыкает
вращаться в абстракциях и двигаться вперед
с помощью понятий без чувственных субстратов, становится
бессознательной мощью, способностью вбирать
в себя все остальное многообразие знаний и наук в разумную
форму, схватывать и удерживать их суть, отбрасывать
внешнее и таким образом извлекать из них логическое,
или, что то же самое, наполнять содержанием
всякой истины абстрактную основу логического, ранее
приобретенную посредством изучения, и придавать логическому
ценность такого всеобщего, которое больше уже
не находится как нечто особенное рядом с другим особенным,
а возвышается над всем этим и составляет его
сущность, то, что абсолютно истинно.


%%% Local Variables:
%%% mode: latex
%%% TeX-master: t
%%% End:
