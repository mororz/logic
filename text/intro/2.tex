Из того, что нами было сказано о \emph{понятии} этой науки
и о том, где оно находит свое обоснование, вытекает, что
общее \emph{деление} может быть здесь лишь \emph{предварительным},
может быть указано как будто лишь постольку, поскольку
автор уже знаком с этой наукой и потому в состоянии
здесь заранее указать \emph{исторически}, к каким
основным различиям определит себя понятие в своем
развитии.

Можно, однако, попытаться заранее объяснить в общем
то, чт\'о требуется для \emph{деления}, хотя и для этого
необходимо прибегать к методу, который приобретает
свою полную ясность и обоснование только в рамках
самой науки.~"--- Итак, прежде всего следует напомнить,
что мы здесь исходим из предпосылки, что \emph{деление} должно
находиться в связи с \emph{понятием} или, вернее, заключаться
в нем самом. Понятие не неопределенно, а \emph{определенно}
в самом себе; деление же выражает в \emph{развитом
виде} эту его \emph{определенность}; оно есть его суждение\endnotemark{},
не суждение \emph{о} каком-нибудь внешне взятом предмете,
а процесс суждения, т.\,е. \emph{процесс определения} понятия
в нем же самом. Прямоугольность, остроугольность и т.\,д.,
так же как и равносторонность и т.\,д., по каковым определениям
делят треугольники, заключаются не в определенности
самого треугольника, т.\,е. не в том, чт\'о обычно
называют понятием треугольника, точно так же как те
определения, по которым животных делят на млекопитающих,
птиц и т.\,д., а эти классы~"--- на дальнейшие
роды, заключаются не в том, чт\'о принимают за понятие
животного вообще или за понятие млекопитающего, птицы
и т.\,д. Такие определения берутся из другого источника,
из эмпирического созерцания; они примыкают
к упомянутым выше так называемым понятиям извне.
В философской же трактовке деления само понятие должно
показать себя содержащим источник этого деления.

\endnotetext{
  Einteilung~"--- деление. Urteil~"--- суждение (Ur-teil~"--- буквально
  <<пра-часть>>). Об этимологии Urteil см.: <<Энциклопедия философских
  наук>> \S\,166. \emph{Гегель}. Соч., т.\,1. М., 1929, стр.\,273.
}

Но само понятие логики показано во введении как
результат науки, лежащей по ту сторону ее, и, стало
быть, принимается здесь равным образом как \emph{предпосылка}.
Логика согласно этому определилась как наука
чистого мышления, имеющая своим принципом \emph{чистое
знание}, не абстрактное, а конкретное, живое единство,
полученное благодаря тому, что противоположность между
сознанием о некоем субъективно \emph{для себя сущем} и
сознанием о некоем втором таком же \emph{сущем}~"--- о некоем
объективном,~"--- знают как преодоленную в этом единстве,
знают бытие как чистое понятие в самом себе, а чистое
понятие~"--- как истинное бытие. Следовательно, это те два
\emph{момента}, которые содержатся в логическом. Но их теперь
знают как существующие \emph{нераздельно}, а не~"--- в отличие
от сознания~"--- как \emph{существующие} каждое \emph{также} и \emph{само
по себе}. Только благодаря тому, что их в то же время
знают как \emph{отличные друг от друга} (однако не как сущие
сами по себе), их единство не абстрактно, мертвенно, неподвижно,
а конкретно.

Это единство составляет логический принцип также
и в качестве \emph{стихии}, так что развитие указанного выше
различия, которое сразу же имеется в ней, совершается
только \emph{внутри} этой стихии. В самом деле, так как деление
(Einteilung), как было сказано, есть \emph{суждение}
(Urteil) понятия, полагание уже имманентного ему определения
и, стало быть, его различия, то это полагание
не должно пониматься как новое разложение указанного
конкретного единства на его определения, которые
должны были бы считаться существующими сами по
себе, ибо это было бы здесь бесполезным возвращением
к прежней точке зрения, к противоположности сознания.
Противоположность эта скорее уже преодолена; указанное
единство остается стихией [логического], и из него
уже больше не выходит различение, осуществляемое делением
и вообще развитием. Тем самым определения,
которые прежде (на \emph{пути к истине}), с какой бы точки
зрения их ни определяли, были для себя \emph{сущими}, как,
например, некое субъективное и некое объективное, или
же мышление и бытие, или понятие и реальность, \emph{теперь
в их истине}, т.\,е. в их единстве, низведены на степень
\emph{форм}. Сами они поэтому в своем различии остаются
\emph{в себе} всем понятием в целом, и последнее полагается в делении
только под своими собственными определениями.

Таким образом, все понятие в целом должно рассматриваться,
во-первых, как \emph{сущее} понятие и, во-вторых,
как \emph{понятие}; в первом случае оно \emph{есть} только понятие
\emph{в себе}, понятие реальности или бытия; во втором случае
оно есть понятие как таковое, \emph{для себя сущее} понятие
(каково оно~"--- назовем конкретные формы~"--- в мыслящем
человеке, но также, хотя и не как \emph{сознаваемое}, а тем
более не как понятие, которое \emph{знают}, в ощущающем животном
и в органической индивидуальности вообще; понятием
же \emph{в себе} оно бывает лишь в неорганической
природе).~"--- Согласно этому, логику следовало бы прежде
всего делить на логику \emph{понятия} как \emph{бытия} и понятия
\emph{как понятия}, или, пользуясь обычными, хотя и самыми
неопределенными, а потому и самыми многозначными
выражениями, на \emph{объективную} и \emph{субъективную} логику.

Сообразно же с лежащей в основе стихией единства
понятия в самом себе и, следовательно, нераздельности
его определений, последние, поскольку они \emph{различны},
поскольку понятие полагается в их \emph{различии}, должны
также находиться по крайней мере в \emph{соотношении} друг
с другом. Отсюда получается некая сфера \emph{опосредствования},
понятие как система \emph{рефлективных определений},
т.\,е. как система бытия, переходящего во \emph{внутри-себя}-бытие
понятия, понятие, которое, таким образом, еще не
положено, \emph{как таковое}, для себя, а обременено в то же
время непосредственным бытием как чем-то также внешним
ему. Это~"--- \emph{учение о сущности}, находящееся посредине
между учением о бытии и учением о понятии.~"---
В общем делении нашего логического произведения оно
помещено еще в \emph{объективной} логике, поскольку, хотя
сущность и есть уже внутреннее, но характер \emph{субъекта}
следует непременно сохранить за понятием.

В новейшее время Кант\footnote{Я напоминаю, что в настоящем сочинении я потому так
  часто принимаю в соображение кантовскую философию (некоторым
  это может казаться излишним), что, как бы ни рассматривали
  другие, а также и мы в настоящем сочинении ее более конкретные
  определения и отдельные части изложения, она составляет
  основу и исходный пункт новейшей немецкой философии,
  и эту ее заслугу не могут умалить имеющиеся в ней недостатки.
  Ее следует часто принимать во внимание в объективной логике
  также и потому, что она подвергает тщательному рассмотрению
  важные, \emph{более определенные} стороны логического, между тем как
  позднейшие изложения философии уделяли ему мало внимания
  и нередко только выказывали по отношению к нему грубое, но
  не оставшееся без возмездия, презрение. То философствование,
  которое у нас более всего распространено, \emph{не идет} дальше кантовских
  выводов, согласно которым разум не способен познать
  никакого истинного содержания и в отношении абсолютной истины
  следует отсылать к вере. Но это философствование непосредственно
  начинает с того, что у Канта составляет вывод, и этим
  сразу отбрасывает предшествующие построения, из которых вытекает
  указанный вывод и которые составляют философское познание.
  Кантовская философия служит, таким образом, подушкой
  для лености мысли, успокаивающейся на том, что все уже доказано
  и решено. За познанием и определенным содержанием мышления,
  которых не найти в таком бесплодном и мертвенном
  (trockenen) успокоении, следует поэтому обращаться к указанным
  предшествующим построениям.} противопоставил тому, чт\'о
обычно называлось логикой, еще одну, а именно \emph{трансцендентальную
логику}. То, чт\'о мы здесь назвали \emph{объективной}
логикой, отчасти соответствовало бы тому, чт\'о
у него составляет \emph{трансцендентальную логику}. Он указывает
следующие различия между ней и тем, чт\'о он
называет общей логикой: трансцендентальная логика ($\alpha$)
рассматривает те понятия, которые относятся к \emph{предметам}
a priori и, следовательно, не абстрагируется от всякого
\emph{содержания} объективного познания, или, как он это
выражает иначе, она заключает в себе правила чистого
мышления о каком бы то ни было \emph{предмете} и ($\beta$) в то
же время исследует происхождение нашего познания,
поскольку познание нельзя приписать предметам. Исключительно
на эту вторую сторону направлен философский
интерес Канта. Основная его мысль~"--- это то, что \emph{категории}
следует признать чем-то принадлежащим самосознанию,
как \emph{субъективному} <<Я>>. В силу этого определения
воззрение [Канта] не выходит за пределы сознания
и его противоположности и кроме эмпирической стороны
чувства и созерцания имеет еще нечто такое, что не положено
мыслящим самосознанием и не определено им,~"---
\emph{вещь в себе}, нечто чуждое и внешнее мышлению, хотя
нетрудно усмотреть, что такого рода абстракция, как
\emph{вещь в себе}, сама есть лишь продукт мышления и притом
только абстрагирующего мышления. Если другие
кантианцы\endnotemark{} выразились об определении \emph{предмета} через
<<Я>> в том смысле, что объективирование этого <<Я>> следует
рассматривать как некую первоначальную и необходимую
деятельность сознания, так что в этой первоначальной
деятельности еще нет представления о самом
<<Я>>, каковое представление есть только некое сознание
указанного сознания или даже объективирование этого сознания,
то эта объективирующая деятельность, освобожденная
от противоположности сознания, оказывается при
более тщательном рассмотрении тем, чт\'о можно считать
вообще \emph{мышлением}, как таковым\footnote{Если выражение
  <<объективирующая деятельность>> <<Я>> может напомнить о
  других продуктах духа, например о продуктах \emph{фантазии},
  то следует отметить, что речь идет о том, как определяют
  предмет, поскольку его содержательные моменты \emph{не} принадлежат
  области \emph{чувства} и \emph{созерцания}. Такой предмет есть некая
  \emph{мысль}, и определить его означает отчасти впервые его продуцировать,
  отчасти же, поскольку он нечто предположенное, иметь
  о нем еще другие мысли, мысленно развивать его дальше.}. Но эта деятельность
не должна была бы больше называться сознанием; сознание
заключает в себе противоположность <<Я>> и его
предмета, а этой противоположности нет в указанной
первоначальной деятельности. Название <<сознание>> набрасывает
тень субъективности на эту деятельность еще
больше, чем выражение <<мышление>>, которое, однако,
следует здесь понимать вообще в абсолютном смысле как
мышление \emph{бесконечное}, не обремененное конечностью сознания,
короче говоря, под этим выражением следует
понимать \emph{мышление, как таковое}.

\endnotetext{
  Имеются в виду Фихте и его единомышленники.
}

Так как интерес кантовской философии был направлен
на так называемое \emph{трансцендентальное} в определениях
мысли, то рассмотрение самих этих определений не
привело к содержательным заключениям. Вопрос о том,
чт\'о они такое сами в себе, помимо их абстрактного, одинакового
у всех них отношения к <<Я>>, каковы их определенность
в сравнении друг с другом и их отношение
друг к другу, не был у Канта предметом рассмотрения;
поэтому указанная философия нисколько не способствовала
познанию их природы. Единственно интересное,
имеющее отношение к этому вопросу, мы находим в критике
идей. Но для действительного прогресса философии
было необходимо, чтобы интерес мышления был привлечен
к рассмотрению формальной стороны, <<Я>>, сознания,
как такового, т.\,е. абстрактного отношения некоего субъективного
знания к некоему объекту, чтобы таким образом
было начато познание \emph{бесконечной формы}, т.\,е. понятия.
Однако, чтобы достигнуть этого познания, нужно
было еще отбросить упомянутую выше конечную определенность,
в которой форма представлена как <<Я>>, сознание.
Форма, мысленно извлеченная таким образом
в своей чистоте, содержит в самой себе процесс \emph{определения}
себя, т.\,е. сообщения себе содержания и притом сообщения
себе содержания в его необходимости~"--- в виде
системы определений мысли.

Объективная логика, таким образом, занимает скорее
место прежней \emph{метафизики}, каковая была высившимся
над миром научным зданием, которое должно было быть
воздвигнуто только \emph{мыслями}.~"--- Если примем во внимание
последнюю форму развития этой науки\endnotemark{}, то мы
должны сказать, во-первых, что объективная логика непосредственно
занимает место \emph{онтологии}~"--- той части указанной
метафизики, которая должна была исследовать
природу ens [сущего] вообще; <<ens>> охватывает как \emph{бытие},
\emph{так} и \emph{сущность}, для различения которых немецкий
язык, к счастью, сохранил разные выражения.~"--- Но тогда
объективная логика постольку охватывает и остальные
части метафизики, поскольку метафизика стремилась
постигнуть посредством чистых форм мысли особенные
субстраты, заимствованные ею первоначально из [области]
представления,~"--- душу, мир, бога,~"--- и поскольку
\emph{определения мышления} составляли \emph{суть} ее способа рассмотрения.
Но логика рассматривает эти формы вне связи
с указанными субстратами, с субъектами \emph{представления},
рассматривает их природу и ценность, взятые
сами по себе. Указанная метафизика не сделала этого
и навлекла на себя справедливый упрек в том, что она
пользовалась ими \emph{без критики}, без предварительного исследования,
способны ли они и как они способны быть,
по выражению Канта, определениями вещи в себе или,
вернее сказать, разумного.~"--- Объективная логика есть
поэтому подлинная критика их, критика, рассматривающая
их не сообразно абстрактной форме априорности,
противопоставляя ее апостериорному, а их самих в их
особом содержании.

\endnotetext{
  Гегель имеет в виду метафизику X. Вольфа и его последователей.
}

\emph{Субъективная логика}~"---- это логика \emph{понятия}~"--- сущности,
которая сняла свое отношение к некоторому бытию
или, иначе говоря, к своей видимости и которая теперь
уже не внешняя в своем определении, а есть свободное,
самостоятельное, определяющее себя внутри себя
субъективное, или, вернее, есть сам \emph{субъект}.~"--- Так как
выражение <<субъективное>> приводит к недоразумениям,
поскольку оно может быть понято в смысле чего-то случайного
и произвольного, равно как вообще в смысле
определений, относящихся к форме \emph{сознания}, то не следует
здесь придавать особое значение различию между
субъективным и объективным, которое позднее будет более
подробно разъяснено при изложении самой логики.

Логика, следовательно, хотя и распадается вообще на
\emph{объективную} и \emph{субъективную} логику, все же имеет, точнее,
следующие три части:
\begin{enumerate}[label=\Roman*.]
\item \emph{Логику бытия},
\item \emph{Логику сущности} и
\item \emph{Логику понятия}.
\end{enumerate}

%%% Local Variables:
%%% mode: latex
%%% TeX-master: "../../main"
%%% TeX-engine: xetex
%%% End:
