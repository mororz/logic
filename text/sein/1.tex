Только в новейшее время зародилось сознание, что
трудно найти \emph{начало} в философии, и причина этой трудности,
равно как и возможность устранить ее были предметом
многократного обсуждения. Начало философии
должно быть или чем-то \emph{опосредствованным} или чем-то
\emph{непосредственным}; и легко показать, что оно не может
быть ни тем, ни другим; стало быть, и тот и другой способ
начинать находит свое опровержение.

Правда, \emph{принцип} какой-нибудь философии также
означает некое начало, но не столько субъективное,
сколько \emph{объективное} начало, начало \emph{всех вещей}. Принцип
есть некое определенное содержание~"--- вода, единое,
нус, идея, субстанция, монада\endnotemark{} и т.\,д.; или, если он касается
природы познавания и, следовательно, должен
быть скорее лишь неким критерием, чем неким объективным
определением~"--- мышлением, созерцанием, ощущением,
Я, самой субъективностью,~"--- то и здесь интерес
направлен на определение содержания. Вопрос же о начале,
как таковом, оставляется без внимания и считается
безразличным как нечто субъективное в том смысле, что
дело идет о случайном способе начинать изложение, стало
быть, и потребность найти то, с чего следует начинать,
представляется незначительной по сравнению с потребностью
найти принцип, ибо кажется, что единственно
лишь в нем заключается \emph{главный} интерес, интерес
к тому, чт\'о такое \emph{истина}, чт\'о такое \emph{абсолютное основание}
всего.

\endnotetext{
  Имеются в виду философские учения Фалеса (вода), Парменида
  (единое, или одно), Анаксагора (нус), Платона (идея), Спинозы
  (субстанция), Лейбница (монада).
}

Но нынешнее затруднение с началом проистекает из
более широкой потребности, еще незнакомой тем, для
кого важно догматическое доказательство своего принципа
или скептические поиски субъективного критерия
для опровержения догматического философствования, и
совершенно отрицаемой теми, кто, как бы выпаливая из
пистолета\endnotemark{}, прямо начинает с своего внутреннего откровения,
с веры, интеллектуального созерцания и т.\,д. и
хочет отделаться от \emph{метода} и логики. Если прежнее абстрактное
мышление сначала интересуется только принципом
как \emph{содержанием}, в дальнейшем же развитии вынуждено
обратить внимание и на другую сторону, на способы
\emph{познавания}, то [теперешнее мышление] понимает
также и \emph{субъективную} деятельность как существенный
момент объективной истины, и возникает потребность в
соединении метода с содержанием, \emph{формы с принципом}.
Таким образом, \emph{принцип} должен быть также началом,
а то, чт\'о представляет собой prius для мышления,~"--- \emph{первым}
в \emph{движении} мышления.

\endnotetext{
  Ср. <<Феноменология духа>>: <<\dots и тем вдохновением, которое
  начинает сразу же, как бы выстрелом из пистолета, с абсолютного знания>> (стр.\,14).
}

Здесь мы должны только рассмотреть, как выступает
\emph{логическое} начало. Мы уже указали, что его можно понимать
двояко~"--- как результат, полученный опосредствованно,
или как подлинное начало, взятое непосредственно.
Вопрос, представляющийся столь важным для
нынешней культуры, есть ли знание истины непосредственное,
всецело зачинающее знание, некая вера или
же опосредствованное знание,~"--- этот вопрос не должен
здесь обсуждаться. Поскольку его можно рассматривать
\emph{предварительно}, мы это сделали в другом месте (в моей
<<Энциклопедии философских наук>>, изд. 3-е, <<Предварительное
понятие>>, \S\,61 и сл.). Здесь мы приведем оттуда
лишь следующее замечание: \emph{нет} ничего ни на небе,
ни в природе, ни в духе, ни где бы то ни было, чт\'о не
содержало бы в такой же мере непосредственность, в какой
и опосредствование, так что эти два определения
оказываются \emph{нераздельными} и \emph{неразделимыми}, а указанная
противоположность [между ними]~"--- чем-то ничтожным.
Что же касается \emph{научного рассмотрения}, то в каждом
логическом предложении мы встречаем определения
непосредственности и опосредствования и, следовательно,
рассмотрение их противоположности и их истины. Поскольку
в отношении мышления, знания, познавания эта
противоположность получает более конкретный вид непосредственного
или опосредствованного \emph{знания}, постольку
природа познавания вообще рассматривается в рамках
науки логики, а познание в его дальнейшей конкретной
форме~"--- в науке о духе и феноменологии духа.
Но желать еще \emph{до} науки получить полную ясность относительно
познавания~"--- значит требовать, чтобы оно
рассматривалось \emph{вне} науки; во всяком случае научно
нельзя это сделать \emph{вне} науки, а здесь мы стремимся единственно
лишь к научности.

Начало есть \emph{логическое} начало, поскольку оно должно
быть сделано в стихии свободно для себя сущего мышления,
в \emph{чистом знании}. \emph{Опосредствовано} оно, стало быть,
тем, что чистое знание есть последняя, абсолютная
истина \emph{сознания}. Мы отметили во введении, что \emph{феноменология
духа} есть наука о сознании, изображение
того, что сознание имеет своим результатом \emph{понятие}
науки, т.\,е. чистое знание. Постольку логика имеет своей
предпосылкой науку об охватывающем явления духе,
содержащую и показывающую необходимость точки зрения,
представляющей собой чистое знание, равно как и
его опосредствование вообще, и тем самым дающую доказательство
ее истинности. В этой науке о духе, охватывающем
явления, исходят из эмпирического, \emph{чувственного}
сознания, которое и есть настоящее, \emph{непосредственное}
знание; там же разъясняется, чт\'о верного в этом
непосредственном знании. Другое сознание, как, например,
вера в божественные истины, внутренний опыт, знание
через внутреннее откровение и т.\,д., оказывается
после небольшого размышления очень неподходящим
для того, чтобы его приводить в качестве примера непосредственного
знания. В феноменологии духа непосредственное
сознание есть первое и непосредственное также
и в науке, и, стало быть, служит предпосылкой; в логике
же предпосылкой служит то, чт\'о оказалось результатом
указанного исследования,~"--- идея как чистое знание. \emph{Логика
есть чистая наука}, т.\,е. чистое знание во всем объеме
своего развития. Но эта идея определилась в указанном
результате как достоверность, ставшая истиной, достоверность,
которая, с одной стороны, уже больше не
противостоит предмету, а вобрала его внутрь себя, знает
его в качестве самой себя\endnote{Ср. <<Феноменология духа>>, стр.\,427--428.}
и которая, с другой стороны,
отказалась от знания о себе как о чем-то таком, чт\'о противостоит
предметному и чт\'о есть лишь его уничтожение,
отчуждена от этой субъективности и есть единство со
своим отчуждением\endnote{
  Ср. <<Феноменология духа>>: <<Преодоление предмета сознания
  следует понимать не как одностороннее в том смысле, что он
  оказался возвращающимся в самость, а определеннее~"--- в том
  смысле, что предмет, как таковой, представляется сознанию исчезающим,
  и кроме того еще, что именно отрешение самосознания
  устанавливает вещность и что это отрешение имеет не только
  негативное, но и положительное значение>> (стр.\,422).
}.

Для того чтобы, исходя из этого определения чистого
знания, начало оставалось имманентным науке о чистом
знании, не надо делать ничего другого, как рассматривать
или, вернее, отстранив всякие размышления, всякие
мнения, которых придерживаются вне этой науки, лишь
воспринимать то, \emph{чт\'о имеется налицо}.

Чистое знание как \emph{слившееся} в это \emph{единство}, сняло
всякое отношение к другому и к опосредствованию; оно
есть то, чт\'о лишено различий; это лишенное различий,
следовательно, само перестает быть знанием; теперь
имеется только \emph{простая непосредственность}.

<<Простая непосредственность>> сама есть выражение
рефлексии и имеет в виду отличие от опосредствованного.
В своем истинном выражении простая непосредственность
есть поэтому \emph{чистое бытие}. Подобно тому как \emph{чистое}
знание не должно означать ничего другого, кроме
знания, как такового, взятого совершенно абстрактно, так
и чистое бытие не должно означать ничего другого, кроме
\emph{бытия} вообще; \emph{бытие}~"--- и ничего больше, бытие без
всякого дальнейшего определения и наполнения.

Здесь бытие~"--- начало, возникшее через опосредствование
и притом через опосредствование, которое есть
в то же время снимание самого себя; при этом предполагается,
что чистое знание есть результат конечного знания,
сознания. Но если не делать никакого предположения,
а само начало брать \emph{непосредственно}, то начало будет
определяться только тем, что оно есть начало логики,
мышления, взятого само по себе. Имеется лишь решение,
которое можно рассматривать и как произвол,
а именно решение рассматривать \emph{мышление, как таковое}.
Таким образом, начало должно быть \emph{абсолютным}, или, чт\'о
здесь то же самое, абстрактным, началом; оно, таким
образом, \emph{ничего не} должно \emph{предполагать}, ничем не должно
быть опосредствовано и не должно иметь какое-либо
основание; оно само, наоборот, должно быть основанием
всей науки. Оно поэтому должно быть \emph{чем-то} (ein) всецело
непосредственным или, вернее, лишь \emph{самим} (das)
\emph{непосредственным}. Как оно не может иметь какое-либо
определение по отношению к иному, так оно не может
иметь какое-либо определение внутри себя, какое-либо
содержание, ибо содержание было бы различением и соотнесением
разного, было бы, следовательно, неким опосредствованием.
Итак, начало~"--- \emph{чистое бытие}.

Изложив то, что прежде всего относится лишь к самому
этому наипростейшему, логическому началу, можно
привести еще и другие соображения. Однако они не
столько могут служить разъяснением и подтверждением
данного выше простого изложения (которое само по себе
закончено), сколько вызываются лишь представлениями
и соображениями, которые могут нам мешать еще до
того, как приступим к делу, но с которыми, как и со
всеми другими предрассудками, предшествующими [изучению
науки], должно быть покончено в самой науке, и
поэтому, собственно говоря, здесь следовало бы, указывая
на это, лишь призвать [читателя] к терпению.

Понимание того, что абсолютно истинное есть, несомненно,
результат и что, наоборот, всякий результат
предполагает некое первое истинное, которое, однако,
именно потому, что оно есть первое, не необходимо, если
рассматривать его объективно, и которое с субъективной
стороны не познано,~"--- это понимание привело в новейшее
время к мысли, что философия должна начинать
лишь с чего-то \emph{гипотетически} и \emph{проблематически} истинного
и что поэтому философствование может быть сначала
лишь исканием. Этот взгляд Рейнгольд многократно
отстаивал в последние годы своего философствования, и
необходимо отдать справедливость этому взгляду и признать,
что в его основе лежит истинный интерес к спекулятивной
природе философского \emph{начала}. Разбор этого
взгляда дает в то же время повод предварительно разъяснять
смысл логического развития вообще, ибо указанный
взгляд с самого начала принимает во внимание это
движение вперед. И притом этот взгляд представляет
себе развитие так, что в философии движение вперед
есть скорее возвращение назад и обоснование, только благодаря
которому и делается вывод, что то, с чего начали,
есть не просто принятое произвольно, а в самом деле
есть отчасти \emph{истинное}, отчасти \emph{первое истинное}.

Нужно признать весьма важной мысль (более определенной
она будет в самой логике), что движение вперед
есть \emph{возвращение назад} в \emph{основание}, к \emph{первоначальному}
и \emph{истинному}, от которого зависит то, с чего начинают,
и которое на деле порождает начало.~"--- Так, сознание
на своем пути от непосредственности, которой оно
начинает, приводится обратно к абсолютному знанию
как к своей внутренней \emph{истине}. Это \emph{последнее}, основание,
и есть то, из чего происходит первое, выступившее
сначала как непосредственное.~"--- Так, в еще большей
мере, абсолютный дух, оказывающийся конкретной и последней
высшей истиной всякого бытия, познается как
свободно отчуждающий себя в \emph{конце} развития и отпускающий
себя, чтобы принять образ \emph{непосредственного}
бытия, познается как решающийся сотворить мир, в котором
содержится все то, чт\'о заключалось в развитии,
предшествовавшем этому результату, и чт\'о благодаря
этому обратному положению превращается вместе со
своим началом в нечто зависящее от результата как от
принципа. Главное для науки не столько то, что началом
служит нечто исключительно непосредственное, а то, что
вся наука в целом есть в самом себе круговорот, в котором
первое становится также и последним, а последнее~"---
также и первым.

Поэтому оказывается, с другой стороны, столь же необходимым
рассматривать как \emph{результат} то, во что движение
возвращается как в свое \emph{основание}. С этой точки
зрения первое есть также и основание, а последнее нечто
производное; так как исходят из первого и с помощью
правильных заключений приходят к последнему как
к основанию, то это основание есть результат. Далее,
\emph{поступательное движение} от того, чт\'о составляет начало,
следует рассматривать как дальнейшее его определение,
так что начало продолжает лежать в основе всего последующего
и не исчезает из него. Движение вперед состоит
не в том, что выводится лишь нечто \emph{иное} или совершается
переход в нечто истинно иное, а, поскольку такой переход
имеет место, он снова снимает себя. Таким образом,
начало философии есть наличная и сохраняющаяся
на всех последующих этапах развития основа, есть то,
чт\'о остается всецело имманентным своим дальнейшим
определениям.

Благодаря именно такому движению вперед начало
утрачивает все одностороннее, которое оно имеет в этой
определенности, заключающейся в том, что оно есть нечто
непосредственное и абстрактное вообще; оно становится
чем-то опосредствованным, и линия продвижения
науки тем самым превращается \emph{в круг}. В то же время
оказывается, что то, чт\'о составляет начало, будучи еще
неразвитым, бессодержательным, по-настоящему еще не
познается в начале и что лишь наука, и притом во всем
ее развитии, есть завершенное, содержательное и теперь
только истинно обоснованное познание его.

Но то обстоятельство, что только \emph{результат} оказывается
абсолютным основанием, не означает, что поступательное
движение этого познавания есть нечто предварительное
или проблематическое и гипотетическое движение.
Это движение познавания должно определяться
природой вещей и самого содержания. Указанное выше
начало не есть ни нечто произвольное и принятое лишь
временно, ни нечто предположенное как появляющееся
произвольно и в результате просьбы, относительно чего
впоследствии все же оказывается, что поступили правильно,
сделав его началом. Здесь дело обстоит не так,
как в тех построениях, которые приходится делать для
доказательства геометрической теоремы: что касается таких
построений, то после того, как приведены доказательства,
выясняется, что мы хорошо сделали, что провели
именно эти линии и что затем в самом доказательстве
начали со сравнения этих линий или углов между
собой: от самого проведения этих линий или от сравнения
их между собой это не ясно. Таким образом, в сам\'ой
чистой науке дано \emph{основание} того, что в ней начинают
с чистого бытия. Это чистое бытие есть то единство,
в которое возвращается чистое знание, или же, если еще
считать чистое знание как форму отличным от его единства,
то чистое бытие есть также его содержание. Именно
в этом отношении \emph{чистое бытие}, это абсолютно непосредственное
есть также и абсолютно опосредствованное. Но
столь же существенно, чтобы оно было взято только
в своей односторонности как чисто непосредственное
\emph{именно потому}, что оно здесь берется как начало. Поскольку
оно не было бы этой чистой неопределенностью,
поскольку оно было бы определенным, мы бы его брали
как опосредствованное, уже развитое далее; всякое определенное
содержит некое \emph{иное}, присоединяющееся к чему-то
первому. Следовательно, \emph{природа самог\'о начала} требует,
чтобы оно было бытием и больше ничем. Бытие
поэтому не нуждается для своего вхождения в философию
ни в каких других приготовлениях, ни в каких посторонних
размышлениях или исходных пунктах.

Из того, что начало есть начало философии, также
нельзя, собственно говоря, почерпать какое-либо \emph{более
точное} его \emph{определение} или какое-либо \emph{положительное}
содержание для этого начала. Ибо здесь в сам\'ом начале,
где еще нет самой сути, философия есть пустое слово или
какое-то принятое [как предпосылка] необоснованное
представление. Чистое знание дает лишь следующее отрицательное
определение: начало должно быть \emph{абстрактным}
началом. Поскольку чистое бытие берется как \emph{содержание}
чистого знания, последнее должно отступить
от своего содержания, дать ему действовать самостоятельно
и больше не определять его.~"--- Иначе говоря, так как
чистое бытие следует рассматривать как единство, в котором
знание, достигнув своей высшей точки единения
с объектом, совпадает с ним, то знание исчезло в этом
единстве, ничем не отличается от него и, следовательно,
не оставило для него никакого определения. Да и вне
этого [знания] нет никакого нечто или содержания, которым
можно было бы пользоваться, чтобы, начав с него,
иметь его в качестве более определенного начала.

Но и определение \emph{бытия}, принятое ранее в качестве
начала, можно было бы опустить, так что оставалось бы
лишь требование~"--- иметь некоторое чистое начало. В таком
случае не было бы ничего другого, кроме самог\'о
\emph{начала}, и нам следовало бы посмотреть, чт\'о оно такое.~"---
Эту позицию можно было бы в то же время милостиво
предложить тем, кто, с одной стороны, по каким-то соображениям
недоволен, что начинают с бытия, и еще более
недоволен результатом, к которому приходит это
бытие,~"--- переходом бытия в ничто, а с другой стороны,
вообще не желает знать о каком-либо другом начале
науки, кроме некоего \emph{представления} как \emph{предпосылки}~"---
представления, которое затем \emph{анализируется}, так что результат
такого анализа служит первым определенным
понятием в науке. Также и при этом способе действия
мы не имели бы никакого особого предмета, потому что
начало как начало \emph{мышления} должно быть совершенно
абстрактным, совершенно всеобщим, должно быть просто
формой без всякого содержания; у нас, таким образом,
не было бы ничего другого, кроме представления только
о начале, как таковом. Нам, стало быть, следует лишь
посмотреть, чт\'о мы имеем в этом представлении.

Пока что есть ничто, и должно возникнуть нечто. Начало
есть не чистое ничто, а такое ничто, из которого
должно произойти нечто; бытие, стало быть, уже содержится
и в начале. Начало, следовательно, содержит и то
и другое, бытие и ничто; оно единство бытия и ничто,
иначе говоря, оно небытие, которое есть в то же время
бытие, и бытие, которое есть в то же время небытие.

Далее, бытие и ничто имеются в начале как \emph{различные},
ибо начало указывает на нечто иное; оно небытие,
соотнесенное с бытием как с чем-то иным; начала еще
\emph{нет}, оно лишь направляется к бытию. Следовательно, начало
содержит бытие как такое бытие, которое отдаляется
от небытия, иначе говоря, снимает его как нечто противоположное
ему.

Но, далее, то, чт\'о начинается, уже \emph{есть}, но в такой
же мере его еще и \emph{нет}. Следовательно, противоположности,
бытие и небытие, находятся в нем в непосредственном
соединении, иначе говоря, начало есть их \emph{неразличенное
единство}.

Стало быть, анализ начала дал бы нам понятие единства
бытия и небытия или, выражая это в более рефлектированной
форме, понятие единства различенности и
неразличенности, или, иначе, понятие тождества тождества
и нетождества\endnotemark{}. Это понятие можно было бы рассматривать
как первую, самую чистую, т.\,е. самую абстрактную
дефиницию абсолютного, и оно в самом деле
было бы таковой, если бы дело шло вообще о форме дефиниций
и о наименовании абсолютного. В этом смысле
указанное абстрактное понятие было бы первой дефиницией
этого абсолютного, а все дальнейшие определения
лишь его более определенными и богатыми дефинициями.
Но пусть те, кто потому недоволен \emph{бытием} как началом,
что оно переходит в ничто и что из этого возникает единство
бытия и ничто, подумают, будут ли они более довольны
таким началом, которое начинается с представления
о \emph{начале}, и анализом этого представления, который,
конечно, правилен, но точно так же приводит
к единству бытия и ничто,~"--- пусть подумают, будут ли
они более довольны этим, нежели тем, что в качестве
начала берется бытие.

\endnotetext{
  <<Тождество тождества и нетождества>>~"--- очень характерное
  для Гегеля выражение, которое подчеркивает (в отличие от трактовки
  Шеллингом тождества противоположностей как непосредственного)
  то, что в тождестве противоположных определений
  в снятом виде сохраняется и различие между ними. Это выражение
  встречается уже в ранней работе Гегеля <<Различие между
  философскими системами Фихте и Шеллинга>> (1801) (Hegels
  Werke. Hrsg. von Lasson. Bd. 1, S. 77).
}

Но необходимо сделать еще одно замечание об этом
способе рассмотрения. Указанный анализ предполагает,
что представление о начале известно; таким образом мы
поступили здесь по примеру других наук. Эти другие
науки предполагают существование своего предмета и
предлагают признавать, что каждый имеет о нем одно и
то же представление и может найти в нем приблизительно
те же определения, которые они то тут, то там приводят
и указывают посредством анализа, сравнения и прочих
рассуждений о нем. Но то, чт\'о представляет собой абсолютное
начало, также должно быть чем-то ранее известным;
если оно есть конкретное и, следовательно, многообразно
определенное внутри себя, то это \emph{соотношение},
которое оно есть внутри себя, предполагается чем-то
известным; оно, следовательно, выдается за нечто \emph{непосредственное,
но на самом деле оно не есть таковое}, ибо
оно лишь соотношение различенных [моментов], стало
быть, содержит \emph{опосредствование}. Далее, в конкретном
появляются случайность и произвольность анализа и разных
способов определения. Какие в конце концов получатся
определения, это зависит от того, чт\'о каждый \emph{находит}
уже наличным в своем непосредственном случайном
представлении. Содержащееся в некоем конкретном,
в некоем синтетическом единстве соотношение есть \emph{необходимое}
соотношение лишь постольку, поскольку оно
заранее не находится, а порождено собственным движением
моментов, которое возвращает их в это единство,
движением, представляющим собой противоположность
аналитическому способу рассмотрения, действованию,
внешнему самой вещи, совершающемуся в субъекте.

Это влечет за собой также и следующий, более определенный
вывод: то, с чего следует начинать, не может
быть чем-то конкретным, чем-то таким, чт\'о содержит некое
соотношение \emph{внутри самого себя}. Ибо такое предполагает,
что внутри него имеется некое опосредствование
и переход от некоего первого к некоему другому,
результатом чего было бы конкретное, ставшее простым.
Но начало не должно само уже быть неким первым \emph{и}
неким иным; в том, что есть внутри себя некоторое первое
и некоторое иное, уже содержится совершившееся
продвижение (Fortgegangensein). То, с чего начинают,
само начало, д\'олжно поэтому брать как нечто неподдающееся
анализу, д\'олжно брать в его простой, ненаполненной
непосредственности, следовательно, \emph{как бытие}, как
то, чт\'о совершенно пусто.

Если кто-то выведенный из терпения рассматриванием
абстрактного начала скажет, что нужно начинать не
с начала, а прямо с самой \emph{сути}, то [мы на это ответим],
что суть эта не что иное, как указанное пустое бытие,
ибо, чт\'о такое суть, это должно выясниться именно только
в ходе самой науки и не может предполагаться известным
до нее.

Какую бы иную форму мы ни брали, чтобы получить
другое начало, нежели пустое бытие, это другое начало
все равно будет страдать указанным недостатком. Тем,
кто остается недовольным этим началом, мы предлагаем
самим взяться за решение этой задачи: пусть попробуют
начинать как-нибудь иначе, чтобы при этом избежать
этих недостатков.

Но нельзя совсем не упомянуть об оригинальном начале
философии, приобретшем большую известность в новейшее
время, о начале с <<Я>>\endnote{Имеется в виду философия Фихте.}.
Оно получилось отчасти
на основании того соображения, что из первого истинного
должно быть выведено всё дальнейшее, а отчасти из
потребности, чтобы \emph{первое} истинное было чем-то известным
и, более того, чем-то \emph{непосредственно достоверным}.
Это начало, вообще говоря, не случайное представление,
которое у одного субъекта может быть таким-то, а у другого
иным. В самом деле, <<Я>>, это непосредственное самосознание,
прежде всего само проявляется отчасти как
нечто непосредственное, отчасти как нечто в гораздо более
высоком смысле известное, чем какое-либо иное представление.
Все иное известное, хотя и принадлежит
к <<Я>>, однако еще есть содержание, отличное от него
и тем самым случайное; <<Я>>, напротив, есть простая достоверность
самого себя. Но <<Я>> вообще есть \emph{в то же
время} и нечто конкретное или, вернее, <<Я>> есть самое
конкретное~"--- сознание себя как бесконечно многообразного
мира. Для того чтобы <<Я>> было началом и основанием
философии, требуется обособление этого конкретного,
требуется тот абсолютный акт, которым <<Я>> очищается
от самого себя и вступает в свое сознание как
абстрактное <<Я>>. Но оказывается, что это чистое <<Я>> \emph{не}
есть ни непосредственное, ни то известное, обыденное
<<Я>> нашего сознания, из которого непосредственно и для
каждого человека должна исходить наука. Этот акт был
бы, собственно говоря, не чем иным, как возвышением
до точки зрения чистого знания, при которой исчезает
различие между субъективным и объективным. Но если
требовать, чтобы это возвышение было столь \emph{непосредственным},
то такое требование будет субъективным постулатом.
Для того, чтобы оно оказалось истинным требованием,
следовало бы показать и представить движение
конкретного <<Я>> в нем самом, по его собственной необходимости,
от непосредственного сознания к чистому знанию.
Без этого объективного движения чистое знание, и
в том случае, когда его определяют как \emph{интеллектуальное
созерцание}, являет себя как произвольная точка зрения,
или даже как одно из эмпирических \emph{состояний} сознания,
относительно которого важно решить, не обстоит
ли дело так, что один человек \emph{находит} или может вызвать
его в себе, а другой~"--- нет. Но так как это чистое
<<Я>> должно быть сущностным чистым знанием, чистое
же знание непосредственно не имеется в индивидуальном
сознании, его лишь полагает в нем абсолютный акт
самовозвышения, то теряется как раз то преимущество,
которое, как утверждают, возникает из этого начала философии,
а именно то, что это начало есть нечто безусловно
известное, чт\'о каждый непосредственно находит
в себе и чт\'о он может сделать исходным пунктом дальнейших
размышлений; в своей абстрактной сущностности
указанное чистое <<Я>> есть скорее нечто неизвестное
обыденному сознанию, нечто такое, чего оно не находит
наличным в себе. Тем самым обнаруживается скорее
вред иллюзии, будто речь идет о чем-то известном, о <<Я>>
эмпирического самосознания, между тем как на самом
деле речь идет о чем-то далеком этому сознанию. Определение
чистого знания как <<Я>> заставляет непрерывно
вспоминать о субъективном <<Я>>, об ограниченности которого
следует забыть, и сохраняет представление, будто
положения и отношения, которые получаются в дальнейшем
развитии <<Я>>, содержатся в обыденном сознании
и будто их можно там найти, ведь именно относительно
него их высказывают. Это смешение порождает
вместо непосредственной ясности скорее лишь еще более
кричащую путаницу и полную дезориентацию, а уж
в умах людей посторонних оно вызывало грубейшие недоразумения.


Что же касается, далее, \emph{субъективной} определенности
<<Я>> вообще, то верно, что чистое знание освобождает
<<Я>> от его ограниченного смысла, заключающегося в том,
что в объекте оно имеет свою непреодолимую противоположность.
Но как раз по этой же причине было бы по
меньшей мере \emph{излишне} сохранять еще эту субъективную
позицию и определение чистой сущности как <<Я>>. Следует,
однако, прибавить, что это определение не только
влечет за собой указанную выше вредную двусмысленность,
но, как оказывается при более пристальном рассмотрении,
оно остается и субъективным <<Я>>. Действительное
развитие науки, которая исходит из <<Я>>, показывает,
что объект имеет и сохраняет в ней постоянное
для <<Я>> определение \emph{иного}, что, следовательно, <<Я>>, из
которого исходят, не есть чистое знание, поистине преодолевшее
противоположность сознания, а еще погружено
в явлении.

При этом необходимо сделать еще следующее важное
замечание: если <<Я>> действительно могло бы быть \emph{в себе}
определено как чистое знание или интеллектуальное
созерцание и признано началом, то ведь для науки главное
не то, чт\'о существует \emph{в себе} или \emph{внутренне}, а наличное
бытие внутреннего \emph{в мышлении} и та \emph{определенность},
которую такое внутреннее имеет в этом наличном бытии.
Но то, чт\'о в \emph{начале} науки \emph{имеется} от интеллектуального
созерцания или~"--- если предмет такого созерцания получает
название вечного, божественного, абсолютного,~"--- от
вечного или абсолютного, может быть только первым, непосредственным,
простым определением. Какое бы ему
ни дали более богатое [содержанием] название, чем то,
которое выражает лишь <<бытие>>, во внимание может
быть принято только то, каким образом такого рода абсолютное
входит в \emph{мыслящее} знание и в словесное выражение
этого знания. Интеллектуальное созерцание есть,
правда, решительный отказ от опосредствования и от доказывающей,
внешней рефлексии. Но то, чт\'о оно выражает
помимо простой непосредственности, есть нечто
конкретное, нечто содержащее в себе разные определения.
Однако выражение и изображение такого конкретного
есть, как мы уже указали, опосредствующее движение,
начинающее с \emph{одного} из определений и переходящее
к другому определению, хотя бы это другое и возвратилось
к первому; это~"--- движение, которое в то же время
не должно быть произвольным или ассерторическим.
Поэтому в таком изображении \emph{начинают} не с самог\'о
конкретного, а только с простого непосредственного, от
которого берет свое начало движение. Кроме того, если
делают началом конкретное, то недостает доказательства,
в котором нуждается соединение определений, содержащихся
в конкретном.

Следовательно, если в выражении <<абсолютное>> или
<<вечное>>, или <<бог>> (а самое бесспорное право имел бы
бог~"--- начинать именно с него), если в созерцании их или
мысли о них \emph{имеется больше содержания}, чем в чистом
бытии, то нужно, чтобы то, чт\'о \emph{содержится} в них, лишь
\emph{проникло} в знание мыслящее, а не представляющее; как
бы ни было богато заключающееся в них содержание,
определение, которое \emph{первым} проникает в знание, есть
нечто простое; ибо лишь в простом нет ничего более,
кроме чистого начала; только непосредственное просто,
ибо лишь в непосредственном нет еще перехода от одного
к другому. Итак, что бы ни высказывали о бытии в более
богатых формах представления об абсолютном или боге
или что бы в них ни содержалось, в начале это лишь
пустое слово и только бытие. Это простое, не имеющее
в общем никакого дальнейшего значения, это пустое
есть, стало быть, безусловно начало философии.

Это воззрение само столь просто, что указанное начало,
как таковое, не нуждается ни в каком подготовлении
или дальнейшем введении, и целью этого нашего
предварительного рассуждения о нем могло быть не введение
этого начала, а скорее устранение всего предварительного.


%%%Local Variables:
%%% mode: latex
%%% TeX-master: "../../main"
%%% End:
