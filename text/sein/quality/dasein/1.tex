Наличное бытие есть \emph{определенное} бытие; его определенность
есть \emph{сущая} определенность, \emph{качество}. Своим
качеством \emph{нечто} противостоит иному, оно изменчиво\endnotemark{}
и \emph{конечно}, определено всецело отрицательно не только в
отношении иного, но и в самом себе. Это его отрицание
прежде всего по отношению к конечному нечто есть \emph{бесконечное};
абстрактная противоположность, в которой выступают
эти определения, разрешается в лишенную противоположности
бесконечность, в \emph{для-себя-бытие}.

\endnotetext{Ein Anderes~"--- иное; veränderlich (изменчиво)~"--- буквально:
  способно стать иным.}

Таким образом, исследование наличного бытия распадается
на следующие три раздела:

\begin{enumerate}[label=\Alph*)]
\item \emph{Наличное бытие, как таковое},
\item \emph{Нечто и иное, конечность},
\item \emph{Качественная бесконечность}.
\end{enumerate}


%%% Local Variables:
%%% mode: latex
%%% TeX-master: "../../../main"
%%% End:
