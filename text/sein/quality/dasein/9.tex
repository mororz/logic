<<\emph{В себе}>>, в которое нечто рефлектировано внутри себя
из своего бытия-для-иного, уже не есть абстрактное
<<в себе>>, а как отрицание своего бытия-для-иного опосредствовано
последним, которое составляет, таким образом,
его момент. <<В себе>> есть не только непосредственное
тождество [самого] нечто с собой, но и такое тождество,
благодаря которому нечто есть и \emph{в самом себе} то,
чт\'о оно есть \emph{в себе}; бытие-для-иного есть \emph{в нем}, потому
что <<\emph{в себе}>> есть его снятие, есть [выхождение] \emph{из него}
в себя, но уже и потому, что оно абстрактно, следовательно,
в своем существе обременено отрицанием, бытием-для-иного.
Здесь имеется не только качество и реальность,
сущая определенность, но и \emph{в-себе-сущая} определенность,
и ее развитие состоит в \emph{полагании} ее как этой
рефлектированной в себя определенности.

1. Качество, которое есть <<в себе>> в простом нечто
и сущностно находится в единстве с другим моментом
этого нечто, с \emph{в-нем-бытием}, можно назвать его \emph{определением},
поскольку это слово в более точном его значении
отличают от \emph{определенности} вообще. Определение есть
утвердительная определенность как в-себе-бытие, которому
нечто в своем наличном бытии, противодействуя своей
переплетенности с иным, которым оно было бы определено,
остается адекватным, сохраняясь в своем равенстве
с собой и проявляя это равенство в своем бытии-для-иного.
Нечто \emph{осуществляет} (erfüllt) свое определение\endnotemark{}, поскольку
дальнейшая определенность, многообразно вырастающая
прежде всего благодаря его отношению к иному,
становится его полнотой (Fülle) в соответствии с его
в-себе-бытием. Определение подразумевает, что то, чт\'о
нечто есть \emph{в себе}, есть также и \emph{в нем}.

\endnotetext{Bestimmung имеет двоякое значение — определение и предназначение.}

Мыслящий разум~"--- вот \emph{определение человека}; мышление
вообще есть его простая \emph{определенность}, ею человек
отличается от животного; он есть мышление \emph{в себе},
поскольку мышление отличается и от его бытия-для-иного,
от его собственной природности и чувственности,
которыми он непосредственно связан с иным. Но мышление
есть и \emph{в нем}: сам человек есть мышление, он \emph{налично
сущ} как мыслящий, оно его существование и действительность;
и далее, так как мышление имеется в его наличном
бытии, а его наличное бытие~"--- в мышлении, то
оно \emph{конкретно}, его следует брать имеющим содержание
и наполненным, оно мыслящий разум и таким образом
оно \emph{определение} человека. Но даже это определение
опять-таки дано лишь \emph{в себе} как долженствование, т.\,е.
оно вместе с включенным в его <<в-себе>> наполнением
дано в форме <<в-себе>> вообще, \emph{в противоположность} не
включенному в него наличному бытию, которое в то же
время еще есть внешне противостоящая [ему] чувственность
и природа.

2. Наполнение в-себе-бытия определенностью также
отлично от той определенности, которая есть лишь бытие-для-иного
и остается вне определения. В самом деле, в
области [категорий] качества различия сохраняют даже
в своей снятости непосредственное, качественное бытие в
отношении друг друга. То, чт\'о нечто имеет \emph{в самом себе},
разделяется таким именно образом, и с этой стороны
есть внешнее наличное бытие этого нечто, каковое наличное
бытие есть также \emph{его} наличное бытие, но не принадлежит
его в-себе-бытию. Определенность, таким образом,
есть \emph{свойство}.

Обладая тем или иным свойством, нечто подвергается
воздействию внешних влияний и обстоятельств. Это внешнее
соотношение, от которого зависит свойство, и определяемость
иным представляется чем-то случайным. Но
качество [всякого] нечто в том-то и состоит, чтобы быть
предоставленным этой внешности и обладать некоторым
\emph{свойством}.

Поскольку нечто изменяется, изменение относится
к свойству, которое есть \emph{в} нечто то, чт\'о становится иным.
Само нечто сохраняет себя в изменении, которое затрагивает
только эту непрочную поверхность его инобытия,
а не его определение.

Определение и свойство, таким образом, отличны
друг от друга; со стороны своего определения нечто безразлично
к своему свойству. Но то, чт\'о нечто имеет
\emph{в самом себе}, есть связующий их средний термин этого
силлогизма. Но \emph{бытие-в-нечто} (das Am-Etwas-Sein) оказалось,
напротив, распадающимся на указанные два крайних
термина. Простой средний термин есть \emph{определенность},
как таковая; к ее тождеству принадлежит и определение,
и свойство. Но определение переходит само по
себе в свойство и свойство [само по себе]~"--- в определение.
Это вытекает из предыдущего; связь, говоря более
точно, такова: поскольку то, чт\'о нечто \emph{есть в себе}, есть
также и \emph{в нем}, оно обременено бытием-для-иного; определение,
как таковое, открыто, следовательно, отношению
к иному. Определенность есть в то же время момент, но
вместе с тем содержит качественное различие~"--- она
отличается от в-себе-бытия, есть отрицание [данного] нечто,
другое наличное бытие. Определенность, охватывающая
таким образом иное, соединенная с в-себе-бытием,
вводит инобытие во в-себе-бытие или, иначе говоря, в
определение, которое в силу этого низводится до свойства.~"---
Наоборот, бытие-для-иного, изолированное и положенное
само по себе как свойство, есть в нем то же, что
иное, как таковое, иное в самом себе, т.\,е. иное самого
себя; но в таком случае оно есть \emph{соотносящееся с собой}
наличное бытие, есть, таким образом, в-себе-бытие с некоторой
определенностью, стало быть, \emph{определение}.~"---
Следовательно, поскольку оба должны быть сохранены
друг вне друга, свойство, представляющееся основанным
в некотором внешнем, в ином вообще, \emph{зависит} также и
от определения, и идущий от чуждого процесс определения
в то же время определен собственным, имманентным
определением [данного] нечто. Но, кроме того, свойство
принадлежит к тому, чт\'о нечто есть в себе; вместе со
своим свойством изменяется и нечто.

Это изменение [данного] нечто уже не первое его
изменение исключительно со стороны его бытия-для-иного;
первое изменение было только в себе сущим изменением,
принадлежащим внутреннему понятию; теперь
же изменение есть и положенное в нечто.~"--- Само нечто
определено далее, и отрицание положено как имманентное
ему, как его развитое \emph{внутри-себя-бытие}.

Переход определения и свойства друг в друга~"--- это
прежде всего снятие их различия; тем самым положено
наличное бытие или нечто вообще, а так как оно результат
указанного различия, заключающего в себе также и
качественное инобытие, то имеются два нечто, но не
только вообще иные по отношению друг к другу~"--- в таком
случае это отрицание оказалось бы еще абстрактным
и относилось бы лишь к сравниванию их [между собой]~"---
теперь это отрицание имеется как \emph{имманентное} этим
нечто. Как \emph{налично сущие} они безразличны друг к другу.
Но это утверждение их уже не есть непосредственное,
каждое из них соотносится с самим собой \emph{через посредство}
снятия того инобытия, которое в определении рефлектировано
во в-себе-бытие.

Таким образом, нечто относится к иному \emph{из самого
себя}, ибо инобытие положено в нем как его собственный
момент; его внутри-себя-бытие заключает в себе отрицание,
через посредство которого оно теперь вообще обладает
своим утвердительным наличным бытием. Но от
последнего иное отлично также качественно и, стало быть,
положено вне нечто. Отрицание своего иного есть лишь
качество [данного] нечто, ибо оно нечто именно как это
снятие своего иного. Только этим, собственно говоря,
иное само противопоставляет себя наличному бытию;
иное противопоставляется первому нечто лишь внешне,
иначе говоря, так как они на самом деле находятся во
взаимной связи безусловно, т.\,е. по своему понятию, то
их связь заключается в том, что наличное бытие \emph{перешло}
в инобытие, нечто~"--- в иное и что нечто в той же мере,
что и иное, есть иное. Поскольку же внутри-себя-бытие
есть небытие инобытия, которое в нем содержится, но в
то же время как сущее отлично от него, постольку само
нечто есть отрицание, \emph{прекращение в нем иного}: оно положено
как относящееся к нему отрицательно и тем самым
сохраняющее себя;~"--- это иное, внутри-себя-бытие
[данного] нечто как отрицание отрицания есть его \emph{в-себе-бытие},
и в то же время это снятие дано \emph{в нем} как простое
отрицание, а именно как отрицание им внешнего
ему другого нечто. Именно \emph{одна} их определенность, с
одной стороны, тождественна с внутри-себя-бытием этих
нечто как отрицание отрицания, а с другой, поскольку
эти отрицания противостоят одно другому как другие
нечто, она, исходя из них же самих, смыкает их и точно
так же отделяет их друг от друга, так как каждое из них
отрицает иное; это \emph{граница}.

3. \emph{Бытие-для-иного} есть неопределенная, утвердительная
общность [всякого] нечто со своим иным; в границе
выдвигается \emph{небытие}-для-иного, качественное отрицание
иного, недопускаемого вследствие этого к рефлектированному
в себя нечто. Следует присмотреться к развитию
(Entwicklung) этого понятия, каковое развитие, впрочем,
скорее оказывается запутанностью (Verwicklung) и противоречием.
Противоречие сразу же имеется в том, что граница
как рефлектированное в себя отрицание [данного] нечто
содержит в себе \emph{идеально} моменты нечто и иного, и в то же
время они как различенные моменты положены в сфере
наличного бытия как \emph{реально, качественно различные}.

$\alpha$) Нечто, следовательно, есть непосредственное соотносящееся
с собой наличное бытие и имеет границу прежде
всего как границу в отношении иного; она небытие
иного, а не самого нечто; последнее ограничивает в ней
свое иное.~"--- Но иное само есть некоторое нечто вообще;
стало быть, граница, которую нечто имеет в отношении
иного, есть также граница иного как нечто, граница этого
нечто, посредством которой оно не допускает к себе
первое нечто как \emph{свое} иное, или, иначе говоря, она есть
\emph{небытие этого первого нечто}; таким образом, она есть не
только небытие иного, но и небытие как одного, так и
другого нечто и, значит, небытие [всякого] \emph{нечто} вообще.

Но по своей сущности граница есть также и небытие
иного; таким образом, нечто в то же время \emph{есть} благодаря
своей границе. Будучи ограничивающим, нечто,
правда, низводится до того, что само оно оказывается
ограничиваемым, однако его граница как прекращение
иного в нем в то же время сама есть лишь бытие этого
нечто: \emph{благодаря ей нечто есть то, чт\'о оно есть, имеет в
ней свое качество}.~"--- Это отношение есть внешнее проявление
того, что граница есть простое или \emph{первое} отрицание,
иное же есть в то же время отрицание отрицания,
внутри-себя-бытие [данного] нечто.

Нечто как непосредственное наличное бытие есть, следовательно,
граница в отношении другого нечто, но оно
имеет ее \emph{в самом себе} и есть нечто через ее опосредствование,
которое в той же мере есть его небытие. Граница~"---
это опосредствование, через которое нечто и иное
\emph{и есть и не есть}.

$\beta$) Поскольку же нечто и \emph{есть и не есть} в своей границе
и эти моменты суть некоторое непосредственное, качественное
различие, постольку отсутствие наличного
бытия [данного] нечто и его наличное бытие оказываются
друг вне друга. Нечто имеет свое наличное бытие \emph{вне}
(или, как это также представляют себе, \emph{внутри}) своей
границы; точно так же и иное, так как оно есть нечто,
находится вне ее. Она \emph{середина между ними}, в которой
они прекращаются. Они имеют свое \emph{наличное бытие по
ту сторону} друг друга и \emph{их границы}; граница как небытие
каждого из них есть иное обоих.

В соответствии с таким различием между нечто и его
границей \emph{линия} представляется линией лишь вне своей
границы, точки; \emph{плоскость} представляется плоскостью
вне линии; \emph{тело} представляется телом лишь вне ограничивающей
его плоскости.~"--- Именно с этой стороны граница
схватывается прежде всего представлением, этим
вовне-себя-бытием понятия, и с этой же стороны она
берется преимущественно в пространственных предметах.

$\gamma$) Но, кроме того, нечто, как оно есть вне границы,
есть неограниченное нечто, лишь наличное бытие вообще.
Так оно не отличается от своего иного; оно лишь
наличное бытие, имеет, следовательно, одно и то же определение
со своим иным; каждое из них есть лишь нечто
вообще или, иначе говоря, каждое есть иное; оба суть,
таким образом, \emph{одно и то же}. Но это их сначала лишь
непосредственное наличное бытие теперь положено с
определенностью как границей, в которой оба суть то,
чт\'о они суть, отличные друг от друга. Но она точно так
же, как и наличное бытие, есть \emph{общее} им обоим различие,
их единство и различие. Это двоякое тождество
обоих~"--- наличное бытие и граница~"--- подразумевает, что
нечто имеет свое наличное бытие только в границе и
что, так как и граница и непосредственное наличное бытие
в то же время отрицают друг друга, то нечто, которое
есть только в своей границе, в такой же мере отделяет
себя от самого себя и по ту сторону себя указывает
на свое небытие и выражает свое небытие как свое бытие,
переходя, таким образом, в это бытие. Чтобы применить
это к предыдущему примеру, следует сказать, что по
одному определению нечто есть то, чт\'о оно есть, только
в своей границе; в таком случае \emph{точка} есть граница
\emph{линии} не только в том смысле, что линия лишь прекращается
в точке и как наличное бытие находится вне точки;
\emph{линия} есть граница \emph{плоскости} не только в том смысле,
что плоскость лишь прекращается в линии (это точно
так же применимо к \emph{плоскости} как к границе \emph{тела}),
но и в том смысле, что в точке линия также и \emph{начинается};
точка есть абсолютное начало линии. Даже и в том
случае, когда линию представляют себе продолженной
в обе ее стороны безгранично, или, как обычно выражаются,
бесконечно, точка составляет ее \emph{элемент}, подобно
тому как линия составляет элемент плоскости, а плоскость~"---
элемент тела. Эти \emph{границы} суть \emph{принцип} того,
чт\'о они ограничивают, подобно тому как единица, например
как сотая, есть граница, но также и элемент целой
сотни.

Другое определение~"--- беспокойство, присущее [всякому]
нечто и состоящее в том, что в своей границе, в
которой оно имманентно, нечто есть \emph{противоречие}, заставляющее
его выходить за свои пределы. Так, диалектика
самой точки~"--- это стать линией; диалектика линии~"---
стать плоскостью, диалектика плоскости~"--- стать целокупным
пространством. Вторая дефиниция линии, плоскости
и всего пространства гласит поэтому, что через
\emph{движение} точки возникает линия, через движение линии
возникает плоскость и т.\,д. Но на это \emph{движение} точки,
линии и т.\,д. смотрят как на нечто случайное или как
на нечто такое, чт\'о мы только представляем себе. Однако
такой взгляд опровергается, собственно говоря, уже тем,
что определения, из которых, согласно этой дефиниции,
возникают линии и т.\,д., суть их \emph{элементы} и \emph{принципы},
а последние в то же время суть не что иное, как и их границы;
возникновение, таким образом, рассматривается
не как случайное или лишь представляемое. Что точка,
линия, поверхность сами по себе, противореча себе, суть
начала, которые сами отталкиваются от себя, и что точка,
следовательно, из себя самой, через свое понятие,
переходит в линию, \emph{движется в себе} и заставляет возникнуть
линию и т.\,д.,~"--- это заключено в понятии границы,
имманентной [данному] нечто. Однако само применение
следует рассматривать там, где будем трактовать
о пространстве; чтобы здесь бегло указать на это применение,
скажем, что точка есть совершенно абстрактная
граница, но \emph{в некотором наличном бытии}; последнее берется
здесь еще совершенно неопределенно; оно есть так
называемое абсолютное, т.\,е. абстрактное \emph{пространство},
совершенно непрерывное вне-друг-друга-бытие. Тем, что
граница не абстрактное отрицание, а отрицание в \emph{этом
наличном бытии}, тем, что она \emph{пространственная} определенность,~"---
точка пространственна, представляет собой
противоречие между абстрактным отрицанием и непрерывностью
и, стало быть, совершающийся и совершившийся
переход в линию и т.\,д., ибо на самом деле \emph{нет}
ни точки, ни линии, ни поверхности.

Нечто вместе со своей имманентной границей, полагаемое
как противоречие самому себе, в силу которого
оно выводится и гонится дальше себя, есть \emph{конечное}.


%%% Local Variables:
%%% mode: latex
%%% TeX-master: "../../../../main"
%%% End:
