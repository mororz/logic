Долженствование играло последнее время большую
роль в философии, особенно в том, чт\'о касается морали,
и в метафизике вообще как последнее и абсолютное понятие
о тождестве в-себе-бытия, или соотношения \emph{с самим
собой}, и \emph{определенности}, или границы.

Ты \emph{можешь, потому что ты должен}\endnotemark{}~"--- это выражение,
которое должно было много значить, содержится в
понятии долженствования. Ибо долженствование есть выход
за предел; граница в нем снята, в-себе-бытие долженствования
есть, таким образом, тождественное соотношение
с собой, стало быть, есть абстракция \emph{возможности}
(des Könnens).~"--- Но столь же правильно и обратное: \emph{ты
не можешь именно потому, что ты должен}. Ибо в долженствовании
содержится также и предел как предел; указанный
формализм возможности имеет в этом пределе некоторую
противостоящую себе реальность, некоторое качественное
инобытие, и их соотношение есть противоречие,
следовательно, означает не быть в состоянии или, вернее,
невозможность.

\endnotetext{Из сатирического стихотворения Ф.~Шиллера <<Die Philosophen>>
  (<<Философы>>), в котором он высмеял точку зрения кантианца
  Карла Шмида (1761--1812):

  \begin{verse}
    Auf theoretischem Feld ist weiter nichts mehr zu finden, \\
    Aber der praktische Satz gilt doch: du kannst, denn du sollst!
  \end{verse}

  (<<В области теоретического разума нельзя ничего больше найти,
  но остается в силе положение практического разума: ты можешь,
  потому что ты должен!>>)}

В долженствовании начинается выхождение за конечность,
бесконечность. Долженствование есть то, чт\'о в
дальнейшем развитии оказывается со стороны указанной
невозможности прогрессом в бесконечность.

Мы можем здесь более подробно подвергнуть критике
два предрассудка относительно формы \emph{предела} и \emph{долженствования}.
Во-первых, обычно придают \emph{большое} значение
пределам мышления, разума и т.\,д. и утверждают, что
\emph{невозможно} выйти за эти пределы. В этом утверждении
сказывается отсутствие сознания того, что если нечто определено
как предел, мы тем самым уже вышли за этот
предел. Ибо некоторая определенность, граница, определена
как предел лишь в противоположность к его иному вообще
как к его \emph{неограниченному}; иное некоторого предела
как раз и есть \emph{выход} за этот предел. Камень, металл
не выходят за свой предел, потому что \emph{для них} он не есть
предел. Но если при таких общих положениях рассудочного
мышления, как утверждение о невозможности выйти
за предел, мышление не хочет рассматривать то, чт\'о содержится
в понятии, то можно сослаться на действительность,
в которой подобного рода положения оказываются
самым что ни на есть недействительным. Именно вследствие
того, что мышление \emph{должно} быть чем-то более высоким,
чем действительность, \emph{должно} оставаться вдали от
нее, в более высоких областях, в силу того, следовательно,
что само оно определено как некоторое \emph{долженствование},~"---
именно поэтому оно, с одной стороны, не движется
к понятию, а, с другой, оказывается в такой же мере неистинным
по отношению к действительности, в какой оно
неистинно по отношению к понятию.~"--- Так как камень
не мыслит и даже не ощущает, то его ограниченность не
есть \emph{для него} предел, т.\,е. она не есть в нем отрицание
для ощущения, представления, мышления и т.\,д., которыми
он не обладает. Но даже и камень как некоторое нечто
заключает в себе различие между своим определением,
или своим в-себе-бытием, и своим наличным бытием,
и постольку он тоже выходит за свой предел; понятие,
которое он есть в себе, содержит тождество с его иным.
Если он способное к окислению [химическое] основание, то
он окисляется, нейтрализуется и т.\,д. В окислении, нейтрализации
и т.\,д. его предел~"--- иметь наличное бытие
лишь как [химическое] основание~"--- снимается; он выходит
за этот предел; и точно так же и кислота снимает
свой предел~"--- быть лишь кислотой,~"--- и в ней, равно как
и в щелочном основании, имеется до такой степени \emph{долженствование}~"---
выйти за свой предел, что только силой
их можно заставить оставаться~"--- безводными, т.\,е. в чистом
виде, не нейтральными~"--- кислотой и щелочным основанием.

Но если некоторое существование содержит понятие
не только как абстрактное в-себе-бытие, но и как для себя
сущую целокупность, как влечение, как жизнь, ощущение,
представление и т.\,д., то оно само из самого себя осуществляет
[стремление] быть за своим пределом и выходить
за свой предел. Растение выходит за предел~"--- быть
зародышем, и точно так же за предел~"--- быть цветком, плодом,
листом; зародыш становится развитым растением,
цветок отцветает и т.\,д. То, чт\'о ощущает в пределе голода,
жажды и т.\,д., есть стремление выйти за этот предел,
и оно осуществляет этот выход. Оно ощущает \emph{боль}, и
ощущение боли есть прерогатива ощущающей природы.
В его самости (Selbst) есть некоторое отрицание, и это
отрицание определено в его чувстве \emph{как некоторый предел}
именно потому, что ощущающее [существо] обладает
чувством своей \emph{самости}, которая есть целокупность, находящаяся
за пределом указанной определенности. Если бы
оно не находилось за пределом этой определенности, оно
не ощущало бы ее как свое отрицание и не испытывало
бы боли.~"--- Но разум, мышление не может, дескать, выйти
за предел, он, который есть \emph{всеобщее}, само по себе находящееся
за пределом особенности, \emph{как таковой}, т.\,е.
\emph{всякой} особенности, есть лишь выход за предел.~"--- Правда,
не всякий выход за предел и не всякое нахождение за
пределом есть истинное освобождение от него, истинное
утверждение; уже само долженствование есть такое несовершенное
выхождение [за предел], есть вообще абстракция.
Но указания на совершенно абстрактное всеобщее
достаточно, чтобы противостоять такому же абстрактному
заверению, будто нельзя выйти за предел, или, пожалуй,
достаточно уже указания на бесконечное вообще,
чтобы противостоять заверению, будто нельзя выйти за
пределы конечного.

Можно при этом упомянуть об одном кажущемся остроумным
замечании Лейбница, что если бы магнит обладал
сознанием, то он считал бы свое направление к северу
определением своей воли, законом своей свободы\endnotemark{}. Скорее
верно другое. Если бы магнит обладал сознанием и, значит,
волей и свободой, то он был бы мыслящим, тем самым
пространство было бы для него как \emph{всеобщее} пространство
объемлющим \emph{все} направления, и потому \emph{одно}
направление к северу было бы скорее пределом для его
свободы, так же как для человека быть удерживаемым
на одном месте есть предел, а для растения~"--- нет.

\endnotetext{Лейбниц приводит этот пример с магнитной стрелкой
  в <<Теодицее>> (часть I, \S\,50). Замечание по поводу этого примера
  см. у Г.\,В.~Плеханова в работе <<К вопросу о развитии монистического
  взгляда на историю>> (<<Избранные философские произведения
  в пяти томах>>, т.\,I.~М., 1956, стр.\,591--592).}

С другой стороны, \emph{долженствование} есть выхождение
за предел, но такое, которое само есть лишь \emph{конечное выхождение}.
Оно имеет поэтому свое место и свою силу в
области конечного, где оно твердо держится в-себе-бытия
против ограниченного и утверждает его как правило и
сущностное против ничтожного. Долг (die Pflicht) есть
\emph{долженствование}, обращенное против отдельной воли, против
эгоистического вожделения и произвольного интереса;
воле, поскольку она в своей подвижности может изолироваться
от того, чт\'о истинно, напоминают о нем как
о некотором долженствовании. Те, кто ставит долженствование
[как принцип] морали так высоко и полагает, будто
непризнание долженствования чем-то последним и
истинным приводит к разрушению нравственности, равно
как резонеры, рассудок которых доставляет себе постоянное
удовлетворение тем, что он имеет возможность выставлять
против всего существующего какое-нибудь долженствование
и тем самым свое притязание на лучшее
знание, и которые поэтому в такой же мере не желают,
чтобы их лишили долженствования, не замечают, что для
интересующих их конечных областей жизни долженствование
полностью признается.~"--- Но в самой действительности
вовсе не обстоит так печально с разумностью и законом,
чтобы они только были \emph{долженствующими} быть,~"---
дальше этого не идет лишь абстракция в-себе-бытия,~"--- и
точно так же неверно, что долженствование, взятое в самом
себе, постоянно и~"--- чт\'о то же самое~"--- что конечность
абсолютна. Кантовская и фихтевская философии
выдают \emph{долженствование} за высший пункт разрешения
противоречий разума, но это скорее точка зрения, не желающая
выйти из области конечного и, следовательно, из
противоречия.


%%% Local Variables:
%%% mode: latex
%%% TeX-master: "../../../../main"
%%% End:
