Наличное бытие определенно; нечто имеет некоторое
качество, и в нем оно не только определенно, но и ограниченно;
его качество есть его граница; обремененное
границей, нечто сначала остается утвердительным,
спокойным наличным бытием. Но это отрицание, когда
оно развито так, что противоположность между наличным
бытием данного нечто и отрицанием как имманентной ему
границей сама есть его внутри-себя-бытие и данное нечто,
таким образом, есть лишь становление в самом
себе,~-- это отрицание составляет в таком случае его конечность.

Когда мы говорим о вещах, что \emph{они конечны}, то разумеем
под этим, что они не только имеют некоторую
определенность, что качество дано не только как реальность
и в-себе-сущее определение, что они не только
ограничены,~-- в этом случае они еще обладают наличным
бытием вне своей границы,~-- но что скорее небытие составляет
их природу, их бытие. Конечные вещи \emph{суть}, но
их соотношение с самими собой состоит в том, что они
соотносятся с самими собой как \emph{отрицательные}, что они
именно в этом соотношении с самими собой гонят себя
дальше себя, дальше своего бытия. Они \emph{суть}, но истиной
этого бытия служит их \emph{конец}. Конечное не только изменяется,
как нечто вообще, а \emph{преходит}; и не только возможно,
что оно преходит, так что оно могло бы быть,
не преходя, но бытие конечных вещей, как таковое, состоит
в том, что они содержат зародыш прехождения как
свое внутри-себя-бытие, что час их рождения есть час
их смерти.


%%% Local Variables:
%%% mode: latex
%%% TeX-master: "../../../../main"
%%% End:
