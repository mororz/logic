Хотя абстрактно это противоречие сразу же содержится
в том, что \emph{нечто} конечно, или, иначе говоря, что
конечное \emph{есть}, однако \emph{нечто} или бытие уже не положено
абстрактно, а рефлектировано в себя и развито как внутри-себя-бытие,
имеющее в себе некоторое определение и
свойство и, еще определеннее, границу в самом себе, которая,
будучи имманентно этому нечто и составляя качество
его внутри-себя-бытия, есть конечность. Мы должны посмотреть,
какие моменты содержатся в этом понятии конечного
нечто.

Определение и свойство оказались \emph{сторонами} для
внешней рефлексии. Но первое уже содержало инобытие
как принадлежащее к <<в себе>> [данного] нечто. Внешность
инобытия находится, с одной стороны, в собственном
внутреннем [данного] нечто, а, с другой, она как
внешность остается отличной от этого внутреннего, она
еще внешность, как таковая, но \emph{в} (an) нечто. Но так как,
далее, инобытие как \emph{граница} само определено как отрицание
отрицания, то имманентное [данному] нечто инобытие
положено как соотношение обеих сторон, и единство
[этого] нечто с собой~-- последнему принадлежит и
определение, и свойство~-- есть его обращенное против
самого себя соотношение, отрицающее в нем его имманентную
границу соотнесением его в-себе-сущего определения
с этой границей. Тождественное себе внутри-себя-бытие
соотносится, таким образом, с самим собой как со
своим собственным небытием, однако как отрицание отрицания,
как отрицающее это свое небытие, которое в то
же время сохраняет в нем наличное бытие, ибо оно качество
его внутри-себя-бытия. Собственная граница [данного]
нечто, положенная им, таким образом, как такое сущностное
в то же время отрицательное, есть не только
граница, как таковая, а \emph{предел}. Но предел есть не только
положенное как подвергнутое отрицанию; отрицание
обоюдоостро, поскольку положенное им как подвергнутое
отрицанию есть \emph{граница}. А именно граница есть вообще
то, чт\'о обще для нечто и иного; она есть также определенность
\emph{в-себе-бытия} определения, как такового. Следовательно,
это в-себе-бытие как отрицательное соотношение
со своей границей, также отличной от него, с собой
как пределом, есть \emph{долженствование}.

Для того чтобы граница, которая вообще есть во [всяком]
нечто, была пределом, нечто должно в то же время
внутри самого себя \emph{переступать ее}, в самом себе соотноситься
\emph{с ней как с некоторым не-сущим}. Наличное бытие
[данного] нечто находится в состоянии спокойствия и равнодушия,
как бы \emph{рядом} со своей границей. Но нечто переступает
свою границу лишь постольку, поскольку оно
есть ее снятость, отрицательное по отношению к ней в-себе-бытие.
А так как граница в самом \emph{определении} существует
как предел, то нечто тем самым переступает
\emph{через само себя}.

Долженствование содержит, следовательно, двоякое
определение: \emph{во-первых}, как в-себе-сущее определение,
противостоящее отрицанию, а \emph{во-вторых}, как некое небытие,
которое как предел отлично от него, но в то же
время само есть в-себе-сущее определение.

Итак, конечное определилось как соотношение его
определения с границей; определение есть в этом соотношении
\emph{долженствование}, а граница~-- \emph{предел}. Оба суть,
таким образом, моменты конечного; стало быть, оба, и
долженствование, и предел, сами конечны. Но лишь предел
\emph{положен} как конечное; долженствование ограничено
лишь в себе, стало быть, для нас. Через свое соотношение
с границей, ему самому уже имманентной, оно
ограничено, но эта его ограниченность скрыта во в-себе-бытии,
ибо по своему наличному бытию, т.\,е. по своей
определенности, противостоящей пределу, долженствование
положено как в-себе-бытие.

То, чт\'о должно быть, \emph{есть} и вместе с тем \emph{не есть}.
Если бы оно \emph{было}, оно тогда не только \emph{должно было бы
быть}. Следовательно, долженствование имеет по существу
своему некоторый предел. Этот предел не есть нечто
чуждое; \emph{то, чт\'о лишь} должно быть, есть \emph{определение},
которое теперь положено так, как оно есть в самом деле,
а именно как то, чт\'о есть вместе с тем лишь некоторая
определенность.

В-себе-бытие, присущее [данному] нечто в его определении,
низводит себя, следовательно, до \emph{долженствования}
тем, что то, чт\'о составляет его в-себе-бытие, дано в одном
и том же отношении как \emph{небытие}; и притом так, что во
внутри-себя-бытии, в отрицании отрицания, указанное
выше в-себе-бытие как одно отрицание (то, что отрицает)
есть единство с другим отрицанием, которое как качественно
другое есть в то же время граница, благодаря чему
указанное единство дано как \emph{соотношение} с ней. Предел
конечного не есть нечто внешнее; его собственное определение
есть также его предел; и предел есть и он сам, и
долженствование; он есть то, чт\'о обще обоим, или, вернее,
то, в чем оба тождественны.

Но, далее, как долженствование конечное выходит \emph{за}
свой предел; та же самая определенность, которая есть
его отрицание, также снята и, таким образом, есть его
в-себе-бытие; его граница также не есть его граница.

Следовательно, как \emph{долженствование} нечто \emph{выше своего
предела}, но и наоборот, лишь \emph{как долженствование}
оно имеет свой \emph{предел}; и то и другое нераздельны. Нечто
имеет предел постольку, поскольку оно в своем определении
имеет отрицание, а определение есть также и снятость
предела.


%%% Local Variables:
%%% mode: latex
%%% TeX-master: "../../../../main"
%%% End:
