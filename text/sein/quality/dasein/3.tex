Из становления возникает наличное бытие. Наличное
бытие есть простое единство (Einssein) бытия и ничто.
Из-за этой простоты оно имеет форму чего-то \emph{непосредственного}.
Его опосредствование, становление, находится
позади него; это опосредствование сняло себя, и наличное
бытие предстает поэтому как некое первое, из которого
исходят. Оно прежде всего в одностороннем определении
\emph{бытия}; другое содержащееся в нем определение,
\emph{ничто}, равным образом проявится в нем как противостоящее
первому.

Оно не просто бытие, а \emph{наличное бытие}; взятое этимологически,
Dasein означает бытие в каком-то \emph{месте}; но
представление о пространстве здесь не приложимо. Наличное
бытие есть вообще по своему становлению \emph{бытие}
с некоторым \emph{небытием}, так что это небытие принято в
простое единство с бытием. \emph{Небытие}, принятое в бытие
таким образом, что конкретное целое имеет форму бытия,
непосредственности, составляет \emph{определенность}, как таковую.

\emph{Целое} также имеет форму, т.\,е. \emph{определенность} бытия,
так как и бытие обнаружило себя в становлении только
как снятый, отрицательно определенный момент\endnotemark{}; но
таково оно \emph{для нас в нашей рефлексии}; оно еще не \emph{положено}
в самом себе. Определенность же наличного бытия,
как таковая, есть положенная определенность, на что
указывает и термин <<\emph{наличное} бытие>>.~"--- Следует всегда
строго различать между тем, чт\'о есть для нас, и тем,
чт\'о положено; лишь то, чт\'о \emph{положено} в каком-то понятии,
входит в рассмотрение, развивающее это понятие,
входит в его содержание. Определенность же, еще не положенная
в нем самом~"--- все равно, касается ли она
природы самого понятия или она есть внешнее сравнение,~"---
принадлежит нашей рефлексии; обращая внимание
на определенность этого рода, можно лишь уяснить
или предварительно наметить путь, который обнаруживается
в самом развитии [понятия]. Что целое, единство
бытия и ничто, имеет \emph{одностороннюю определенность
бытия},~"--- это внешняя рефлексия. В отрицании же, в
нечто и \emph{ином} и т.\,д., это единство дойдет до того, что
окажется \emph{положенным}.~"--- Следовало здесь обратить внимание
на это различие; но давать себе отчет обо всем,
чт\'о рефлексия может позволить себе заметить,~"--- излишне;
это привело бы к слишком пространному изложению,
к предвосхищению того, чт\'о должно вытекать из самого
предмета (Sache). Хотя такого рода рефлексии и могут
облегчить обзор целого и тем самым и понимание, однако
они невыгодны тем, что выглядят неоправданными
утверждениями, основаниями и основами последующего.
Не надо поэтому придавать им большее значение, чем то,
которое они должны иметь, и надлежит отличать их от
того, чт\'о составляет момент в развитии самого предмета.

\endnotetext{Эта фраза в издании 1833\,г. (воспроизведенном Глокнером
  в 1928\,г.) дана с несколько необычными знаками препинания, что
  дало повод Лассону изменить пунктуацию, прибавив тире перед
  словами <<denn Sein hat\dots>>. Тем самым глагол <<ist>> приобрел
  в главном предложении значение связки, тогда как согласно
  пунктуации, даваемой в издании 1833\,г., его следует понимать
  в смысле самостоятельного глагола (<<имеется в форме>> или <<имеет
  форму>>). Если принять пунктуацию Лассона, то всю эту фразу
  надо перевести так: <<Это \emph{целое} также в форме, т.\,е. \emph{определенности}
  бытия (так как и бытие обнаружило себя в становлении
  имеющим характер всего лишь момента) есть нечто снятое, отрицательно
  определенное>>. Сопоставление этого места с серединой
  данного абзаца (<<что целое, единство бытия и ничто, имеет одностороннюю
  определенность бытия,~"--- это внешняя рефлексия>>)
  заставляет предпочесть интерпретацию Б.\,Г.~Столпнера, которая
  и принята в настоящем томе.}

Наличное бытие соответствует \emph{бытию} предшествующей
сферы; однако бытие есть неопределенное, поэтому
в нем не получается никаких определений. Наличное же
бытие есть определенное бытие, \emph{конкретное}; поэтому в
нем сразу же выявляется несколько определений, различенные
отношения его моментов.


%%% Local Variables:
%%% mode: latex
%%% TeX-master: "../../../../main"
%%% End:
