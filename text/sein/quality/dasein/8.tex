1. \emph{Во-первых}, нечто и иное суть налично сущие, или
\emph{нечто}.

\emph{Во-вторых}, каждое из них есть также \emph{иное}. Безразлично,
которое из них мы называем сначала и лишь потому
именуем \emph{нечто} (по-латыни, когда они встречаются
в одном предложении, оба называются aliud, или <<один
другого>>~"--- alius alium, а когда речь идет об отношении
взаимности, аналогичным выражением служит alter alterum).
Если мы одно наличное бытие называем $A$, а другое
$B$, то $B$ определено ближайшим образом как иное.
Но точно так же $A$ есть иное этого $B$. Оба одинаково суть
\emph{иные}. Для фиксирования различия и того нечто, которое
следует брать как утвердительное, служит [слово] <<это>>\endnotemark{}.
Но <<это>> как раз и выражает, что такое различение и выделение
одного нечто есть субъективное обозначение,
имеющее место вне самого нечто. В этом внешнем показывании
и заключается вся определенность; даже выражение
<<это>> не содержит никакого различия; всякое и каждое
нечто есть столь же <<\emph{это}>>, сколь и иное. \emph{Считается}, что
словом <<это>> выражают нечто совершенно определенное;
но при этом упускают из виду, что язык как произведение
рассудка выражает лишь всеобщее; исключение составляет
только \emph{имя} единичного предмета, но индивидуальное
имя есть нечто бессмысленное в том смысле, что оно
не выражает всеобщего, и по этой же причине оно представляется
чем-то лишь положенным, произвольным, как
и на самом деле собственные имена могут быть произвольно
приняты, даны или также изменены.

\endnotetext{Ср. раздел <<Чувственная достоверность или <<это>> и мнение>>
  в <<Феноменологии духа>> (стр.\,51--59).}

Итак, инобытие представляется определением, чуждым
определенному таким образом наличному бытию,
или, иначе говоря, иное выступает \emph{вне} данного наличного
бытия; отчасти так, что наличное бытие определяет
себя как иное только через \emph{сравнение}, производимое некоторым
третьим, отчасти так, что это наличное бытие
определяет себя как другое только из-за иного, находящегося
вне его, но само по себе оно не таково. В то же
время, как мы уже отметили, каждое наличное бытие
определяет себя и для представления в равной мере как
другое наличное бытие, так что не остается ни одного
наличного бытия, которое было бы определено лишь как
наличное бытие и не было бы вне некоторого наличного
бытия, следовательно, само не было бы некоторым иным.

Оба определены и как \emph{нечто} и как \emph{иное}, они, значит,
\emph{одно и то же}, и между ними еще нет никакого различия.
Но эта \emph{тождественность} определений также имеет мест
только во внешней рефлексии, в \emph{сравнении} их друг с другом;
но в том виде, в каком вначале положено \emph{иное}, оно
само по себе, правда, соотносится с нечто, однако оно
также и \emph{само по себе находится вне последнего}.

\emph{В-третьих}, следует поэтому брать \emph{иное} как изолированное,
в соотношении с самим собой, брать \emph{абстрактно}
как иное, как τό έτερον Платона, который противопоставляет
его \emph{единому} как один из моментов целокупности
и таким образом приписывает \emph{иному} свойственную ему
\emph{природу}. Таким образом, \emph{иное}, понимаемое лишь как таковое,
есть не иное некоторого нечто, а иное в самом
себе, т.\,е. иное самого себя.~"--- \emph{Физическая природа} есть
по своему определению такое иное; она есть \emph{иное духа}.
Это ее определение есть, таким образом, вначале одна
лишь относительность, которая выражает не какое-то
качество самой природы, а лишь внешнее ей соотношение.
Но так как дух есть истинное нечто, а природа
поэтому есть в самой себе лишь то, что она есть по отношению
к духу, то, поскольку она берется сама по себе,
ее качество состоит именно в том, что она в самой себе
есть иное, \emph{вовне-себя-сущее} (в определениях пространства,
времени, материи).

Иное само по себе есть иное по отношению к самому
себе (an ihm selbst) и, следовательно, иное самого себя,
таким образом, иное иного,~"--- следовательно, всецело неравное
внутри себя, отрицающее себя, \emph{изменяющееся}.
Но точно так же оно остается тождественным с собой,
ибо то, во что оно изменилось, есть \emph{иное}, которое помимо
этого не имеет никаких других определений. А то, чт\'о
изменяется, определено быть иным не каким-нибудь
другим образом, а тем же самым; оно поэтому \emph{соединяется}
в том ином \emph{лишь с самим собой}. Таким образом, оно
положено как рефлектированное в себя со снятием инобытия;
оно есть \emph{тождественное} с собой нечто, по отношению
к которому, следовательно, инобытие, составляющее
в то же время его момент, есть нечто отличное от него,
не принадлежащее ему самому как такому нечто.

2. Нечто \emph{сохраняется} в отсутствии своего наличного
бытия (Nichtdasein), оно по своему существу \emph{едино} с ним
и по своему существу \emph{не едино} с ним. Оно, следовательно,
\emph{соотносится} со своим инобытием; оно не есть только
свое инобытие. Инобытие в одно и то же время и содержится
в нем, и еще \emph{отделено} от него. Оно \emph{бытие-для-иного}.

Наличное бытие, как таковое, есть непосредственное,
безотносительное; иначе говоря, оно имеется в определении
\emph{бытия}. Но наличное бытие как включающее в себя
небытие есть \emph{определенное} бытие, подвергшееся внутри
себя отрицанию, а затем ближайшим образом~"--- иное; но
так как оно в то же время и сохраняется, подвергнув себя
отрицанию, то оно есть лишь \emph{бытие-для-иного}.

Оно сохраняется в отсутствии своего наличного бытия
и есть бытие; но не бытие вообще, а как соотношение
с собой в \emph{противоположность} своему соотношению с иным,
как равенство с собой в противоположность своему неравенству.
Такое бытие есть \emph{в-себе-бытие}.

Бытие-для-иного и в-себе-бытие составляют \emph{оба момента}
[всякого] нечто. Здесь имеются \emph{две пары} определений:
1) \emph{нечто} и \emph{иное}; 2) \emph{бытие-для-иного} и \emph{в-себе-бытие}.
В первых имеется безотносительность их определенности:
нечто и иное расходятся. Но их истина~"--- это соотношение
между ними; бытие-для-иного и в-себе-бытие
суть поэтому указанные определения, положенные как \emph{моменты}
одного и того же, как определения, которые суть
соотношения и остаются в своем единстве, в единстве наличного
бытия. Каждое из них, следовательно, в то же время
содержит в себе и свой отличный от себя момент.

Бытие и ничто в том их единстве, которое есть наличное
бытие, уже более не бытие и ничто: таковы они только
вне своего единства. Таким образом, в их беспокойном
единстве, в становлении, они суть возникновение и
прехождение.~"--- Бытие во [всяком] нечто есть \emph{в-себе-бытие}.
Бытие, соотношение с собой, равенство с собой,
теперь уже не непосредственное, оно соотношение с собой
лишь как небытие инобытия (как рефлектированное
в себя наличное бытие).~"--- И точно так же небытие как
момент [всякого] нечто в этом единстве бытия и небытия
есть не отсутствие наличного бытия вообще, а иное,
и, говоря определеннее, по \emph{различению} его и бытия оно
есть в то же время \emph{соотношение} с отсутствием своего
наличного бытия, бытие-для-иного.

Тем самым \emph{в-себе-бытие} есть, во-первых, отрицательное
соотношение с отсутствием наличного бытия, оно
имеет инобытие вовне себя и противоположно ему;
поскольку нечто есть в \emph{себе}, оно лишено инобытия и бытия
для иного. Но, во-вторых, оно имеет небытие и в самом
себе, ибо оно само есть \emph{не-бытие} бытия-для-иного.

Но \emph{бытие-для-иного} есть, во-первых, отрицание простого
соотношения бытия с собой, соотношения, которым
ближайшим образом должно быть наличное бытие и нечто;
поскольку нечто есть в ином или для иного, оно лишено
собственного бытия. Но, во-вторых, оно не отсутствие
наличного бытия как чистое ничто. Оно отсутствие
наличного бытия, указывающее на в-себе-бытие как на
свое рефлектированное в себя бытие, как и наоборот,
в-себе-бытие указывает на бытие-для-иного.

3. Оба момента суть определения одного и того же, а
именно определения [всякого] нечто. Нечто есть \emph{в себе},
поскольку оно ушло из бытия-для-иного, возвратилось
в себя. Но нечто имеет также определение или обстоятельство
\emph{в себе} (an sich) (здесь ударение падает на <<в>>)
или \emph{в самом себе} (an ihm), поскольку это обстоятельство
есть \emph{в нем} (an ihm) внешним образом, есть бытие-для-иного.

Это ведет к некоторому дальнейшему определению.
\emph{В-себе-бытие} и бытие-для-иного прежде всего различны,
но то, что нечто имеет \emph{то же самое}, чт\'о \emph{оно есть в себе}
(an sich), также и \emph{в самом себе} (an ihm), и,наоборот,то,
что оно есть как бытие-для-иного, оно есть и в себе~"---
в этом состоит тождество в-себе-бытия и бытия-для-иного,
согласно определению, что само нечто есть тождество
обоих моментов и что они, следовательно, в нем нераздельны.~"---
Формально это тождество получается уже в
сфере наличного бытия, но более определенное выражение
оно получит при рассмотрении сущности и затем
при рассмотрении отношения \emph{внутреннего} (Innerlichkeit)
и \emph{внешнего} (Äusserlichkeit), а определеннее всего~"--- при
рассмотрении идеи как единства понятия и действительности.~"---
Полагают, что словами <<\emph{в себе}>> и <<\emph{внутреннее}>>
высказывают нечто возвышенное; однако то, чт\'о нечто
есть \emph{только в себе}, есть также \emph{только в нем}; <<в себе>>
есть лишь абстрактное и, следовательно, внешнее определение.
Выражения <<\emph{в нем} ничего нет>>, <<\emph{в этом} что-то
есть>> имеют, хотя и смутно, тот смысл, что то, чт\'о в чем-то
есть, принадлежит также и к его \emph{в-себе-бытию}, к его
внутренней, истинной ценности.

Можно отметить, что здесь уясняется смысл \emph{вещи-в-себе},
которая есть очень простая абстракция, но в продолжение
некоторого времени слыла очень важным определением,
как бы чем-то изысканным, так же как положение
о том, что мы не знаем, каковы вещи в себе,
признавалось большой мудростью.~"--- Вещи называются
вещами-в-себе, поскольку мы абстрагируемся от всякого
бытия-для-иного, т.\,е. вообще поскольку мы их мыслим
без всякого определения, как ничто. В этом смысле нельзя,
разумеется, знать, \emph{что такое} вещь-\emph{в-себе}. Ибо вопрос:
\emph{что такое?}~"--- требует, чтобы были указаны \emph{определения};
но так как те вещи, определения которых следовало бы
указать, должны быть в то же время \emph{вещами-в-себе}, т.\,е.
как раз без всякого определения, то в вопрос необдуманно
включена невозможность ответить на него или же
дают только нелепый ответ на него.~"--- Вещь-в-себе есть
то же самое, что то абсолютное, о котором знают только
то, что все в нем едино. Мы поэтому знаем очень хорошо,
чт\'о представляют собой эти вещи-в-себе; они, как
таковые, не что иное, как лишенные истинности, пустые
абстракции. Но что такое поистине вещь-в-себе, что поистине
есть в себе,~"--- изложением этого служит логика,
причем, однако, под <<в-себе>> понимается нечто лучшее,
чем абстракция, а именно то, чт\'о нечто есть в своем
понятии; но понятие конкретно внутри себя постижимо
как понятие вообще и внутри себя познаваемо как определенное
и как связь своих определений.

В-себе-бытие имеет своим противостоящим моментом
прежде всего бытие-для-иного; но в-себе-бытию противопоставляется
также и \emph{положенностъ} (Gesetztsein). Это
выражение, правда, подразумевает также и бытие-для-иного,
но оно определенно разумеет уже происшедший
поворот\endnotemark{} от того, чт\'о не есть в себе, к тому, чт\'о есть
его в-себе-бытие, в чем оно \emph{положительно}. \emph{В-себе-бытие}
следует обычно понимать как абстрактный способ выражения
понятия; \emph{полагание}, собственно говоря, относится
уже к сфере сущности, объективной рефлексии; основание
\emph{полагает} то, чт\'о им обосновывается; причина, больше того,
\emph{производит} действие, наличное бытие, самостоятельность
которого \emph{непосредственно} отрицается и смысл которого
заключается в том, что оно имеет свою \emph{суть} (Sache),
свое бытие в ином. В сфере бытия наличное бытие \emph{происходит}
только из становления, иначе говоря, вместе с
нечто положено иное, вместе с конечным~"--- бесконечное,
но конечное не производит бесконечного, не \emph{полагает} его.
В сфере бытия \emph{самоопределение} (\emph{Sichbestimmen}) понятия
само есть лишь \emph{в-себе}~"--- и в таком случае оно называется
переходом. Рефлектирующие определения бытия,
как, например, нечто и иное или конечное и бесконечное,
хотя по своему существу и указывают друг на друга,
или даны как бытие-для-иного, также считаются как
\emph{качественные} существующими сами по себе; иное есть,
конечное считается точно так же \emph{непосредственно сущим}
и пребывающим само по себе, как и бесконечное; их
смысл представляется завершенным также и без их иного.
Напротив, положительное и отрицательное, причина
и действие, хотя они также берутся как изолированно
сущие, все же не имеют никакого смысла друг без друга;
они \emph{сами} светятся друг в друге, каждое из них светится
в своем ином.~"--- В разных сферах определения и
в особенности в развитии изложения, или, точнее, в движении
понятия к своему изложению существенно всегда
надлежащим образом различать между тем, чт\'о еще есть
\emph{в себе}, и тем, чт\'о \emph{положено}, например определения, как
они суть в понятии и каковы они, будучи положенными
или сущими-для-иного. Это~"--- различение, относящееся
только к диалектическому развитию, различение, которого
не знает метафизическое философствование, в том
числе и критическая философия; дефиниции метафизики,
равно как и ее предпосылки, различения и выводы, имеют
целью утверждать и выявлять лишь \emph{сущее} и притом
\emph{в-себе-сущее}.

\endnotetext{Zurückbengung (поворот) имеет то же значение, что и
  рефлексия.}

В единстве [всякого] нечто с собой \emph{бытие-для-иного}
тождественно со своим <<\emph{в себе}>>; \emph{в этом случае} бытие-для-иного
есть \emph{в} [самом] нечто. Рефлектированная таким
образом в себя определенность тем самым есть вновь
\emph{простое сущее}, есть, следовательно, вновь качество~"---
\emph{определение}.


%%% Local Variables:
%%% mode: latex
%%% TeX-master: "../../../../main"
%%% End:
