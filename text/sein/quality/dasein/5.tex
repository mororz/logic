<<Реальность>> может показаться многозначным словом,
так как оно употребляется для обозначения разных
и даже противоположных определений. В философском
смысле говорят, например, о \emph{чисто эмпирической} реальности
как о лишенном ценности наличном бытии. Но
когда говорят о мыслях, понятиях, теориях, что они \emph{лишены
реальности}, то это означает, что у них нет \emph{действительности},
хотя \emph{в себе}, или в понятии, идея, например
платоновской республики, может, дескать, быть истинной.
Здесь не отрицается за идеей ее ценность, и \emph{наряду}
с реальностью допускают и ее. Но сравнительно с так называемыми
\emph{голыми} идеями, с \emph{голыми} понятиями реальное
считается единственно истинным.~-- Смысл, в котором
внешнему наличному бытию приписывается решение
вопроса об истинности того или иного содержания,
столь же односторонен, как односторонне представление,
будто для идеи, сущности или даже внутреннего чувства
безразлично внешнее наличное бытие, и еще в большей
мере односторонне мнение о том, что они тем превосходнее,
чем более они отдалены от реальности.

Рассматривая термин <<реальность>>, следует коснуться
прежнего метафизического \emph{понятия бога}, из которого
исходило прежде всего так называемое онтологическое
доказательство бытия бога. Бога определяли как \emph{совокупность
всех реальностей}, и об этой совокупности говорилось,
что она не заключает в себе никакого противоречия,
что ни одна из реальностей не снимает другую; ибо
реальность следует, мол, понимать лишь как некоторое
совершенство, как нечто \emph{утвердительное}, не содержащее
никакого отрицания. Реальности, стало быть, не противоположны
и не противоречат друг другу.

При таком понимании реальности предполагают, что
она остается и тогда, когда мысленно устраняют всякое
отрицание; однако этим снимается всякая определенность
реальности. Реальность есть качество, наличное
бытие; тем самым она содержит момент отрицательности,
и лишь благодаря этому она есть то определенное, которое
она есть. В так называемом \emph{эминентном смысле}\endnotemark{}
или как \emph{бесконечная}~-- в обычном значении этого слова,
т.\,е. в том смысле, в котором ее будто бы следует понимать,~--
она становится неопределенной и теряет свое
значение. Божественная благость, утверждали, есть благость
не в обычном смысле, а в эминентном; она не отлична
от справедливости, а \emph{умеряется} (\emph{лейбницевское}
примиряющее выражение) ею, как и, наоборот, справедливость
умеряется благостью; таким образом, благость
уже перестает быть благостью и справедливость~-- справедливостью.
Могущество [бога], говорят, умеряется [его]
мудростью, но в таком случае оно уже не могущество,
как таковое, ибо оно было бы подчинено мудрости; мудрость
[бога], утверждают, расширяется до могущества, но
в таком случае она исчезает как мудрость, определяющая
цель и меру. Истинное понятие бесконечного и его
\emph{абсолютное} единство~-- понятие, к которому мы придем
позднее,~-- нельзя понимать как \emph{умерение, взаимное ограничение}
или \emph{смешение}; это~-- поверхностное, окутанное
неопределенным туманом соотношение, которым может
удовлетворяться лишь чуждое понятия представление,~--
Реальность, как ее берут в указанной выше дефиниции
бога, т.\,е. реальность как определенное качество, выведенное
за пределы своей определенности, перестает быть реальностью;
оно превращается в абстрактное бытие; бог как
\emph{чисто} реальное во всем реальном или как \emph{совокупность}
всех реальностей так же лишен определения и содержания,
как и пустое абсолютное, в котором все есть одно.

\endnotetext{Термин \emph{эминентный} (eminenter) заимствован из средневековой
  философии и обозначает высшую степень какого-нибудь
  качества или свойства.}

Если же, напротив, брать реальность в ее определенности,
то ввиду того, что она содержит как нечто сущностное
момент отрицательности, совокупность всех реальностей
становится также совокупностью всех отрицаний,
совокупностью всех противоречий, прежде всего
абсолютным \emph{могуществом}, в котором все определенное
поглощается; но так как само это могущество имеется
лишь постольку, поскольку оно имеет против себя нечто,
еще не снятое им, то, когда его мыслят как могущество,
ставшее осуществленным, беспредельным, оно превращается
в абстрактное ничто. То реальное во всяком реальном,
\emph{бытие} во всяком \emph{наличном бытии}, которое будто
бы выражает понятие бога, есть не что иное, как абстрактное
бытие, то же, что и ничто.

Определенность есть отрицание, положенное как утвердительное,
это~-- положение Спинозы: omnis determinatio
est negatio\endnotemark{}. Это чрезвычайно важное положение;
только надо сказать, что отрицание, как таковое, есть
бесформенная абстракция. Но не следует обвинять спекулятивную
философию в том, что для нее отрицание
или ничто есть нечто последнее; оно не есть для нее
последнее, как и реальность не есть для нее истинное.

\endnotetext{Выражение <<omnis determinate est negatio>> у Спинозы
  нигде нет. У него встречается <<determinate est negatio>>~-- в
  письме Яриху Иеллесу от 2 июня 1674\,г. (см.~Б.~Спиноза. Избранные
  произведения в двух томах, т.\,2. М., 1957, стр.\,568). Из текста
  письма явствует, что determinatio означает в данном случае не
  <<определение>>, а <<ограничение>>.}

Необходимым выводом из положения о том, что определенность
есть отрицание, является \emph{единство спинозовской
субстанции} или то, что существует лишь одна субстанция.
\emph{Мышление} и \emph{бытие}, или протяжение, эти два определения,
рассматриваемые Спинозой, должны были быть
сведены им в одно в этом единстве, ибо как определенные
реальности они отрицания, бесконечность которых
есть их единство; согласно спинозовской дефиниции, о
которой будет сказано ниже, бесконечность [всякого] нечто
есть его утверждение. Он понимал поэтому оба определения
как атрибуты, т.\,е. как такие, которые не имеют
отдельного существования, в-себе-и-для-себя-бытия, а даны
лишь как снятые, как моменты; или, правильнее
сказать, они для него даже и не моменты, ибо субстанция
совершенно лишена определений в самой себе, а
атрибуты, равно как и модусы, суть различения, делаемые
внешним рассудком.~-- Точно так же несовместима с этим
положением субстанциальность индивидов. Индивид есть
соотношение с собой в силу того, что он ставит границы
всему иному; но тем самым эти границы суть также
и границы его самого, суть соотношения с иным; он
не имеет своего наличного бытия в самом себе. Индивид,
правда, есть нечто \emph{большее}, чем только во всех отношениях
ограниченное, но это <<большее>> относится к другой
сфере~-- понятия; в метафизике бытия он всецело определен;
и против того, чтобы индивид, чтобы конечное, как
таковое, существовало в себе и для себя, выступает определенность
в своем существе как отрицание и увлекает
конечное в то же отрицательное движение рассудка, которое
заставляет все исчезать в абстрактном единстве,
в субстанции.

Отрицание непосредственно противостоит реальности;
в дальнейшем, в сфере собственно рефлектированных
определений, оно противопоставляется \emph{положительному},
которое есть рефлектирующая в отрицание реальность,~--
реальность, в которой \emph{светится} то отрицательное, которое
еще скрыто в реальности, как таковой.

Качество есть \emph{свойство} прежде всего лишь в том
смысле, что оно в некотором \emph{внешнем соотношении} показывает
себя \emph{имманентным определением}. Под свойствами,
например трав, понимают определения, которые
не только вообще \emph{свойственны} тому или иному нечто,
а свойственны ему постольку, поскольку благодаря им
оно присущим ему образом \emph{сохраняет} себя в соотношении
с иным, не дает воли внутри себя посторонним
положенным в нем воздействиям, а само \emph{показывает} в
ином \emph{силу} своих собственных определений, хотя и не
отстраняет от себя этого иного. Напротив, более спокойные
определенности, как, например, фигура, внешний вид,
не называют свойствами, как, впрочем, и не качествами,
поскольку их представляют себе изменчивыми, не тождественными
с \emph{бытием}.

Qualierung или Inqualierung~-- термин философии
Якоба Бёме, проникающей вглубь, но в смутную глубь,~--
означает движение того или иного качества (кислого,
терпкого, горячего и т.\,д.) в самом себе, поскольку оно
в своей отрицательной природе (в своей Qual\endnotemark{}) выделяется
из другого и укрепляется, поскольку оно вообще
есть свое собственное беспокойство в самом себе, сообразно
которому оно порождает и сохраняет себя лишь
в борьбе.

\endnotetext{Я.~Бёме этимологически связывал немецкое слово Qual
  (м\'ука) с латинским словом qualitas (качество). Искусственно
  образованные Бёме слова Qualierung и Inqualierung можно было
  бы перевести буквально <<качествование>> и <<вкачествование>>.}


%%% Local Variables:
%%% mode: latex
%%% TeX-master: "../../../../main"
%%% End:
