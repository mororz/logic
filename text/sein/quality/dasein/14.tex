Долженствование, взятое само по себе, содержит предел,
а предел~-- долженствование. Их взаимоотношение
есть само конечное, содержащее их оба в своем внутри-себя-бытии.
Эти моменты его определения качественно
противоположны; предел определен как отрицание долженствования,
а долженствование~-- как отрицание предела.
Таким образом, конечное есть внутреннее противоречие
с собой; оно снимает себя, преходит. Но этот его результат,
отрицательное вообще, есть $\alpha$) само его \emph{определение};
ибо оно есть отрицательное отрицательного. Конечное,
таким образом, не прешло в прехождении; оно
прежде всего стало лишь некоторым \emph{другим} конечным,
которое, однако, есть также прехождение как переход в
некоторое другое конечное и т.\,д., можно сказать до \emph{бесконечности}.
Но $\beta$) рассматривая ближе этот результат,
мы убеждаемся, что в своем прехождении, этом отрицании
самого себя, конечное достигло своего в-себе-бытия, оно
в этом прехождении \emph{слилось с самим собой}. Каждый из
его моментов содержит именно этот результат; долженствование
выходит за предел, т.\,е. за себя само; но выход
за себя, или его иное, есть лишь сам предел. Предел же
указывает на непосредственный выход самого себя к
своему иному, которое есть долженствование, а последнее
есть то же раздвоение \emph{в-себе-бытия} и \emph{наличного бытия},
чт\'о и предел, есть то же, что и он; выходя за себя,
оно поэтому точно так же лишь сливается с собой. Это
\emph{тождество с собой}, отрицание отрицания, есть утвердительное
бытие, есть, таким образом, иное конечного, долженствующего
иметь своей определенностью первое отрицание;
это иное есть \emph{бесконечное}.


%%% Local Variables:
%%% mode: latex
%%% TeX-master: "../../../../main"
%%% End:
