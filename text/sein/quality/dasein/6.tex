В наличном бытии мы различили его определенность
как качество; в качестве как налично сущем \emph{есть} различие~"---
различие реальности и отрицания. Насколько
эти различия имеются в наличном бытии, настолько же
они ничтожны и сняты. Сама реальность содержит отрицание,
есть наличное, а не неопределенное, абстрактное
бытие. И точно так же отрицание есть наличное бытие;
оно не абстрактное, как считают, ничто, оно здесь положено
так, как оно есть в себе, как сущее, принадлежащее
к наличному бытию. Таким образом, качество вообще не
отделено от наличного бытия, которое есть лишь определенное,
качественное бытие.

Это снятие различения есть больше, чем только отказ
от него и еще одно внешнее отбрасывание его или простой
возврат к простому началу, к наличному бытию,
как таковому. Различие не может быть отброшено, ибо
оно \emph{есть}. Фактическое, стало быть, то, что имеется, есть
наличное бытие вообще, различие в нем и снятие этого
различия; не наличное бытие, лишенное различий, как
вначале, а наличное бытие как \emph{снова} равное самому себе
\emph{благодаря снятию различия}, как простота наличного
бытия, \emph{опосредствованная} этим снятием. Эта снятость
различия есть отличительная определенность наличного
бытия. Таким образом, оно есть \emph{внутри-себя-бытие}; наличное
бытие есть \emph{налично сущее}, \emph{нечто}.

Нечто есть \emph{первое отрицание отрицания} как простое
сущее соотношение с собой. Наличное бытие, жизнь,
мышление и т.\,д. в своей сущности определяют себя как
\emph{налично сущее, живое, мыслящее} (<<Я>>) и т.\,д. Это определение
в высшей степени важно, если хотят идти дальше
наличного бытия, жизни, мышления и т.\,д., а также
божественности (вместо бога) как всеобщностей. Представление
справедливо считает \emph{нечто реальным}. Однако
\emph{нечто} есть еще очень поверхностное определение, подобно
тому как \emph{реальность} и \emph{отрицание}, наличное бытие и
его определенность, хотя они уже не пустые бытие и
ничто, однако суть совершенно абстрактные определения.
Поэтому они и самые ходячие выражения, и философски
необразованная рефлексия чаще всего пользуется
ими, втискивает в них свои различения и мнит, будто
имеет в них что-то вполне добротное и строго определенное.~"---
Отрицание отрицания как \emph{нечто} есть лишь начало
субъекта,~"--- внутри-себя-бытие, еще совершенно неопределенное.
В дальнейшем оно определяет себя прежде
всего как сущее для себя, продолжая определять себя до
тех пор, пока оно не получит лишь в понятии конкретную
напряженность субъекта. В основе всех этих определений
лежит отрицательное единство с собой. Но при этом
следует различать между отрицанием как \emph{первым}, как
отрицанием \emph{вообще}, и вторым, отрицанием отрицания,
которое есть конкретная, \emph{абсолютная} отрицательность,
так Же как первое отрицание есть, напротив, лишь
\emph{абстрактная} отрицательность.

\emph{Нечто} есть \emph{сущее} как отрицание отрицания; ибо последнее~"---
это восстановление простого соотношения с собой;
но тем самым нечто есть также и \emph{опосредствование
себя} с \emph{самим собой}. Уже в простоте [всякого] нечто, а
затем еще определеннее в для-себя-бытии, субъекте
и т.\,д. имеется опосредствование себя с самим собой; оно
имеется уже и в становлении, но в нем оно лишь совершенно
абстрактное опосредствование. В нечто опосредствование
с собой \emph{положено}, поскольку нечто определено
как простое \emph{тождественное}.~"--- Можно обратить внимание
на то, что вообще имеется опосредствование, в противовес
принципу утверждаемой чистой непосредственности
знания, из которой опосредствование будто бы исключено;
но в дальнейшем нет нужды обращать особое внимание
на момент опосредствования, ибо он находится везде
и всюду, в каждом понятии.

Это опосредствование с собой, которое нечто есть \emph{в
себе}, взятое лишь как отрицание отрицания, своими сторонами
не имеет каких-либо конкретных определений;
так оно сводится в простое единство, которое есть \emph{бытие}.
Нечто \emph{есть}, и оно ведь \emph{есть} также налично сущее; далее,
оно есть \emph{в себе} также и \emph{становление}, которое, однако,
уже не имеет своими моментами только бытие и ничто.
Один из них~"--- бытие~"--- есть теперь наличное бытие, я,
далее, налично сущее; второй есть также нечто \emph{налично
сущее}, но определенное как отрицательность, присущая
нечто (Negatives des Etwas),~"--- \emph{иное}. Нечто как становление
есть переход, моменты которого сами суть нечто и
который поэтому есть \emph{изменение},~"--- становление, ставшее
уже \emph{конкретным}.~"--- Но нечто изменяется сначала лишь
в своем понятии; оно, таким образом, еще не \emph{положено}
как опосредствующее и опосредствованное; вначале оно
положено как просто сохраняющее себя в своем соотношении
с собой, а его отрицательность~"--- как некоторое
такое же качественное, как только \emph{иное} вообще.


%%% Local Variables:
%%% mode: latex
%%% TeX-master: "../../../../main"
%%% End:
