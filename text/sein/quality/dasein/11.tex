Мысль о конечности вещей влечет за собой эту скорбь
по той причине, что конечность эта есть доведенное до
крайности качественное отрицание и что в простоте такого
определения им уже не оставлено никакого утвердительного
бытия, \emph{отличного} от их определения к гибели.
Ввиду этой качественной простоты отрицания, возвратившегося
к абстрактной противоположности между
ничто и прехождением, с одной стороны, и бытием~-- с
другой, конечность есть наиболее упрямая категория рассудка;
отрицание вообще, свойство, граница уживаются
со своим иным~-- с наличным бытием; даже от абстрактного
ничто самого по себе как от абстракции отказываются;
но конечность есть \emph{фиксированное в себе} отрицание
и поэтому резко противостоит своему утвердительному.
Конечное, правда, позволяет привести себя в движение,
оно само и состоит в том, что оно определено к
своему концу, но только к своему концу; оно скорее есть
отказ от того, чтобы его утвердительно приводили к его
утвердительному, к бесконечному, чтобы его приводили
в связь с последним. Оно, стало быть, положено нераздельным
со своим ничто, и этим отрезан путь к какому
бы то ни было его примирению со своим иным, с утвердительным.
Определение конечных вещей не простирается
далее их \emph{конца}. Рассудок никак не хочет отказаться
от этой скорби о конечности, делая небытие определением
вещей и вместе с тем \emph{непреходящим} и \emph{абсолютным}.
Их преходящность (Vergänglichkeit) могла бы прейти
лишь в ином, в утвердительном; тогда их конечность
отделилась бы от них; но она есть их неизменное качество,
т.\,е. не переходящее в свое иное, т.\,е. в свое утвердительное;
\emph{таким образом она вечна}.

Это весьма важное наблюдение; но что конечное
абсолютно~-- это такая точка зрения, которую, разумеется,
вряд ли какое-либо философское учение или какое-либо
воззрение или рассудок позволят навязать себе;
скорее в утверждении о конечном определенно содержится
противоположный взгляд: конечное есть ограниченное,
преходящее; конечное есть \emph{только} конечное, а не непреходящее;
это заключается непосредственно в его определении
и выражении. Но важно знать, настаивает ли это
воззрение на том, чтобы мы не шли дальше \emph{бытия конечности}
и рассматривали \emph{преходящность} как сохраняющуюся,
или же [на том, что] \emph{преходящность} и \emph{прехождение
преходят}? Что это не имеет места, фактически утверждается
как раз тем воззрением на конечное, которое
делает \emph{прехождение последним} [моментом] в конечном.
Оно определенно утверждает, что конечное не уживается
и несоединимо с бесконечным, что конечное полностью
противоположно бесконечному. Бесконечному приписывается
бытие, абсолютное бытие; конечное, таким
образом, остается по отношению к нему фиксированным
как его отрицательное; несоединимое с бесконечным, оно
остается абсолютно у себя; оно могло бы получить утвердительность
от утвердительного, от бесконечного и таким
образом оно прешло бы; но как раз соединение с последним
объявляется невозможным. Если верно, что оно по
отношению к бесконечному не остается неизменным, а
преходит, то, как мы сказали раньше, последний [момент]
в нем есть именно его прехождение, а не утвердительное,
которым могло бы быть лишь прехождение прехождения.
Если же конечное преходит не в утвердительном, а его конец
понимается как \emph{ничто}, то мы снова оказались бы у того
первого, абстрактного ничто, которое само давно прешло.

Однако у этого ничто, которое должно быть \emph{только}
ничто и которому в то же время приписывают некоторое
существование, а именно существование в мышлении,
представлении или речи, мы встречаем то же самое противоречие,
которое только что было указано у конечного,
с той лишь разницей, что в абстрактном ничто это противоречие
только \emph{встречается}, а в конечности оно \emph{решительно
выражено}. Там оно представляется субъективным,
здесь же утверждают, что конечное противостоит бесконечному
\emph{вечно}, \emph{есть} в себе ничтожное и дано \emph{как} в себе
ничтожное. Это нужно осознать; и развертывание конечного
показывает, что оно в самом себе как это противоречие
рушится внутри себя, но при этом действительно
разрешает указанное противоречие, [обнаруживая], что
оно не только преходяще и преходит, но что прехождение,
ничто не есть последний момент, а само преходит.


%%% Local Variables:
%%% mode: latex
%%% TeX-master: "../../../../main"
%%% End:
