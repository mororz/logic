a) Нечто \emph{и иное}; они ближайшим образом безразличны
друг к другу; иное также есть непосредственно налично
сущее, нечто; отрицание, таким образом, имеет место
вне их обоих. Нечто есть \emph{в себе}, противостоящее своему
\emph{бытию-для-иного}. Но определенность принадлежит также
к его <<в-себе>> и есть

b) его \emph{определение}, переходящее также в \emph{свойство}
(Beschaffenheit), которое, будучи тождественным с первым,
составляет имманентное и в то же время подвергшееся
отрицанию бытие-для-иного, \emph{границу} [всякого]
нечто, которая

c) есть имманентное определение самого нечто, а нечто,
следовательно, есть \emph{конечное}.

В начале главы, где мы рассматривали \emph{наличное
бытие} вообще, последнее как взятое первоначально имело
определение \emph{сущего}. Моменты его развития, качество
и нечто, имеют поэтому также утвердительную определенность.
Напротив, в начале этого раздела развивается
заключающееся в наличном бытии отрицательное определение,
которое там еще было только отрицанием вообще,
\emph{первым} отрицанием, а теперь определено как \emph{внутри-себя-бытие}
[всякого] нечто, как отрицание отрицания.



%%% Local Variables:
%%% mode: latex
%%% TeX-master: "../../../../main"
%%% End:
