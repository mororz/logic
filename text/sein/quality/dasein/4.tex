Ввиду непосредственности, в которой бытие и ничто
едины в наличном бытии, они не выходят за пределы
друг друга; насколько наличное бытие есть сущее, настолько
же оно есть небытие, определено. Бытие не есть
\emph{всеобщее}, определенность не есть \emph{особенное}. Определенность
еще \emph{не отделилась от бытия}; правда, она уже не
будет отделяться от него, ибо лежащее отныне в основе
истинное есть единство небытия с бытием; на этом единстве
как на основе зиждутся все дальнейшие определения.
Но соотношение здесь определенности с бытием
есть непосредственное единство обоих, так что еще не
положено никакого различения их.

Определенность как изолированная сама по себе, как
\emph{сущая} определенность, есть \emph{качество}~"--- нечто совершенно
простое, непосредственное. \emph{Определенность} вообще есть
более общее, которое точно так же может быть и количественным,
и далее определенным. Ввиду этой простоты
нечего более сказать о качестве, как таковом.

Но наличное бытие, в котором содержатся и ничто, и
бытие, само служит масштабом для односторонности качества
как лишь \emph{непосредственной} или \emph{сущей} определенности.
Качество должно быть положено и в определении
ничто, благодаря чему непосредственная или \emph{сущая} определенность
полагается как некая различенная, рефлектированная
определенность и, таким образом, ничто как то,
чт\'о определенно в некоторой определенности, есть также
нечто рефлектированное, некое \emph{отрицание}. Качество, взятое
таким образом, чтобы оно, будучи различенным, считалось
\emph{сущим}, есть \emph{реальность}; оно же, обремененное некоторым
отрицанием, есть \emph{отрицание} вообще, а также
некоторое качество, считающееся, однако, недостатком и
определяющееся в дальнейшем как граница, предел.

Оба суть наличное бытие; но в \emph{реальности} как качестве,
в котором акцентируется то, что оно \emph{сущее}, скрыто
то обстоятельство, что оно содержит определенность, следовательно,
и отрицание; реальность считается поэтому
лишь чем-то положительным, из которого исключены
отрицание, ограниченность, недостаток. Отрицание только
как недостаток было бы то же, что и ничто; но оно
наличное бытие, качество, только определяемое посредством
небытия.


%%% Local Variables:
%%% mode: latex
%%% TeX-master: "../../../../main"
%%% End:
