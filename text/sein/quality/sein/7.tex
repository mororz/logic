Единство, моменты которого, бытие и ничто, даны как
неразделимые, в то же время отлично от них самих и таким
образом есть в отношении их некое \emph{третье}, которое
в своей самой характерной форме есть \emph{становление}. \emph{Переход}
есть то же, чт\'о и становление, с той лишь разницей,
что оба [момента], от одного из которых совершается переход
к другому, в становлении представляют себе скорее
как находящиеся в покое друг вне друга, а переход~-- как
совершающийся \emph{между} ними. Где бы и как бы ни шла
речь о бытии или ничто, непременно должно наличествовать
это третье; ведь бытие и ничто существуют не сами
по себе, а лишь в становлении, в этом третьем. Но это
третье имеет многоразличные эмпирические образы, которые
абстракция оставляет в стороне или которыми она
пренебрегает, чтобы фиксировать каждый из ее продуктов~--
бытие и ничто~-- особо и показать их защищенными
от перехода. В противовес такому простому способу
абстрагирования следует столь же просто сослаться лишь
на эмпирическое существование, в котором сама эта абстракция
есть лишь нечто, обладает наличным бытием.
Или же фиксировать разделение неразделимых должны
другие формы рефлексии. В таком определении само по
себе имеется его противоположность, так что и не восходя
к природе вещей и не апеллируя к ней, можно изобличить
это определение рефлексии в нем самом, беря его
так, как оно само себя дает, и в нем самом обнаруживая
его иное. Было бы тщетно стараться как бы схватить все
извороты, все неожиданные мысли рефлексии и ее рассуждения,
чтобы лишить ее возможности пользоваться теми
лазейками и увертками, при помощи которых она
скрывает от себя свое противоречие с самой собой. Поэтому
я и отказываюсь принимать во внимание те многочисленные,
так называющие себя возражения и опровержения,
которые приводились против того [взгляда], что
ни бытие, ни ничто не есть нечто истинное, а что их истина~--
это только становление. Культура мысли, требующаяся
для того, чтобы усмотреть ничтожность этих опровержений,
или, вернее, чтобы отогнать от самого себя такие
неожиданные мысли, достигается лишь благодаря критическому
познанию форм рассудка. Но те, кто щедрее всего
на подобного рода возражения, сразу нападают со своими
соображениями на первые положения, не давая себе
труда до или после этого путем дальнейшего изучения
логики помочь себе осознать природу этих плоских соображений.

Здесь следует рассмотреть некоторые явления, возникающие
от того, что изолируют друг от друга бытие и ничто
и полагают одно вне сферы другого, так что тем самым
отрицается переход.

Парменид признавал только бытие и был как нельзя
более последователен, говоря в то же время о ничто, что
его \emph{вовсе нет}; имеется лишь бытие. Бытие, взятое совершенно
отдельно, есть неопределенное, следовательно, никак
не соотносится с иным; поэтому кажется, что, исходя
\emph{из этого начала}, а именно из самого бытия, нельзя
\emph{двигаться дальше}, что, для того чтобы двинуться дальше,
надо присоединить к нему \emph{извне} нечто чуждое. Дальнейшее
движение, [выражаемое положением о том], что бытие
есть то же самое, что ничто, представляется, стало
быть, как второе, абсолютное начало~-- как переход, стоящий
отдельно и внешне примыкающий к бытию. Бытие
вообще не было бы абсолютным началом, если бы у него
была какая-нибудь определенность; оно тогда зависело бы
от иного и не было бы непосредственным, не было бы началом.
Если же оно неопределенно и тем самым есть
истинное начало, то у него и нет ничего такого, с помощью
чего оно переходило бы в иное, оно в то же время
есть и \emph{конец}. Столь же мало может что-либо вырваться
из него, как и ворваться в него; у Парменида, как и у
Спинозы, нет продвижения от бытия или абсолютной субстанции
к отрицательному, конечному. Если же все-таки
совершается такое продвижение (что, исходя из бытия,
лишенного соотношений и, стало быть, лишенного продвижения,
можно, как мы заметили, осуществить только
внешне), то это движение есть второе, новое начало. Так,
у Фихте его абсолютнейшее, безусловное основоположение
$A = A$ есть \emph{полагание}; второе основоположение~--
\emph{противополагание}; это второе основоположение, согласно
Фихте, \emph{отчасти} обусловлено, \emph{отчасти} безусловно (оно,
следовательно, есть противоречие внутри себя). Это~--
продвижение внешней рефлексии, которое снова так же
отрицает то, с чего оно начинает как с чего-то абсолютного,~--
противополагание есть отрицание первого тождества,~--
как тотчас же определенно делает свое второе
безусловное обусловленным. Но если бы [здесь] поступательное
движение, т.\,е. снятие первого начала, было вообще
правомерно, то в самом этом первом должна была
бы заключаться возможность соотнесения с ним некоего
иного; оно, стало быть, должно было бы быть чем-то \emph{определенным}.
Однако \emph{бытие} или даже абсолютная субстанция
не выдает себя за таковое. Напротив. Оно есть \emph{непосредственное},
еще всецело \emph{неопределенное}.

Самые красноречивые, быть может, забытые описания
причины того, почему невозможно от некоторой абстракции
прийти к чему-то дальнейшему и к их объединению,
дает Якоби в интересах своей полемики с кантовским априорным
\emph{синтезом} самосознания в своей статье <<О предпринятой
критицизмом попытке довести разум до рассудка>>
(Jac. Werke, III Bd.). Он ставит (стр.\,113) задачу
так, что требуется в чем-то \emph{чистом}, будь то чистое сознание,
чистое пространство или чистое время, обнаружить
возникновение или порождение некоего синтеза. <<Пространство
есть \emph{одно}, время есть \emph{одно}, сознание есть \emph{одно};
скажите же, каким образом какое-либо из этих трех
<<одно>> в самом себе, в своей \emph{чистоте} приобретает характер
многообразия? Каждое из них есть лишь нечто \emph{одно}
и не есть \emph{никакое иное}: одинаковость (Einerleiheit),
<<этот>>, <<эта>>, <<это>> в их \emph{тождестве} (eine Der-Die-Das-Selbigkeit)
без того, чт\'о присуще <<этому>>, <<этой>>, <<этому>>
(ohne Derheit, Dieheit, Dasheit), ибо оно еще дремлет
вместе с <<этот>>, <<эта>>, <<это>> в бесконечности $= 0$ неопределенного,
из которой еще только должно произойти
все и всякое \emph{определенное}! Чем вносится \emph{конечность} в
эти три бесконечности? Что оплодотворяет a priori пространство
и время числом и мерой и превращает их в
нечто \emph{чистое многообразное}? Что приводит в колебание
\emph{чистую спонтанность} (<<Я>>) (Ich)? Каким образом его
чистая гласная получает согласную, или, лучше сказать,
каким образом приостанавливается, прерывая само себя,
его \emph{беззвучное} непрерывное \emph{дуновение}, чтобы приобрести
по крайней мере некоторый род гласной, некоторое
\emph{ударение}?>>~-- Как видно, Якоби очень определенно признавал
абсурдность (Unwesen) абстракции, будь она так
называемое абсолютное, т.\,е. лишь абстрактное, пространство,
или такое же время, или такое же чистое сознание,
<<Я>>. Он настаивает на этом, чтобы доказать, что
продвижение к иному~-- к условию синтеза~-- и к самому
синтезу невозможно. Этот интересующий нас синтез не
следует понимать как связь \emph{внешне} уже имеющихся определений,~--
отчасти дело идет о порождении некоторого
второго, присоединяющегося к некоторому первому, о
порождении некоторого определенного, присоединяющегося
к неопределенному первоначальному, отчасти же об
\emph{имманентном} синтезе, синтезе a priori,~-- о в-себе-и-для-себя-сущем
единстве различных [моментов]. \emph{Становление}
и есть этот имманентный синтез бытия и ничто. Но так
как синтезу ближе всего по смыслу внешнее сведение
вместе [определений], находящихся во внешнем отношении
друг к другу, то справедливо перестали пользоваться названиями
<<синтез>>, <<синтетическое единство>>.~-- Якоби
спрашивает, \emph{каким образом} чистая гласная <<Я>> получает
согласную, чт\'о вносит определенность в неопределенность?
На вопрос: чт\'о?~-- было бы нетрудно ответить, и
Кант по-своему дал ответ на этот вопрос. А вопрос: \emph{как}?
означает: каким способом, по каким отношениям и т.\,п.,
и требует, стало быть, указать некоторую особую категорию;
но о способе, о рассудочных категориях здесь не может
быть и речи. Вопрос: как? сам представляет собой
одну из дурных манер рефлексии, которая спрашивает о
постижимости, но при этом берет предпосылкой свои застывшие
категории и тем самым знает наперед, что она
вооружена против ответа на то, о чем она спрашивает.
Более высокого смысла, заключенного в вопросе о \emph{необходимости}
синтеза, он не имеет также и у Якоби, ибо
последний, как сказано, крепко держится за абстракции,
защищая утверждение о невозможности синтеза. С особенной
наглядностью он описывает (стр.\,147) процедуру,
посредством которой достигают абстракции пространства.
<<Я должен на столь долгое время стараться начисто забыть,
что я когда-либо что-нибудь видел, слышал, к чему-либо
прикасался, причем я определенно не должен
делать исключения и для самого себя. Я должен начисто,
начисто, начисто забыть всякое движение, и это последнее
\emph{забвение} я должен осуществить самым старательным образом
именно потому, что оно всего труднее. И все вообще
я должен всецело и полностью \emph{удалить}, как я его уже
мысленно устранил, и ничего не должен сохранить, кроме
одного лишь \emph{насильственно} остановленного созерцания
одного лишь бесконечного \emph{неизменного пространства}.
Я поэтому не вправе \emph{снова в него мысленно включать} самого
себя как нечто отличное от него и, однако, связанное
с ним; я не вправе просто давать себя \emph{окружить} и
\emph{проникнуться} им, а должен полностью \emph{перейти} в него,
стать с ним единым, превратиться в него; я не должен
ничего оставить от себя, кроме самого \emph{этого моего созерцания},
чтобы рассматривать это созерцание как истинно
самостоятельное, независимое, единое и единственное
представление>>.

При такой совершенно абстрактной чистоте непрерывности,
т.\,е. при этой неопределенности и пустоте представления,
безразлично, будем ли мы называть эту абстракцию
пространством, чистым созерцанием или чистым
мышлением; все это~-- то же самое, чт\'о индус называет
\emph{брамой}, когда он, оставаясь внешне неподвижным и не
побуждаемым никакими ощущениями, представлениями,
фантазиями, вожделениями и т.\,д., годами смотрит лишь
на кончик своего носа и лишь говорит внутренне, в себе,
<<ом, ом, ом>>, или вообще ничего не говорит. Это заглушённое,
пустое сознание, понимаемое как сознание, есть
\emph{бытие}.

В этой пустоте, говорит далее Якоби, с ним происходит
противоположное тому, чт\'о должно было бы произойти
с ним согласно уверению Канта; он ощущает себя не
каким-то \emph{множественным} и \emph{многообразным}, а, наоборот,
единым без всякой множественности, без всякого многообразия;
более того: <<Я сама \emph{невозможность}, \emph{уничтожение}
всякого многообразного и множественного\dots Исходя из
своей чистой, совершенно простой и неизменной сущности,
я \emph{не в состоянии} хоть что-нибудь \emph{восстановить} или
вызвать в себе как призрак\dots Таким образом, в этой чистоте
все внеположное и рядоположное, всякое покоящееся
на нем многообразие и множественность обнаруживаются
как \emph{чистая невозможность}>> (стр.\,149).

Эта невозможность есть не что иное, как тавтология,
она означает, что я упорно держусь абстрактного единства
и исключаю всякую множественность и всякое многообразие,
пребываю в том, чт\'о лишено различий и неопределенно,
и отвращаю свой взор от всего различенного
и определенного. В такую же абстракцию Якоби превращает
кантовский априорный синтез самосознания, т.\,е. деятельность
этого единства, состоящую в том, что оно расщепляет
себя и в этом расщеплении сохраняет само себя.
Этот <<синтез \emph{в себе}>>, <<\emph{первоначальное суждение}>> он односторонне
превращает (стр.\,125) в <<\emph{связку в себе}~-- в
[словечко] <<\emph{есть}>>, <<\emph{есть}>>, <<\emph{есть}>>, без начала и конца и
без <<что>>, <<кто>> и <<какие>>. Это продолжающееся до бесконечности
повторение повторения~-- единственное занятие,
функция и произведение наичистейшего синтеза; сам
синтез есть само голое, чистое, абсолютное повторение>>.
Или, в самом деле, так как в нем нет никакого перерыва
(Absatz), т.\,е. никакого отрицания, различения, то он не
повторение, а только неразличенное простое бытие.~-- Но
есть ли это еще синтез, если Якоби опускает как раз то,
благодаря чему единство есть синтетическое единство?

Если Якоби так укрепился в абсолютном, т.\,е. абстрактном
пространстве, времени, а также сознании, то
прежде всего следует сказать, что он таким образом обитает
и удерживается в чем-то \emph{эмпирически} ложном. \emph{Нет},
т.\,е. эмпирически не существует, такого пространства и
времени, которые были бы чем-то неограниченно пространственным
и временн\'ым, которые не были бы в своей
непрерывности наполнены многообразно ограниченным
наличным бытием и изменением, так что эти границы и
изменения нераздельно и неотделимо принадлежат пространственности
и временности. И точно так же сознание
наполнено определенными чувствами, представлениями,
желаниями и т.\,д.; оно существует нераздельно от какого
бы то ни было особого содержания.~-- Эмпирический
\emph{переход} и без того понятен сам собой; сознание может,
правда, сделать своим предметом и содержанием пустое
пространство, пустое время и само пустое сознание, или
чистое бытие, но оно на этом не останавливается и не
только выходит, но вырывается из такой пустоты, устремляясь
к лучшему, т.\,е. к каким-то образом более конкретному
содержанию, и, как бы плохо ни было в остальном
то или иное содержание, оно постольку лучше и истиннее;
именно такого рода содержание есть синтетическое
содержание вообще, синтетическое в более всеобщем
смысле. Так, Пармениду приходится иметь дело с видимостью
и мнением~-- с противоположностью бытия и
истины; так же Спинозе~-- с атрибутами, модусами, протяжением,
движением, рассудком, волей и т.\,д. Синтез
содержит и показывает неистинность указанных выше
абстракций; в нем они находятся в единстве со своим
иным, следовательно, даны не как сами по себе существующие,
не как абсолютные, а всецело как относительные.

Но речь идет не о показывании эмпирической ничтожности
пустого пространства и т.\,д. Сознание может, конечно,
путем абстрагирования наполнить себя и таким неопределенным
[содержанием], и фиксированные абстракции~--
это \emph{мысли} о чистом пространстве, чистом времени, чистом
сознании, чистом бытии. Должна быть показана
ничтожность мысли о чистом пространстве и т.\,д., т, е.
ничтожность чистого пространства \emph{самого по себе} и т.\,д.,
т.\,е. должно быть показано, что оно, как таковое, уже
есть своя противоположность, что в него самого уже проникла
его противоположность, что оно уже само по себе
есть совершившийся выход (das Herausgegangensein) из
самого себя~-- определенность.

Но это происходит непосредственно в них же. Они,
как подробно описывает Якоби, суть результаты абстракции,
ясно определены как \emph{неопределенное}, которое~-- если
обратиться к его простейшей форме~-- есть бытие. Но
именно эта \emph{неопределенность} и есть то, чт\'о составляет
его определенность; ибо неопределенность противоположна
определенности; она, стало быть, как противоположное,
сама есть определенное, или отрицательное, и
притом чистое, совершенно абстрактное отрицательное.
Эта-то неопределенность или абстрактное отрицание, которое
бытие имеет таким образом в самом себе, и есть то,
чт\'о высказывает и внешняя, и внутренняя рефлексия,
приравнивая бытие к ничто, объявляя его пустым порождением
мысли, ничем.~-- Или можно это выразить
иначе: так как бытие есть то, чт\'о лишено определений, то
оно не (утвердительная) определенность, которая оно
есть, не бытие, а ничто.

В чистой рефлексии начала, каковым в этой логике
является \emph{бытие}, как таковое, переход еще скрыт. Так как
\emph{бытие} положено лишь как непосредственное, то \emph{ничто}
выступает в нем наружу лишь непосредственно. Но все
последующие определения, как, например, \emph{наличное бытие},
более конкретны; в последнем уже \emph{положено} то, чт\'о
содержит и порождает противоречие указанных выше
абстракций, а потому и их переход. Напоминание о том,
что бытие как указанное простое, непосредственное есть
результат полной абстракции и, стало быть, уже потому
абстрактная отрицательность, ничто,~-- это напоминание
оставлено за порогом науки, которая в своих пределах,
особенно в разделе о \emph{сущности}, изобразит эту одностороннюю
\emph{непосредственность} как нечто опосредствованное,
где \emph{положено} бытие как \emph{существование}, а также основание~--
то, что опосредствует это бытие.

С помощью этого напоминания можно представить
или даже, как это называют, \emph{объяснить}\endnotemark{} и \emph{сделать постижимым}
переход бытия в ничто как нечто даже легкое
и тривиальное: бытие, сделанное [нами] началом науки,
есть, разумеется, ничто, ибо абстрагироваться можно от
всего, а когда мы от всего абстрагировались, остается ничто.
Но, можно продолжить, тем самым начало [здесь] не
нечто утвердительное, не бытие, а как раз ничто, и ничто
оказывается в таком случае и \emph{концом}; оно оказывается
этим концом в такой же мере, как непосредственное бытие,
и даже в еще большей мере, чем последнее. Проще
всего дать такому резонерству полную волю и посмотреть,
каковы результаты, которыми оно кичится. То обстоятельство,
что согласно этому ничто оказалось бы результатом
этого резонерства и что теперь следует начинать (как в
китайской философии) с ничто~-- ради этого не стоило бы
и пальцем шевельнуть, ибо раньше, чем мы шевельнули
бы им, это ничто точно так же превратилось бы в бытие
(см. выше: В. Ничто). Но, далее, если бы предполагали
такое абстрагирование от \emph{всего}, а ведь это все есть \emph{сущее},
то следует отнестись к нему более серьезно; результат
абстрагирования от всего сущего~-- это прежде всего абстрактное
бытие, \emph{бытие} вообще; так, в космологическом
доказательстве бытия бога из случайного бытия мира
(в этом доказательстве возвышаются над таким бытием)
бытие поднимается нами выше и приобретает определение
\emph{бесконечного бытия}. Но, разумеется, \emph{можно} абстрагироваться
и от этого чистого бытия, присоединить и бытие
ко всему, от чего уже абстрагировались; тогда остается
ничто. Затем, если решить забыть \emph{мышление} об этом ничто,
т.\,е. о его переходе в бытие, или если бы ничего не
знали об этом, \emph{можно} продолжать в стиле этой возможности;
а именно можно (слава богу!) абстрагироваться также
и от этого ничто (сотворение мира и в самом деле есть
абстрагирование от ничто), и тогда остается не ничто,
ибо как раз от него абстрагируются, а снова приходят к
бытию.~-- Эта \emph{возможность} дает внешнюю игру абстрагирования,
причем само абстрагирование есть лишь одностороннее
действование отрицательного. Сама эта возможность
состоит прежде всего в том, что для нее бытие так
же безразлично, как и ничто, и что в какой мере каждое
из них исчезает, в такой же мере и возникает; но столь
же безразлично, отправляться ли от действования ничто
или от ничто; действование ничто, т.\,е. одно лишь абстрагирование,
есть нечто истинное не больше и не меньше,
чем чистое ничто.

\endnotetext{В <<Феноменологии духа>> Гегель называет объяснение тавтологическим
  движением рассудка, <<которое не только ничего не
  объясняет, но отличается такой ясностью, что, собираясь сказать
  что-нибудь отличное от уже сказанного, оно скорее ничего не
  высказывает, а лишь повторяет то же самое>> (стр.\,85).}

Ту диалектику, в соответствии с которой Платон трактует
в <<Пармениде>> единое, также следует признать больше
диалектикой внешней рефлексии. Бытие и единое
суть оба элеатские формы, представляющие собой одно и
то же. Но их следует также отличать друг от друга. Такими
и берет их Платой в упомянутом диалоге. Удалив
из единого разнообразные определения целого и частей,
бытия в себе и бытия в ином и т.\,д., определения фигуры,
времени и т.\,д., он приходит к выводу, что единому не
присуще бытие, ибо бытие присуще некоторому нечто не
иначе, как в соответствии с одним из указанных способов
(р.\,141, е, Vol. Ill, ed. Steph.)\endnotemark{}. Затем Платон рассматривает
положение: \emph{единое есть}; и у него можно проследить,
каким образом, согласно этому положению, совершается
переход к \emph{небытию} единого: это происходит путем
\emph{сравнения} обоих определений предположенного положения:
\emph{единое есть}. В этом положении содержится единое
\emph{и} бытие, и <<единое есть>> содержит нечто большее, чем
если бы мы сказали лишь: <<единое>>. В том, что они \emph{различны},
раскрывается содержащийся в положении момент
отрицания. Ясно, что этот путь имеет некое предположение
и есть некоторая внешняя рефлексия.

\endnotetext{См. \emph{Платон}. Сочинения в трех томах, т.\,2. М., 1970, стр.\,428.}

Так же как единое приведено здесь в связь с бытием,
так и бытие, которое должно быть фиксировано абстрактно,
\emph{особо}, самым простым образом, не пускаясь в
мышление, раскрывается в связи, содержащей противоположное
тому, чт\'о должно утверждаться. Бытие, взятое
так, как оно есть непосредственно, принадлежит некоторому
\emph{субъекту}, есть нечто высказанное, обладает вообще
некоторым эмпирическим \emph{наличным бытием} и потому
стоит на почве предельного и отрицательного. В каких
бы терминах или оборотах ни выражал себя рассудок,
когда он отвергает единство бытия и ничто и ссылается
на то, чт\'о, дескать, непосредственно наличествует, он как
раз в этом опыте не найдет ничего другого, кроме \emph{определенного}
бытия, бытия с некоторым пределом или отрицанием,~--
не найдет ничего другого, кроме того единства,
которого не признает. Утверждение о непосредственном
бытии сводится таким образом к [утверждению] о
некотором эмпирическом существовании, от \emph{раскрытия}
которого оно не может отказаться, так как оно ведь желает
держаться именно непосредственности, существующей
вне мышления.

Точно так же обстоит дело и с \emph{ничто}, только противоположным
образом, и эта рефлексия известна и довольно
часто применялась к нему. Ничто, взятое в своей непосредственности,
оказывается \emph{сущим}, ибо по своей природе
оно то же самое, что и бытие. Мы мыслим ничто, представляем
его себе, говорим о нем; стало быть, оно \emph{есть};
ничто имеет свое бытие в мышлении, представлении,
речи и т.\,д. Но, кроме того, это бытие также и отлично
от него; поэтому, хотя и говорят, что ничто есть в мышлении,
представлении, но это означает, что не \emph{оно есть}, не
ему, как таковому, присуще бытие, а лишь мышление или
представление есть это бытие. При таком различении
нельзя также отрицать, что ничто находится в \emph{соотношении}
с некоторым бытием; но в этом соотношении, хотя оно
и содержит также различие, имеется единство с бытием.
Как бы ни высказывались о ничто или показывали его,
оно оказывается связанным или, если угодно, соприкасающимся
с некоторым бытием, оказывается неотделимым
от некоторого бытия, а именно находящимся в некотором
\emph{наличном бытии}.

Однако при таком показывании ничто в некотором
наличном бытии обычно все еще предстает следующее
отличие его от бытия: наличное бытие ничто (des Nichts)
вовсе-де не присуще ему самому, оно, само по себе взятое,
не имеет в себе бытия, оно не \emph{есть} бытие, как таковое;
ничто есть-де лишь отсутствие бытия; так, тьма~-- это
лишь \emph{отсутствие} света, холод~-- отсутствие тепла и т.\,д.
Тьма имеет-де значение лишь в отношении к глазу, во
внешнем сравнении с положительным, со светом, и точно
так же холод есть нечто лишь в нашем ощущении; свет
же, тепло, как и бытие, суть сами по себе объективное,
реальное, действенное, обладают совершенно другим качеством
и достоинством, чем указанные отрицательные
[моменты], чем ничто. Часто приводят как очень важное
соображение и значительное знание утверждение, что
тьма есть \emph{лишь отсутствие} света, холод~-- \emph{лишь отсутствие}
тепла. Относительно этого остроумного соображения
можно, оставаясь в этой области эмпирических предметов,
с эмпирической точки зрения заметить, что в
самом деле тьма оказывается действенным в свете, обусловливая
то, что свет становится цветом\endnotemark{} и лишь благодаря
этому сообщая ему зримость, ибо, как мы сказали
раньше, в чистом свете так же ничего не видно, как и в
чистой тьме. А зримость есть такая действенность в глазу,
в которой указанное отрицательное принимает такое
же участие, как и свет, считающийся реальным, положительным;
и точно так же холод дает себя достаточно почувствовать
воде, нашему ощущению и т.\,д., и если мы
ему отказываем в так называемой объективной реальности,
то от этого в нем ничего не убывает. Но, далее, достойно
порицания то, что здесь, как и выше, говорят о
чем-то отрицательном, обладающем определенным содержанием,
идут дальше самого ничто, по сравнению с которым
у бытия не больше и не меньше пустой абстрактности.~--
Однако следует тотчас же брать холод, тьму
и тому подобные определенные отрицания сами по себе и
посмотреть, чт\'о этим положено в отношении их всеобщего
определения, с которым мы теперь имеем дело. Они
должны быть не ничто вообще, а ничто света, тепла и т.\,д.,
ничто чего-то определенного, какого-то содержания; они,
таким образом, если можно так выразиться, определенные,
содержательные ничто. Но определенность, как мы
это еще увидим дальше, сама есть отрицание; таким образом,
они отрицательные ничто; но отрицательное ничто
есть нечто утвердительное. Превращение ничто через его
определенность (которая раньше представала как некое
\emph{наличное бытие} в субъекте или в чем бы то ни было другом)
в некоторое утвердительное представляется сознанию,
застревающему в рассудочной абстракции, как верх
парадоксальности; как ни прост взгляд, что отрицание
отрицания есть положительное, он, быть может, именно
из-за самой этой его простоты представляется чем-то
тривиальным, с которым гордому рассудку нет поэтому
надобности считаться, хотя это имеет свое основание,~--
а между тем оно не только имеет свое основание, но благодаря
всеобщности таких определений обладает бесконечным
распространением и всеобщим применением, так
что все же следовало бы с ним считаться.

\endnotetext{Гегель излагает теорию цвета Гёте.}

Относительно определения перехода бытия и ничто
друг в друга можно еще заметить, что этот переход следует
постигать, не прибегая к дальнейшим определениям
рефлексии. Он непосредствен и всецело абстрактен из-за
абстрактности переходящих моментов, т.\,е. вследствие
того, что в этих моментах еще не положена определенность
другого, посредством чего они переходили бы друг
в друга; ничто еще не \emph{положено} в бытии, хотя бытие есть
\emph{по своему существу} ничто, и наоборот. Поэтому недопустимо
применять здесь дальнейшие определенные опосредствования
и понимать бытие и ничто находящимися
в каком-то отношении,~-- этот переход еще не отношение.
Недозволительно, стало быть, говорить: ничто~-- \emph{основание}
бытия или бытие~-- \emph{основание} ничто; ничто~-- \emph{причина}
бытия и т.д.; или сказать: переход в ничто возможен
лишь \emph{при условии}, что нечто \emph{есть}, или: переход в бытие
возможен лишь \emph{при условии}, что есть небытие. Род соотношения
не может получить дальнейшее определение,
если бы не были в то же время далее определены соотносящиеся
\emph{стороны}. Связь основания и следствия и т.\,д.
имеет своими сторонами, которые она связывает, уже не
просто бытие и просто ничто, а непременно такое бытие,
которое есть основание, и нечто такое, что, хотя и есть
лишь нечто положенное, несамостоятельное, однако не
есть абстрактное ничто.

%%% Local Variables:
%%% mode: latex
%%% TeX-master: "../../../main"
%%% End:
