Становление есть нераздельность бытия и ничто~-- не
единство, абстрагирующееся от бытия и ничто; как единство
\emph{бытия} и \emph{ничто} оно есть это \emph{определенное} единство,
или, иначе говоря, такое единство, в котором \emph{есть} и бытие,
и ничто. Но так как каждое из них, и бытие, и ничто,
нераздельно от своего иного, то \emph{их нет}. Они, следовательно,
\emph{суть} в этом единстве, но как исчезающие, лишь
как \emph{снятые}. Теряя свою \emph{самостоятельность}, которая, как
первоначально представлялось, была им присуща, они
низводятся до \emph{моментов, еще различимых}, но в то же
время снятых.

Взятые со стороны этой своей различимости, каждый
из них есть \emph{в этой различимости} единство с \emph{иным}. Становление
содержит, следовательно, бытие и ничто как \emph{два
таких единства}, каждое из которых само есть единство
бытия и ничто. Одно из них есть бытие как непосредственное
бытие и как соотношение с ничто; другое есть ничто
как непосредственное ничто и как соотношение с бытием.
Определения обладают в этих единствах неодинаковой
ценностью.

Становление дано, таким образом, в двояком определении;
в одном определении ничто есть непосредственное,
т.\,е. определение начинает с ничто, соотносящегося
с бытием, т.\,е. переходящего в него; в другом бытие дано
как непосредственное, т.\,е. определение начинает с бытия,
переходящего в ничто,~-- \emph{возникновение} и \emph{прехождение}.

Оба суть одно и то же, становление, и даже как эти
направления, различенные таким образом, они друг друга
проникают и парализуют. Одно есть \emph{прехождение}; бытие
переходит в ничто; но ничто есть точно так же и своя
противоположность, переход в бытие, возникновение. Это
возникновение есть другое направление; ничто переходит
в бытие, но бытие точно так же и снимает само себя и
есть скорее переход в ничто, есть прехождение.~-- Они не
снимают друг друга, одно внешне не снимает другое,
каждое из них снимает себя в себе самом (an sich selbst)
и есть в самом себе (an ihm selbst) своя противоположность.


%%% Local Variables:
%%% mode: latex
%%% TeX-master: "../../../main"
%%% End:
