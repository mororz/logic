Равновесие, в которое приводят себя возникновение
и прехождение,~-- это прежде всего само становление. Но
становление точно так же сходится (geht zusammen) в
\emph{спокойное единство}. Бытие и ничто находятся в становлении
лишь как исчезающие; становление же, как таковое,
имеется лишь благодаря их разности. Их исчезание
есть поэтому исчезание становления, иначе говоря, исчезание
самого исчезания. Становление есть неустойчивое
беспокойство, которое оседает, переходя в некоторый
спокойный результат.

Это можно было бы выразить и так: становление есть
исчезание бытия в ничто и ничто~-- в бытие, и исчезание
бытия и ничто вообще; но в то же время оно основывается
на различии последних. Оно, следовательно, противоречит
себе внутри самого себя, так как соединяет
в себе нечто противоположное себе; но такое соединение
разрушает себя.

Этот результат есть исчезновение (Verschwundensein),
но не как \emph{ничто}; в последнем случае он был бы лишь возвратом
к одному из уже снятых определений, а не результатом
ничто и \emph{бытия}. Этот результат есть ставшее
спокойной простотой единство бытия и ничто. Но спокойная
простота есть \emph{бытие}, однако бытие уже более не для
себя, а бытие как определение целого.

Становление как переход в такое единство бытия и
ничто, которое дано как \emph{сущее} или, иначе говоря, имеет
вид одностороннего \emph{непосредственного} единства этих моментов,
есть \emph{наличное бытие}.


%%% Local Variables:
%%% mode: latex
%%% TeX-master: "../../../main"
%%% End:
