\emph{Ничто} обычно противопоставляют [всякому] \emph{нечто}; но
нечто есть уже определенное сущее, отличающееся от
другого нечто; таким образом и ничто, противопоставляемое
[всякому] нечто, есть ничто какого-нибудь нечто,
определенное ничто. Но здесь д\'олжно брать ничто в его
неопределенной простоте.~-- Если бы кто-нибудь считал
более правильным противопоставлять бытию не ничто,
а \emph{небытие}, то, имея в виду результат, нечего было бы
возразить против этого, ибо в \emph{небытии} содержится соотношение
с \emph{бытием}; оно и то и другое, бытие и его отрицание,
выраженные в \emph{одном}, ничто, как оно есть в становлении.
Но прежде всего речь должна идти не о форме
противопоставления, т.\,е. одновременно и о форме \emph{соотношения},
а об абстрактном, непосредственном отрицании,
о ничто, взятом чисто само по себе, о безотносительном
отрицании,~-- что, если угодно, можно было бы выразить
также и простым \emph{не}.

Простую мысль о \emph{чистом бытии} как об абсолютном
и как о единственной истине впервые высказали \emph{элеаты},
особенно Парменид, который в дошедших до нас фрагментах
высказал ее с чистым воодушевлением мышления,
в первый раз постигшего себя в своей абсолютной абстрактности:
\emph{только бытие есть, а ничто вовсе нет}.~--
В восточных системах, особенно в буддизме, \emph{ничто}, пустота,
составляет, как известно, абсолютный принцип.~--
Глубокий мыслитель Гераклит выдвигал против указанной
простой и односторонней абстракции более высокое,
целокупное понятие становления и говорил: \emph{бытия нет
точно так же, как нет ничто}, или, выражая эту мысль
иначе, \emph{все течет}, т.\,е. все есть \emph{становление}.~-- Общедоступные
изречения, в особенности восточные, гласящие,
что все, что есть, имеет зародыш своего уничтожения
в самом своем рождении, а смерть, наоборот, есть вступление
в новую жизнь, выражают в сущности то же единение
бытия и ничто. Но эти выражения предполагают
субстрат, в котором совершается переход: бытие и ничто
обособлены друг от друга во времени, представлены как
чередующиеся в нем, а не мыслятся в их абстрактности,
и поэтому мыслятся не так, чтобы они сами по себе были,
одним и тем же.

Ex nihilo nihil fit~-- это одно из положений, которым
в метафизике приписывалось большое значение. В этом
положении можно либо усматривать лишь бессодержательную
тавтологию: ничто есть ничто; либо, если действительным
смыслом этого положения должно быть
[высказывание о] \emph{становлении}, то следует сказать, что
так как из \emph{ничего становится} только \emph{ничто}, то на самом
деле здесь нет речи о \emph{становлении}, ибо ничто так и
остается здесь ничем. Становление означает, что ничто
не остается ничем, а переходит в свое иное, в бытие.~--
Если позже метафизика, особенно христианская, отвергла
положение о том, что из ничего ничего не происходит,
то она этим утверждала, что ничто переходит в \emph{бытие};
как бы она ни брала последнее положение~-- в виде ли
синтеза или просто в виде представления,~-- даже в самом
несовершенном соединении имеется точка, в которой
бытие и ничто встречаются и их различие исчезает.~--
Положение: \emph{из ничего ничего не происходит, ничто есть
именно ничто}, приобретает свое настоящее значение благодаря
тому, что противопоставляется \emph{становлению} вообще
и, следовательно, также сотворению мира из ничего.
Те, кто высказывает и даже горячо отстаивает положение:
ничто есть именно ничто, не сознают, что они тем
самым соглашаются с абстрактным \emph{пантеизмом} элеатов
и по сути дела также и со спинозовским пантеизмом.
Философское воззрение, которое считает принципом положение
<<бытие~-- это только бытие, ничто~-- это только
ничто>>, заслуживает названия системы тождества; это
абстрактное тождество составляет сущность пантеизма.

Если вывод, что бытие и ничто суть одно и то же,
взятый сам по себе, кажется удивительным или парадоксальным,
то не следует больше обращать на это внимания;
скорее приходится удивляться удивлению тех, кто
показывает себя таким новичком в философии и забывает,
что в этой науке встречаются совсем иные определения,
чем определения обыденного сознания и так называемого
здравого человеческого рассудка, который не
обязательно здравый, а бывает и рассудком, возвышающимся
до абстракций и до веры в них или, вернее, до
суеверного отношения к абстракциям. Было бы нетрудно
показать это единство бытия и ничто на любом примере,
во \emph{всякой} действительной вещи или мысли. О бытии и
ничто следует сказать то же, чт\'о было сказано выше
о непосредственности и опосредствовании (заключающем
в себе некое соотношение \emph{друг с другом} (aufeinander)
и, значит, \emph{отрицание}), а именно, что \emph{нет ничего ни
на небе, ни на земле, что не содержало бы в себе и бытие
и ничто}. Разумеется, так как при этом речь заходит
о \emph{каком-то нечто} и \emph{действительном}, то в этом нечто указанные
определения наличествуют уже не в той совершенной
неистинности, в какой они выступают как бытие
и ничто, а в некотором дальнейшем определении и понимаются,
например, как \emph{положительное} и \emph{отрицательное};
первое есть положенное, рефлектированное бытие, а последнее
есть положенное, рефлектированное ничто; но
положительное и отрицательное содержат как свою абстрактную
основу: первое~-- бытие, а второе~-- ничто.~--
Так, в самом боге качество, \emph{деятельность}, \emph{творение}, \emph{могущество}
и т.\,д. содержат как нечто сущностное определение
отрицательного,~-- они создают некое \emph{иное}. Но эмпирическое
пояснение указанного утверждения примерами
было бы здесь совершенно излишне. Так как это
единство бытия и ничто раз навсегда лежит в основе как
первая истина и составляет стихию всего последующего,
то помимо самого становления все дальнейшие логические
определения: наличное бытие, качество, да и вообще
все понятия философии служат примерами этого единства.
А так называющий себя обыденный или здравый
человеческий рассудок, поскольку он отвергает нераздельность
бытия и ничто, пусть попытается отыскать
пример, в котором одно оказалось бы отделенным от
другого (нечто от границы, предела, или бесконечное,
бог, как мы только что упомянули, от деятельности).
Только пустые порождения мысли (Gedankendinge)~--
бытие и ничто~-- только сами они и суть такого рода
раздельные, и их-то этот рассудок предпочитает истине,
нераздельности того и другого, которую мы всюду имеем
перед собой.

Нашим намерением не может быть предупреждать все
случаи, когда обыденное сознание сбивается с толку при
рассмотрении подобного рода логических положений, ибо
случаи эти неисчислимы. Мы можем коснуться лишь некоторых
из них. Одной из причин такой путаницы служит,
между прочим, то обстоятельство, что сознание
привносит в такие абстрактные логические положения
представления о некотором конкретном нечто и забывает,
что речь идет вовсе не о нем, а лишь о чистых абстракциях
бытия и ничто, и что только их необходимо придерживаться.

Бытие и небытие суть одно и то же; \emph{следовательно},
одно и то же, существую ли я или не существую, существует
ли или не существует этот дом, обладаю ли я или
не обладаю ста талерами. Это умозаключение или применение
указанного положения совершенно меняет его
смысл. В указанном положении говорится о чистых абстракциях
бытия и ничто; применение же делает из них
определенное бытие и определенное ничто. Но об определенном
бытии, как уже сказано, здесь речь не идет.
Определенное, конечное бытие~-- это такое бытие, которое
соотносится с другим бытием: оно содержание, находящееся
в отношении необходимости с другим содержанием,
со всем миром. Имея в виду взаимоопределяющую
связь целого, метафизика могла выставить~-- в сущности
говоря, тавтологическое~-- утверждение, что если бы была
уничтожена одна пылинка, то обрушилась бы вся Вселенная.
В примерах, приводимых против рассматриваемого
нами положения, представляется небезразличным,
существует ли нечто или его нет, не из-за бытия или
небытия, а из-за его \emph{содержания}, связывающего его
с другим содержанием. Когда \emph{предполагается} некое определенное
содержание, какое-то определенное наличное
бытие, то это наличное бытие, потому что оно \emph{определенное},
находится в многообразном соотношении с другим
содержанием. Для него небезразлично, имеется ли
другое содержание, с которым оно соотносится, или его
нет, ибо только через такое соотношение оно по своему
существу есть то, что оно есть. То же самое имеет место
и в \emph{представлении} (поскольку мы берем небытие в более
определенном смысле~-- как представление в противоположность
действительности), в связи с которым небезразлично,
имеется ли бытие или отсутствие содержания,
которое как определенное представляется соотнесенным
с другим содержанием.

Это соображение касается того, что составляет один
из главных моментов в кантовской критике онтологического
доказательства бытия бога, которую, однако, мы
здесь рассматриваем лишь в отношении встречающегося
в ней различения между бытием и ничто вообще и между
\emph{определенными} бытием или небытием.~-- Как известно,
это так называемое доказательство заранее предполагает
понятие существа, которому присущи все реальности и,
следовательно, также существование, каковое также было
принято за одну из реальностей. Кантова критика напирает,
главным образом, на то, что \emph{существование} или бытие
(которые здесь считаются равнозначными) не есть
\emph{свойство} или \emph{реальный предикат}, т.\,е. не есть понятие
чего-то такого, что можно прибавить к \emph{понятию} какой-нибудь
вещи\footnotemark{}.~-- Кант хочет этим сказать, что бытие не
есть определение содержания.~-- Стало быть, продолжает
он, действительное не содержит в себе чего-либо большего,
чем возможное; сто действительных талеров не содержат
в себе ни на йоту больше, чем сто возможных
талеров, а именно первые не имеют другого определения
содержания, чем последние. Для этого, рассматриваемого
как изолированное, содержания в самом деле
безразлично, быть или не быть; в нем нет никакого различия
бытия или небытия, это различие вообще не затрагивает
его: сто талеров не сделаются меньше, если их
нет, и больше, если они есть. Различие должно прийти
откуда-то извне.~-- <<Но,~-- напоминает Кант,~-- мое имущество
больше при наличии ста действительных талеров,
чем при одном лишь понятии их (т.\,е. возможности их).
В самом деле, в случае действительности \emph{предмет} не
только аналитически содержится в моем понятии, \emph{но и
прибавляется синтетически к моему понятию} (которое
служит \emph{определением} моего \emph{состояния}), нисколько не увеличивая
эти мыслимые сто талеров этим бытием вне
моего понятия>>\endnotemark{}.

\footnotetext{Kants Kritik der г. Vera. 2te Aufl. S. 628 ff
  \endnote{<<Критика чистого разума>>, стр.\,521.}.}

\endnotetext{<<Критика чистого разума>>, стр.\,522. Курсив Гегеля.}

Здесь \emph{предполагаются}~-- если сохранить выражения
Канта, не свободные от запутывающей тяжеловесности,~--
двоякого рода состояния: одно, которое Кант называет
понятием и под которым следует понимать представление,
и другое~-- состояние имущества. Для одного, как и
для другого,~-- для имущества, как и для представления,
сто талеров суть определение содержания, или, как
выражается Кант, <<они прибавляются к нему \emph{синтетически}>>.
Я как \emph{обладатель} ста талеров или как необладатель
их или же я как \emph{представляющий} себе сто талеров
или не представляющий себе их~-- это, конечно, разное
содержание. Выразим это в более общем виде:
абстракции бытия и ничто перестают быть абстракциями,
когда они получают определенное содержание; в этом
случае бытие есть реальность, определенное бытие ста
талеров, ничто есть отрицание, определенное небытие
этих талеров. Само же это определение содержания, сто
талеров, рассматриваемое также абстрактно, само по себе,
остается без изменений, одним и тем же и в том, и в другом
случае. Но когда, далее, бытие берется как имущественное
состояние, сто талеров вступают в связь с некоторым
состоянием, и для последнего такого рода определенность,
которую они составляют, не безразлична; их
бытие или небытие есть лишь \emph{изменение}; они перенесены
в сферу \emph{наличного бытия}. Поэтому, если против
единства бытия и ничто возражают, что, мол, не безразлично,
имеется ли то-то (100 талеров) или не имеется,
то заблуждаются, относя различие между моим \emph{обладанием}
и \emph{необладанием} ста талерами только за счет бытия
или небытия. Это заблуждение, как мы показали, основано
на односторонней абстракции, опускающей \emph{определенное
наличное бытие}, которое имеется в такого рода
примерах, и удерживающей лишь бытие и небытие, так
же как и, наоборот, превращающей абстрактное бытие
и [абстрактное] ничто, которое должно постигнуть, в определенное
бытие и ничто, в наличное бытие. Лишь \emph{наличное
бытие} содержит реальное различие между бытием
и ничто, а именно \emph{нечто} и \emph{иное}.~-- Это реальное различие
предстает перед представлением вместо абстрактного
бытия и чистого ничто и лишь мнимого различия между
ними.

Как выражается Кант, <<посредством существования
нечто вступает в контекст совокупного опыта>>. <<Благодаря
этому мы получаем одним предметом \emph{восприятия}
больше, но наше \emph{понятие} о предмете этим не обогащается>>\endnotemark{}.~--
Это, как вытекает из предыдущего разъяснения,
означает следующее: посредством существования, главным
образом потому, что нечто есть определенное существование,
оно находится в связи с \emph{иным}, и, между
прочим, также с неким воспринимающим.~-- Понятие ста
талеров, говорит Кант, не обогащается от того, что их
воспринимают. \emph{Понятием} Кант здесь называет означенные
выше \emph{изолированно} представляемые сто талеров.
В такой изолированности они, правда, суть некоторое
эмпирическое содержание, но содержание оторванное, не
связанное с \emph{иным} и не имеющее определенности в отношении
иного. Форма тождества с собой лишает их соотношения
с иным и делает их безразличными к тому,
восприняты ли они или нет. Но это так называемое \emph{понятие}
ста талеров~-- ложное понятие; форма простого
соотношения с собой не принадлежит самому такому
ограниченному, конечному содержанию; она форма, приданная
ему субъективным рассудком и заимствованная
им у этого рассудка; сто талеров~-- это не нечто соотносящееся
с собой, а нечто изменчивое и преходящее.

\endnotetext{Это не цитаты, а свободное изложение мысли Канта.
Ср. <<Критика чистого разума>>, стр.\,523.}

Мышлению или представлению, перед которыми предстает
лишь какое-то определенное бытие~-- наличное бытие,~--
следует указать на упомянутое выше начало
науки, положенное Парменидом, который свое представление
и тем самым и представление последующих поколений
очистил и возвысил до \emph{чистой мысли}, до бытия,
как такового, и этим создал стихию науки.~-- То, чт\'о
составляет \emph{первый шаг в науке}, должно было явить себя
\emph{первым} и \emph{исторически}. И \emph{единое} или \emph{бытие} в учении
элеатов мы должны рассматривать как первый шаг знания
о мысли; \emph{вода}\endnote{Имеется в виду учение Фалеса о воде как первоначале
всего сущего.} и тому подобные материальные начала,
хотя, \emph{по мнению} выдвигавших их философов, представляли
собой всеобщее, однако как материи они не чистые
мысли; \emph{числа}\endnote{Имеется в виду учение пифагорейцев о числах как сущности вещей.}
же~-- это не первая простая и не
остающаяся самой собой мысль, а мысль, всецело внешняя
самой себе.

Отсылку от \emph{отдельного конечного} бытия к бытию, как
таковому, взятому в его совершенно абстрактной всеобщности,
следует рассматривать как самое первое теоретическое
и даже практическое требование. А именно, если
поднимают шумиху вокруг этих ста талеров, утверждая,
что для моего имущественного состояния не безразлично,
обладаю ли я ими или нет, и тем более не безразлично,
существую ли я или нет, существует ли иное или нет, то
не говоря уже о том, что бывают такие имущественные
состояния, для которых такое обладание ста талерами будет
безразлично,~-- можно напомнить, что человек должен
подняться в своем образе мыслей до такой абстрактной
всеобщности, при которой ему в самом деле будет
безразлично, существуют ли или не существуют эти сто
талеров, каково бы ни было их количественное соотношение
с его имущественным состоянием, как ему будет
столь же безразлично, существует ли он или нет, т.\,е. существует
ли он или нет в конечной жизни (ибо имеется
в виду некое состояние, определенное бытие) и т.\,д. Даже
si fractus illabatur orbis, impavidum ferient ruinae\endnotemark{},
сказал один римлянин, а тем более должно быть присуще
такое безразличие христианину.

\endnotetext{<<Если бы на него обрушился весь мир, он без страха встретил
  бы смерть под его развалинами>>~-- строки из оды Горация
  <<Iiistum et tenacem propositi virum>> (Horatius. Carmina III,~3).
  Гораций рисует в этой оде образ справедливого и постоянного
  в своих намерениях человека, который ничего не боится и которого
  ничто не может вывести из душевного равновесия.}

Следует еще отметить непосредственную связь между
возвышением над ста талерами и вообще над конечными
вещами и онтологическим доказательством и упомянутой
кантовской критикой его. Эта критика показалась всем
убедительной благодаря приведенному ею популярному
примеру; кто же не знает, что сто действительных талеров
отличны от ста лишь возможных талеров? Кто не
знает, что они составляют разницу в моем имущественном
состоянии? Так как на примере ста талеров обнаруживается
таким образом эта разница, то понятие, т.\,е.
определенность содержания как пустая возможность, и
бытие отличны друг от друга; \emph{стало быть}, и понятие бога
отлично от его бытия, и так же как я из возможности
ста талеров не могу вывести их действительность, точно
так же не могу из понятия бога <<вылущить>> (herausklauben)
его существование; а в таком вылущивании существования
бога из его понятия и состоит-де онтологическое
доказательство. Но если несомненно верно, что понятие
отлично от бытия, то бог еще более отличен от ста
талеров и других конечных вещей. В том и состоит \emph{дефиниция
конечных вещей}, что в них понятие и бытие различны,
понятие и реальность, душа и тело отделимы друг
от друга, и потому преходящи и смертны; напротив, абстрактная
дефиниция бога состоит именно в том, что его
понятие и его бытие \emph{нераздельны} и \emph{неотделимы}. Истинная
критика категорий и разума заключается как раз в
том, чтобы сделать познание этого различия ясным и
удерживать его от применения к богу определений и соотношений
конечного.


%%% Local Variables:
%%% mode: latex
%%% TeX-master: "../../../main"
%%% End:
