\emph{Снятие} (Aufheben) и \emph{снятое} (\emph{идеальное}~-- ideelle)~--
одно из важнейших понятий философии, одно из главных
определений, которое встречается решительно всюду и
смысл которого следует точно понять и в особенности отличать
от ничто.~-- Оттого, что нечто снимает себя, оно
не превращается в ничто. Ничто есть \emph{непосредственное};
снятое же есть нечто \emph{опосредствованное}: оно не-сущее, но
как \emph{результат}, имевший своим исходным пунктом некоторое
бытие, поэтому оно \emph{еще} имеет \emph{в себе определенность,
от которой оно происходит}.

Aufheben имеет в немецком языке двоякий смысл: оно
означает сохранить, \emph{удержать} и в то же время прекратить,
\emph{положить конец}. Само сохранение уже заключает в
себе отрицательное в том смысле, что для того, чтобы
удержать нечто, его лишают непосредственности и тем самым
наличного бытия, открытого для внешних воздействий.
Таким образом, снятое есть в то же время и сохраненное,
которое лишь потеряло свою непосредственность,
но от этого не уничтожено.~-- Указанные два определения
\emph{снятия} можно лексически привести как два \emph{значения}
этого слова, но должно представляться странным, что в
языке одно и то же слово обозначает два противоположных
определения. Для спекулятивного мышления отрадно
находить в языке слова, имеющие в самих себе спекулятивное
значение; в немецком языке много таких слов.
Двоякий смысл латинского слова tollere (ставший знаменитым
благодаря остроумному выражению Цицерона:
tollendum esse Octavium)
\endnote{Латинское слово tollere многозначно. Среди его значений
  имеются и такие противоположные, как \emph{возносить}, \emph{возвеличивать}
  и \emph{убирать}, \emph{устранять}. Поэтому приведенное Гегелем выражение
  Цицерона двусмысленно: оно может быть понято и как
  <<д\'олжно вознести Октавия>> и как <<д\'олжно убрать Октавия>>.}
не идет так далеко: утвердительное
определение доходит лишь до [понятия] возвышения.
Нечто снято лишь постольку, поскольку оно вступило
в единство со своей противоположностью; для него,
взятого в этом более точном определении как нечто рефлектированное,
подходит название \emph{момента}. \emph{Вес} и \emph{расстояние}
от некоторой точки называются в рычаге его механическими
\emph{моментами} из-за \emph{тождественности} оказываемого
ими действия при всем прочем различии менаду
такой реальностью, как вес, и такой идеальностью, как чисто
пространственное определение, линия (см. <<Энциклопедию
философских наук>>, изд. 3-е, \S\,261, примечание).~--
Еще чаще придется обращать внимание на то, что в
философской терминологии рефлектированные определения
обозначены латинскими терминами
\endnote{Гегель имеет в виду прежде всего латинское слово momentum
  (момент), которое он употребляет в значении, близком к Aufgehobenes (снятое).}
либо потому,
что в родном языке для них нет терминов, либо же, если,
как в данном случае, в нем имеются такие термины, потому,
что термин, которым располагает родной язык,
больше напоминает о непосредственном, а иностранный
термин~-- больше о рефлектированном.

Более точный смысл и выражение, которые бытие и
ничто получают, поскольку они стали теперь \emph{моментами},
должны выявиться при рассмотрении наличного бытия,
как единства, в котором они сохранены. Бытие есть бытие
и ничто есть ничто лишь в их отличии друг от друга;
но в их истине, в их единстве, они исчезли как эти
определения и суть теперь иное. Бытие и ничто суть одно
и то же; \emph{именно потому, что они одно и то же, они
уже не бытие и ничто} и имеют различное определение:
в становлении они были возникновением и прехождением;
в наличном бытии как по-иному определенном единстве
они опять-таки по-иному определенные моменты.
Это единство остается отныне их основой, которую они
уже больше не покинут, чтобы не возвращаться к абстрактному
значению бытия и ничто.


%%% Local Variables:
%%% mode: latex
%%% TeX-master: "../../../main"
%%% End:
