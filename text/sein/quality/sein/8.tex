Из предшествующего ясно видно, как обстоит дело с
диалектикой, отрицающей \emph{начало мира} и [возможность]
его гибели, с диалектикой, которая должна была доказать
\emph{вечность} материи, т.\,е. с диалектикой, отрицающей \emph{становление},
возникновение или прехождение вообще.~--
Кантовскую антиномию конечности или бесконечности
мира в пространстве и времени мы подробнее рассмотрим
ниже, когда будем рассматривать понятие количественной
бесконечности.~-- Указанная простая, тривиальная диалектика
основана на отстаивании противоположности
между бытием и ничто. Невозможность начала мира или
чего бы то ни было доказывается следующим образом.

Нет ничего такого, что могло бы иметь начало, ни поскольку
нечто есть, ни поскольку его нет; в самом деле,
поскольку оно есть, оно уже не начинается, а поскольку
его нет, оно также не начинается.~-- Если бы мир или нечто
имели начало, то он имел бы начало в ничто, но в
ничто нет начала или, иначе говоря, ничто не есть начало,
ведь начало заключает в себе некое бытие, а ничто не
содержит никакого бытия. Ничто~-- это только ничто.
В основании, причине и т.\,д.~-- если так определяют ничто,~--
содержится некое утверждение, бытие. На этом же
основании нечто не может и прекратиться. Ибо в таком
случае бытие должно было бы содержать ничто, но бытие~--
это только бытие, а не своя противоположность.

Ясно, что против становления или начала и прекращения,
против этого \emph{единства} бытия и ничто здесь не
приводится никакого доказательства, а его лишь ассерторически
отрицают и приписывают истинность бытию и
ничто в их отдельности друг от друга.~-- Однако эта диалектика
по крайней мере последовательнее рефлектирующего
представления. Последнее считает полной истиной,
что бытие и ничто существуют лишь раздельно, а, с
другой стороны, признает начало и прекращение столь
же истинными определениями; но, признавая их, оно
фактически принимает нераздельность бытия и ничто.

Разумеется, при предположении абсолютной раздельности
бытия и ничто начало или становление есть~--
это можно весьма часто слышать~-- нечто \emph{непостижимое}.
Ведь те, кто делает это предположение, упраздняют начало
или становление, которое, однако, они \emph{снова} допускают,
и это противоречие, которое они сами же создают
и разрешение которого они делают невозможным, они называют
\emph{непостижимостью}.

Изложенное выше и есть та же диалектика, какой
пользуется рассудок против даваемого высшим анализом
понятия \emph{бесконечно малых величин}. Это понятие будет
подробнее рассмотрено ниже.~-- Величины эти определены
как величины, \emph{существующие в своем исчезновении}~--
не \emph{до} своего исчезновения, ибо в таком случае они конечные
величины, и не \emph{после} своего исчезновения, ибо в таком
случае они ничто. Против этого чистого понятия
было выдвинуто постоянно повторявшееся возражение,
что такие величины суть \emph{либо} нечто, \emph{либо} ничто и что
нет \emph{промежуточного состояния} (<<состояние>> здесь неподходящее,
варварское выражение) между бытием и небытием.~--
При этом опять-таки признают абсолютную раздельность
бытия и ничто. Но против этого было показано,
что бытие и ничто суть на самом деле одно и то же или,
говоря языком выдвигающих это возражение, \emph{нет} ничего
такого, что не было бы \emph{промежуточным состоянием между
бытием и ничто}. Математика обязана своими самыми блестящими
успехами тому, что она приняла то определение,
которого не признает рассудок.

Приведенное рассуждение, делающее ложное предположение
об абсолютной раздельности бытия и небытия и
не идущее дальше этого предположения, следует называть
не \emph{диалектикой}, а \emph{софистикой}. В самом деле, софистика
есть резонерство, исходящее из необоснованного предположения,
истинность которого признается без критики и
необдуманно. Диалектикой же мы называем высшее разумное
движение, в котором такие кажущиеся безусловно
раздельными [моменты] переходят друг в друга благодаря
самим себе, благодаря тому, чт\'о они суть, и предположение
[об их раздельности] снимается. Диалектическая, имманентная
природа самого бытия и ничто в том и состоит,
что они свое единство~-- становление~-- обнаруживают
как свою истину.


%%% Local Variables:
%%% mode: latex
%%% TeX-master: "../../../main"
%%% End:
