Следует еще указать и на другую причину, усиливающую
неприязнь к положению о бытии и ничто. Эта причина~"--- то,
что вывод, вытекающий из рассмотрения бытия
и ничто, несовершенно выражен в положении: \emph{бытие
и ничто~"--- одно и то же}. Ударение падает преимущественно
на <<одно и то же>>, как и вообще в суждении, поскольку
в нем лишь предикат высказывает, чт\'о \emph{представляет
собой} субъект [суждения]. Поэтому кажется, будто смысл
[вывода]~"--- в отрицании различия, которое, однако, в то
же время непосредственно имеется в положении, ибо оно
высказывает \emph{оба} определения, бытие и ничто, и содержит
их как различные.~"--- И не в том смысл этого положения,
что следует от них абстрагироваться и удерживать лишь
единство. Подобный смысл сам обнаруживал бы свою односторонность,
так как то, от чего якобы д\'олжно отвлекаться,
все же имеется и названо в положении.~"--- Итак,
поскольку положение: \emph{бытие и ничто~"--- одно и то же}, высказывает
тождество этих определений, но на самом деле
также содержит эти два определения как различные, постольку
оно противоречиво в самом себе и разлагает себя.
Если выразиться более точно, то здесь дано положение,
которое, как обнаруживается при более тщательном
рассмотрении, устремлено к тому, чтобы заставить само
себя исчезнуть. Но тем самым в нем самом совершается
то, чт\'о должно составить его настоящее содержание, а
именно \emph{становление}.

Рассматриваемое нами положение, таким образом, \emph{содержит}
вывод, оно \emph{в самом себе} есть этот вывод. Но здесь
мы должны обратить внимание на следующий недостаток:
сам вывод не \emph{выражен} в положении; только внешняя рефлексия
познает его в нем.~"--- По этому поводу следует
уже в самом начале сделать общее замечание, что положение
в \emph{форме суждения} не пригодно для выражения
спекулятивных истин. Знакомство с этим обстоятельством
могло бы устранить многие недоразумения относительно
спекулятивных истин. Суждение есть отношение \emph{тождества}
между субъектом и предикатом, при этом абстрагируются
от того, что у субъекта еще многие [другие] определенности,
чем те, которыми обладает предикат, и от
того, что предикат шире субъекта. Но если содержание спекулятивно,
то и \emph{нетождественное} в субъекте и предикате
составляет существенный момент, однако в суждении это
не выражено. Парадоксальный и странный свет, в котором
не освоившимся со спекулятивным мышлением представляются
многие положения новейшей философии, часто
зависит от формы простого суждения, когда она применяется
для выражения спекулятивных выводов.

Чтобы выразить спекулятивную истину, указанный недостаток
устраняют прежде всего тем, что к положению
прибавляют противоположное положение: \emph{бытие и ничто
не одно и то же}, каковое положение также было высказано
выше. Но тогда возникает еще другой недостаток, а
именно: эти положения не связаны между собой и, стало
быть, излагают содержание лишь в антиномии, между
тем как их содержание касается одного и того же, и определения,
выраженные в этих двух положениях, должны
быть безусловно соединены,~"--- получится соединение, которое
может быть высказано лишь как некое \emph{беспокойство
несовместимых} друг с другом [определений], как \emph{некое
движение}. Самая обычная несправедливость, совершаемая
по отношению к спекулятивному содержанию, заключается
в том, что его делают односторонним, т.\,е. выпячивают
лишь одно из положений, на которые оно может
быть разложено. Нельзя в таком случае отрицать, что это
положение [действительно] утверждается; \emph{но насколько
правильно то, чт\'о в нем указывается, настолько же оно
и ложно}, ибо раз из области спекулятивного берут одно
положение, то следовало бы по меньшей мере точно так
же обратить внимание и на другое положение и указать
его.~"--- При этом нужно еще особо отметить, так сказать,
злополучное слово <<единство>>. <<Единство>> еще в большей
мере, чем <<тождество>>, обозначает субъективную
рефлексию. Оно берется главным образом как соотношение,
получающееся из \emph{сравнивания}, из внешней рефлексии."
Поскольку последняя находит в двух \emph{разных предметах}
одно и то же, единство имеется таким образом, что
при этом предполагается полное \emph{безразличие} самих сравниваемых
предметов к этому единству, так что это сравнивание
и единство вовсе не касаются самих предметов
и суть некое внешнее для них действование и определение.
<<Единство>> выражает поэтому совершенно \emph{абстрактное}
<<одно и то же>> и звучит тем резче и более странно,
чем больше те предметы, о которых оно высказывается,
являют себя просто различными. Постольку было бы поэтому
лучше вместо <<единства>> говорить лишь <<\emph{нераздельность}>>
и <<\emph{неразделимостъ}>>; но эти слова не выражают
того, чт\'о есть \emph{утвердительного} в соотношении целого.

Таким образом, полный, истинный результат, выявившийся
здесь, это~"--- \emph{становление}, которое не есть ЛИШЬ
одностороннее или абстрактное единство бытия и ничто.
Становление состоит в следующем движении: чистое бытие
непосредственно и просто; оно поэтому в такой же
мере есть чистое ничто; различие между ними \emph{есть}, но
в такой же мере \emph{снимает себя} и \emph{не есть}. Результат, следовательно,
утверждает также и различие между бытием и
ничто, но как такое различие, которое только \emph{предполагается}
(gemeinten).

\emph{Предполагают}, что бытие есть скорее всецело иное,
чем ничто, и ничего нет яснее того, что они абсолютно
различны, и, кажется, ничего нет легче, чем указать их
различие. Но столь же легко убедиться в том, что это невозможно,
что это различие \emph{невыразимо}. Пусть \emph{те, кто
настаивает на различии между бытием и ничто}, возьмут
на себя труд \emph{указать, в чем оно состоит}. Если бы бытие
и ничто различала какая-нибудь определенность, то они,
как мы уже говорили, были бы определенным бытием и
определенным ничто, а не чистым бытием и чистым ничто,
каковы они еще здесь. Поэтому различие между ними
совершенно пусто, каждое из них в равной мере есть
неопределенное. Это различие имеется поэтому не в них
самих, а лишь в чем-то третьем, в \emph{предполагании} (Meinen).
Однако предполагание есть форма субъективного,
которое не имеет касательства к этому изложению. Но
третье, в котором имеют свое существование бытие и ничто,
должно иметь место и здесь; и оно, действительно,
имело здесь место; это~"--- \emph{становление}. В нем они имеются
как различные; становление имеется лишь постольку, поскольку
они различны. Это третье есть нечто иное, чем
они. Они существуют лишь в ином. Это также означает,
что они не существуют особо (für sich). Становление есть
существование (Bestehen) бытия в той же мере, что и существование
небытия, иначе говоря, их существование
есть лишь их бытие в \emph{одном}; именно это их существование
и есть то, чт\'о также снимает их различие.

Требование указать различие между бытием и ничто
заключает в себе и требование сказать, чт\'о же такое
\emph{бытие} и чт\'о такое \emph{ничто}. Пусть те, кто отказывается признать,
что и бытие, и ничто есть лишь \emph{переход} одного в
другое, и утверждает о бытии и ничто то и се,~"--- пусть
они укажут, о \emph{чем} они говорят, т.\,е. пусть дадут \emph{дефиницию}
бытия и ничто и пусть докажут, что она правильна.
Без удовлетворения этого первого требования старой науки,
логические правила которой они в других случаях
признают и применяют, все их утверждения о бытии и
ничто не более как заверения, лишенные научной значимости.
Если, например, раньше говорили, что существование,
поскольку прежде всего его считают равнозначным
бытию, есть \emph{дополнение} к \emph{возможности}, то этим предполагается
другое определение~"--- возможность, и бытие выражено
не в своей непосредственности и даже не как нечто
самостоятельное, а как обусловленное. Для обозначения
\emph{опосредствованного} бытия мы сохраним выражение
\emph{существование}. Правда, люди представляют себе бытие,~"---
скажем, прибегая к образу чистого света, как ясность непомутненного
в\'идения, а ничто~"--- как чистую ночь, и связывают
их различие с этой хорошо знакомой чувственной
разницей. Однако на самом деле, если точнее представить
себе и это в\'идение, то легко заметить, что в абсолютной
ясности мы столь же много и столь же мало видим, как
и в абсолютной тьме, что и то и другое в\'идение есть чистое
в\'идение, т.\,е. ничегоневидение. Чистый свет и чистая
тьма~"--- это две пустоты, которые суть одно и то же. Лишь
в определенном свете~"--- а свет определяется тьмой,~"--- следовательно,
в помутненном свете, и точно так же лишь
в определенной тьме~"--- а тьма определяется светом,~"--- в
освещенной тьме можно что-то различать, так как лишь
помутненный свет и освещенная тьма имеют различие в
самих себе и, следовательно, суть определенное бытие,
\emph{наличное бытие}.


%%% Local Variables:
%%% mode: latex
%%% TeX-master: "../../../main"
%%% End:
