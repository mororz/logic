Бытие, \emph{во-первых}, определено вообще по отношению
к иному.

Оно, \emph{во-вторых}, определяет себя внутри самого себя.

\emph{В-третьих}, если отбросить это предварительное деление,
бытие есть та абстрактная неопределенность и непосредственность,
в которой оно должно служить началом.


Согласно \emph{первому} определению бытие отделяет себя
от \emph{сущности}, показывая в дальнейшем своем развитии
свою целокупность лишь как одну сферу понятия и
противопоставляя ей как момент некоторую другую
сферу.

Согласно \emph{второму} определению оно есть сфера, в которую
входят определения и все движение его рефлексии.
В ней бытие полагает себя в трех следующих определениях:
\begin{enumerate}[label=\Roman*.]
\item как \emph{определенность}, как таковая: \emph{качество};
\item как \emph{снятая} определенность: \emph{величина}, \emph{количество};
\item как \emph{качественно} определенное \emph{количество}: \emph{мера}.
\end{enumerate}

Это деление, как сказано во введении относительно
всех этих делений вообще, есть только предварительное
перечисление. Его определения должны еще возникнуть
из движения самого бытия, дать себе через это движение
дефиницию и обоснование. Об отклонении этого деления
от обычного перечня категорий, а именно как количества,
качества, отношения и модальности, которые, впрочем,
у Канта, надо полагать, служили только заглавиями
для его категорий, а на самом деле сами суть категории,
только более всеобщие,~"--- об этом отклонении здесь не стоит
говорить, так как все изложение покажет, каковы вообще
наши отклонения от обычного порядка и значения категорий.


Здесь можно отметить лишь следующее: определение
\emph{количества} обычно приводят раньше определения \emph{качества},
и притом это делается, как в большинстве случаев,
без какого-либо обоснования. Мы уже показали, что началом
служит бытие, \emph{как таковое}, значит, качественное
бытие. Из сравнения качества с количеством легко увидеть,
что по своей природе качество есть первое. Ибо
количество есть качество, ставшее уже отрицательным;
\emph{величина} есть определенность, которая больше не едина
с бытием, а уже отлична от него, она снятое, ставшее
безразличным качество. Она включает в себя изменчивость
бытия, не изменяя самой вещи, бытия, определением
которого она служит; качественная же определенность
едина со своим бытием, она не выходит за его
пределы и не находится внутри его, а есть его \emph{непосредственная}
ограниченность. Поэтому качество как непосредственная
определенность есть первая определенность,
и с него следует начинать.

\emph{Мера} есть \emph{отношение}, но не отношение вообще,
а определенна отношение качества и количества друг
к другу; категории, которые Кант объединяет под названием
<<отношение>>, займут свое место совсем в другом
разделе. Меру можно, если угодно, рассматривать и как
некоторую модальность. Но так как у Канта модальность
уже не есть определение содержания, а касается лишь
отношения содержания к мышлению, к субъективному,
то это~"--- совершенно чужеродное, сюда не принадлежащее
отношение.

\emph{Третье} определение \emph{бытия} входит в раздел о качестве,
ибо бытие как абстрактная непосредственность низводит
себя до единичной определенности, противостоящей
внутри его сферы другим его определенностям.




%%% Local Variables:
%%% mode: latex
%%% TeX-master: "../../main"
%%% TeX-engine: xetex
%%% End:
