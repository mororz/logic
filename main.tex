% \documentclass[a4paper, 12pt]{article}

\documentclass[12pt]{memoir}

%%%%%%%%%%%%%%%%%%%%%%%%%%%%%%%%%%%%%%%% 
%%%%%%%%%%%%%%%%%%%%%%%%%%%%%%%%%%%%%%%%
%%%%%%%%%% General packages %%%%%%%%%%%%
%%%%%%%%%%%%%%%%%%%%%%%%%%%%%%%%%%%%%%%%
%%%%%%%%%%%%%%%%%%%%%%%%%%%%%%%%%%%%%%%%

\usepackage{hyperref}
\hypersetup{
  pdfborder=false
}

%%%%%%%%%%%%%%%%%%%%%%%%%%%%%%%%%%%%%%%% 
%%%%%%%%%%%%%%%%%%%%%%%%%%%%%%%%%%%%%%%%
%%%%%%%%%%%%%%% Language %%%%%%%%%%%%%%%
%%%%%%%%%%%%%%%%%%%%%%%%%%%%%%%%%%%%%%%%
%%%%%%%%%%%%%%%%%%%%%%%%%%%%%%%%%%%%%%%%

\usepackage[babelshorthands]{polyglossia}
\setmainlanguage[indentfirst=true]{russian}
\setotherlanguages{german, greek, latin}
\setlanghyphenmins{russian}{2}{3}

%%%%%%%%%%%%%%%%%%%%%%%%%%%%%%%%%%%%%%%%
%%%%%%%%%%%%%%%%%%%%%%%%%%%%%%%%%%%%%%%%
%%%%%%%%%%%%%%% Geometry %%%%%%%%%%%%%%%
%%%%%%%%%%%%%%%%%%%%%%%%%%%%%%%%%%%%%%%%
%%%%%%%%%%%%%%%%%%%%%%%%%%%%%%%%%%%%%%%%

\setstocksize{297mm}{210mm}
\settrimmedsize{\stockheight}{\stockwidth}{*}
\settypeblocksize{52pc}{28pc}{*}
\setlrmargins{11pc}{*}{*}
\setulmargins{7pc}{*}{*}
\checkandfixthelayout

\makepagestyle{mystyle}
\makeevenhead{mystyle}{}{}{}
\makeoddhead{mystyle}{}{}{}
\makeevenfoot{mystyle}{\thepage}{}{}
\makeoddfoot{mystyle}{}{}{\thepage}

\pagestyle{mystyle}

%%%%%%%%%%%%%%%%%%%%%%%%%%%%%%%%%%%%%%%%
%%%%%%%%%%%%%%%%%%%%%%%%%%%%%%%%%%%%%%%%
%%%%%%%%%%%%%%%%% Font %%%%%%%%%%%%%%%%%
%%%%%%%%%%%%%%%%%%%%%%%%%%%%%%%%%%%%%%%%
%%%%%%%%%%%%%%%%%%%%%%%%%%%%%%%%%%%%%%%%

\usepackage{fontspec}

\defaultfontfeatures{
	Path = fonts/EB_Garamond/static/,
	Extension = .ttf,
	Numbers = Lowercase
}

\setmainfont{EBGaramond-Regular}[
	ItalicFont = EBGaramond-Italic,
	BoldFont = EBGaramond-Bold,
	BoldItalicFont = EBGaramond-BoldItalic
]

%%%%%%%%%%%%%%%%%%%%%%%%%%%%%%%%%%%%%%%%
%%%%%%%%%%%%%%%%%%%%%%%%%%%%%%%%%%%%%%%%
%%%%%%%%%%%%%% Typography %%%%%%%%%%%%%%
%%%%%%%%%%%%%%%%%%%%%%%%%%%%%%%%%%%%%%%%
%%%%%%%%%%%%%%%%%%%%%%%%%%%%%%%%%%%%%%%%

\parskip = 0 pt % do not allow empty space between paragraphs

% use asteriks for autor's notes
\usepackage[symbol]{footmisc}
\renewcommand{\thefootnote}{\fnsymbol{footnote}}


\usepackage{pifont}

\usepackage{microtype}
%\tolerance = 5000
\midsloppy

\usepackage[all]{nowidow}
\setnoclub
\setnowidow

\usepackage{enotez}
\setenotez{
	list-name=Примечания,
	backref=true,
	totoc=chapter
}

\DeclareInstance{enotez-list}{custom}{paragraph}{
	format = \normalfont
}

%%%%%%%%%%%%%%%%%%%%%%%%%%%%%%%%%%%%%%%%
%%%%%%%%%%%%%%%%%%%%%%%%%%%%%%%%%%%%%%%%
%%%%%%%%%%%%%%%% Custom %%%%%%%%%%%%%%%%
%%%%%%%%%%%%%%%%%%%%%%%%%%%%%%%%%%%%%%%%
%%%%%%%%%%%%%%%%%%%%%%%%%%%%%%%%%%%%%%%%

\usepackage{enumitem} % for lists

\newcommand{\signature}[1]{
  \begin{flushright}
    \emph{#1}
  \end{flushright}
}

\setsecnumdepth{none} % don't print chapter numbers

\makechapterstyle{mychapterstyle}{
	\chapterstyle{default}
	\setlength\beforechapskip{14pc}
	\setlength\afterchapskip{1pc}
	\renewcommand{\chaptitlefont}{
		\bfseries\centering\Large\MakeTextUppercase
	}
}
	
\chapterstyle{mychapterstyle}
\aliaspagestyle{chapter}{mystyle}



%%% Local Variables:
%%% mode: latex
%%% TeX-master: "main"
%%% TeX-engine: xetex
%%% End:


\begin{document}

\chapter{Предисловие к первому изданию}

Полное изменение, которое претерпел у нас за последние
лет двадцать пять характер философского мышления,
более высокая точка зрения на само себя, которой
в этот период достигло самосознание духа, до сих пор
еще оказали мало влияния на облик \emph{логики}.

То, чт\'о до этого времени называлось метафизикой,
подверглось, так сказать, радикальному искоренению и
исчезло из области наук. Где теперь мы услышим или
где теперь смеют еще раздаваться голоса прежней онтологии,
рациональной психологии, космологии или даже
прежней естественной теологии? Где теперь будут интересоваться
такого рода исследованиями, как, например,
об имматериальности души, о механических и конечных
причинах? Да и прежние доказательства бытия бога излагаются
лишь исторически или в целях назидания и ради
возвышения духа. Это факт, что интерес отчасти к содержанию,
отчасти к форме прежней метафизики, а отчасти
к обоим вместе утрачен. Насколько удивительно, когда
для народа стали непригодными, например, наука о
его государственном праве, его убеждения, его нравственные
привычки и добродетели, настолько же удивительно
по меньшей мере, когда народ утрачивает свою метафизику,
когда дух, занимающийся своей чистой сущностью,
уже не имеет в нем действительного существования.

Экзотерическое учение кантовской философии, гласящее,
что \emph{рассудок не вправе перешагивать область
опыта} и что иначе познавательная способность становится
\emph{теоретическим разумом}, который сам по себе порождает
только \emph{химеры},~-- это учение оправдывало с научной
стороны отказ от спекулятивного мышления. Содействовали
этому популярному учению и вопли новейшей
педагогики (требование времени, направляющее взор
людей на непосредственные нужды) о том, что, подобно
тому как главное для познания~-- опыт, так и для преуспеяния
в общественной и частной жизни теоретическое
понимание даже вредно, а существенно, единственно
полезно~-- упражнение и вообще практическое образование.~--
Таким образом, поскольку наука и здравый человеческий
смысл способствовали крушению метафизики,
казалось, что в результате их общих усилий возникло
странное зрелище~-- \emph{образованный народ без метафизики},
нечто вроде храма, в общем-то разнообразно украшенного,
но без святыни. Теология, которая в прежние времена
была хранительницей спекулятивных таинств и
(правда, зависимой) метафизики, отказалась от этой науки,
заменив ее чувствованиями, практически общедоступными
поучениями и учено-историческими сведениями.
Этой перемене соответствует то обстоятельство, что, с
другой стороны, исчезли те \emph{одинокие}, которые приносились
в жертву своим народом и удалялись из мира, дабы
существовали созерцание вечного и жизнь, посвященная
единственно лишь этому созерцанию не ради какой-то
выгоды, а ради благодати. Это~-- исчезновение, которое
в другой связи можно рассматривать как явление, по
своему существу тождественное с вышеупомянутым. Казалось,
таким образом, что, после того как был рассеян
этот мрак, это бесцветное занятие самим собой ушедшего
в себя духа, существование превратилось в светлый, радостный
мир цветов, среди которых, как известно, нет
\emph{черных}.

\emph{Логика} испытала не столь печальную участь, как метафизика.
Предрассудок, будто логика \emph{научает мыслить},~--
в этом раньше видели ее пользу и, стало быть,
ее цель (это похоже на то, как если бы сказали, что
только благодаря изучению анатомии и физиологии мы
научаемся переваривать пищу и двигаться),~-- этот предрассудок
давно уже исчез, и дух практичности уготовлял
ей, по-видимому, не лучшую участь, чем ее сестре. Тем
не менее, вероятно ввиду приносимой ею некоторой формальной
пользы, ей было еще оставлено место среди наук,
и ее даже сохранили в качестве предмета публичного
преподавания. Но этот лучший удел касается только ее
внешней участи, ибо ее форма и содержание остались
такими же, какими они по давней традиции передавались
от поколения к поколению, причем, однако, при
этой передаче ее содержание делалось все более и более
тощим и скудным; в ней еще не чувствуется тот новый
дух, который выявился в науке не менее, чем в действительности.
Но совершенно тщетно желание сохранить
формы прежнего образования, когда изменилась субстанциальная
форма духа. Они представляют собой увядшие
листья, спадающие под напором новых почек, образовавшихся
у их основания.

\emph{Игнорирование} этой общей перемены начинает постепенно
исчезать также и в научной области. Незаметно
эти новые представления стали привычными даже
противникам, они усвоили их, и если они все еще высказывают
пренебрежение к источнику этих представлений
и лежащим в их основе принципам и оспаривают
их, то зато им приходится мириться с выводами и они
оказываются не в силах противиться влиянию последних.
Помимо того что все больше и больше слабеет их
отрицательное отношение [к указанным представлениям],
эти противники не знают иного способа придать своим работам
положительное значение, кроме как вместе с другими
начинать говорить языком новых представлений.

С другой стороны, уже прошло, по-видимому, время
брожения, с которого начинается всякое новое творчество.
Первоначально это творчество относится с фантастической
враждебностью к существующей обширной систематизации
прежнего принципа; отчасти оно опасается
также, что потеряется в пространных частностях, отчасти
же страшится труда, требуемого для научной разработки,
и, чувствуя потребность в такой разработке, хватается
сначала за пустой формализм. Ввиду этого требование,
чтобы содержание подверглось обработке и было
развито, становится еще более настоятельным. В формировании
той или иной эпохи, как и в формировании отдельного
человека, бывает период, когда речь идет главным
образом о приобретении и утверждении принципа в его
неразвитой еще напряженности. Однако более высокое
требование состоит в том, чтобы этот принцип стал
наукой.

Но, что бы ни было уже сделано в других отношениях
для сути и формы науки, логическая наука, составляющая
собственно метафизику или чистую, спекулятивную
философию, до сих пор находилась еще в большем
пренебрежении. Чт\'о я разумею более конкретно под этой
наукой и ее точкой зрения, я указал предварительно во
\emph{введении}. Необходимость вновь начать в этой науке с
самого начала, природа самого предмета и отсутствие таких
подготовительных работ, которые можно было бы
использовать для предпринятого [нами] преобразования,~--
пусть все эти обстоятельства будут приняты во внимание
справедливыми критиками, если окажется, что и многолетний
труд [автора] не смог сообщить этой попытке
большее совершенство. Важно иметь в виду, что дело
идет о том, чтобы дать новое понятие научного рассмотрения.
Философия, поскольку она должна быть наукой,
не может, как я указал в другом
месте\footnotemark{},
для этой цели
заимствовать свой метод у такой подчиненной науки, как
математика, и точно так же она не может довольствоваться
категорическими заверениями внутреннего созерцания
или пользоваться рассуждениями, основывающимися
на внешней рефлексии. Только \emph{природа содержания}
может быть тем, что \emph{развертывается} в научном познании,
причем именно лишь эта \emph{собственная рефлексия} содержания
полагает и \emph{порождает} само \emph{определение содержания}\endnotemark{}.

\endnotetext{
См. <<Феноменология духа>>: <<Наука должна организоваться
только собственной жизнью понятия\dots Содержание показывает,
что его определенность не принята от другого и не пристегнута
[к нему], но оно само сообщает ее себе и, исходя из себя, определяет
себя в качестве момента и устанавливает себе место внутри
целого>> (\emph{Гегель}. Соч., т.\,IV, М.,~1959,~стр.\,28. Все дальнейшие
цитаты из <<Феноменологии духа>> даны по этому изданию).
}

\footnotetext{
<<Феноменология духа>>. Предисловие к первому изданию.
Подлинное развитие сказанного~-- познание метода, место которого в
самой логике\endnotemark{}
}

\endnotetext{
Это примечание прибавлено Гегелем в 1831\,г. при подготовке
второго издания <<Науки логики>>.
}

\emph{Рассудок дает определения} и твердо держится их;
\emph{разум} же отрицателен и \emph{диалектичен}, ибо он обращает
определения рассудка в ничто; он положителен, ибо порождает
\emph{всеобщее} и постигает в нем особенное. Подобно
тому как рассудок обычно понимается как нечто обособленное
от разума вообще, так и диалектический разум
обычно принимается за нечто обособленное от положительного
разума. Но в своей истине разум есть \emph{дух}, который
выше их обоих; он рассудочный разум или разумный
рассудок. Он есть отрицательное (das Negative), то,
чт\'о составляет качество и диалектического разума, и рассудка.
Этот дух отрицает простое (das Einfache) и тем
самым полагает определенное различие, которым занимается
рассудок; он также разлагает это различие, тем
самым он диалектичен. Однако он не задерживается на
этом нулевом результате, а выступает в нем и как нечто
положительное, и, таким образом, восстанавливает первоначальное
простое, но как всеобщее, которое конкретно
внутри себя. Под конкретно всеобщее не подводится то
или другое данное особенное, а в указанном процессе
определения и в разлагании его уже определилось вместе
с тем и особенное. Это духовное движение, дающее
себе в своей простоте свою определенность, а в ней~-- и
равенство с самим собой, это движение, представляющее
собой, стало быть, имманентное развитие понятия, есть
абсолютный метод познания и вместе с тем имманентная
душа самого содержания.~-- Я утверждаю, что философия
способна быть объективной, доказательной наукой лишь
на этом конструирующем себя пути.~-- Таким способом
я попытался в <<Феноменологии духа>> изобразить \emph{сознание}.
Сознание есть дух, как конкретное знание, и притом
погрязшее во внешнем. Но движение форм этого
предмета, подобно развитию всякой природной и духовной
жизни, покоится только на природе \emph{чистых сущностей},
составляющих содержание логики. Сознание как
дух, который охватывает лишь явления и который освобождается
на своем пути от своей непосредственности и
сращенности с внешним, становится чистым знанием,
дающим себе в качестве предмета указанные чистые
сущности, как они суть сами по себе. Они чистые мысли,
мыслящий свою сущность дух. Их самодвижение есть
их духовная жизнь и представляет собой то, чт\'о конституирует
науку и изображением чего она является.

Этим указано [внутреннее] отношение науки, которую
я называю \emph{феноменологией духа}, к логике. Что же касается
внешнего отношения между ними, то я полагал,
что за первой частью <<Системы науки>>\footnotemark{}, содержащей
феноменологию, последует вторая часть, которая должна
была содержать логику и обе реальные дисциплины
философии~-- философию природы и философию духа,~--
так что этой частью заканчивалась бы система науки.
Но необходимость расширить объем логики, взятой сама
по себе, побудила меня выпустить ее в свет отдельно;
она, таким образом, составляет, согласно этому расширенному
плану, первое продолжение <<Феноменологии
духа>>. Позднее я разработаю обе названные выше реальные
философские науки. Этот первый том <<Логики>> содержит
первую книгу~-- \emph{учение о бытии}, вторую книгу~--
\emph{учение о сущности}, как второй раздел первого тома; второй
же том будет содержать \emph{субъективную логику}, или
\emph{учение о понятии}.

\footnotetext{
Бамберг и Вюрцбург в издательстве Гёбгарда,~1807. Во втором
издании, которое появится в свет в ближайшую пасху, это
название будет исключено. Вместо указываемой далее предполагавшейся
второй части, которая должна была содержать все другие
философские науки, я выпустил после этого в свет <<Энциклопедию
философских наук>>, вышедшую в прошлом году третьим
изданием.\endnotemark{}
}

\endnotetext{
Это примечание прибавлено Гегелем в 1831\,г. при подготовке
второго издания <<Науки логики>>. Упоминаемое здесь второе
издание <<Феноменологии духа>> он подготовить не успел, исправив
лишь 36 страниц <<Предисловия>>. С этими исправлениями <<Феноменология>>
была издана уже после его смерти, в 1832 г. <<Энциклопедия
философских наук>> впервые была издана в 1817\,г.,
третье издание~-- в 1830\,г.
}

\signature{Нюрнберг,~22~марта~1812\,г.}

%%% Local Variables:
%%% mode: latex
%%% TeX-master: "../main"
%%% End:


\chapter{Предисловие ко второму изданию}

К этой новой редакции <<Науки логики>>, первый том
которой теперь выходит в свет, я, должен сказать, приступил
с полным сознанием как трудности предмета
самого по себе, а затем и его изложения, так и несовершенства
его редакции в первом издании. Сколько я ни старался
после дальнейших многолетних занятий этой наукой
устранить это несовершенство, я все же чувствую,
что у меня достаточно причин просить читателя быть ко
мне снисходительным. Право же на такое снисхождение
дает мне прежде всего то обстоятельство, что для содержания
я нашел в прежней метафизике и прежней логике
преимущественно только внешний материал. Хотя эти
науки разрабатывались повсеместно и часто,~-- последняя
из указанных наук разрабатывается еще и поныне,~-- все
же эта разработка мало касалась спекулятивной стороны;
в целом скорее повторялся тот же самый материал, который
попеременно то разжижался до тривиальной поверхностности,
то расширялся благодаря тому, что снова вытаскивался
старый балласт, так что от таких, часто лишь
совершенно механических, стараний философское содержание
ничего не могло выиграть. Изображение царства
мысли философски, т.\,е. в его собственной имманентной
деятельности, или, что то же самое, в его необходимом
развитии, должно было поэтому явиться новым предприятием,
и притом начинающим все с самого начала. Указанный
же приобретенный материал~-- известные уже
формы мысли~-- должен рассматриваться как в высшей
степени важный подсобный материал (Vorlage) и даже
как необходимое условие, как заслуживающая нашу признательность
предпосылка, хотя этот материал лишь кое-где
дает нам слабую нить или мертвые кости скелета, к
тому же еще перемешанные между собой в беспорядке.

Формы мысли выявляются и отлагаются прежде всего
в человеческом \emph{языке}. В наше время мы должны неустанно
напоминать, что человек отличается от животного
именно тем, что он мыслит. Во все, что для человека
становится чем-то внутренним, вообще представлением, во
все, что он делает своим, проник язык, а все то, что он
превращает в язык и выражает в языке, содержит, в скрытом
ли, спутанном или более разработанном виде, некоторую
категорию; в такой мере естественно для него логическое,
или, правильнее сказать, последнее есть сама
присущая ему \emph{природа}. Но если противопоставлять природу
вообще как физическое духовному, то следовало бы
сказать, что логическое есть, вернее, сверхприродное, проникающее
во все естественные отношения человека, в его
чувства, созерцания, вожделения, потребности, влечения
и тем только и превращающее их, хотя лишь формально,
в нечто человеческое, в представления и цели. Если язык
богат логическими выражениями, и притом специальными
и отвлеченными, для [обозначения] самих определений
мысли, то это его преимущество. Из предлогов и членов
речи многие уже выражают отношения, основывающиеся
на мышлении; китайский язык, говорят, в своем развитии
вовсе не достиг этого или достиг в незначительной степени.
Но эти грамматические частицы выполняют всецело
служебную роль, они только немногим более отделены от
соответствующих слов, чем глагольные приставки, знаки
склонения и т.\,д. Гораздо важнее, если в данном языке
определения мысли выражены в виде существительных и
глаголов и таким образом отчеканены так, что получают
предметную форму. Немецкий язык обладает в этом отношении
большими преимуществами перед другими современными
языками; к тому же многие из его слов имеют
еще ту особенность, что обладают не только различными,
но и противоположными значениями, так что нельзя не
усмотреть в этом спекулятивный дух этого языка: мышление
может только радовать, когда оно неожиданно
сталкивается с такого рода словами и обнаруживает, что
соединение противоположностей~-- результат спекуляции,
который для рассудка представляет собой бессмыслицу,~--
наивно выражено уже лексически в виде \emph{одного} слова,
имеющего противоположные значения. Поэтому философия
вообще не нуждается в особой терминологии; приходится,
правда, заимствовать некоторые слова из иностранных
языков; эти слова, однако, благодаря частому
употреблению уже получили в нашем языке право гражданства,
и аффектированный пуризм был бы менее всего
уместен здесь, где в особенности важна суть дела.~--
Успехи образования вообще, и в частности наук, даже
эмпирических наук и наук о чувственно воспринимаемом,
в общем двигаясь в рамках самых обычных категорий
(например, категорий целого и частей, вещи и ее свойств
и т.\,п.), постепенно выдвигают и более высокие отношения
мысли или по крайней мере поднимают их до большей
всеобщности и тем самым заставляют обращать на
них больше внимания. Если, например, в физике получило
преобладание такое определение мысли, как <<сила>>, то
в новейшее время самую значительную роль играет категория
\emph{полярности}\endnotemark{}, которую, впрочем, слишком à tort et à
travers [без разбора] втискивают во все, даже в учение
о свете; полярность есть определение такого различия, в
котором различаемые [моменты] \emph{неразрывно} связаны
друг с другом. То обстоятельство, что таким способом
отошли от формы абстракции, от тождества, которое
сообщает некоей определенности, например силе, самостоятельность,
и вместо этого была выделена и стала привычным
представлением другая форма определения~--
форма различия, которое в то же время сохраняется в
тождестве как нечто нераздельное,~-- это обстоятельство
бесконечно важно. Рассмотрение природы благодаря самой
реальности, в которой удерживаются ее предметы,
необходимо заставляет фиксировать категории, которые
уже нельзя более игнорировать в ней, хотя при этом имеет
место величайшая непоследовательность в отношении
других категорий, за которыми \emph{также} сохраняют их значимость,
и это рассмотрение не допускает того, чтобы,
как это легче происходит в сфере духовного, переходили
от противоположности к абстракциям и всеобщностям.

\endnotetext{
  Гегель имеет в виду <<закон полярности>>, сформулированный
  в натурфилософии Шеллинга, в частности в работе <<О мировой душе>> (1798).
}

Но хотя, таким образом, логические предметы, равно
как и их словесные выражения, суть по крайней мере
нечто всем известное в области образования, однако, как
я сказал в другом месте\endnotemark{}, то, что \emph{известно} (bekannt),
еще не есть поэтому \emph{познанное} (erkannt); между тем
требование продолжать заниматься тем, чт\'о уже известно,
может даже вывести из терпения,~-- а чт\'о более известно,
чем определения мысли, которыми мы пользуемся
повсюду, которые мы произносим в каждом предложении?
Это предисловие и имеет своей целью указать общие
моменты движения познания, исходящего из этого известного,
общие моменты отношения научного мышления
к этому естественному мышлению. Этого указания вместе
с тем, что содержится в прежнем \emph{введении}, достаточно
для того, чтобы дать общее представление о смысле логического
познания, то общее представление, которое с самого
начала желают получить о науке, еще до того представления,
которое составляет суть дела.

\endnotetext{
  В предисловии к <<Феноменологии духа>>: <<Известное
  вообще~-- оттого, что оно \emph{известно,еще не познано}>> (стр. 16).
}

Прежде всего следует рассматривать как бесконечный
прогресс то обстоятельство, что формы мышления были
высвобождены из того материала, в который они погружены
при сознающем себя созерцании, представлении,
равно как и в нашем вожделении и волении или, вернее,
также и в представляющем вожделении и волении (а ведь
нет человеческого вожделения или воления без представления),
что эти всеобщности были выделены в нечто самостоятельное
и, как мы это видим у Платона, а главным
образом у Аристотеля, были сделаны предметом самостоятельного
рассмотрения; этим , начинается их познание.
<<Лишь после того,~-- говорит Аристотель,~-- как было налицо
почти все необходимое и требующееся для жизненных
удобств и сношений, люди стали добиваться философского
познания>>\endnote{Вольный перевод из <<Метафизики>> Аристотеля А 2 982 b
  (M.~-- Л., 1934, стр.\,22).}. <<В Египте,~-- замечает он перед
тем,~-- математические науки рано развились, ибо там
жречество было рано поставлено в условия, дававшие
ему досуг>>\endnote{<<Метафизика>> А 1 981 b (стр.\,20).}.
Действительно, потребность заниматься чистыми
мыслями предполагает длинный путь, который
человеческий дух должен был пройти, она, можно сказать,
есть потребность уже удовлетворенной потребности
в необходимости отсутствия потребностей, которой человеческий
дух должен был достигнуть,~-- потребность абстрагироваться
от материала созерцания, воображения
и т.\,д., от конкретных интересов вожделения, влечения,
воли, в каковом материале закутаны определения мысли.
В тихой обители пришедшего к самому себе и лишь в себе
пребывающего мышления умолкают интересы, движущие
жизнью народов и индивидов. <<Во многих отношениях,~--
говорит Аристотель в той же связи,~-- человеческая природа
зависима, но эта наука, которой ищут не для какого-нибудь
употребления, есть единственная наука, свободная
сама по себе, и потому кажется, будто она не есть человеческое
достояние>>\endnotemark{}. Философия вообще в своих мыслях
еще имеет дело с конкретными предметами~-- богом, природой,
духом; логика же занимается этими предметами
всецело лишь в их полной абстрактности. Логика поэтому~--
обычно предмет изучения для юношества, каковое
еще не вступило в круг интересов повседневной жизни,
пользуется по отношению к этим интересам досугом и
лишь для своей субъективной цели должно заниматься
приобретением средств и возможностей для проявления
своей активности в сфере объектов указанных интересов,
причем и этими объектами оно должно заниматься теоретически.
К этим \emph{средствам}, в противоположность указанному
выше представлению Аристотеля, причисляют и науку
логики; занятия ею~-- это предварительная работа,
место для этой работы~-- школа, лишь после которой
должно следовать настоящее дело жизни и деятельность
для достижения действительных целей. В жизни уже
\emph{пользуются} категориями; они лишаются чести рассматриваться
особо и низводятся до \emph{служения} духовной выработке
живого содержания, созданию и сообщению друг
другу представлений, относящихся к этому содержанию.
С одной стороны, они ввиду своей всеобщности служат
\emph{сокращениями} (ведь какое бесконечное множество частностей
внешнего существования и деятельности объемлют
собой представления: битва, война, народ или море, животное
и т.\,д.; какое бесконечное множество представлений,
видов деятельности, состояний и т.\,д. должно быть
сведено к таким \emph{простым} представлениям, как бог или
любовь и т.\,д.). С другой стороны, они служат для более
точного определения и нахождения \emph{предметных отношений},
причем, однако, содержание и цель, правильность
и истинность вмешивающегося сюда мышления ставятся
в полную зависимость от наличествующего, и определениям
мысли самим по себе не приписывается никакой
определяющей содержание действенности. Такое применение
категорий, которое в прежнее время называлось
естественной логикой, носит бессознательный характер;
и если научная рефлексия отводит им в духе роль служебных
средств, то этим мышление превращается вообще
в нечто подчиненное другим духовным определениям.
О наших ощущениях, влечениях, интересах мы, правда,
не говорим, что они нам служат, мы считаем их самостоятельными
способностями и силами; так что мы сами
суть те, кто ощущает так-то, желает и хочет того-то,
полагает свой интерес в том-то. С другой стороны, можно
прийти к сознанию того, что мы скорее служим нашим
чувствам, влечениям, страстям, интересам и тем более
привычкам, чем обладаем ими; ввиду же нашего внутреннего
единства с ними нам еще менее может [прийти в голову],
что они нам служат средствами. Мы скоро обнаруживаем,
что такие определения души и духа суть \emph{особенные}
в противоположность \emph{всеобщности}, в качестве каковой
мы себя сознаем и в которой заложена наша свобода,
и начинаем думать, что мы скорее находимся в плену у
этих особенностей, что они приобретают власть над нами.
После этого мы тем менее можем считать, что формы
мысли, которые проходят через все наши представления,~--
будут ли последние чисто теоретическими или
содержащими материал, принадлежащий ощущениям,
влечениям, воле,~-- служат нам, что мы обладаем ими, а
не наоборот, они нами. Чт\'о остается на \emph{нашу} долю против
них, каким образом можем \emph{мы}, могу я возвышать
себя \emph{над} ними как нечто более всеобщее, когда они сами
суть всеобщее, как таковое? Когда мы предаемся какому-нибудь
ощущению, какой-нибудь цели, интересу и чувствуем
себя в них ограниченными, несвободными, то областью,
в которую мы в состоянии выбраться из них и тем
самым вновь прийти к свободе, является эта область достоверности
самого себя, область чистой абстракции,
мышления. Или, когда мы хотим говорить о \emph{вещах}, их
\emph{природу} или их \emph{сущность} мы равным образом называем
\emph{понятием}, которое существует только для мышления; о
понятиях же вещей мы имеем гораздо меньшее основание
сказать, что мы ими владеем или что определения
мысли, комплекс которых они составляют, служат нам;
напротив, наше мышление должно ограничивать себя сообразно
им и наш произвол или свобода не должны
переделывать их по-своему. Стало быть, поскольку субъективное
мышление есть наиболее характерная для нас
деятельность, а объективное понятие вещей составляет
самое суть (Sache), то мы не можем выходить за пределы
указанной деятельности, не можем стать выше ее, и столь
же мало мы можем выходить за пределы природы вещей
(Natur der Dinge). Последнее определение мы можем, однако,
оставить в стороне; оно совпадает с первым постольку,
поскольку оно есть некое отношение наших мыслей
к самой вещи (Sache), но только дало бы оно нам
нечто пустое, ибо мы этим признали бы вещь правилом
для наших понятий, а между тем вещь может быть для
нас не чем иным, как нашим понятием о ней. Если критическая
философия понимает отношение между этими
\emph{тремя} терминами так, что мы ставим \emph{мысли} между \emph{нами}
и \emph{вещами}, как средний термин, в том смысле, что этот
средний термин скорее отгораживает \emph{нас} от \emph{вещей},
вместо того чтобы соединять нас с ними, то этому взгляду
следует противопоставить простое замечание, что как раз
эти вещи, которые будто бы стоят на другом конце, по
ту сторону нас и по ту сторону соотносящихся с ними
мыслей, сами суть мысленные вещи и как совершенно
неопределенные они суть лишь \emph{одна} мысленная вещь
(так называемая вещь в себе), пустая абстракция.

\endnotetext{
  <<Метафизика>> А 2 982 b (стр.\,22). Гегель переставил предложения
  в этой цитате.
}

Все же сказанного нами будет вполне достаточно для
уяснения той точки зрения, согласно которой исчезает
отношение, выражающееся в том, что определения мысли
берутся только как нечто полезное и как средства. Более
важное значение имеет находящийся в связи с указанным
отношением взгляд, согласно которому их обычно понимают
как внешние формы.~-- Пронизывающая все наши
представления, цели, интересы и поступки деятельность
мышления происходит, как сказано, бессознательно (естественная
логика) ; то, чт\'о наше сознание имеет перед
собой, согласно этому взгляду,~-- это содержание, предметы
представлений, то, чем проникнут интерес; определения
же мысли суть \emph{формы}, которые только \emph{касаются
содержания}, а не составляют самого содержания. Но если
верно то, что мы указали выше и с чем в общем соглашаются,
а именно, если верно, что \emph{природа}, особая \emph{сущность},
истинно \emph{сохраняющееся} и \emph{субстанциальное} при
всем многообразии и случайности явлений и преходящем
проявлении есть \emph{понятие} вещи, \emph{всеобщее в самой этой
вещи} (как, например, каждый человеческий индивид,
хотя и бесконечно своеобразен, все же имеет в себе prius
[первичное] всего своего своеобразия, prius, состоящее в
том, что он в этом своеобразии есть \emph{человек}, так же как
каждое отдельное \emph{животное} имеет prius, состоящее в том,
что оно животное), то нельзя сказать, чт\'о осталось бы от
такого индивида~-- какими бы многообразными прочими
предикатами он ни был наделен,~-- если бы от него была
отнята эта основа (хотя последнюю тоже можно назвать
предикатом). Непременная основа, понятие, всеобщее, которое
и есть сама мысль, поскольку только при слове
<<мысль>> можно отвлечься от представления,~-- это всеобщее
нельзя рассматривать \emph{лишь} как безразличную форму
\emph{при} некотором содержании. Но эти мысли обо всех природных
и духовных вещах, само субстанциальное \emph{содержание},
представляют собой еще такое содержание, которое
заключает в себе многообразные определенности и
еще имеет в себе различие души и тела; понятия и соотносимой
с ним реальности; более глубокой основой служит
душа, взятая сама по себе, чистое понятие~-- сердцевина
предметов, их простой жизненный пульс, равно как
и жизненный пульс самог\'о субъективного мышления о
них. Задача и состоит в том, чтобы осознать эту \emph{логическую}
природу, которая одушевляет дух, движет и действует
в нем. Инстинктивная деятельность отличается от
руководимой интеллектом и свободной деятельности
вообще тем, что последняя осуществляется сознательно;
поскольку содержание побудительного мотива выключается
из непосредственного единства с субъектом и доведено
до предметности, возникает свобода духа, который, будучи
в инстинктивной деятельности мышления связанным своими
категориями, расщепляется на бесконечно многообразный
материал. В этой сети завязываются там и сям
более прочные узлы, служащие опорными и направляющими
пунктами жизни и сознания духа; эти узлы обязаны
своей прочностью и мощью именно тому, что они, доведенные
до сознания, суть в себе и для себя сущие понятия
его сущности. Важнейший пункт, уясняющий природу
духа,~-- это отношение не только того, чт\'о он есть \emph{в себе},
к тому, чт\'о он есть \emph{в действительности}, но и того, чем он
\emph{себя знает}; так как дух есть по своей сущности сознание,
то это знание себя есть основное определение его \emph{действительности}.
Следовательно, высшая задача логики~--
очистить категории, действующие лишь инстинктивно как
влечения и осознаваемые духом прежде всего разрозненно,
тем самым как изменчивые и путающие друг друга,
доставляющие ему таким образом разрозненную и
сомнительную действительность, и этим очищением возвысить
его в них к свободе и истине.

То, на что мы указали как на начало науки, огромная
ценность которого, взятого само по себе и в то же время
как условие истинного познания, признано было уже ранее,
а именно рассмотрение понятий и вообще моментов
понятия, определений мысли, прежде всего как формы,
отличные от содержания и лишь касающиеся его,~-- это
рассмотрение тотчас же проявляет себя в себе самом
неадекватным отношением к истине, признаваемой предметом
и целью логики. Ибо, беря их просто как формы,
как отличные от содержания, принимают, что им присуще
определение, характеризующее их как конечные и делающее
их неспособными схватить истину, которая бесконечна
в себе. Если истинное в каком-либо отношении и
сочетается вновь с ограниченностью и конечностью, то
это есть момент его отрицания, его неистинности и недействительности,
именно его конца, а не утверждения, каковое
оно есть как истинное. По отношению к убожеству
чисто формальных категорий инстинкт здравого смысла
почувствовал себя, наконец, столь окрепшим, что он презрительно
предоставляет их познание школьной логике
и школьной метафизике, пренебрегая в то же время той
ценностью, которую осознание этих нитей имеет уже
само по себе, и не сознавая, что, когда он ограничивается
инстинктивным действием естественной логики, а тем
более когда он обдуманно (reflektiert) отвергает знание и
познание самих определений мысли, он рабски служит
неочищенному и, стало быть, несвободному мышлению.
Простым основополагающим определением или общим
формальным определением совокупности таких форм служит
\emph{тождество}, которое в логике этой совокупности форм
признается законом, $A = A$, законом противоречия. Здравый
смысл в такой мере потерял свое почтительное отношение
к школе, которая обладает такими законами истины
и в которой их продолжают излагать, что он из-за
этих законов насмехается над ней и считает невыносимым
человека, который, руководясь такими законами,
умеет высказывать такого рода истины: растение есть
растение, наука есть наука и \emph{т.\,д. до бесконечности}.
Относительно формул, служащих правилами умозаключения,
которое на самом деле представляет собой одно из
главных применений рассудка, также упрочилось столь
же справедливое сознание, что они безразличные средства,
которые по меньшей мере приводят и к заблуждению
и которыми пользуется софистика; что, как бы мы ни
определяли истину, они непригодны для более высокой
истины, например религиозной; что они вообще касаются
лишь правильности познания, а не истины, хотя было бы
несправедливо отрицать, что в познании у них есть такая
область, где они должны обладать значимостью, и что
они в то же время~-- существенный материал для мышления
разума.

Недостаточность этого способа рассмотрения мышления,
оставляющего в стороне истину, может быть восполнена
лишь тем, что к мысленному рассмотрению привлекается
не только то, чт\'о обычно считается внешней формой,
но и содержание. Вскоре само собой обнаруживается,
что то, чт\'о в ближайшей самой обычной рефлексии отделяют
от формы как содержание, в самом деле не должно
быть бесформенным, лишенным определений внутри себя,
ибо в таком случае оно было бы лишь пустотой, скажем
абстракцией вещи в себе; что оно, наоборот, обладает в
самом себе формой и, более того, только благодаря ей
одушевлено и обладает содержимым (Gehalt), и что именно
она сама превращается лишь в видимость некоего содержания,
а тем самым и в видимость чего-то внешнего
по отношению к этой видимости [содержания]. С этим
введением содержания в логическое рассмотрение предметом
[логики] становятся не \emph{вещи} (Dinge), а \emph{суть}
(Sache), \emph{понятие} вещей. Однако при этом нам могут также
напомнить, что \emph{имеется} множество понятий, множество
сутей. Чем же ограничивают это множество,~-- об
этом мы отчасти уже сказали выше, а именно, что понятие
как мысль вообще, как всеобщее есть беспредельное сокращение
по сравнению с единичностью вещей, которые в своей
множественности предстают неопределенному созерцанию
и представлению. Отчасти же \emph{какое-то} понятие (ein Begriff)
есть в то же время, во-первых, \emph{само} понятие (der
Begriff) в самом себе, а последнее имеется только в единственном
числе и составляет субстанциальную основу; но,
во-вторых, оно есть некоторое \emph{определенное} понятие, каковая
определенность в нем есть то, чт\'о выступает как
содержание; определенность же понятия есть определение
формы указанного субстанциального единства, момент
формы как целокупности, момент \emph{самого понятия}, составляющего
основу определенных понятий. Это понятие чувственно
не созерцается и не представляется; оно только
предмет, продукт и содержание \emph{мышления} и в себе и для
себя сущая суть, логос, разум того, чт\'о есть, истина того,
чт\'о носит название вещей. Уж менее всего д\'олжно оставлять
вне науки логики логос. Поэтому не должно быть
делом произвола, вводить ли его в науку или оставлять его
за ее пределами. Если те определения мысли, которые суть
только внешние формы, рассматриваются истинно в них
самих, то из этого может следовать лишь то, что они конечны,
что их якобы самостоятельное бытие (Für-sich-sein-Sollen)
неистинно и что их истина~-- понятие. Поэтому,
имея дело с определениями мысли, которые вообще
пронизывают наш дух инстинктивно и бессознательно и
которые остаются беспредметными, незамеченными, даже
когда они проникают в язык, логическая наука будет также
реконструкцией тех определений мысли, которые выделены
рефлексией и фиксированы ею как субъективные,
внешние формы по отношению к материалу и содержанию.

Нет ни одного предмета, который, сам по себе взятый,
поддавался бы столь строгому, имманентно пластическому
изложению, как развитие мышления в его необходимости;
нет ни одного предмета, который в такой мере требовал
бы такого изложения; в этом отношении наука о нем
должна была бы превосходить даже математику, ибо ни
один предмет не имеет в самом себе этой свободы и независимости.
Такой способ изложения требовал бы, как это
по-своему происходит при математическом выведении,
чтобы ни на одной ступени развития не встречались определения
мысли и рефлексии, которые не возникали бы
непосредственно на этой ступени, а переходили бы в нее
из предшествующих ступеней. Но, конечно, приходится
вообще отказаться от такого абстрактного совершенства
изложения. Уже одно то обстоятельство, что наука должна
начинать с абсолютно простого и, стало быть, наиболее
всеобщего и пустого, требует, чтобы способ изложения [ее]
допускал только такие совершенно простые выражения
[для уяснения] простого без какого-либо добавления хотя
бы одного слова; единственно, чт\'о по существу дела требовалось
бы,~-- это отрицательные рефлексии, которые старались
бы не допускать и удалять то, чт\'о обычно могло
бы сюда привнести представление или неупорядоченное
мышление. Однако подобные вторжения (Einfälle) в простой,
имманентный ход развития [мысли] сами по себе
случайны и старания предотвратить их, стало быть, столь
же случайны; кроме того, было бы тщетно стремиться
предупредить \emph{все} такого рода вторжения именно потому,
что они не касаются существа дела, и было бы по крайней
мере недостаточным то, что желательно для удовлетворения
требования систематичности. Но свойственные
нашему нынешнему сознанию беспокойство и разбросанность
также не позволяют нам не принимать во внимание
более или менее доступные всем рефлексии и вторжения
[в мысль]. Пластический способ изложения требует к тому
же пластической способности восприятия и понимания\endnotemark{};
но таких пластических юношей и мужей, каких
придумывает Платон, таких слушателей, столь спокойно
следящих лишь за существом дела, сами отрекаясь от
высказывания \emph{собственных} рефлексий и взбредших на ум
соображений (Einfälle), при помощи которых доморощенному
мышлению не терпится показать себя, нельзя
было бы выставить в современном диалоге; еще в меньшей
степени можно было бы рассчитывать на таких читателей.
Наоборот, я слишком часто встречал чрезмерно
ярых противников, не способных сообразить такой простой
вещи, что взбредшие им в голову мысли и возражения
содержат категории, которые суть предпосылки и которые
сами должны быть подвергнуты критическому рассмотрению,
прежде чем пользоваться ими. Неспособность
осознавать это заходит невероятно далеко; она приводит
к основному недоразумению, к тому плохому, т.\,е. необразованному
способу рассуждения, когда при рассмотрении
какой-то категории мыслят \emph{нечто иное}, а не самое
эту категорию. Такая неспособность осознавать тем более
не может быть оправдана, что это \emph{иное} представляет собой
другие определения мысли и понятия, а в системе
логики именно эти другие категории также должны были
найти свое место и быть там самостоятельно рассмотрены.
Более всего это бросается в глаза в подавляющей части
возражений и нападок, вызванных первыми понятиями
или положениями логики,~-- \emph{бытием}, \emph{ничто} и \emph{становлением};
последнее, хотя оно и само есть простое определение,
тем не менее бесспорно~-- простейший анализ показывает
это~-- содержит указанные два определения в качестве
моментов. Основательность, по-видимому, требует,
чтобы прежде всего было вполне исследовано начало как
основа, на которой зиждется все остальное, и даже требует,
чтобы не шли дальше, прежде чем оно не окажется
прочным, и чтобы, напротив, если окажется, что это не
так, было отвергнуто все следующее за ним. Эта основательность
обладает также и тем преимуществом, что она
необычайно облегчает дело мышления: она имеет перед
собой все [дальнейшее] развитие заключенным в этом зародыше
и считает, что покончила со всем [исследованием],
если покончила с этим зародышем, а с ним легче всего
справиться потому, что он есть наипростейшее, само простое;
главным образом именно легкостью требуемой работы
подкупает эта столь самодовольная основательность.
Это ограничение [критики] простым предоставляет свободный
простор произволу мышления, которое само не
желает оставаться простым, а приводит относительно
этого простого свои соображения. Будучи вполне вправе
сначала заниматься \emph{только} принципом и, стало быть, не
вдаваться в [рассмотрение] \emph{дальнейшего}, эта основательность
сама действует обратно этому, привлекая к рассмотрению
\emph{дальнейшее}, т.\,е. другие категории, чем ту, которая
представляет собой только принцип, другие предпосылки
и предубеждения. Такие предпосылки, как то, что бесконечность
отлична от конечности, содержание не то, что
форма, внутреннее не то, что внешнее, а опосредствование
точно так же не есть непосредственность (как будто
кто-то этого не знает), приводятся в виде наставлений и
их не столько доказывают, сколько рассказывают, уверяя,
что это так. В таком наставлении как образе действий
есть~-- иначе это нельзя назвать~-- нечто глупое. По сути
же дела здесь отчасти неправомерно то, что такого рода
[положения] только служат предпосылкой и сразу принимаются;
отчасти же и в еще большей мере здесь имеется
незнание того, что потребность и дело логического мышления~--
именно исследовать, может ли быть истинным
конечное без бесконечного и, равным образом, может ли
быть \emph{чем-то истинным}, а также \emph{чем-то действительным}
такая абстрактная бесконечность, лишенное формы содержание
и лишенная содержания форма, такое внутреннее
само по себе, не имеющее никакого внешнего проявления,
и т.\,д. Но эта культура и дисциплина мышления, благодаря
которым достигается его пластическое отношение
[к предмету научного рассмотрения] и преодолевается
нетерпение вторгающейся рефлексии, приобретаются
единственно лишь движением вперед, изучением и проделыванием
всего пути развития.

\endnotetext{
  О пластичности как особенности греческого духа и искусства
  см.: \emph{Гегель}. Эстетика, т.\,3. М., 1970, стр.\,113--114.
}

Тому, кто в новейшее время работает над самостоятельным
построением философской науки, можно при
упоминании о платоновском изложении напомнить рассказ
о том, что Платон семь раз перерабатывал свои книги
о государстве. Напоминание об этом и сравнение (поскольку,
по-видимому, это напоминание заключает в себе
таковое) могли бы только еще в большей мере вызвать
желание, чтобы автору произведения, которое, как принадлежащее
нынешнему миру, имеет перед собой для
разработки более глубокий принцип, более трудный предмет
и более богатый по объему материал, был предоставлен
свободный досуг для переработки его не семь, а семьдесят
семь раз. Но, принимая во внимание, что труд
писался в условиях, диктовавшихся внешней необходимостью,
что широта и многосторонность присущих нашему
времени интересов неизбежно отрывали от работы
над ним, что автор даже испытывал сомнение, оставляют
ли еще повседневная суета и оглушающая болтливость
самомнения, довольная тем, что она ограничивается этой
суетой, возможность для участия в бесстрастной тишине
чисто мыслящего познания,~-- принимая во внимание все
это, автор, рассматривая свой труд под углом зрения
величия задачи, должен довольствоваться тем, чем этот
труд мог стать в таких условиях.

\signature{Берлин, 7 ноября 1831\,г.}


%%% Local Variables:
%%% mode: latex
%%% TeX-master: "../main"
%%% End:


\chapter{Введение}

\section{Общее понятие логики}

Ни в какой другой науке не чувствуется столь сильно
потребность начинать с самой сути дела, без предварительных
размышлений, как в науке логики. В каждой
другой науке рассматриваемый ею предмет и научный
метод различаются между собой; равным образом и содержание
[этих наук] не начинает абсолютно с самого
начала, а зависит от других понятий и связано с окружающим
его иным материалом. Вот почему за этими науками
признается право говорить лишь при помощи лемм
о почве, на которой они стоят, и о ее связи, равно как
и о методе, прямо применять предполагаемые известными
и принятыми формы дефиниций и т.\,п. и пользоваться
для установления своих всеобщих понятий и основных
определений обычным способом рассуждения.

Логика же, напротив, не может брать в качестве
предпосылки ни одной из этих форм рефлексии или
правил и законов мышления, ибо сами они составляют
часть ее содержания и сначала должны получить свое
обоснование внутри нее. Но в ее содержание входит не
только указание научного метода, но и вообще само
\emph{понятие науки}, причем это понятие составляет ее конечный
результат: она поэтому не может заранее сказать,
чт\'о она такое, лишь все ее изложение порождает это
знание о ней самой как ее итог (Letztes) и завершение.
И точно так же ее предмет, \emph{мышление} или, говоря определеннее,
\emph{мышление, постигающее в понятиях}, рассматривается
по существу внутри нее; понятие этого мышления
образуется в ходе ее развертывания и, стало быть,
не может быть предпослано. То, чт\'о мы предпосылаем
здесь в этом введении, не имеет поэтому своей целью
обосновать, скажем, понятие логики или дать наперед
научное обоснование ее содержания и метода, а имеет
своей целью с помощью некоторых разъяснений и размышлений
в рассуждающем и историческом духе растолковать
представлению ту точку зрения, с которой
следует рассматривать эту науку.

Если вообще логику признают наукой о мышлении,
то под этим понимают, что это мышление составляет
\emph{голую форму} некоторого познания, что логика абстрагируется
от всякого \emph{содержания}, и так называемая вторая
\emph{составная часть} всякого познания, \emph{материя}, должна быть
дана откуда-то извне, что, следовательно, логика, от которой
эта материя совершенно независима, может только
указать формальные условие истинного познания, но не
может содержать самое реальную истину, не может даже
быть \emph{путем} к реальной истине, так как именно суть истины,
содержание, находится вне ее.

Но, во-первых, неудачно уже утверждение, что логика
абстрагируется от всякого \emph{содержания}, что она только
учит правилам мышления, не имея возможности вдаваться
в рассмотрение мыслимого и его характера. В самом
деле, если, как утверждают, ее предмет~-- мышление
и правила мышления, то она непосредственно в них имеет
свое, ей лишь свойственное содержание; в них она имеет
также и вторую составную часть познания, некую материю,
характер которой ее интересует.

Во-вторых, вообще представления, на которых до сих
пор основывалось понятие логики, отчасти уже сошли
со сцены, отчасти им пора полностью исчезнуть, пора,
чтобы понимание этой науки исходило из более высокой
точки зрения и чтобы она приобрела совершенно измененный
вид.

Понятие логики, которого придерживались до сих пор,
основано на раз навсегда принятом обыденным сознанием
предположении о раздельности \emph{содержания} познания
и его \emph{формы}, или, иначе сказать, \emph{истины} и \emph{достоверности}.
Предполагается, \emph{во-первых}, что материя познавания существует
сама по себе вне мышления как некий готовый
мир, что мышление, взятое само по себе, пусто, что оно
примыкает к этой материи как некая форма извне, наполняется
ею, лишь в ней обретает некоторое содержание
и благодаря этому становится реальным познанием.

\emph{Во-вторых}, эти две составные части (ибо предполагается,
что они находятся между собой в отношении составных
частей и познание составляется из них механически
или в лучшем случае химически)\endnotemark{} находятся,
согласно этому воззрению, в следующей иерархии: объект
есть нечто само по себе завершенное, готовое,
нисколько не нуждающееся для своей действительности
в мышлении, тогда как мышление есть нечто ущербное,
которому еще предстоит восполнить себя в некоторой
материи, и притом оно должно сделать себя адекватным
своей материи в качестве мягкой неопределенной формы.
Истина есть соответствие мышления предмету, и для
того чтобы создать такое соответствие~-- ибо само по
себе оно не дано как нечто наличное;~-- мышление должно
подчиняться предмету, сообразоваться с ним.

\endnotetext{
  В томе 3 <<Науки логики>> Гегель дает развернутое определение этим
  понятиям (раздел второй, главы первая и вторая).
}

\emph{В-третьих}, так как различие материи и формы, предмета
и мышления не оставляется в указанной туманной
неопределенности, а берется более определенно, то каждая
из них есть отделенная от другой сфера. Поэтому
мышление, воспринимая и формируя материю, не выходит
за свои пределы, воспринимание ее и сообразование
с ней остается видоизменением его самого, и от этого
оно не становится своим иным; а сознающий себя процесс
определения уж во всяком случае принадлежит
лишь исключительно мышлению. Следовательно, даже в
своем отношении к предмету оно не выходит из самого
себя, не переходит к предмету; последний остается как
вещь в себе просто чем-то потусторонним мышлению.

Эти взгляды на отношение между субъектом и объектом
выражают собой те определения, которые составляют
природу нашего обыденного сознания, охватывающего
лишь явления. Но когда эти предрассудки переносятся
в область разума, как будто и в нем имеет место то же
самое отношение, как будто это отношение истинно само
по себе, они представляют собой заблуждения, опровержением
которых, проведенным через все части духовного
и природного универсума, служит философия или, вернее,
они суть заблуждения, от которых следует освободиться
до того, как приступают к философии, так как они преграждают
вход в нее.

В этом отношении прежняя метафизика имела более
возвышенное понятие о мышлении, чем то, которое сделалось
ходячим в новейшее время. А именно она исходила
из того, что действительно истинное (das wahrhaft
Wahre) в предметах~-- это то, чт\'о познается мышлением
о них и в них; следовательно, действительно истинны
не предметы в своей непосредственности, а лишь предметы,
возведенные в форму мышления, предметы как
мыслимые. Эта метафизика, стало быть, считала, что
мышление и определения мышления не нечто чуждое
предметам, а скорее их сущность, иначе говоря, что \emph{вещи}
и \emph{мышление} о них сами по себе соответствуют друг
другу (как и немецкий язык выражает их сродство)\endnotemark{},
что мышление в своих имманентных определениях и
истинная природа вещей составляют одно содержание.

\endnotetext{
  Dinge (вещи) и Denken (мышление) имеют некоторое
  сходство в своем звучании и начертании, на что Гегель и намекает.
}

Но философией овладел \emph{рефлектирующий} рассудок.
Мы должны точно знать, чт\'о означает это выражение,
которое часто употребляется просто как эффектное
словечко (Schlagwort). Под ним следует вообще понимать
абстрагирующий и, стало быть, разделяющий рассудок,
который упорствует в своих разделениях. Обращенный
против разума, он ведет себя как \emph{обыкновенный здравый
смысл} и отстаивает свой взгляд, согласно которому истина
покоится на чувственной реальности, мысли суть
\emph{только} мысли в том смысле, что лишь чувственное восприятие
сообщает им содержательность (Gehalt) и реальность,
а разум, поскольку он остается сам по себе,
порождает лишь химеры\endnotemark{}. В этом отречении разума от
самого себя утрачивается понятие истины, разум ограничивают
познанием только субъективной истины, только
явления, только чего-то такого, чему не соответствует
природа самой вещи; \emph{знание} низведено до уровня \emph{мнения}.

\endnotetext{
  Ср. замечание Гегеля об <<экзотерическом учении кантовской
  философии>> в начале Предисловия к первому изданию.
}

Однако это направление, принятое познанием и представляющееся
потерей и шагом назад, имеет более глубокое
основание, на котором вообще покоится возведение
разума в более высокий дух новейшей философии. А
именно основание указанного, ставшего всеобщим, представления
следует искать в понимании того, что определения
рассудка \emph{необходимо сталкиваются} с самими собой.~--
Уже названная нами рефлексия заключается в
том, что выходят \emph{за пределы} конкретно непосредственного
и \emph{определяют} и \emph{разделяют} его. Но равным образом
она должна выходить и \emph{за пределы} этих своих \emph{разделяющих}
определений, и прежде всего \emph{соотносить} их.
В стадии (auf dem Standpunkte) этого соотнесения выступает
наружу их столкновение. Это осуществляемое
рефлексией соотнесение само по себе есть дело разума;
возвышение над указанными определениями, которое
приходит к пониманию их столкновения, есть большой
отрицательный шаг к истинному понятию разума. Но
это не доведенное до конца понимание приводит к ошибочному
взгляду, будто именно разум впадает в противоречие
с собой; оно не признает, что противоречие как
раз и есть возвышение разума над ограниченностью
рассудка и ее устранение. Вместо того чтобы сделать
отсюда последний шаг вверх, познание неудовлетворительности
рассудочных определений отступает к чувственному
существованию, ошибочно полагая, что в нем оно
найдет устойчивость и согласие. Но так как, с другой
стороны, это познание знает себя как познание только
явлений, то оно тем самым соглашается, что чувственное
существование неудовлетворительно, но вместе с тем
предполагает, что, хотя вещи в себе и не познаются,
однако внутри сферы явлений познание правильное; как
будто различны только \emph{роды предметов}, и один род предметов,
а именно вещи в себе, не познается, другой же
род предметов, а именно явления, познается. Это похоже
на то, как если бы мы приписывали кому-нибудь правильное
уразумение, но при этом прибавили бы, что он,
однако, способен уразуметь не истинное, а только ложное.
Так же как это было бы нелепо, столь же нелепо
истинное познание, не познающее предмета, как он есть
в себе.

\emph{Критика форм рассудка} привела к указанному выше
выводу, что эти формы не \emph{применимы к вещам в себе}.
Это может иметь только тот смысл, что эти формы суть
в самих себе нечто неистинное. Но так как все еще считают
их значимыми для субъективного разума и для
опыта, то критика ничего не изменила в них самих, а
оставляет их для субъекта в том же виде, в каком они
прежде имели значение для объекта. Но если они недостаточны
для познания вещи в себе, то рассудок, которому,
как утверждают, они принадлежат, еще в меньшей
степени должен был бы принимать их и довольствоваться
ими. Если они не могут быть определениями \emph{вещи в
себе}, то они еще в меньшей степени могут быть определениями
рассудка, за которым мы должны были бы
признать по крайней мере достоинство некоторой вещи
в себе. Определения конечного и бесконечного одинаково
сталкиваются между собой, будем ли мы применять их
к времени и пространству, к миру или они будут признаны
определениями внутри духа, точно так же как черное
и белое все равно образуют серое, смешаем ли мы их
на стене или только на палитре. Если наше представление
о \emph{мире} расплывается, когда мы на него переносим
определения бесконечного и конечного, то сам \emph{дух}, содержащий
в себе эти два определения, должен в еще большей
мере оказаться чем-то внутренне противоречивым,
чем-то расплывающимся. Свойство материи или предмета,
к которым мы стали бы их применять или в которых они
находятся, не может составлять [здесь] какое-либо различие,
ибо предмет внутренне противоречив только из-за
указанных определений и согласно им.

Указанная критика, стало быть, отдалила формы
объективного мышления только от вещи, но оставила их
в субъекте в том виде, в каком она их нашла. А именно,
она не рассмотрела этих форм, взятых сами по себе, со
свойственным им содержанием, а прямо заимствовала их
при помощи лемм из субъективной логики. Таким образом,
не было речи о выведении их из (an) них самих
или хотя бы о выведении их как субъективно-логических
форм, тем более не было речи о диалектическом их рассмотрении.

Более последовательно проведенный трансцендентальный
идеализм признал ничтожность сохраненного
еще критической философией призрака \emph{вещи в себе}, этой
абстрактной, оторванной от всякого содержания тени,
и он поставил себе целью окончательно его уничтожить\endnotemark{}.
Кроме того, эта философия положила начало попытке
дать разуму развернуть свои определения из самого себя.
Но субъективная позиция этой попытки не позволила
завершить ее. В дальнейшем отказались от этой позиции,
а с ней и от указанной начатой попытки и от разработки
чистой науки.

\endnotetext{
  Гегель имеет в виду философию Фихте.
}

Но совершенно не принимая во внимание метафизического
значения, рассматривают то, чт\'о обычно понимают
под логикой. Эта наука в том состоянии, в каком
она еще находится, лишена, правда, того содержания,
которое признается в обыденном сознании реальностью
и некоей истинной вещью (Sache). Однако не поэтому
она формальная наука, лишенная всякой содержательной
истины. В том материале, который в ней не находят и
отсутствием которого обычно объясняют ее неудовлетворительность,
мы, впрочем, не должны искать сферу истины.
Причина бессодержательности логических форм
скорее только в способе их рассмотрения и трактовки.
Так как они в качестве застывших определений лишены
связи друг с другом и не удерживаются в органическом
единстве, то они мертвые формы и в них не обитает дух,
составляющий их живое конкретное единство. Но тем
самым им недостает подлинного содержания (Inhalt)~--
материи, которая была бы в самой себе содержанием
(Gehalt). Содержание, которого мы не находим в логических
формах, есть не что иное, как некоторая прочная
основа и сращение (Konkretion) этих абстрактных определений,
и обычно ищут для них такую субстанциальную
сущность вне логики. Но сам логический разум и
есть то субстанциальное или реальное, которое удерживает
в себе все абстрактные определения, и он есть
их подлинное, абсолютно конкретное единство. Нет,
следовательно, надобности далеко искать то, чт\'о обычно
называют материей. Если логика, как утверждают, лишена
содержания, то это вина не предмета логики, а только
способа его понимания.

Это размышление приводит нас к необходимости указать
ту точку зрения, с которой мы должны рассматривать
логику, поскольку эта точка зрения отличается от
прежней трактовки этой науки и есть единственно истинная
точка зрения, которой она впредь должна придерживаться
раз и навсегда.

В <<Феноменологии духа>> я представил сознание в
его поступательном движении от первой непосредственной
противоположности между ним и предметом до
абсолютного знания. Этот путь проходит через все формы
\emph{отношения сознания к объекту} и имеет своим
результатом \emph{понятие науки}. Это понятие, следовательно
(независимо от того, что оно возникает в рамках самой
логики), не нуждается здесь в оправдании, так как оно
его получило уже там; и оно не может иметь никакого
другого оправдания, кроме этого его порождения сознанием,
для которого все его собственные образы разрешаются
в это понятие, как в истину. Резонерское обоснование
или разъяснение понятия науки может самое большее
привести лишь к тому, что понятие станет объектом
представления и о нем будут получены исторические
сведения; но дефиниция науки, или, точнее, логики,
имеет свое \emph{доказательство} исключительно в указанной
необходимости ее происхождения. Та дефиниция, которой
какая-либо наука начинает абсолютно с самого начала,
не может содержать ничего другого, кроме определенного
корректного выражения того, что как \emph{известное}
и \emph{общепризнанное представляют себе} в качестве предмета
и цели этой науки. Что в качестве таковых представляют
себе именно это, [а не другое], это есть историческое
уверение, относительно которого можно сослаться
лишь на то или иное признанное или, собственно говоря,
можно только в виде просьбы предложить, чтобы считали
то или иное признанным. Вовсе не удивительно, что
один отсюда, другой оттуда приводит какой-нибудь случай
или пример, показывающий, что под таким-то выражением
нужно понимать еще нечто большее и иное и
что, стало быть, в его дефиницию следует включить еще
одно более частное или более общее определение и с
этим должна быть согласована и наука.~-- При этом от
резонерства зависит, до какой границы и в каком объеме
те или иные определения должны быть включены или
исключены; само же резонерство имеет перед собой на
выбор самые многообразные и самые различные воззрения,
застывшее определение которых может в конце
концов давать только произвол. При этом способе начинать
науку с ее дефиниции нет и речи о потребности
показать \emph{необходимость ее предмета} и, следовательно,
также ее самой.

Итак, в настоящем произведении понятие чистой науки
и его дедукция берутся как предпосылка постольку,
поскольку феноменология духа есть не что иное, как дедукция
его. Абсолютное знание есть \emph{истина} всех способов
сознания, потому что, как показало [описанное в <<Феноменологии
духа>>] движение сознания, лишь в абсолютном
знании полностью преодолевается разрыв между
\emph{предметом} и \emph{достоверностью самого себя}, и истина стала
равной этой достоверности, так же как и эта достоверность
стала равной истине.

Чистая наука, стало быть, предполагает освобождение
от противоположности сознания [и его предмета]. Она
содержит в себе \emph{мысль, поскольку мысль есть также и вещь}
(Sache) \emph{сама по себе}, или \emph{содержит вещь самое по себе},
поскольку вещь \emph{есть также и чистая мысль}. В качестве
\emph{науки} истина есть чистое развивающееся самосознание
и имеет образ самости, [что выражается в том], что \emph{в себе
и для себя сущее есть осознанное} (gewusster) \emph{понятие},
а \emph{понятие, как таковое, есть в себе и для себя сущее}.
Это объективное мышление и есть \emph{содержание} чистой
науки. Она поэтому в такой мере не формальна, в такой
мере не лишена материи для действительного и истинного
познания, что скорее лишь ее содержание и есть абсолютно
истинное или (если еще угодно пользоваться
словом <<материя>>) подлинная материя, но такая материя,
для которой форма не есть нечто внешнее, так как
эта материя есть скорее чистая мысль и, следовательно,
есть сама абсолютная форма. Логику, стало быть, следует
понимать как систему чистого разума, как царство
чистой мысли. \emph{Это царство есть истина, какова она без
покровов, в себе и для себя самой}. Можно поэтому выразиться
так: это содержание есть \emph{изображение бога, каков
он в своей вечной сущности до сотворения природы и
какого бы то ни было конечного духа}.

Анаксагор восхваляется как тот, кто впервые высказал
ту мысль, что \emph{нус, мысль}, есть первоначало (Prinzip)
мира, что необходимо определить сущность мира как
мысль. Он этим положил основу интеллектуального
воззрения на Вселенную, чистой формой которого должна
быть \emph{логика}. В ней мы имеем дело не с мышлением \emph{о}
чем-то таком, чт\'о лежало бы в основе и существовало
бы особо, вне мышления, не с формами, которые будто
бы дают только \emph{признаки} истины; необходимые формы
и собственные определения мышления суть само содержание
и сама высшая истина.

Для того чтобы представление по крайней мере понимало,
в чем дело, следует отбросить мнение, будто истина
есть нечто осязаемое. Подобную осязаемость вносят, например,
даже еще в платоновские идеи, имеющие бытие
в мышлении бога, [толкуя их так], как будто они существующие
вещи, но существующие в некоем другом мире
или области, вне которой находится мир действительности,
обладающий отличной от этих идей субстанциальностью,
реальной только благодаря этому отличию.
Платоновская идея есть не что иное, как всеобщее, или,
говоря более определенно, понятие предмета; лишь
в своем понятии нечто обладает действительностью; поскольку
же оно отлично от своего понятия, оно перестает
быть действительным и есть нечто ничтожное; осязаемость
и чувственное вовне-себя-бытие принадлежат этой
ничтожной стороне.~-- Но, с другой стороны, можно сослаться
на собственные представления обычной логики;
в ней ведь принимается, что, например, дефиниции содержат
не определения, относящиеся лишь к познающему
субъекту, а определения предмета, составляющие его
самую существенную, неотъемлемую природу. Или [другой
пример]: когда умозаключают от данных определений
к другим, считают, что выводы не нечто внешнее и чуждое
предмету, а скорее принадлежат самому предмету,
что этому мышлению соответствует бытие.~-- Вообще при
употреблении форм понятия, суждения, умозаключения,
дефиниции, деления т.\,д. исходят из того, что они формы
не только сознающего себя мышления, но и предметного
смысла (Verstandes).~-- <<Мышление>> есть выражение, которое
содержащееся в нем определение приписывает
преимущественно сознанию. Но так как говорят, что
\emph{в предметном мире есть смысл} (Verstand), \emph{разум}, что
дух и природа имеют \emph{всеобщие законы}, согласно которым
протекает их жизнь и совершаются их изменения, то
признают, что определения мысли обладают также
и объективными ценностью и существованием.

Критическая философия, правда, уже превратила
\emph{метафизику} в \emph{логику}; однако подобно позднейшему
идеализму\endnote{Имеется в виду субъективный идеализм Фихте.}
она из страха перед объектом придала, как мы
уже сказали выше, логическим определениям преимущественно
субъективное значение; в то же время они
тем самым остаются обремененными объектом, которого
они избегали, и в них оставались как нечто потустороннее
вещь в себе\endnote{В философии Канта.},
бесконечный импульс\endnote{В философии Фихте.}. Но освобождение
от противоположности сознания [и его предмета],
которое наука должна иметь возможность предположить,
возвышает определения мысли над этим робким, незавершенным
взглядом и требует, чтобы их рассматривали
такими, каковы они в себе и для себя, без такого рода
ограничения и отношения, требует, чтобы их рассматривали
как логическое, как чисто разумное.

Кант в одном месте\endnotemark{} считает счастьем для логики,
а именно для того агрегата определений и положений,
который обычно носит название логики, то, что она
сравнительно с другими науками достигла столь раннего
завершения; со времени Аристотеля она, по его словам,
не сделала ни одного шага назад, но также и ни одного
шага вперед; последнего она не сделала потому, что она,
судя по всему, казалась законченной и завершенной.~--
Но если со времени Аристотеля логика не подверглась
никаким изменениям,~-- и в самом деле при рассмотрении
новых учебников логики мы убеждаемся, что
изменения сводятся часто больше всего лишь к сокращениям,~--
то мы отсюда должны сделать скорее тот вывод,
что она тем более нуждается в полной переработке; ибо
двухтысячелетняя непрерывная работа духа должна
была ему доставить более высокое сознание о своем мышлении
и о своей чистой сущности в самой себе. Сравнение
образов, до которых поднялись дух практического и религиозного
миров и дух науки во всякого рода реальном
и идеальном сознании, с образом, который носит логика
(его сознание о своей чистой сущности), являет столь
огромное различие, что даже при самом поверхностном
рассмотрении не может не бросаться тотчас же в глаза,
что это последнее сознание совершенно не соответствует
тем взлетам и недостойно их.

\endnotetext{
  <<Критика чистого разума>>, предисловие ко второму изданию,
  стр.\,VIII (\emph{И. Кант}. Сочинения в шести томах, т.\,3. М., 1964,
  стр.\,82. Все дальнейшие цитаты из сочинений Канта даны по
  этому изданию).
}

И в самом деле, потребность в преобразовании логики
чувствовалась давно. Следует сказать, что в той форме
и с тем содержанием, с каким логика излагается в учебниках,
она сделалась предметом презрения. Ее еще тащат
за собой больше из-за смутного чувства, что совершенно
без логики не обойтись, и из-за сохранившегося еще привычного,
традиционного представления о ее важности,
нежели из убеждения, что то обычное содержание и занятие
теми пустыми формами ценны и полезны.

Расширение, которое она получила в продолжение
некоторого времени благодаря [добавлению] психологического,
педагогического и даже физиологического материала,
в дальнейшем почти все признали искажением.
Большая часть этих психологических, педагогических,
физиологических наблюдений, законов и правил все равно,
даны ли они в логике или в какой-либо другой науке,
сама по себе должна представляться очень плоской и тривиальной.
А уж такие, например, правила, что следует
продумывать и подвергать критическому разбору прочитанное
в книгах или слышанное, что тот, кто плохо видит,
должен помочь своим глазам, надевая очки (правила,
дававшиеся в учебниках по так называемой прикладной
логике и притом с серьезным видом разделенные на
параграфы, дабы люди достигли истины),~-- такие правила
должны казаться излишними всем, кроме разве автора
учебника или преподавателей, не знающих, как растянуть
слишком краткое и мертвенное содержание
логики\endnotemark{}.

\endnotetext{
  Гегель имеет в виду сочинения Христиана Вольфа (1679--1754)
  и его последователей. В первом издании <<Науки логики>>
  (Нюрнберг,~1812) к этому месту имелось следующее примечание:
  <<Одно только что появившееся исследование этой науки~-- <<Система
  логики>> Фриза (Fries)~-- возвращается к антропологическим
  основам. Поверхностность представления или мнения самого по
  себе, составляющего исходный пункт этой <<Системы>>, а также
  ее обоснования избавляет меня от труда уделять какое-либо
  внимание этому незначительному произведению (Erscheinung)>>.
  <<Система логики>> Я. Фриза (1773--1843) вышла в 1811\,г.
}

Что же касается этого содержания, то мы уже указали
выше, почему оно так плоско. Его застывшие
определения считаются незыблемыми и ставятся лишь
во внешнее отношение друг с другом. Оттого, что в суждениях
и умозаключениях оперируют главным образом
количественной стороной определений и исходят из нее,
все оказывается покоящимся на внешнем различии, на
голом сравнении, все становится совершенно аналитическим
способом [рассуждения] и лишенным понятия
вычислением. Дедукция так называемых правил и законов,
в особенности законов и правил умозаключения,
немногим лучше, чем перебирание палочек разной длины
для сортирования их по величине или чем детская игра,
состоящая в подборе подгоняемых друг к другу частей
различным образом разрезанных картинок.~-- Поэтому не
без основания приравнивали это мышление к счету и в
свою очередь счет~-- к этому мышлению. В арифметике
числа берутся как нечто лишенное понятия, как нечто такое,
что помимо своего равенства или неравенства, т.\,е.
помимо своего совершенно внешнего отношения, не имеет
никакого значения,~-- берутся как нечто такое, что
ни само по себе, ни в своих отношениях не есть мысль.
Когда мы механически вычисляем, что три четверти,
помноженные на две трети, дают половину, то это действие
содержит примерно столь же много или столь же
мало мыслей, как и соображение о том, возможен ли в
данной фигуре тот или другой вид умозаключения.

Дабы эти мертвые кости логики оживотворились духом
и получили, таким образом, содержимое и содержание,
ее \emph{методом} должен быть тот, который единственно
только и способен сделать ее чистой наукой. В том состоянии,
в котором она находится, нет даже предчувствия
научного метода. Она имеет, можно сказать, форму опытной
науки. Опытные науки для того, чем они должны
быть, нашли свой особый метод, метод дефиниции и классификации
своего материала, насколько это возможно. Чистая
математика также имеет свой метод, который подходит
для ее абстрактных предметов и для количественного
определения, единственно в котором она их рассматривает.
Главное об этом методе и вообще о подчиненном
характере той научности, которая возможна в математике,
я высказал в предисловии к <<Феноменологии духа>>,
но он будет рассмотрен нами более подробно в рамках
самой логики. Спиноза, Вольф и другие впали в соблазн
применить этот метод также и к философии и сделать
внешнее движение лишенного понятая количества движением
понятия, что само по себе противоречиво. До сих
пор философия еще не нашла своего метода. Она смотрела
с завистью на системное построение математики и,
как мы сказали, заимствовала у нее ее метод или обходилась
методом тех наук, которые представляют собой
лишь смесь данного материала, исходящих из опыта положений
и мыслей, или выходила из затруднения тем,
что просто отбрасывала всякий метод. Но раскрытие того,
чт\'о единственно только и может быть истинным методом
философской науки, составляет предмет самой логики,
ибо метод есть осознание формы внутреннего самодвижения
ее содержания. В <<Феноменологии духа>> я дал
образчик этого метода применительно к более конкретному
предмету, к \emph{сознанию}\footnotemark{}. Там я показал формы сознания,
каждая из которых при своей реализации разрешает
(auflöst) в то же время самое себя, имеет своим
результатом свое собственное отрицание,~-- и тем самым
перешла в некоторую более высокую форму. Единственное,
что \emph{нужно для научного прогресса} и к совершенно
\emph{простому} пониманию чего следует главным образом стремиться,~--
это познание логического положения о том, что
отрицательное равным образом и положительно или,
иначе говоря, противоречащее себе не переходит в нуль,
в абстрактное ничто, а по существу лишь в отрицание
своего \emph{особенного} содержания, или, другими словами,
такое отрицание есть не отрицание всего, а \emph{отрицание
определенной вещи}, которая разрешает самое себя, стало
быть, такое отрицание есть определенное отрицание и,
следовательно, результат содержит по существу то, из
чего он вытекает; это есть, собственно говоря, тавтология,
ибо в противном случае он был бы чем-то непосредственным,
а не результатом. Так как то, чт\'о получается
в качестве результата, отрицание, есть \emph{определенное отрицание},
то оно имеет некоторое \emph{содержание}. Оно новое
\emph{понятие}, но более высокое, более богатое понятие, чем
предыдущее, ибо оно обогатилось его отрицанием или противоположностью;
оно, стало быть, содержит предыдущее
понятие, но содержит больше, чем только его, и есть
единство его и его противоположности.~-- Таким путем
должна вообще образоваться система понятий,~-- и в неудержимом,
чистом, ничего не принимающем в себя
извне движении получить свое завершение.

\footnotetext{
  Позднее же~-- применительно и к другим конкретным предметам
  и соответственно частям философии.
}

Я, разумеется, не могу полагать, что метод которому
я следовал в этой системе логики или, вернее, которому
следовала в самой себе эта система, не допускает
еще значительного усовершенствования, многочисленных
улучшений в частностях, но в то же время я знаю, что
он единственно истинный. Это само по себе явствует уже
из того, что он не есть нечто отличное от своего предмета
и содержания, ибо именно содержание внутри себя,
\emph{диалектика, которую он имеет в самом себе}, движет вперед
это содержание. Ясно, что нельзя считать научными
какие-либо способы изложения, если они не следуют движению
этого метода и не соответствуют его простому
ритму, ибо движение этого метода есть движение самой
сути дела.

В соответствии с этим методом я напоминаю, что
подразделения и заглавия книг, разделов и глав, данные
в настоящем сочинении, равно как и связанные с ними
объяснения, делаются для предварительного обзора и что
они, собственно говоря, имеют значение лишь с \emph{исторической}
точки зрения. Они не входят в содержание и корпус
науки, а суть сопоставления, произведенные внешней
рефлексией, которая уже ознакомилась со всем изложением
в целом, заранее знает поэтому последовательность
его моментов и указывает их еще до того, как они будут
выведены из самой сути дела.

В других науках такие предварительные определения
и подразделения, взятые сами по себе, также представляют
собой не что иное, как такие внешние указания;
но даже внутри самой науки они не поднимаются выше
такого характера. Даже в логике говорится, например:
<<У логики две главные части, общая часть и методика>>.
А затем в общей части мы без дальнейших объяснений
встречаем такие, скажем, \emph{заголовки}, как <<Законы мышления>>,
и далее \emph{первая глава}: <<О понятиях>>. \emph{Первый раздел}:
<<О ясности понятий>> и т.\,д. Эти определения и подразделения,
даваемые без всякой дедукции и обоснования,
образуют остов системы и всю связь подобных наук.
Такого рода логика видит свое призвание в провозглашении
того, что понятия и истины должны быть \emph{выведены}
из принципов; но когда речь идет о том, чт\'о она
называет методом, нет и намека на мысль о выведении.
Порядок состоит здесь примерно в сопоставлении однородного,
в рассмотрении более простого до [рассмотрения]
сложного и в других внешних соображениях, В отношении
же внутренней необходимой связи дело ограничивается
перечнем определений тех или иных разделов, и
переход осуществляется лишь так, что ставят теперь:
<<Вторая глава>> или пишут: <<Мы переходим теперь
к суждениям>> и т.\,д.

Заглавия и подразделения, встречающиеся в настоящей
системе, сами по себе также не имеют никакого другого
значения, помимо указания на последующее содержание.
Но, кроме того, при рассмотрении самой сути дела
должны найти место \emph{необходимость} связи и \emph{имманентное
возникновение} различий, ибо они входят в собственное
развитие определения понятия.

То, с помощью чего понятие ведет само себя дальше,
это~-- указанное выше \emph{отрицательное}, которое оно имеет
в самом себе; это составляет подлинно диалектическое.
\emph{Диалектика}, которая рассматривалась как некая обособленная
часть логики и относительно цели и точки зрения
которой господствовало, можно сказать, полное
непонимание, оказывается благодаря этому совсем в другом
положении. \emph{Платоновская} диалектика даже в <<Пармениде>>,
а в других произведениях еще более непосредственно,
с одной стороны, также имеет своей целью
только разбор и опровержение ограниченных утверждений
через них же самих, с другой стороны, вообще имеет
своим результатом ничто. Обычно видят в диалектике
лишь внешнее и отрицательное действие, не относящееся
к самой сути дела, вызываемое только тщеславием
как некоторой субъективной страстью колебать и разлагать
прочное и истинное, или видят в ней по меньшей
мере действие, приводящее к ничто как к тому, чт\'о составляет
тщету диалектически рассматриваемого предмета.

Кант отвел диалектике более высокое место, и это
одна из величайших его заслуг: он освободил ее от видимости
произвола, которая, согласно обычному представлению,
присуща ей, и изложил ее как \emph{необходимую деятельность
разума}. Пока ее считали только умением проделывать
фокусы и вызывать иллюзии, до тех пор просто
предполагалось, что она ведет фальшивую игру и вся
ее сила зиждется на том, что ей удается прикрыть обман,
и выводы, к которым она приходит, получаются хитростью
и представляют собой субъективную видимость.
Диалектические рассуждения Канта в разделе об антиномиях
чистого разума не заслуживают, правда, большой
похвалы, если присмотреться к ним пристальнее, как мы
в дальнейшем это сделаем в настоящем произведении
более обстоятельно; однако общая идея, из которой он
исходил и которой придавал большое значение,~-- это
\emph{объективность видимости} и \emph{необходимость противоречия},
свойственного \emph{природе} определений мысли; прежде всего,
правда, это касалось того способа, каким разум применяет
эти определения к \emph{вещам в себе}; но ведь именно то,
чт\'о они суть в разуме и по отношению к тому, чт\'о есть
в себе, и есть их природа. Этот результат, \emph{понимаемый
с его положительной стороны}, есть не что иное, как их
внутренняя \emph{отрицательность}, их движущая сама себя
душа, вообще принцип всякой природной и духовной
жизненности. Но так как Кант не идет дальше абстрактно
-отрицательной стороны диалектического, то выводом
оказывается лишь известное утверждение, что разум неспособен
познать бесконечное~-- странный вывод: сказать,
что так как бесконечное есть разумное, то разум
не способен познать разумное.

В этом диалектическом, как мы его берем здесь, и,
следовательно, в постижении противоположностей в их
единстве, или, иначе говоря, в постижении положительного
в отрицательном, состоит \emph{спекулятивное}. Это важнейшая,
но для еще неискушенной, несвободной способности
мышления труднейшая сторона. Если эта способность
мышления еще не избавила себя от чувственно
конкретных представлений и от резонерства, то она должна
сначала упражняться в абстрактном мышлении,
удерживать понятия в их \emph{определенности} и научиться
познавать, исходя из них. Изложение логики, имеющее
в виду эту цель, должно было бы придерживаться в своем
методе упомянутых выше подразделений, а в отношении
ближайшего содержания~-- определений, даваемых отдельным
понятиям, не вдаваясь [пока] в диалектическое.
Внешне оно стало бы похожим на обычное изложение
этой науки, впрочем, по содержанию и отличалось бы от
него и все еще служило бы к тому, чтобы упражнять
абстрактное, хотя и не спекулятивное мышление; а ведь
[обычная] логика, которая стала популярной благодаря
психологическим и антропологическим добавлениям, не
достигает даже и этой цели. То изложение логики доставляло
бы уму образ методически упорядоченного целого,
хотя сама душа этого построения~-- метод,~-- имеющая
свою жизнь в диалектическом, в нем не обнаруживалась
бы.

Что касается \emph{образования и отношения индивида
к логике}, то я в заключение еще отмечу, что эта наука,
подобно грамматике, выступает в двух видах или имеет
двоякого рода ценность. Она нечто одно для тех, кто
только приступает к ней и вообще к наукам, и нечто
другое для тех, кто возвращается к ней от них. Тот, кто
только начинает знакомиться с грамматикой, находит
в ее формах и законах сухие абстракции, случайные правила
и вообще множество обособленных друг от друга
определений, показывающих лишь ценность и значение
того, чт\'о заключается в их непосредственном смысле;
сначала познание не познает в них ничего кроме них.
Напротив, кто владеет каким-нибудь языком и в то же
время знает и другие языки, которые он сопоставляет
с ним, только тот и может почувствовать дух и образованность
народа в грамматике его языка; эти же правила
и формы имеют теперь для него наполненную содержанием,
живую ценность. Он в состоянии через грамматику
познать выражение духа вообще~-- логику. Точно
так же тот, кто только приступает к науке, находит сначала
в логике изолированную систему абстракций, ограничивающуюся
самой собой, не захватывающую других
знаний и наук. В сопоставлении с богатством представления
о мире, с реально выступающим содержанием
других наук и в сравнении с обещанием абсолютной
науки раскрыть \emph{сущность} этого богатства, \emph{внутреннюю
природу} духа и мира, \emph{истину}, эта наука в ее абстрактном
виде, в бесцветной, холодной простоте ее чистых
определений кажется скорее исполняющей все что угодно,
только не это обещание, и противостоящей этому богатству
как лишенная содержания. При первом знакомстве
с логикой ее значение ограничивают только ею самой;
ее содержание признается только изолированным
занятием определениями мысли, \emph{наряду} с которым другие
научные занятия имеют собственный самостоятельный
материал и содержание, на которые логическое оказывает
разве что формальное влияние, и притом такое
влияние, которое скорее осуществляется само собой и
в отношении которого можно, конечно, в крайнем случае
обойтись без научной формы и ее изучения. Другие
науки отбросили в целом метод, придерживающийся строгих
правил и дающий ряд дефиниций, аксиом, теорем и
их доказательств и т.\,д.; так называемая естественная
логика приобретает в них силу самостоятельно и обходится
без особого, направленного на само мышление познания.
Кроме того, материал и содержание этих наук,
взятые сами по себе, остаются независимыми от логического
и они более привлекательны и для ощущения, чувства,
представления и всякого рода практических интересов.

Таким образом, логику приходится, конечно, первоначально
изучать как нечто такое, чт\'о мы, правда, понимаем
и постигаем, но в чем мы не находим сначала широты,
глубины и более значительного смысла. Лишь
на основе более глубокого знания других наук логическое
возвышается для субъективного духа не только как
абстрактно всеобщее, но и как всеобщее, охватывающее
собой также богатство особенного, подобно тому как одно
и то же нравоучительное изречение в устах юноши, понимающего
его совершенно правильно, не имеет [для
него] той значимости и широты, которые оно имеет
для духа умудренного житейским опытом зрелого мужа;
для последнего этот опыт раскрывает всю силу заключенного
в таком изречении содержания. Таким образом,
логическое получает свою истинную оценку, когда оно
становится результатом опыта наук. Этот опыт являет
духу это логическое как всеобщую истину, являет его
не как некоторое \emph{особое} знание \emph{наряду} с другими материями
и реальностями, а как сущность всего этого прочего
содержания.

Хотя логическое в начале [его] изучения не существует
для духа в этой сознательной силе, он благодаря этому
изучению не в меньшей мере вбирает в себя ту силу,
которая ведет его ко всякой истине. Система логики~--
это царство теней, мир простых сущностей, освобожденных
от всякой чувственной конкретности. Изучение этой
науки, длительное пребывание и работа в этом царстве
теней есть абсолютная культура и дисциплина сознания.
Сознание занимается здесь делом, далеким от чувственных
созерцаний и целей, от чувств, от мира представлений,
имеющих лишь характер мнения. Рассматриваемое
со своей отрицательной стороны, это занятие состоит
в недопущении случайности резонирующего мышления
и произвола, выражающегося в том, что задумываются
над вот этими или противоположными им основаниями
и признают их [правильными].

Но главным образом благодаря этому занятию мысль
приобретает самостоятельность и независимость. Она привыкает
вращаться в абстракциях и двигаться вперед
с помощью понятий без чувственных субстратов, становится
бессознательной мощью, способностью вбирать
в себя все остальное многообразие знаний и наук в разумную
форму, схватывать и удерживать их суть, отбрасывать
внешнее и таким образом извлекать из них логическое,
или, что то же самое, наполнять содержанием
всякой истины абстрактную основу логического, ранее
приобретенную посредством изучения, и придавать логическому
ценность такого всеобщего, которое больше уже
не находится как нечто особенное рядом с другим особенным,
а возвышается над всем этим и составляет его
сущность, то, что абсолютно истинно.


%%% Local Variables:
%%% mode: latex
%%% TeX-master: t
%%% End:


\section{Общее деление логики}

Из того, что нами было сказано о \emph{понятии} этой науки
и о том, где оно находит свое обоснование, вытекает, что
общее \emph{деление} может быть здесь лишь \emph{предварительным},
может быть указано как будто лишь постольку, поскольку
автор уже знаком с этой наукой и потому в состоянии
здесь заранее указать \emph{исторически}, к каким
основным различиям определит себя понятие в своем
развитии.

Можно, однако, попытаться заранее объяснить в общем
то, чт\'о требуется для \emph{деления}, хотя и для этого
необходимо прибегать к методу, который приобретает
свою полную ясность и обоснование только в рамках
самой науки.~-- Итак, прежде всего следует напомнить,
что мы здесь исходим из предпосылки, что \emph{деление} должно
находиться в связи с \emph{понятием} или, вернее, заключаться
в нем самом. Понятие не неопределенно, а \emph{определенно}
в самом себе; деление же выражает в \emph{развитом
виде} эту его \emph{определенность}; оно есть его суждение\endnotemark{},
не суждение \emph{о} каком-нибудь внешне взятом предмете,
а процесс суждения, т.\,е. \emph{процесс определения} понятия
в нем же самом. Прямоугольность, остроугольность и т.\,д.,
так же как и равносторонность и т.\,д., по каковым определениям
делят треугольники, заключаются не в определенности
самого треугольника, т.\,е. не в том, чт\'о обычно
называют понятием треугольника, точно так же как те
определения, по которым животных делят на млекопитающих,
птиц и т.\,д., а эти классы~-- на дальнейшие
роды, заключаются не в том, чт\'о принимают за понятие
животного вообще или за понятие млекопитающего, птицы
и т.\,д. Такие определения берутся из другого источника,
из эмпирического созерцания; они примыкают
к упомянутым выше так называемым понятиям извне.
В философской же трактовке деления само понятие должно
показать себя содержащим источник этого деления.

\endnotetext{
  Einteilung~-- деление. Urteil~-- суждение (Ur-teil~-- буквально
  <<пра-часть>>). Об этимологии Urteil см.: <<Энциклопедия философских
  наук>> \S\,166. \emph{Гегель}. Соч., т.\,1. М., 1929, стр.\,273.
}

Но само понятие логики показано во введении как
результат науки, лежащей по ту сторону ее, и, стало
быть, принимается здесь равным образом как \emph{предпосылка}.
Логика согласно этому определилась как наука
чистого мышления, имеющая своим принципом \emph{чистое
знание}, не абстрактное, а конкретное, живое единство,
полученное благодаря тому, что противоположность между
сознанием о некоем субъективно \emph{для себя сущем} и
сознанием о некоем втором таком же \emph{сущем}~-- о некоем
объективном,~-- знают как преодоленную в этом единстве,
знают бытие как чистое понятие в самом себе, а чистое
понятие~-- как истинное бытие. Следовательно, это те два
\emph{момента}, которые содержатся в логическом. Но их теперь
знают как существующие \emph{нераздельно}, а не~-- в отличие
от сознания~-- как \emph{существующие} каждое \emph{также} и \emph{само
по себе}. Только благодаря тому, что их в то же время
знают как \emph{отличные друг от друга} (однако не как сущие
сами по себе), их единство не абстрактно, мертвенно, неподвижно,
а конкретно.

Это единство составляет логический принцип также
и в качестве \emph{стихии}, так что развитие указанного выше
различия, которое сразу же имеется в ней, совершается
только \emph{внутри} этой стихии. В самом деле, так как деление
(Einteilung), как было сказано, есть \emph{суждение}
(Urteil) понятия, полагание уже имманентного ему определения
и, стало быть, его различия, то это полагание
не должно пониматься как новое разложение указанного
конкретного единства на его определения, которые
должны были бы считаться существующими сами по
себе, ибо это было бы здесь бесполезным возвращением
к прежней точке зрения, к противоположности сознания.
Противоположность эта скорее уже преодолена; указанное
единство остается стихией [логического], и из него
уже больше не выходит различение, осуществляемое делением
и вообще развитием. Тем самым определения,
которые прежде (на \emph{пути к истине}), с какой бы точки
зрения их ни определяли, были для себя \emph{сущими}, как,
например, некое субъективное и некое объективное, или
же мышление и бытие, или понятие и реальность, \emph{теперь
в их истине}, т.\,е. в их единстве, низведены на степень
\emph{форм}. Сами они поэтому в своем различии остаются
\emph{в себе} всем понятием в целом, и последнее полагается в делении
только под своими собственными определениями.

Таким образом, все понятие в целом должно рассматриваться,
во-первых, как \emph{сущее} понятие и, во-вторых,
как \emph{понятие}; в первом случае оно \emph{есть} только понятие
\emph{в себе}, понятие реальности или бытия; во втором случае
оно есть понятие как таковое, \emph{для себя сущее} понятие
(каково оно~-- назовем конкретные формы~-- в мыслящем
человеке, но также, хотя и не как \emph{сознаваемое}, а тем
более не как понятие, которое \emph{знают}, в ощущающем животном
и в органической индивидуальности вообще; понятием
же \emph{в себе} оно бывает лишь в неорганической
природе).~-- Согласно этому, логику следовало бы прежде
всего делить на логику \emph{понятия} как \emph{бытия} и понятия
\emph{как понятия}, или, пользуясь обычными, хотя и самыми
неопределенными, а потому и самыми многозначными
выражениями, на \emph{объективную} и \emph{субъективную} логику.

Сообразно же с лежащей в основе стихией единства
понятия в самом себе и, следовательно, нераздельности
его определений, последние, поскольку они \emph{различны},
поскольку понятие полагается в их \emph{различии}, должны
также находиться по крайней мере в \emph{соотношении} друг
с другом. Отсюда получается некая сфера \emph{опосредствования},
понятие как система \emph{рефлективных определений},
т.\,е. как система бытия, переходящего во \emph{внутри-себя}-бытие
понятия, понятие, которое, таким образом, еще не
положено, \emph{как таковое}, для себя, а обременено в то же
время непосредственным бытием как чем-то также внешним
ему. Это~-- \emph{учение о сущности}, находящееся посредине
между учением о бытии и учением о понятии.~--
В общем делении нашего логического произведения оно
помещено еще в \emph{объективной} логике, поскольку, хотя
сущность и есть уже внутреннее, но характер \emph{субъекта}
следует непременно сохранить за понятием.

В новейшее время Кант\footnote{Я напоминаю, что в настоящем сочинении я потому так
  часто принимаю в соображение кантовскую философию (некоторым
  это может казаться излишним), что, как бы ни рассматривали
  другие, а также и мы в настоящем сочинении ее более конкретные
  определения и отдельные части изложения, она составляет
  основу и исходный пункт новейшей немецкой философии,
  и эту ее заслугу не могут умалить имеющиеся в ней недостатки.
  Ее следует часто принимать во внимание в объективной логике
  также и потому, что она подвергает тщательному рассмотрению
  важные, \emph{более определенные} стороны логического, между тем как
  позднейшие изложения философии уделяли ему мало внимания
  и нередко только выказывали по отношению к нему грубое, но
  не оставшееся без возмездия, презрение. То философствование,
  которое у нас более всего распространено, \emph{не идет} дальше кантовских
  выводов, согласно которым разум не способен познать
  никакого истинного содержания и в отношении абсолютной истины
  следует отсылать к вере. Но это философствование непосредственно
  начинает с того, что у Канта составляет вывод, и этим
  сразу отбрасывает предшествующие построения, из которых вытекает
  указанный вывод и которые составляют философское познание.
  Кантовская философия служит, таким образом, подушкой
  для лености мысли, успокаивающейся на том, что все уже доказано
  и решено. За познанием и определенным содержанием мышления,
  которых не найти в таком бесплодном и мертвенном
  (trockenen) успокоении, следует поэтому обращаться к указанным
  предшествующим построениям.} противопоставил тому, чт\'о
обычно называлось логикой, еще одну, а именно \emph{трансцендентальную
логику}. То, чт\'о мы здесь назвали \emph{объективной}
логикой, отчасти соответствовало бы тому, чт\'о
у него составляет \emph{трансцендентальную логику}. Он указывает
следующие различия между ней и тем, чт\'о он
называет общей логикой: трансцендентальная логика ($\alpha$)
рассматривает те понятия, которые относятся к \emph{предметам}
a priori и, следовательно, не абстрагируется от всякого
\emph{содержания} объективного познания, или, как он это
выражает иначе, она заключает в себе правила чистого
мышления о каком бы то ни было \emph{предмете} и ($\beta$) в то
же время исследует происхождение нашего познания,
поскольку познание нельзя приписать предметам. Исключительно
на эту вторую сторону направлен философский
интерес Канта. Основная его мысль~-- это то, что \emph{категории}
следует признать чем-то принадлежащим самосознанию,
как \emph{субъективному} <<Я>>. В силу этого определения
воззрение [Канта] не выходит за пределы сознания
и его противоположности и кроме эмпирической стороны
чувства и созерцания имеет еще нечто такое, что не положено
мыслящим самосознанием и не определено им,~--
\emph{вещь в себе}, нечто чуждое и внешнее мышлению, хотя
нетрудно усмотреть, что такого рода абстракция, как
\emph{вещь в себе}, сама есть лишь продукт мышления и притом
только абстрагирующего мышления. Если другие
кантианцы\endnotemark{} выразились об определении \emph{предмета} через
<<Я>> в том смысле, что объективирование этого <<Я>> следует
рассматривать как некую первоначальную и необходимую
деятельность сознания, так что в этой первоначальной
деятельности еще нет представления о самом
<<Я>>, каковое представление есть только некое сознание
указанного сознания или даже объективирование этого сознания,
то эта объективирующая деятельность, освобожденная
от противоположности сознания, оказывается при
более тщательном рассмотрении тем, чт\'о можно считать
вообще \emph{мышлением}, как таковым\footnote{Если выражение
  <<объективирующая деятельность>> <<Я>> может напомнить о
  других продуктах духа, например о продуктах \emph{фантазии},
  то следует отметить, что речь идет о том, как определяют
  предмет, поскольку его содержательные моменты \emph{не} принадлежат
  области \emph{чувства} и \emph{созерцания}. Такой предмет есть некая
  \emph{мысль}, и определить его означает отчасти впервые его продуцировать,
  отчасти же, поскольку он нечто предположенное, иметь
  о нем еще другие мысли, мысленно развивать его дальше.}. Но эта деятельность
не должна была бы больше называться сознанием; сознание
заключает в себе противоположность <<Я>> и его
предмета, а этой противоположности нет в указанной
первоначальной деятельности. Название <<сознание>> набрасывает
тень субъективности на эту деятельность еще
больше, чем выражение <<мышление>>, которое, однако,
следует здесь понимать вообще в абсолютном смысле как
мышление \emph{бесконечное}, не обремененное конечностью сознания,
короче говоря, под этим выражением следует
понимать \emph{мышление, как таковое}.

\endnotetext{
  Имеются в виду Фихте и его единомышленники.
}

Так как интерес кантовской философии был направлен
на так называемое \emph{трансцендентальное} в определениях
мысли, то рассмотрение самих этих определений не
привело к содержательным заключениям. Вопрос о том,
чт\'о они такое сами в себе, помимо их абстрактного, одинакового
у всех них отношения к <<Я>>, каковы их определенность
в сравнении друг с другом и их отношение
друг к другу, не был у Канта предметом рассмотрения;
поэтому указанная философия нисколько не способствовала
познанию их природы. Единственно интересное,
имеющее отношение к этому вопросу, мы находим в критике
идей. Но для действительного прогресса философии
было необходимо, чтобы интерес мышления был привлечен
к рассмотрению формальной стороны, <<Я>>, сознания,
как такового, т.\,е. абстрактного отношения некоего субъективного
знания к некоему объекту, чтобы таким образом
было начато познание \emph{бесконечной формы}, т.\,е. понятия.
Однако, чтобы достигнуть этого познания, нужно
было еще отбросить упомянутую выше конечную определенность,
в которой форма представлена как <<Я>>, сознание.
Форма, мысленно извлеченная таким образом
в своей чистоте, содержит в самой себе процесс \emph{определения}
себя, т.\,е. сообщения себе содержания и притом сообщения
себе содержания в его необходимости~-- в виде
системы определений мысли.

Объективная логика, таким образом, занимает скорее
место прежней \emph{метафизики}, каковая была высившимся
над миром научным зданием, которое должно было быть
воздвигнуто только \emph{мыслями}.~-- Если примем во внимание
последнюю форму развития этой науки\endnotemark{}, то мы
должны сказать, во-первых, что объективная логика непосредственно
занимает место \emph{онтологии}~-- той части указанной
метафизики, которая должна была исследовать
природу ens [сущего] вообще; <<ens>> охватывает как \emph{бытие},
\emph{так} и \emph{сущность}, для различения которых немецкий
язык, к счастью, сохранил разные выражения.~-- Но тогда
объективная логика постольку охватывает и остальные
части метафизики, поскольку метафизика стремилась
постигнуть посредством чистых форм мысли особенные
субстраты, заимствованные ею первоначально из [области]
представления,~-- душу, мир, бога,~-- и поскольку
\emph{определения мышления} составляли \emph{суть} ее способа рассмотрения.
Но логика рассматривает эти формы вне связи
с указанными субстратами, с субъектами \emph{представления},
рассматривает их природу и ценность, взятые
сами по себе. Указанная метафизика не сделала этого
и навлекла на себя справедливый упрек в том, что она
пользовалась ими \emph{без критики}, без предварительного исследования,
способны ли они и как они способны быть,
по выражению Канта, определениями вещи в себе или,
вернее сказать, разумного.~-- Объективная логика есть
поэтому подлинная критика их, критика, рассматривающая
их не сообразно абстрактной форме априорности,
противопоставляя ее апостериорному, а их самих в их
особом содержании.

\endnotetext{
  Гегель имеет в виду метафизику X. Вольфа и его последователей.
}

\emph{Субъективная логика}~--- это логика \emph{понятия}~-- сущности,
которая сняла свое отношение к некоторому бытию
или, иначе говоря, к своей видимости и которая теперь
уже не внешняя в своем определении, а есть свободное,
самостоятельное, определяющее себя внутри себя
субъективное, или, вернее, есть сам \emph{субъект}.~-- Так как
выражение <<субъективное>> приводит к недоразумениям,
поскольку оно может быть понято в смысле чего-то случайного
и произвольного, равно как вообще в смысле
определений, относящихся к форме \emph{сознания}, то не следует
здесь придавать особое значение различию между
субъективным и объективным, которое позднее будет более
подробно разъяснено при изложении самой логики.

Логика, следовательно, хотя и распадается вообще на
\emph{объективную} и \emph{субъективную} логику, все же имеет, точнее,
следующие три части:
\begin{enumerate}[label=\Roman*.]
\item \emph{Логику бытия},
\item \emph{Логику сущности} и
\item \emph{Логику понятия}.
\end{enumerate}

%%% Local Variables:
%%% mode: latex
%%% TeX-master: "../../main"
%%% TeX-engine: xetex
%%% End:


\book{Книга первая. Учение о бытии}

\section{С чего следует начинать науку?}

Только в новейшее время зародилось сознание, что
трудно найти \emph{начало} в философии, и причина этой трудности,
равно как и возможность устранить ее были предметом
многократного обсуждения. Начало философии
должно быть или чем-то \emph{опосредствованным} или чем-то
\emph{непосредственным}; и легко показать, что оно не может
быть ни тем, ни другим; стало быть, и тот и другой способ
начинать находит свое опровержение.

Правда, \emph{принцип} какой-нибудь философии также
означает некое начало, но не столько субъективное,
сколько \emph{объективное} начало, начало \emph{всех вещей}. Принцип
есть некое определенное содержание~"--- вода, единое,
нус, идея, субстанция, монада\endnotemark{} и т.\,д.; или, если он касается
природы познавания и, следовательно, должен
быть скорее лишь неким критерием, чем неким объективным
определением~"--- мышлением, созерцанием, ощущением,
Я, самой субъективностью,~"--- то и здесь интерес
направлен на определение содержания. Вопрос же о начале,
как таковом, оставляется без внимания и считается
безразличным как нечто субъективное в том смысле, что
дело идет о случайном способе начинать изложение, стало
быть, и потребность найти то, с чего следует начинать,
представляется незначительной по сравнению с потребностью
найти принцип, ибо кажется, что единственно
лишь в нем заключается \emph{главный} интерес, интерес
к тому, чт\'о такое \emph{истина}, чт\'о такое \emph{абсолютное основание}
всего.

\endnotetext{
  Имеются в виду философские учения Фалеса (вода), Парменида
  (единое, или одно), Анаксагора (нус), Платона (идея), Спинозы
  (субстанция), Лейбница (монада).
}

Но нынешнее затруднение с началом проистекает из
более широкой потребности, еще незнакомой тем, для
кого важно догматическое доказательство своего принципа
или скептические поиски субъективного критерия
для опровержения догматического философствования, и
совершенно отрицаемой теми, кто, как бы выпаливая из
пистолета\endnotemark{}, прямо начинает с своего внутреннего откровения,
с веры, интеллектуального созерцания и т.\,д. и
хочет отделаться от \emph{метода} и логики. Если прежнее абстрактное
мышление сначала интересуется только принципом
как \emph{содержанием}, в дальнейшем же развитии вынуждено
обратить внимание и на другую сторону, на способы
\emph{познавания}, то [теперешнее мышление] понимает
также и \emph{субъективную} деятельность как существенный
момент объективной истины, и возникает потребность в
соединении метода с содержанием, \emph{формы с принципом}.
Таким образом, \emph{принцип} должен быть также началом,
а то, чт\'о представляет собой prius для мышления,~"--- \emph{первым}
в \emph{движении} мышления.

\endnotetext{
  Ср. <<Феноменология духа>>: <<\dots и тем вдохновением, которое
  начинает сразу же, как бы выстрелом из пистолета, с абсолютного знания>> (стр.\,14).
}

Здесь мы должны только рассмотреть, как выступает
\emph{логическое} начало. Мы уже указали, что его можно понимать
двояко~"--- как результат, полученный опосредствованно,
или как подлинное начало, взятое непосредственно.
Вопрос, представляющийся столь важным для
нынешней культуры, есть ли знание истины непосредственное,
всецело зачинающее знание, некая вера или
же опосредствованное знание,~"--- этот вопрос не должен
здесь обсуждаться. Поскольку его можно рассматривать
\emph{предварительно}, мы это сделали в другом месте (в моей
<<Энциклопедии философских наук>>, изд. 3-е, <<Предварительное
понятие>>, \S\,61 и сл.). Здесь мы приведем оттуда
лишь следующее замечание: \emph{нет} ничего ни на небе,
ни в природе, ни в духе, ни где бы то ни было, чт\'о не
содержало бы в такой же мере непосредственность, в какой
и опосредствование, так что эти два определения
оказываются \emph{нераздельными} и \emph{неразделимыми}, а указанная
противоположность [между ними]~"--- чем-то ничтожным.
Что же касается \emph{научного рассмотрения}, то в каждом
логическом предложении мы встречаем определения
непосредственности и опосредствования и, следовательно,
рассмотрение их противоположности и их истины. Поскольку
в отношении мышления, знания, познавания эта
противоположность получает более конкретный вид непосредственного
или опосредствованного \emph{знания}, постольку
природа познавания вообще рассматривается в рамках
науки логики, а познание в его дальнейшей конкретной
форме~"--- в науке о духе и феноменологии духа.
Но желать еще \emph{до} науки получить полную ясность относительно
познавания~"--- значит требовать, чтобы оно
рассматривалось \emph{вне} науки; во всяком случае научно
нельзя это сделать \emph{вне} науки, а здесь мы стремимся единственно
лишь к научности.

Начало есть \emph{логическое} начало, поскольку оно должно
быть сделано в стихии свободно для себя сущего мышления,
в \emph{чистом знании}. \emph{Опосредствовано} оно, стало быть,
тем, что чистое знание есть последняя, абсолютная
истина \emph{сознания}. Мы отметили во введении, что \emph{феноменология
духа} есть наука о сознании, изображение
того, что сознание имеет своим результатом \emph{понятие}
науки, т.\,е. чистое знание. Постольку логика имеет своей
предпосылкой науку об охватывающем явления духе,
содержащую и показывающую необходимость точки зрения,
представляющей собой чистое знание, равно как и
его опосредствование вообще, и тем самым дающую доказательство
ее истинности. В этой науке о духе, охватывающем
явления, исходят из эмпирического, \emph{чувственного}
сознания, которое и есть настоящее, \emph{непосредственное}
знание; там же разъясняется, чт\'о верного в этом
непосредственном знании. Другое сознание, как, например,
вера в божественные истины, внутренний опыт, знание
через внутреннее откровение и т.\,д., оказывается
после небольшого размышления очень неподходящим
для того, чтобы его приводить в качестве примера непосредственного
знания. В феноменологии духа непосредственное
сознание есть первое и непосредственное также
и в науке, и, стало быть, служит предпосылкой; в логике
же предпосылкой служит то, чт\'о оказалось результатом
указанного исследования,~"--- идея как чистое знание. \emph{Логика
есть чистая наука}, т.\,е. чистое знание во всем объеме
своего развития. Но эта идея определилась в указанном
результате как достоверность, ставшая истиной, достоверность,
которая, с одной стороны, уже больше не
противостоит предмету, а вобрала его внутрь себя, знает
его в качестве самой себя\endnote{Ср. <<Феноменология духа>>, стр.\,427--428.}
и которая, с другой стороны,
отказалась от знания о себе как о чем-то таком, чт\'о противостоит
предметному и чт\'о есть лишь его уничтожение,
отчуждена от этой субъективности и есть единство со
своим отчуждением\endnote{
  Ср. <<Феноменология духа>>: <<Преодоление предмета сознания
  следует понимать не как одностороннее в том смысле, что он
  оказался возвращающимся в самость, а определеннее~"--- в том
  смысле, что предмет, как таковой, представляется сознанию исчезающим,
  и кроме того еще, что именно отрешение самосознания
  устанавливает вещность и что это отрешение имеет не только
  негативное, но и положительное значение>> (стр.\,422).
}.

Для того чтобы, исходя из этого определения чистого
знания, начало оставалось имманентным науке о чистом
знании, не надо делать ничего другого, как рассматривать
или, вернее, отстранив всякие размышления, всякие
мнения, которых придерживаются вне этой науки, лишь
воспринимать то, \emph{чт\'о имеется налицо}.

Чистое знание как \emph{слившееся} в это \emph{единство}, сняло
всякое отношение к другому и к опосредствованию; оно
есть то, чт\'о лишено различий; это лишенное различий,
следовательно, само перестает быть знанием; теперь
имеется только \emph{простая непосредственность}.

<<Простая непосредственность>> сама есть выражение
рефлексии и имеет в виду отличие от опосредствованного.
В своем истинном выражении простая непосредственность
есть поэтому \emph{чистое бытие}. Подобно тому как \emph{чистое}
знание не должно означать ничего другого, кроме
знания, как такового, взятого совершенно абстрактно, так
и чистое бытие не должно означать ничего другого, кроме
\emph{бытия} вообще; \emph{бытие}~"--- и ничего больше, бытие без
всякого дальнейшего определения и наполнения.

Здесь бытие~"--- начало, возникшее через опосредствование
и притом через опосредствование, которое есть
в то же время снимание самого себя; при этом предполагается,
что чистое знание есть результат конечного знания,
сознания. Но если не делать никакого предположения,
а само начало брать \emph{непосредственно}, то начало будет
определяться только тем, что оно есть начало логики,
мышления, взятого само по себе. Имеется лишь решение,
которое можно рассматривать и как произвол,
а именно решение рассматривать \emph{мышление, как таковое}.
Таким образом, начало должно быть \emph{абсолютным}, или, чт\'о
здесь то же самое, абстрактным, началом; оно, таким
образом, \emph{ничего не} должно \emph{предполагать}, ничем не должно
быть опосредствовано и не должно иметь какое-либо
основание; оно само, наоборот, должно быть основанием
всей науки. Оно поэтому должно быть \emph{чем-то} (ein) всецело
непосредственным или, вернее, лишь \emph{самим} (das)
\emph{непосредственным}. Как оно не может иметь какое-либо
определение по отношению к иному, так оно не может
иметь какое-либо определение внутри себя, какое-либо
содержание, ибо содержание было бы различением и соотнесением
разного, было бы, следовательно, неким опосредствованием.
Итак, начало~"--- \emph{чистое бытие}.

Изложив то, что прежде всего относится лишь к самому
этому наипростейшему, логическому началу, можно
привести еще и другие соображения. Однако они не
столько могут служить разъяснением и подтверждением
данного выше простого изложения (которое само по себе
закончено), сколько вызываются лишь представлениями
и соображениями, которые могут нам мешать еще до
того, как приступим к делу, но с которыми, как и со
всеми другими предрассудками, предшествующими [изучению
науки], должно быть покончено в самой науке, и
поэтому, собственно говоря, здесь следовало бы, указывая
на это, лишь призвать [читателя] к терпению.

Понимание того, что абсолютно истинное есть, несомненно,
результат и что, наоборот, всякий результат
предполагает некое первое истинное, которое, однако,
именно потому, что оно есть первое, не необходимо, если
рассматривать его объективно, и которое с субъективной
стороны не познано,~"--- это понимание привело в новейшее
время к мысли, что философия должна начинать
лишь с чего-то \emph{гипотетически} и \emph{проблематически} истинного
и что поэтому философствование может быть сначала
лишь исканием. Этот взгляд Рейнгольд многократно
отстаивал в последние годы своего философствования, и
необходимо отдать справедливость этому взгляду и признать,
что в его основе лежит истинный интерес к спекулятивной
природе философского \emph{начала}. Разбор этого
взгляда дает в то же время повод предварительно разъяснять
смысл логического развития вообще, ибо указанный
взгляд с самого начала принимает во внимание это
движение вперед. И притом этот взгляд представляет
себе развитие так, что в философии движение вперед
есть скорее возвращение назад и обоснование, только благодаря
которому и делается вывод, что то, с чего начали,
есть не просто принятое произвольно, а в самом деле
есть отчасти \emph{истинное}, отчасти \emph{первое истинное}.

Нужно признать весьма важной мысль (более определенной
она будет в самой логике), что движение вперед
есть \emph{возвращение назад} в \emph{основание}, к \emph{первоначальному}
и \emph{истинному}, от которого зависит то, с чего начинают,
и которое на деле порождает начало.~"--- Так, сознание
на своем пути от непосредственности, которой оно
начинает, приводится обратно к абсолютному знанию
как к своей внутренней \emph{истине}. Это \emph{последнее}, основание,
и есть то, из чего происходит первое, выступившее
сначала как непосредственное.~"--- Так, в еще большей
мере, абсолютный дух, оказывающийся конкретной и последней
высшей истиной всякого бытия, познается как
свободно отчуждающий себя в \emph{конце} развития и отпускающий
себя, чтобы принять образ \emph{непосредственного}
бытия, познается как решающийся сотворить мир, в котором
содержится все то, чт\'о заключалось в развитии,
предшествовавшем этому результату, и чт\'о благодаря
этому обратному положению превращается вместе со
своим началом в нечто зависящее от результата как от
принципа. Главное для науки не столько то, что началом
служит нечто исключительно непосредственное, а то, что
вся наука в целом есть в самом себе круговорот, в котором
первое становится также и последним, а последнее~"---
также и первым.

Поэтому оказывается, с другой стороны, столь же необходимым
рассматривать как \emph{результат} то, во что движение
возвращается как в свое \emph{основание}. С этой точки
зрения первое есть также и основание, а последнее нечто
производное; так как исходят из первого и с помощью
правильных заключений приходят к последнему как
к основанию, то это основание есть результат. Далее,
\emph{поступательное движение} от того, чт\'о составляет начало,
следует рассматривать как дальнейшее его определение,
так что начало продолжает лежать в основе всего последующего
и не исчезает из него. Движение вперед состоит
не в том, что выводится лишь нечто \emph{иное} или совершается
переход в нечто истинно иное, а, поскольку такой переход
имеет место, он снова снимает себя. Таким образом,
начало философии есть наличная и сохраняющаяся
на всех последующих этапах развития основа, есть то,
чт\'о остается всецело имманентным своим дальнейшим
определениям.

Благодаря именно такому движению вперед начало
утрачивает все одностороннее, которое оно имеет в этой
определенности, заключающейся в том, что оно есть нечто
непосредственное и абстрактное вообще; оно становится
чем-то опосредствованным, и линия продвижения
науки тем самым превращается \emph{в круг}. В то же время
оказывается, что то, чт\'о составляет начало, будучи еще
неразвитым, бессодержательным, по-настоящему еще не
познается в начале и что лишь наука, и притом во всем
ее развитии, есть завершенное, содержательное и теперь
только истинно обоснованное познание его.

Но то обстоятельство, что только \emph{результат} оказывается
абсолютным основанием, не означает, что поступательное
движение этого познавания есть нечто предварительное
или проблематическое и гипотетическое движение.
Это движение познавания должно определяться
природой вещей и самого содержания. Указанное выше
начало не есть ни нечто произвольное и принятое лишь
временно, ни нечто предположенное как появляющееся
произвольно и в результате просьбы, относительно чего
впоследствии все же оказывается, что поступили правильно,
сделав его началом. Здесь дело обстоит не так,
как в тех построениях, которые приходится делать для
доказательства геометрической теоремы: что касается таких
построений, то после того, как приведены доказательства,
выясняется, что мы хорошо сделали, что провели
именно эти линии и что затем в самом доказательстве
начали со сравнения этих линий или углов между
собой: от самого проведения этих линий или от сравнения
их между собой это не ясно. Таким образом, в сам\'ой
чистой науке дано \emph{основание} того, что в ней начинают
с чистого бытия. Это чистое бытие есть то единство,
в которое возвращается чистое знание, или же, если еще
считать чистое знание как форму отличным от его единства,
то чистое бытие есть также его содержание. Именно
в этом отношении \emph{чистое бытие}, это абсолютно непосредственное
есть также и абсолютно опосредствованное. Но
столь же существенно, чтобы оно было взято только
в своей односторонности как чисто непосредственное
\emph{именно потому}, что оно здесь берется как начало. Поскольку
оно не было бы этой чистой неопределенностью,
поскольку оно было бы определенным, мы бы его брали
как опосредствованное, уже развитое далее; всякое определенное
содержит некое \emph{иное}, присоединяющееся к чему-то
первому. Следовательно, \emph{природа самог\'о начала} требует,
чтобы оно было бытием и больше ничем. Бытие
поэтому не нуждается для своего вхождения в философию
ни в каких других приготовлениях, ни в каких посторонних
размышлениях или исходных пунктах.

Из того, что начало есть начало философии, также
нельзя, собственно говоря, почерпать какое-либо \emph{более
точное} его \emph{определение} или какое-либо \emph{положительное}
содержание для этого начала. Ибо здесь в сам\'ом начале,
где еще нет самой сути, философия есть пустое слово или
какое-то принятое [как предпосылка] необоснованное
представление. Чистое знание дает лишь следующее отрицательное
определение: начало должно быть \emph{абстрактным}
началом. Поскольку чистое бытие берется как \emph{содержание}
чистого знания, последнее должно отступить
от своего содержания, дать ему действовать самостоятельно
и больше не определять его.~"--- Иначе говоря, так как
чистое бытие следует рассматривать как единство, в котором
знание, достигнув своей высшей точки единения
с объектом, совпадает с ним, то знание исчезло в этом
единстве, ничем не отличается от него и, следовательно,
не оставило для него никакого определения. Да и вне
этого [знания] нет никакого нечто или содержания, которым
можно было бы пользоваться, чтобы, начав с него,
иметь его в качестве более определенного начала.

Но и определение \emph{бытия}, принятое ранее в качестве
начала, можно было бы опустить, так что оставалось бы
лишь требование~"--- иметь некоторое чистое начало. В таком
случае не было бы ничего другого, кроме самог\'о
\emph{начала}, и нам следовало бы посмотреть, чт\'о оно такое.~"---
Эту позицию можно было бы в то же время милостиво
предложить тем, кто, с одной стороны, по каким-то соображениям
недоволен, что начинают с бытия, и еще более
недоволен результатом, к которому приходит это
бытие,~"--- переходом бытия в ничто, а с другой стороны,
вообще не желает знать о каком-либо другом начале
науки, кроме некоего \emph{представления} как \emph{предпосылки}~"---
представления, которое затем \emph{анализируется}, так что результат
такого анализа служит первым определенным
понятием в науке. Также и при этом способе действия
мы не имели бы никакого особого предмета, потому что
начало как начало \emph{мышления} должно быть совершенно
абстрактным, совершенно всеобщим, должно быть просто
формой без всякого содержания; у нас, таким образом,
не было бы ничего другого, кроме представления только
о начале, как таковом. Нам, стало быть, следует лишь
посмотреть, чт\'о мы имеем в этом представлении.

Пока что есть ничто, и должно возникнуть нечто. Начало
есть не чистое ничто, а такое ничто, из которого
должно произойти нечто; бытие, стало быть, уже содержится
и в начале. Начало, следовательно, содержит и то
и другое, бытие и ничто; оно единство бытия и ничто,
иначе говоря, оно небытие, которое есть в то же время
бытие, и бытие, которое есть в то же время небытие.

Далее, бытие и ничто имеются в начале как \emph{различные},
ибо начало указывает на нечто иное; оно небытие,
соотнесенное с бытием как с чем-то иным; начала еще
\emph{нет}, оно лишь направляется к бытию. Следовательно, начало
содержит бытие как такое бытие, которое отдаляется
от небытия, иначе говоря, снимает его как нечто противоположное
ему.

Но, далее, то, чт\'о начинается, уже \emph{есть}, но в такой
же мере его еще и \emph{нет}. Следовательно, противоположности,
бытие и небытие, находятся в нем в непосредственном
соединении, иначе говоря, начало есть их \emph{неразличенное
единство}.

Стало быть, анализ начала дал бы нам понятие единства
бытия и небытия или, выражая это в более рефлектированной
форме, понятие единства различенности и
неразличенности, или, иначе, понятие тождества тождества
и нетождества\endnotemark{}. Это понятие можно было бы рассматривать
как первую, самую чистую, т.\,е. самую абстрактную
дефиницию абсолютного, и оно в самом деле
было бы таковой, если бы дело шло вообще о форме дефиниций
и о наименовании абсолютного. В этом смысле
указанное абстрактное понятие было бы первой дефиницией
этого абсолютного, а все дальнейшие определения
лишь его более определенными и богатыми дефинициями.
Но пусть те, кто потому недоволен \emph{бытием} как началом,
что оно переходит в ничто и что из этого возникает единство
бытия и ничто, подумают, будут ли они более довольны
таким началом, которое начинается с представления
о \emph{начале}, и анализом этого представления, который,
конечно, правилен, но точно так же приводит
к единству бытия и ничто,~"--- пусть подумают, будут ли
они более довольны этим, нежели тем, что в качестве
начала берется бытие.

\endnotetext{
  <<Тождество тождества и нетождества>>~"--- очень характерное
  для Гегеля выражение, которое подчеркивает (в отличие от трактовки
  Шеллингом тождества противоположностей как непосредственного)
  то, что в тождестве противоположных определений
  в снятом виде сохраняется и различие между ними. Это выражение
  встречается уже в ранней работе Гегеля <<Различие между
  философскими системами Фихте и Шеллинга>> (1801) (Hegels
  Werke. Hrsg. von Lasson. Bd. 1, S. 77).
}

Но необходимо сделать еще одно замечание об этом
способе рассмотрения. Указанный анализ предполагает,
что представление о начале известно; таким образом мы
поступили здесь по примеру других наук. Эти другие
науки предполагают существование своего предмета и
предлагают признавать, что каждый имеет о нем одно и
то же представление и может найти в нем приблизительно
те же определения, которые они то тут, то там приводят
и указывают посредством анализа, сравнения и прочих
рассуждений о нем. Но то, чт\'о представляет собой абсолютное
начало, также должно быть чем-то ранее известным;
если оно есть конкретное и, следовательно, многообразно
определенное внутри себя, то это \emph{соотношение},
которое оно есть внутри себя, предполагается чем-то
известным; оно, следовательно, выдается за нечто \emph{непосредственное,
но на самом деле оно не есть таковое}, ибо
оно лишь соотношение различенных [моментов], стало
быть, содержит \emph{опосредствование}. Далее, в конкретном
появляются случайность и произвольность анализа и разных
способов определения. Какие в конце концов получатся
определения, это зависит от того, чт\'о каждый \emph{находит}
уже наличным в своем непосредственном случайном
представлении. Содержащееся в некоем конкретном,
в некоем синтетическом единстве соотношение есть \emph{необходимое}
соотношение лишь постольку, поскольку оно
заранее не находится, а порождено собственным движением
моментов, которое возвращает их в это единство,
движением, представляющим собой противоположность
аналитическому способу рассмотрения, действованию,
внешнему самой вещи, совершающемуся в субъекте.

Это влечет за собой также и следующий, более определенный
вывод: то, с чего следует начинать, не может
быть чем-то конкретным, чем-то таким, чт\'о содержит некое
соотношение \emph{внутри самого себя}. Ибо такое предполагает,
что внутри него имеется некое опосредствование
и переход от некоего первого к некоему другому,
результатом чего было бы конкретное, ставшее простым.
Но начало не должно само уже быть неким первым \emph{и}
неким иным; в том, что есть внутри себя некоторое первое
и некоторое иное, уже содержится совершившееся
продвижение (Fortgegangensein). То, с чего начинают,
само начало, д\'олжно поэтому брать как нечто неподдающееся
анализу, д\'олжно брать в его простой, ненаполненной
непосредственности, следовательно, \emph{как бытие}, как
то, чт\'о совершенно пусто.

Если кто-то выведенный из терпения рассматриванием
абстрактного начала скажет, что нужно начинать не
с начала, а прямо с самой \emph{сути}, то [мы на это ответим],
что суть эта не что иное, как указанное пустое бытие,
ибо, чт\'о такое суть, это должно выясниться именно только
в ходе самой науки и не может предполагаться известным
до нее.

Какую бы иную форму мы ни брали, чтобы получить
другое начало, нежели пустое бытие, это другое начало
все равно будет страдать указанным недостатком. Тем,
кто остается недовольным этим началом, мы предлагаем
самим взяться за решение этой задачи: пусть попробуют
начинать как-нибудь иначе, чтобы при этом избежать
этих недостатков.

Но нельзя совсем не упомянуть об оригинальном начале
философии, приобретшем большую известность в новейшее
время, о начале с <<Я>>\endnote{Имеется в виду философия Фихте.}.
Оно получилось отчасти
на основании того соображения, что из первого истинного
должно быть выведено всё дальнейшее, а отчасти из
потребности, чтобы \emph{первое} истинное было чем-то известным
и, более того, чем-то \emph{непосредственно достоверным}.
Это начало, вообще говоря, не случайное представление,
которое у одного субъекта может быть таким-то, а у другого
иным. В самом деле, <<Я>>, это непосредственное самосознание,
прежде всего само проявляется отчасти как
нечто непосредственное, отчасти как нечто в гораздо более
высоком смысле известное, чем какое-либо иное представление.
Все иное известное, хотя и принадлежит
к <<Я>>, однако еще есть содержание, отличное от него
и тем самым случайное; <<Я>>, напротив, есть простая достоверность
самого себя. Но <<Я>> вообще есть \emph{в то же
время} и нечто конкретное или, вернее, <<Я>> есть самое
конкретное~"--- сознание себя как бесконечно многообразного
мира. Для того чтобы <<Я>> было началом и основанием
философии, требуется обособление этого конкретного,
требуется тот абсолютный акт, которым <<Я>> очищается
от самого себя и вступает в свое сознание как
абстрактное <<Я>>. Но оказывается, что это чистое <<Я>> \emph{не}
есть ни непосредственное, ни то известное, обыденное
<<Я>> нашего сознания, из которого непосредственно и для
каждого человека должна исходить наука. Этот акт был
бы, собственно говоря, не чем иным, как возвышением
до точки зрения чистого знания, при которой исчезает
различие между субъективным и объективным. Но если
требовать, чтобы это возвышение было столь \emph{непосредственным},
то такое требование будет субъективным постулатом.
Для того, чтобы оно оказалось истинным требованием,
следовало бы показать и представить движение
конкретного <<Я>> в нем самом, по его собственной необходимости,
от непосредственного сознания к чистому знанию.
Без этого объективного движения чистое знание, и
в том случае, когда его определяют как \emph{интеллектуальное
созерцание}, являет себя как произвольная точка зрения,
или даже как одно из эмпирических \emph{состояний} сознания,
относительно которого важно решить, не обстоит
ли дело так, что один человек \emph{находит} или может вызвать
его в себе, а другой~"--- нет. Но так как это чистое
<<Я>> должно быть сущностным чистым знанием, чистое
же знание непосредственно не имеется в индивидуальном
сознании, его лишь полагает в нем абсолютный акт
самовозвышения, то теряется как раз то преимущество,
которое, как утверждают, возникает из этого начала философии,
а именно то, что это начало есть нечто безусловно
известное, чт\'о каждый непосредственно находит
в себе и чт\'о он может сделать исходным пунктом дальнейших
размышлений; в своей абстрактной сущностности
указанное чистое <<Я>> есть скорее нечто неизвестное
обыденному сознанию, нечто такое, чего оно не находит
наличным в себе. Тем самым обнаруживается скорее
вред иллюзии, будто речь идет о чем-то известном, о <<Я>>
эмпирического самосознания, между тем как на самом
деле речь идет о чем-то далеком этому сознанию. Определение
чистого знания как <<Я>> заставляет непрерывно
вспоминать о субъективном <<Я>>, об ограниченности которого
следует забыть, и сохраняет представление, будто
положения и отношения, которые получаются в дальнейшем
развитии <<Я>>, содержатся в обыденном сознании
и будто их можно там найти, ведь именно относительно
него их высказывают. Это смешение порождает
вместо непосредственной ясности скорее лишь еще более
кричащую путаницу и полную дезориентацию, а уж
в умах людей посторонних оно вызывало грубейшие недоразумения.


Что же касается, далее, \emph{субъективной} определенности
<<Я>> вообще, то верно, что чистое знание освобождает
<<Я>> от его ограниченного смысла, заключающегося в том,
что в объекте оно имеет свою непреодолимую противоположность.
Но как раз по этой же причине было бы по
меньшей мере \emph{излишне} сохранять еще эту субъективную
позицию и определение чистой сущности как <<Я>>. Следует,
однако, прибавить, что это определение не только
влечет за собой указанную выше вредную двусмысленность,
но, как оказывается при более пристальном рассмотрении,
оно остается и субъективным <<Я>>. Действительное
развитие науки, которая исходит из <<Я>>, показывает,
что объект имеет и сохраняет в ней постоянное
для <<Я>> определение \emph{иного}, что, следовательно, <<Я>>, из
которого исходят, не есть чистое знание, поистине преодолевшее
противоположность сознания, а еще погружено
в явлении.

При этом необходимо сделать еще следующее важное
замечание: если <<Я>> действительно могло бы быть \emph{в себе}
определено как чистое знание или интеллектуальное
созерцание и признано началом, то ведь для науки главное
не то, чт\'о существует \emph{в себе} или \emph{внутренне}, а наличное
бытие внутреннего \emph{в мышлении} и та \emph{определенность},
которую такое внутреннее имеет в этом наличном бытии.
Но то, чт\'о в \emph{начале} науки \emph{имеется} от интеллектуального
созерцания или~"--- если предмет такого созерцания получает
название вечного, божественного, абсолютного,~"--- от
вечного или абсолютного, может быть только первым, непосредственным,
простым определением. Какое бы ему
ни дали более богатое [содержанием] название, чем то,
которое выражает лишь <<бытие>>, во внимание может
быть принято только то, каким образом такого рода абсолютное
входит в \emph{мыслящее} знание и в словесное выражение
этого знания. Интеллектуальное созерцание есть,
правда, решительный отказ от опосредствования и от доказывающей,
внешней рефлексии. Но то, чт\'о оно выражает
помимо простой непосредственности, есть нечто
конкретное, нечто содержащее в себе разные определения.
Однако выражение и изображение такого конкретного
есть, как мы уже указали, опосредствующее движение,
начинающее с \emph{одного} из определений и переходящее
к другому определению, хотя бы это другое и возвратилось
к первому; это~"--- движение, которое в то же время
не должно быть произвольным или ассерторическим.
Поэтому в таком изображении \emph{начинают} не с самог\'о
конкретного, а только с простого непосредственного, от
которого берет свое начало движение. Кроме того, если
делают началом конкретное, то недостает доказательства,
в котором нуждается соединение определений, содержащихся
в конкретном.

Следовательно, если в выражении <<абсолютное>> или
<<вечное>>, или <<бог>> (а самое бесспорное право имел бы
бог~"--- начинать именно с него), если в созерцании их или
мысли о них \emph{имеется больше содержания}, чем в чистом
бытии, то нужно, чтобы то, чт\'о \emph{содержится} в них, лишь
\emph{проникло} в знание мыслящее, а не представляющее; как
бы ни было богато заключающееся в них содержание,
определение, которое \emph{первым} проникает в знание, есть
нечто простое; ибо лишь в простом нет ничего более,
кроме чистого начала; только непосредственное просто,
ибо лишь в непосредственном нет еще перехода от одного
к другому. Итак, что бы ни высказывали о бытии в более
богатых формах представления об абсолютном или боге
или что бы в них ни содержалось, в начале это лишь
пустое слово и только бытие. Это простое, не имеющее
в общем никакого дальнейшего значения, это пустое
есть, стало быть, безусловно начало философии.

Это воззрение само столь просто, что указанное начало,
как таковое, не нуждается ни в каком подготовлении
или дальнейшем введении, и целью этого нашего
предварительного рассуждения о нем могло быть не введение
этого начала, а скорее устранение всего предварительного.


%%%Local Variables:
%%% mode: latex
%%% TeX-master: "../../main"
%%% End:


\section{Общее деление бытия}

Бытие, \emph{во-первых}, определено вообще по отношению
к иному.

Оно, \emph{во-вторых}, определяет себя внутри самого себя.

\emph{В-третьих}, если отбросить это предварительное деление,
бытие есть та абстрактная неопределенность и непосредственность,
в которой оно должно служить началом.


Согласно \emph{первому} определению бытие отделяет себя
от \emph{сущности}, показывая в дальнейшем своем развитии
свою целокупность лишь как одну сферу понятия и
противопоставляя ей как момент некоторую другую
сферу.

Согласно \emph{второму} определению оно есть сфера, в которую
входят определения и все движение его рефлексии.
В ней бытие полагает себя в трех следующих определениях:
\begin{enumerate}[label=\Roman*.]
\item как \emph{определенность}, как таковая: \emph{качество};
\item как \emph{снятая} определенность: \emph{величина}, \emph{количество};
\item как \emph{качественно} определенное \emph{количество}: \emph{мера}.
\end{enumerate}

Это деление, как сказано во введении относительно
всех этих делений вообще, есть только предварительное
перечисление. Его определения должны еще возникнуть
из движения самого бытия, дать себе через это движение
дефиницию и обоснование. Об отклонении этого деления
от обычного перечня категорий, а именно как количества,
качества, отношения и модальности, которые, впрочем,
у Канта, надо полагать, служили только заглавиями
для его категорий, а на самом деле сами суть категории,
только более всеобщие,~-- об этом отклонении здесь не стоит
говорить, так как все изложение покажет, каковы вообще
наши отклонения от обычного порядка и значения категорий.


Здесь можно отметить лишь следующее: определение
\emph{количества} обычно приводят раньше определения \emph{качества},
и притом это делается, как в большинстве случаев,
без какого-либо обоснования. Мы уже показали, что началом
служит бытие, \emph{как таковое}, значит, качественное
бытие. Из сравнения качества с количеством легко увидеть,
что по своей природе качество есть первое. Ибо
количество есть качество, ставшее уже отрицательным;
\emph{величина} есть определенность, которая больше не едина
с бытием, а уже отлична от него, она снятое, ставшее
безразличным качество. Она включает в себя изменчивость
бытия, не изменяя самой вещи, бытия, определением
которого она служит; качественная же определенность
едина со своим бытием, она не выходит за его
пределы и не находится внутри его, а есть его \emph{непосредственная}
ограниченность. Поэтому качество как непосредственная
определенность есть первая определенность,
и с него следует начинать.

\emph{Мера} есть \emph{отношение}, но не отношение вообще,
а определенна отношение качества и количества друг
к другу; категории, которые Кант объединяет под названием
<<отношение>>, займут свое место совсем в другом
разделе. Меру можно, если угодно, рассматривать и как
некоторую модальность. Но так как у Канта модальность
уже не есть определение содержания, а касается лишь
отношения содержания к мышлению, к субъективному,
то это~-- совершенно чужеродное, сюда не принадлежащее
отношение.

\emph{Третье} определение \emph{бытия} входит в раздел о качестве,
ибо бытие как абстрактная непосредственность низводит
себя до единичной определенности, противостоящей
внутри его сферы другим его определенностям.




%%% Local Variables:
%%% mode: latex
%%% TeX-master: "../../main"
%%% TeX-engine: xetex
%%% End:


\part{Определенность (качество)}

Бытие есть неопределенное непосредственное. Оно
свободно от определенности по отношению к сущности,
равно как и от всякой определенности, которую оно может
обрести внутри самого себя. Это лишенное рефлексии
бытие есть бытие, как оно есть непосредственно
лишь в самом себе.

Так как оно неопределенно, то оно бескачественное
бытие. Однако \emph{в себе} ему присущ характер неопределенности
лишь в противоположность \emph{определенному} или качественному.
Но бытию вообще противостоит \emph{определенное}
бытие, как таковое, а благодаря этому сама его
неопределенность составляет его качество. Тем самым обнаружится,
что \emph{первое} бытие есть определенное в себе и
что, следовательно,

\emph{во-вторых}, оно переходит в \emph{наличное бытие}, есть \emph{наличное
бытие}, но это последнее как конечное бытие снимает
себя и переходит в бесконечное соотношение бытия
с самим собой,

переходит, \emph{в-третьих}, в \emph{для-себя-бытие}.


%%% Local Variables:
%%% mode: latex
%%% TeX-master: "../../../main"
%%% End:


\chapter{Бытие}

\section{Бытие}

\emph{Бытие}, \emph{чистое бытие}~-- без всякого дальнейшего определения.
В своей неопределенной непосредственности оно
равно лишь самому себе, а также не неравно в отношении
иного, не имеет никакого различия ни внутри себя,
ни по отношению к внешнему. Если бы в бытии было
какое-либо различимое определение или содержание или
же оно благодаря этому было бы положено как отличное
от некоего иного, то оно не сохранило бы свою чистоту.
Бытие есть чистая неопределенность и пустота.~-- В нем
\emph{нечего} созерцать, если здесь может идти речь о созерцании,
иначе говоря, оно есть только само это чистое, пустое
созерцание. В нем также нет ничего такого, что
можно было бы мыслить, иначе говоря, оно равным образом
лишь это пустое мышление. Бытие, неопределенное
непосредственное, есть на деле \emph{ничто} и не более и
не менее, как ничто.

%%% Local Variables:
%%% mode: latex
%%% TeX-master: "../../../main"
%%% End:


\section{Ничто}

\emph{Ничто}, \emph{чистое ничто}; оно простое равенство с самим
собой, совершенная пустота, отсутствие определений и
содержания; неразличенность в самом себе.~-- Насколько
здесь можно говорить о созерцании или мышлении, следует
сказать, что считается небезразличным, созерцаем
ли мы, или мыслим ли мы нечто или \emph{ничто}. Следовательно,
выражение <<созерцать или мыслить ничто>> что-то
означает. Мы проводим различие между нечто и ничто;
таким образом, ничто \emph{есть} (существует) в нашем созерцании
или мышлении; или, вернее, оно само пустое созерцание
и мышление; и оно есть то же пустое созерцание
или мышление, что и чистое бытие.~-- Ничто есть,
стало быть, то же определение или, вернее, то же отсутствие
определений и, значит, вообще то же, что и чистое
\emph{бытие}.

%%% Local Variables:
%%% mode: latex
%%% TeX-master: "../../../main"
%%% End:


\section{Становление}

\subsection{Единство бытия и ничто}

\emph{Чистое бытие и чистое ничто есть, следовательно,
одно и то же}. Истина~-- это не бытие и не ничто, она
состоит в том, что бытие не переходит, а перешло в ничто,
и ничто не переходит, а перешло в бытие. Но точно так
же истина не есть их неразличенность, она состоит в том,
что \emph{они не одно и то же}, что они \emph{абсолютно различны},
но также нераздельны и неразделимы и что каждое из
них непосредственно \emph{исчезает в своей противоположности}.
Их истина есть, следовательно, это \emph{движение} непосредственного
исчезновения одного в другом: \emph{становление};
такое движение, в котором они оба различны, но
благодаря такому различию, которое столь же непосредственно
растворилось.

%%% Local Variables:
%%% mode: latex
%%% TeX-master: "../../../main"
%%% End:


\subsubsection{Примечание 1. [Противоположность бытия и ничто в представлении]\endnote{
    В квадратных скобках даются заголовки примечаний, приведенные Гегелем лишь в
    оглавлении <<Науки логики>>, но отсутствующие в самом тексте.}}

\emph{Ничто} обычно противопоставляют [всякому] \emph{нечто}; но
нечто есть уже определенное сущее, отличающееся от
другого нечто; таким образом и ничто, противопоставляемое
[всякому] нечто, есть ничто какого-нибудь нечто,
определенное ничто. Но здесь д\'олжно брать ничто в его
неопределенной простоте.~-- Если бы кто-нибудь считал
более правильным противопоставлять бытию не ничто,
а \emph{небытие}, то, имея в виду результат, нечего было бы
возразить против этого, ибо в \emph{небытии} содержится соотношение
с \emph{бытием}; оно и то и другое, бытие и его отрицание,
выраженные в \emph{одном}, ничто, как оно есть в становлении.
Но прежде всего речь должна идти не о форме
противопоставления, т.\,е. одновременно и о форме \emph{соотношения},
а об абстрактном, непосредственном отрицании,
о ничто, взятом чисто само по себе, о безотносительном
отрицании,~-- что, если угодно, можно было бы выразить
также и простым \emph{не}.

Простую мысль о \emph{чистом бытии} как об абсолютном
и как о единственной истине впервые высказали \emph{элеаты},
особенно Парменид, который в дошедших до нас фрагментах
высказал ее с чистым воодушевлением мышления,
в первый раз постигшего себя в своей абсолютной абстрактности:
\emph{только бытие есть, а ничто вовсе нет}.~--
В восточных системах, особенно в буддизме, \emph{ничто}, пустота,
составляет, как известно, абсолютный принцип.~--
Глубокий мыслитель Гераклит выдвигал против указанной
простой и односторонней абстракции более высокое,
целокупное понятие становления и говорил: \emph{бытия нет
точно так же, как нет ничто}, или, выражая эту мысль
иначе, \emph{все течет}, т.\,е. все есть \emph{становление}.~-- Общедоступные
изречения, в особенности восточные, гласящие,
что все, что есть, имеет зародыш своего уничтожения
в самом своем рождении, а смерть, наоборот, есть вступление
в новую жизнь, выражают в сущности то же единение
бытия и ничто. Но эти выражения предполагают
субстрат, в котором совершается переход: бытие и ничто
обособлены друг от друга во времени, представлены как
чередующиеся в нем, а не мыслятся в их абстрактности,
и поэтому мыслятся не так, чтобы они сами по себе были,
одним и тем же.

Ex nihilo nihil fit~-- это одно из положений, которым
в метафизике приписывалось большое значение. В этом
положении можно либо усматривать лишь бессодержательную
тавтологию: ничто есть ничто; либо, если действительным
смыслом этого положения должно быть
[высказывание о] \emph{становлении}, то следует сказать, что
так как из \emph{ничего становится} только \emph{ничто}, то на самом
деле здесь нет речи о \emph{становлении}, ибо ничто так и
остается здесь ничем. Становление означает, что ничто
не остается ничем, а переходит в свое иное, в бытие.~--
Если позже метафизика, особенно христианская, отвергла
положение о том, что из ничего ничего не происходит,
то она этим утверждала, что ничто переходит в \emph{бытие};
как бы она ни брала последнее положение~-- в виде ли
синтеза или просто в виде представления,~-- даже в самом
несовершенном соединении имеется точка, в которой
бытие и ничто встречаются и их различие исчезает.~--
Положение: \emph{из ничего ничего не происходит, ничто есть
именно ничто}, приобретает свое настоящее значение благодаря
тому, что противопоставляется \emph{становлению} вообще
и, следовательно, также сотворению мира из ничего.
Те, кто высказывает и даже горячо отстаивает положение:
ничто есть именно ничто, не сознают, что они тем
самым соглашаются с абстрактным \emph{пантеизмом} элеатов
и по сути дела также и со спинозовским пантеизмом.
Философское воззрение, которое считает принципом положение
<<бытие~-- это только бытие, ничто~-- это только
ничто>>, заслуживает названия системы тождества; это
абстрактное тождество составляет сущность пантеизма.

Если вывод, что бытие и ничто суть одно и то же,
взятый сам по себе, кажется удивительным или парадоксальным,
то не следует больше обращать на это внимания;
скорее приходится удивляться удивлению тех, кто
показывает себя таким новичком в философии и забывает,
что в этой науке встречаются совсем иные определения,
чем определения обыденного сознания и так называемого
здравого человеческого рассудка, который не
обязательно здравый, а бывает и рассудком, возвышающимся
до абстракций и до веры в них или, вернее, до
суеверного отношения к абстракциям. Было бы нетрудно
показать это единство бытия и ничто на любом примере,
во \emph{всякой} действительной вещи или мысли. О бытии и
ничто следует сказать то же, чт\'о было сказано выше
о непосредственности и опосредствовании (заключающем
в себе некое соотношение \emph{друг с другом} (aufeinander)
и, значит, \emph{отрицание}), а именно, что \emph{нет ничего ни
на небе, ни на земле, что не содержало бы в себе и бытие
и ничто}. Разумеется, так как при этом речь заходит
о \emph{каком-то нечто} и \emph{действительном}, то в этом нечто указанные
определения наличествуют уже не в той совершенной
неистинности, в какой они выступают как бытие
и ничто, а в некотором дальнейшем определении и понимаются,
например, как \emph{положительное} и \emph{отрицательное};
первое есть положенное, рефлектированное бытие, а последнее
есть положенное, рефлектированное ничто; но
положительное и отрицательное содержат как свою абстрактную
основу: первое~-- бытие, а второе~-- ничто.~--
Так, в самом боге качество, \emph{деятельность}, \emph{творение}, \emph{могущество}
и т.\,д. содержат как нечто сущностное определение
отрицательного,~-- они создают некое \emph{иное}. Но эмпирическое
пояснение указанного утверждения примерами
было бы здесь совершенно излишне. Так как это
единство бытия и ничто раз навсегда лежит в основе как
первая истина и составляет стихию всего последующего,
то помимо самого становления все дальнейшие логические
определения: наличное бытие, качество, да и вообще
все понятия философии служат примерами этого единства.
А так называющий себя обыденный или здравый
человеческий рассудок, поскольку он отвергает нераздельность
бытия и ничто, пусть попытается отыскать
пример, в котором одно оказалось бы отделенным от
другого (нечто от границы, предела, или бесконечное,
бог, как мы только что упомянули, от деятельности).
Только пустые порождения мысли (Gedankendinge)~--
бытие и ничто~-- только сами они и суть такого рода
раздельные, и их-то этот рассудок предпочитает истине,
нераздельности того и другого, которую мы всюду имеем
перед собой.

Нашим намерением не может быть предупреждать все
случаи, когда обыденное сознание сбивается с толку при
рассмотрении подобного рода логических положений, ибо
случаи эти неисчислимы. Мы можем коснуться лишь некоторых
из них. Одной из причин такой путаницы служит,
между прочим, то обстоятельство, что сознание
привносит в такие абстрактные логические положения
представления о некотором конкретном нечто и забывает,
что речь идет вовсе не о нем, а лишь о чистых абстракциях
бытия и ничто, и что только их необходимо придерживаться.

Бытие и небытие суть одно и то же; \emph{следовательно},
одно и то же, существую ли я или не существую, существует
ли или не существует этот дом, обладаю ли я или
не обладаю ста талерами. Это умозаключение или применение
указанного положения совершенно меняет его
смысл. В указанном положении говорится о чистых абстракциях
бытия и ничто; применение же делает из них
определенное бытие и определенное ничто. Но об определенном
бытии, как уже сказано, здесь речь не идет.
Определенное, конечное бытие~-- это такое бытие, которое
соотносится с другим бытием: оно содержание, находящееся
в отношении необходимости с другим содержанием,
со всем миром. Имея в виду взаимоопределяющую
связь целого, метафизика могла выставить~-- в сущности
говоря, тавтологическое~-- утверждение, что если бы была
уничтожена одна пылинка, то обрушилась бы вся Вселенная.
В примерах, приводимых против рассматриваемого
нами положения, представляется небезразличным,
существует ли нечто или его нет, не из-за бытия или
небытия, а из-за его \emph{содержания}, связывающего его
с другим содержанием. Когда \emph{предполагается} некое определенное
содержание, какое-то определенное наличное
бытие, то это наличное бытие, потому что оно \emph{определенное},
находится в многообразном соотношении с другим
содержанием. Для него небезразлично, имеется ли
другое содержание, с которым оно соотносится, или его
нет, ибо только через такое соотношение оно по своему
существу есть то, что оно есть. То же самое имеет место
и в \emph{представлении} (поскольку мы берем небытие в более
определенном смысле~-- как представление в противоположность
действительности), в связи с которым небезразлично,
имеется ли бытие или отсутствие содержания,
которое как определенное представляется соотнесенным
с другим содержанием.

Это соображение касается того, что составляет один
из главных моментов в кантовской критике онтологического
доказательства бытия бога, которую, однако, мы
здесь рассматриваем лишь в отношении встречающегося
в ней различения между бытием и ничто вообще и между
\emph{определенными} бытием или небытием.~-- Как известно,
это так называемое доказательство заранее предполагает
понятие существа, которому присущи все реальности и,
следовательно, также существование, каковое также было
принято за одну из реальностей. Кантова критика напирает,
главным образом, на то, что \emph{существование} или бытие
(которые здесь считаются равнозначными) не есть
\emph{свойство} или \emph{реальный предикат}, т.\,е. не есть понятие
чего-то такого, что можно прибавить к \emph{понятию} какой-нибудь
вещи\footnotemark{}.~-- Кант хочет этим сказать, что бытие не
есть определение содержания.~-- Стало быть, продолжает
он, действительное не содержит в себе чего-либо большего,
чем возможное; сто действительных талеров не содержат
в себе ни на йоту больше, чем сто возможных
талеров, а именно первые не имеют другого определения
содержания, чем последние. Для этого, рассматриваемого
как изолированное, содержания в самом деле
безразлично, быть или не быть; в нем нет никакого различия
бытия или небытия, это различие вообще не затрагивает
его: сто талеров не сделаются меньше, если их
нет, и больше, если они есть. Различие должно прийти
откуда-то извне.~-- <<Но,~-- напоминает Кант,~-- мое имущество
больше при наличии ста действительных талеров,
чем при одном лишь понятии их (т.\,е. возможности их).
В самом деле, в случае действительности \emph{предмет} не
только аналитически содержится в моем понятии, \emph{но и
прибавляется синтетически к моему понятию} (которое
служит \emph{определением} моего \emph{состояния}), нисколько не увеличивая
эти мыслимые сто талеров этим бытием вне
моего понятия>>\endnotemark{}.

\footnotetext{Kants Kritik der г. Vera. 2te Aufl. S. 628 ff
  \endnote{<<Критика чистого разума>>, стр.\,521.}.}

\endnotetext{<<Критика чистого разума>>, стр.\,522. Курсив Гегеля.}

Здесь \emph{предполагаются}~-- если сохранить выражения
Канта, не свободные от запутывающей тяжеловесности,~--
двоякого рода состояния: одно, которое Кант называет
понятием и под которым следует понимать представление,
и другое~-- состояние имущества. Для одного, как и
для другого,~-- для имущества, как и для представления,
сто талеров суть определение содержания, или, как
выражается Кант, <<они прибавляются к нему \emph{синтетически}>>.
Я как \emph{обладатель} ста талеров или как необладатель
их или же я как \emph{представляющий} себе сто талеров
или не представляющий себе их~-- это, конечно, разное
содержание. Выразим это в более общем виде:
абстракции бытия и ничто перестают быть абстракциями,
когда они получают определенное содержание; в этом
случае бытие есть реальность, определенное бытие ста
талеров, ничто есть отрицание, определенное небытие
этих талеров. Само же это определение содержания, сто
талеров, рассматриваемое также абстрактно, само по себе,
остается без изменений, одним и тем же и в том, и в другом
случае. Но когда, далее, бытие берется как имущественное
состояние, сто талеров вступают в связь с некоторым
состоянием, и для последнего такого рода определенность,
которую они составляют, не безразлична; их
бытие или небытие есть лишь \emph{изменение}; они перенесены
в сферу \emph{наличного бытия}. Поэтому, если против
единства бытия и ничто возражают, что, мол, не безразлично,
имеется ли то-то (100 талеров) или не имеется,
то заблуждаются, относя различие между моим \emph{обладанием}
и \emph{необладанием} ста талерами только за счет бытия
или небытия. Это заблуждение, как мы показали, основано
на односторонней абстракции, опускающей \emph{определенное
наличное бытие}, которое имеется в такого рода
примерах, и удерживающей лишь бытие и небытие, так
же как и, наоборот, превращающей абстрактное бытие
и [абстрактное] ничто, которое должно постигнуть, в определенное
бытие и ничто, в наличное бытие. Лишь \emph{наличное
бытие} содержит реальное различие между бытием
и ничто, а именно \emph{нечто} и \emph{иное}.~-- Это реальное различие
предстает перед представлением вместо абстрактного
бытия и чистого ничто и лишь мнимого различия между
ними.

Как выражается Кант, <<посредством существования
нечто вступает в контекст совокупного опыта>>. <<Благодаря
этому мы получаем одним предметом \emph{восприятия}
больше, но наше \emph{понятие} о предмете этим не обогащается>>\endnotemark{}.~--
Это, как вытекает из предыдущего разъяснения,
означает следующее: посредством существования, главным
образом потому, что нечто есть определенное существование,
оно находится в связи с \emph{иным}, и, между
прочим, также с неким воспринимающим.~-- Понятие ста
талеров, говорит Кант, не обогащается от того, что их
воспринимают. \emph{Понятием} Кант здесь называет означенные
выше \emph{изолированно} представляемые сто талеров.
В такой изолированности они, правда, суть некоторое
эмпирическое содержание, но содержание оторванное, не
связанное с \emph{иным} и не имеющее определенности в отношении
иного. Форма тождества с собой лишает их соотношения
с иным и делает их безразличными к тому,
восприняты ли они или нет. Но это так называемое \emph{понятие}
ста талеров~-- ложное понятие; форма простого
соотношения с собой не принадлежит самому такому
ограниченному, конечному содержанию; она форма, приданная
ему субъективным рассудком и заимствованная
им у этого рассудка; сто талеров~-- это не нечто соотносящееся
с собой, а нечто изменчивое и преходящее.

\endnotetext{Это не цитаты, а свободное изложение мысли Канта.
Ср. <<Критика чистого разума>>, стр.\,523.}

Мышлению или представлению, перед которыми предстает
лишь какое-то определенное бытие~-- наличное бытие,~--
следует указать на упомянутое выше начало
науки, положенное Парменидом, который свое представление
и тем самым и представление последующих поколений
очистил и возвысил до \emph{чистой мысли}, до бытия,
как такового, и этим создал стихию науки.~-- То, чт\'о
составляет \emph{первый шаг в науке}, должно было явить себя
\emph{первым} и \emph{исторически}. И \emph{единое} или \emph{бытие} в учении
элеатов мы должны рассматривать как первый шаг знания
о мысли; \emph{вода}\endnote{Имеется в виду учение Фалеса о воде как первоначале
всего сущего.} и тому подобные материальные начала,
хотя, \emph{по мнению} выдвигавших их философов, представляли
собой всеобщее, однако как материи они не чистые
мысли; \emph{числа}\endnote{Имеется в виду учение пифагорейцев о числах как сущности вещей.}
же~-- это не первая простая и не
остающаяся самой собой мысль, а мысль, всецело внешняя
самой себе.

Отсылку от \emph{отдельного конечного} бытия к бытию, как
таковому, взятому в его совершенно абстрактной всеобщности,
следует рассматривать как самое первое теоретическое
и даже практическое требование. А именно, если
поднимают шумиху вокруг этих ста талеров, утверждая,
что для моего имущественного состояния не безразлично,
обладаю ли я ими или нет, и тем более не безразлично,
существую ли я или нет, существует ли иное или нет, то
не говоря уже о том, что бывают такие имущественные
состояния, для которых такое обладание ста талерами будет
безразлично,~-- можно напомнить, что человек должен
подняться в своем образе мыслей до такой абстрактной
всеобщности, при которой ему в самом деле будет
безразлично, существуют ли или не существуют эти сто
талеров, каково бы ни было их количественное соотношение
с его имущественным состоянием, как ему будет
столь же безразлично, существует ли он или нет, т.\,е. существует
ли он или нет в конечной жизни (ибо имеется
в виду некое состояние, определенное бытие) и т.\,д. Даже
si fractus illabatur orbis, impavidum ferient ruinae\endnotemark{},
сказал один римлянин, а тем более должно быть присуще
такое безразличие христианину.

\endnotetext{<<Если бы на него обрушился весь мир, он без страха встретил
  бы смерть под его развалинами>>~-- строки из оды Горация
  <<Iiistum et tenacem propositi virum>> (Horatius. Carmina III,~3).
  Гораций рисует в этой оде образ справедливого и постоянного
  в своих намерениях человека, который ничего не боится и которого
  ничто не может вывести из душевного равновесия.}

Следует еще отметить непосредственную связь между
возвышением над ста талерами и вообще над конечными
вещами и онтологическим доказательством и упомянутой
кантовской критикой его. Эта критика показалась всем
убедительной благодаря приведенному ею популярному
примеру; кто же не знает, что сто действительных талеров
отличны от ста лишь возможных талеров? Кто не
знает, что они составляют разницу в моем имущественном
состоянии? Так как на примере ста талеров обнаруживается
таким образом эта разница, то понятие, т.\,е.
определенность содержания как пустая возможность, и
бытие отличны друг от друга; \emph{стало быть}, и понятие бога
отлично от его бытия, и так же как я из возможности
ста талеров не могу вывести их действительность, точно
так же не могу из понятия бога <<вылущить>> (herausklauben)
его существование; а в таком вылущивании существования
бога из его понятия и состоит-де онтологическое
доказательство. Но если несомненно верно, что понятие
отлично от бытия, то бог еще более отличен от ста
талеров и других конечных вещей. В том и состоит \emph{дефиниция
конечных вещей}, что в них понятие и бытие различны,
понятие и реальность, душа и тело отделимы друг
от друга, и потому преходящи и смертны; напротив, абстрактная
дефиниция бога состоит именно в том, что его
понятие и его бытие \emph{нераздельны} и \emph{неотделимы}. Истинная
критика категорий и разума заключается как раз в
том, чтобы сделать познание этого различия ясным и
удерживать его от применения к богу определений и соотношений
конечного.


%%% Local Variables:
%%% mode: latex
%%% TeX-master: "../../../main"
%%% End:


\subsubsection{Примечание 2. [Неудовлетворительность выражения: единство, тождество бытия и ничто]}

Следует еще указать и на другую причину, усиливающую
неприязнь к положению о бытии и ничто. Эта причина~-- то,
что вывод, вытекающий из рассмотрения бытия
и ничто, несовершенно выражен в положении: \emph{бытие
и ничто~-- одно и то же}. Ударение падает преимущественно
на <<одно и то же>>, как и вообще в суждении, поскольку
в нем лишь предикат высказывает, чт\'о \emph{представляет
собой} субъект [суждения]. Поэтому кажется, будто смысл
[вывода]~-- в отрицании различия, которое, однако, в то
же время непосредственно имеется в положении, ибо оно
высказывает \emph{оба} определения, бытие и ничто, и содержит
их как различные.~-- И не в том смысл этого положения,
что следует от них абстрагироваться и удерживать лишь
единство. Подобный смысл сам обнаруживал бы свою односторонность,
так как то, от чего якобы д\'олжно отвлекаться,
все же имеется и названо в положении.~-- Итак,
поскольку положение: \emph{бытие и ничто~-- одно и то же}, высказывает
тождество этих определений, но на самом деле
также содержит эти два определения как различные, постольку
оно противоречиво в самом себе и разлагает себя.
Если выразиться более точно, то здесь дано положение,
которое, как обнаруживается при более тщательном
рассмотрении, устремлено к тому, чтобы заставить само
себя исчезнуть. Но тем самым в нем самом совершается
то, чт\'о должно составить его настоящее содержание, а
именно \emph{становление}.

Рассматриваемое нами положение, таким образом, \emph{содержит}
вывод, оно \emph{в самом себе} есть этот вывод. Но здесь
мы должны обратить внимание на следующий недостаток:
сам вывод не \emph{выражен} в положении; только внешняя рефлексия
познает его в нем.~-- По этому поводу следует
уже в самом начале сделать общее замечание, что положение
в \emph{форме суждения} не пригодно для выражения
спекулятивных истин. Знакомство с этим обстоятельством
могло бы устранить многие недоразумения относительно
спекулятивных истин. Суждение есть отношение \emph{тождества}
между субъектом и предикатом, при этом абстрагируются
от того, что у субъекта еще многие [другие] определенности,
чем те, которыми обладает предикат, и от
того, что предикат шире субъекта. Но если содержание спекулятивно,
то и \emph{нетождественное} в субъекте и предикате
составляет существенный момент, однако в суждении это
не выражено. Парадоксальный и странный свет, в котором
не освоившимся со спекулятивным мышлением представляются
многие положения новейшей философии, часто
зависит от формы простого суждения, когда она применяется
для выражения спекулятивных выводов.

Чтобы выразить спекулятивную истину, указанный недостаток
устраняют прежде всего тем, что к положению
прибавляют противоположное положение: \emph{бытие и ничто
не одно и то же}, каковое положение также было высказано
выше. Но тогда возникает еще другой недостаток, а
именно: эти положения не связаны между собой и, стало
быть, излагают содержание лишь в антиномии, между
тем как их содержание касается одного и того же, и определения,
выраженные в этих двух положениях, должны
быть безусловно соединены,~-- получится соединение, которое
может быть высказано лишь как некое \emph{беспокойство
несовместимых} друг с другом [определений], как \emph{некое
движение}. Самая обычная несправедливость, совершаемая
по отношению к спекулятивному содержанию, заключается
в том, что его делают односторонним, т.\,е. выпячивают
лишь одно из положений, на которые оно может
быть разложено. Нельзя в таком случае отрицать, что это
положение [действительно] утверждается; \emph{но насколько
правильно то, чт\'о в нем указывается, настолько же оно
и ложно}, ибо раз из области спекулятивного берут одно
положение, то следовало бы по меньшей мере точно так
же обратить внимание и на другое положение и указать
его.~-- При этом нужно еще особо отметить, так сказать,
злополучное слово <<единство>>. <<Единство>> еще в большей
мере, чем <<тождество>>, обозначает субъективную
рефлексию. Оно берется главным образом как соотношение,
получающееся из \emph{сравнивания}, из внешней рефлексии."
Поскольку последняя находит в двух \emph{разных предметах}
одно и то же, единство имеется таким образом, что
при этом предполагается полное \emph{безразличие} самих сравниваемых
предметов к этому единству, так что это сравнивание
и единство вовсе не касаются самих предметов
и суть некое внешнее для них действование и определение.
<<Единство>> выражает поэтому совершенно \emph{абстрактное}
<<одно и то же>> и звучит тем резче и более странно,
чем больше те предметы, о которых оно высказывается,
являют себя просто различными. Постольку было бы поэтому
лучше вместо <<единства>> говорить лишь <<\emph{нераздельность}>>
и <<\emph{неразделимостъ}>>; но эти слова не выражают
того, чт\'о есть \emph{утвердительного} в соотношении целого.

Таким образом, полный, истинный результат, выявившийся
здесь, это~-- \emph{становление}, которое не есть ЛИШЬ
одностороннее или абстрактное единство бытия и ничто.
Становление состоит в следующем движении: чистое бытие
непосредственно и просто; оно поэтому в такой же
мере есть чистое ничто; различие между ними \emph{есть}, но
в такой же мере \emph{снимает себя} и \emph{не есть}. Результат, следовательно,
утверждает также и различие между бытием и
ничто, но как такое различие, которое только \emph{предполагается}
(gemeinten).

\emph{Предполагают}, что бытие есть скорее всецело иное,
чем ничто, и ничего нет яснее того, что они абсолютно
различны, и, кажется, ничего нет легче, чем указать их
различие. Но столь же легко убедиться в том, что это невозможно,
что это различие \emph{невыразимо}. Пусть \emph{те, кто
настаивает на различии между бытием и ничто}, возьмут
на себя труд \emph{указать, в чем оно состоит}. Если бы бытие
и ничто различала какая-нибудь определенность, то они,
как мы уже говорили, были бы определенным бытием и
определенным ничто, а не чистым бытием и чистым ничто,
каковы они еще здесь. Поэтому различие между ними
совершенно пусто, каждое из них в равной мере есть
неопределенное. Это различие имеется поэтому не в них
самих, а лишь в чем-то третьем, в \emph{предполагании} (Meinen).
Однако предполагание есть форма субъективного,
которое не имеет касательства к этому изложению. Но
третье, в котором имеют свое существование бытие и ничто,
должно иметь место и здесь; и оно, действительно,
имело здесь место; это~-- \emph{становление}. В нем они имеются
как различные; становление имеется лишь постольку, поскольку
они различны. Это третье есть нечто иное, чем
они. Они существуют лишь в ином. Это также означает,
что они не существуют особо (für sich). Становление есть
существование (Bestehen) бытия в той же мере, что и существование
небытия, иначе говоря, их существование
есть лишь их бытие в \emph{одном}; именно это их существование
и есть то, чт\'о также снимает их различие.

Требование указать различие между бытием и ничто
заключает в себе и требование сказать, чт\'о же такое
\emph{бытие} и чт\'о такое \emph{ничто}. Пусть те, кто отказывается признать,
что и бытие, и ничто есть лишь \emph{переход} одного в
другое, и утверждает о бытии и ничто то и се,~-- пусть
они укажут, о \emph{чем} они говорят, т.\,е. пусть дадут \emph{дефиницию}
бытия и ничто и пусть докажут, что она правильна.
Без удовлетворения этого первого требования старой науки,
логические правила которой они в других случаях
признают и применяют, все их утверждения о бытии и
ничто не более как заверения, лишенные научной значимости.
Если, например, раньше говорили, что существование,
поскольку прежде всего его считают равнозначным
бытию, есть \emph{дополнение} к \emph{возможности}, то этим предполагается
другое определение~-- возможность, и бытие выражено
не в своей непосредственности и даже не как нечто
самостоятельное, а как обусловленное. Для обозначения
\emph{опосредствованного} бытия мы сохраним выражение
\emph{существование}. Правда, люди представляют себе бытие,~--
скажем, прибегая к образу чистого света, как ясность непомутненного
в\'идения, а ничто~-- как чистую ночь, и связывают
их различие с этой хорошо знакомой чувственной
разницей. Однако на самом деле, если точнее представить
себе и это в\'идение, то легко заметить, что в абсолютной
ясности мы столь же много и столь же мало видим, как
и в абсолютной тьме, что и то и другое в\'идение есть чистое
в\'идение, т.\,е. ничегоневидение. Чистый свет и чистая
тьма~-- это две пустоты, которые суть одно и то же. Лишь
в определенном свете~-- а свет определяется тьмой,~-- следовательно,
в помутненном свете, и точно так же лишь
в определенной тьме~-- а тьма определяется светом,~-- в
освещенной тьме можно что-то различать, так как лишь
помутненный свет и освещенная тьма имеют различие в
самих себе и, следовательно, суть определенное бытие,
\emph{наличное бытие}.


%%% Local Variables:
%%% mode: latex
%%% TeX-master: "../../../main"
%%% End:


\subsubsection{Примечание 3. [Изолирование этих абстракций]}

Единство, моменты которого, бытие и ничто, даны как
неразделимые, в то же время отлично от них самих и таким
образом есть в отношении их некое \emph{третье}, которое
в своей самой характерной форме есть \emph{становление}. \emph{Переход}
есть то же, чт\'о и становление, с той лишь разницей,
что оба [момента], от одного из которых совершается переход
к другому, в становлении представляют себе скорее
как находящиеся в покое друг вне друга, а переход~-- как
совершающийся \emph{между} ними. Где бы и как бы ни шла
речь о бытии или ничто, непременно должно наличествовать
это третье; ведь бытие и ничто существуют не сами
по себе, а лишь в становлении, в этом третьем. Но это
третье имеет многоразличные эмпирические образы, которые
абстракция оставляет в стороне или которыми она
пренебрегает, чтобы фиксировать каждый из ее продуктов~--
бытие и ничто~-- особо и показать их защищенными
от перехода. В противовес такому простому способу
абстрагирования следует столь же просто сослаться лишь
на эмпирическое существование, в котором сама эта абстракция
есть лишь нечто, обладает наличным бытием.
Или же фиксировать разделение неразделимых должны
другие формы рефлексии. В таком определении само по
себе имеется его противоположность, так что и не восходя
к природе вещей и не апеллируя к ней, можно изобличить
это определение рефлексии в нем самом, беря его
так, как оно само себя дает, и в нем самом обнаруживая
его иное. Было бы тщетно стараться как бы схватить все
извороты, все неожиданные мысли рефлексии и ее рассуждения,
чтобы лишить ее возможности пользоваться теми
лазейками и увертками, при помощи которых она
скрывает от себя свое противоречие с самой собой. Поэтому
я и отказываюсь принимать во внимание те многочисленные,
так называющие себя возражения и опровержения,
которые приводились против того [взгляда], что
ни бытие, ни ничто не есть нечто истинное, а что их истина~--
это только становление. Культура мысли, требующаяся
для того, чтобы усмотреть ничтожность этих опровержений,
или, вернее, чтобы отогнать от самого себя такие
неожиданные мысли, достигается лишь благодаря критическому
познанию форм рассудка. Но те, кто щедрее всего
на подобного рода возражения, сразу нападают со своими
соображениями на первые положения, не давая себе
труда до или после этого путем дальнейшего изучения
логики помочь себе осознать природу этих плоских соображений.

Здесь следует рассмотреть некоторые явления, возникающие
от того, что изолируют друг от друга бытие и ничто
и полагают одно вне сферы другого, так что тем самым
отрицается переход.

Парменид признавал только бытие и был как нельзя
более последователен, говоря в то же время о ничто, что
его \emph{вовсе нет}; имеется лишь бытие. Бытие, взятое совершенно
отдельно, есть неопределенное, следовательно, никак
не соотносится с иным; поэтому кажется, что, исходя
\emph{из этого начала}, а именно из самого бытия, нельзя
\emph{двигаться дальше}, что, для того чтобы двинуться дальше,
надо присоединить к нему \emph{извне} нечто чуждое. Дальнейшее
движение, [выражаемое положением о том], что бытие
есть то же самое, что ничто, представляется, стало
быть, как второе, абсолютное начало~-- как переход, стоящий
отдельно и внешне примыкающий к бытию. Бытие
вообще не было бы абсолютным началом, если бы у него
была какая-нибудь определенность; оно тогда зависело бы
от иного и не было бы непосредственным, не было бы началом.
Если же оно неопределенно и тем самым есть
истинное начало, то у него и нет ничего такого, с помощью
чего оно переходило бы в иное, оно в то же время
есть и \emph{конец}. Столь же мало может что-либо вырваться
из него, как и ворваться в него; у Парменида, как и у
Спинозы, нет продвижения от бытия или абсолютной субстанции
к отрицательному, конечному. Если же все-таки
совершается такое продвижение (что, исходя из бытия,
лишенного соотношений и, стало быть, лишенного продвижения,
можно, как мы заметили, осуществить только
внешне), то это движение есть второе, новое начало. Так,
у Фихте его абсолютнейшее, безусловное основоположение
$A = A$ есть \emph{полагание}; второе основоположение~--
\emph{противополагание}; это второе основоположение, согласно
Фихте, \emph{отчасти} обусловлено, \emph{отчасти} безусловно (оно,
следовательно, есть противоречие внутри себя). Это~--
продвижение внешней рефлексии, которое снова так же
отрицает то, с чего оно начинает как с чего-то абсолютного,~--
противополагание есть отрицание первого тождества,~--
как тотчас же определенно делает свое второе
безусловное обусловленным. Но если бы [здесь] поступательное
движение, т.\,е. снятие первого начала, было вообще
правомерно, то в самом этом первом должна была
бы заключаться возможность соотнесения с ним некоего
иного; оно, стало быть, должно было бы быть чем-то \emph{определенным}.
Однако \emph{бытие} или даже абсолютная субстанция
не выдает себя за таковое. Напротив. Оно есть \emph{непосредственное},
еще всецело \emph{неопределенное}.

Самые красноречивые, быть может, забытые описания
причины того, почему невозможно от некоторой абстракции
прийти к чему-то дальнейшему и к их объединению,
дает Якоби в интересах своей полемики с кантовским априорным
\emph{синтезом} самосознания в своей статье <<О предпринятой
критицизмом попытке довести разум до рассудка>>
(Jac. Werke, III Bd.). Он ставит (стр.\,113) задачу
так, что требуется в чем-то \emph{чистом}, будь то чистое сознание,
чистое пространство или чистое время, обнаружить
возникновение или порождение некоего синтеза. <<Пространство
есть \emph{одно}, время есть \emph{одно}, сознание есть \emph{одно};
скажите же, каким образом какое-либо из этих трех
<<одно>> в самом себе, в своей \emph{чистоте} приобретает характер
многообразия? Каждое из них есть лишь нечто \emph{одно}
и не есть \emph{никакое иное}: одинаковость (Einerleiheit),
<<этот>>, <<эта>>, <<это>> в их \emph{тождестве} (eine Der-Die-Das-Selbigkeit)
без того, чт\'о присуще <<этому>>, <<этой>>, <<этому>>
(ohne Derheit, Dieheit, Dasheit), ибо оно еще дремлет
вместе с <<этот>>, <<эта>>, <<это>> в бесконечности $= 0$ неопределенного,
из которой еще только должно произойти
все и всякое \emph{определенное}! Чем вносится \emph{конечность} в
эти три бесконечности? Что оплодотворяет a priori пространство
и время числом и мерой и превращает их в
нечто \emph{чистое многообразное}? Что приводит в колебание
\emph{чистую спонтанность} (<<Я>>) (Ich)? Каким образом его
чистая гласная получает согласную, или, лучше сказать,
каким образом приостанавливается, прерывая само себя,
его \emph{беззвучное} непрерывное \emph{дуновение}, чтобы приобрести
по крайней мере некоторый род гласной, некоторое
\emph{ударение}?>>~-- Как видно, Якоби очень определенно признавал
абсурдность (Unwesen) абстракции, будь она так
называемое абсолютное, т.\,е. лишь абстрактное, пространство,
или такое же время, или такое же чистое сознание,
<<Я>>. Он настаивает на этом, чтобы доказать, что
продвижение к иному~-- к условию синтеза~-- и к самому
синтезу невозможно. Этот интересующий нас синтез не
следует понимать как связь \emph{внешне} уже имеющихся определений,~--
отчасти дело идет о порождении некоторого
второго, присоединяющегося к некоторому первому, о
порождении некоторого определенного, присоединяющегося
к неопределенному первоначальному, отчасти же об
\emph{имманентном} синтезе, синтезе a priori,~-- о в-себе-и-для-себя-сущем
единстве различных [моментов]. \emph{Становление}
и есть этот имманентный синтез бытия и ничто. Но так
как синтезу ближе всего по смыслу внешнее сведение
вместе [определений], находящихся во внешнем отношении
друг к другу, то справедливо перестали пользоваться названиями
<<синтез>>, <<синтетическое единство>>.~-- Якоби
спрашивает, \emph{каким образом} чистая гласная <<Я>> получает
согласную, чт\'о вносит определенность в неопределенность?
На вопрос: чт\'о?~-- было бы нетрудно ответить, и
Кант по-своему дал ответ на этот вопрос. А вопрос: \emph{как}?
означает: каким способом, по каким отношениям и т.\,п.,
и требует, стало быть, указать некоторую особую категорию;
но о способе, о рассудочных категориях здесь не может
быть и речи. Вопрос: как? сам представляет собой
одну из дурных манер рефлексии, которая спрашивает о
постижимости, но при этом берет предпосылкой свои застывшие
категории и тем самым знает наперед, что она
вооружена против ответа на то, о чем она спрашивает.
Более высокого смысла, заключенного в вопросе о \emph{необходимости}
синтеза, он не имеет также и у Якоби, ибо
последний, как сказано, крепко держится за абстракции,
защищая утверждение о невозможности синтеза. С особенной
наглядностью он описывает (стр.\,147) процедуру,
посредством которой достигают абстракции пространства.
<<Я должен на столь долгое время стараться начисто забыть,
что я когда-либо что-нибудь видел, слышал, к чему-либо
прикасался, причем я определенно не должен
делать исключения и для самого себя. Я должен начисто,
начисто, начисто забыть всякое движение, и это последнее
\emph{забвение} я должен осуществить самым старательным образом
именно потому, что оно всего труднее. И все вообще
я должен всецело и полностью \emph{удалить}, как я его уже
мысленно устранил, и ничего не должен сохранить, кроме
одного лишь \emph{насильственно} остановленного созерцания
одного лишь бесконечного \emph{неизменного пространства}.
Я поэтому не вправе \emph{снова в него мысленно включать} самого
себя как нечто отличное от него и, однако, связанное
с ним; я не вправе просто давать себя \emph{окружить} и
\emph{проникнуться} им, а должен полностью \emph{перейти} в него,
стать с ним единым, превратиться в него; я не должен
ничего оставить от себя, кроме самого \emph{этого моего созерцания},
чтобы рассматривать это созерцание как истинно
самостоятельное, независимое, единое и единственное
представление>>.

При такой совершенно абстрактной чистоте непрерывности,
т.\,е. при этой неопределенности и пустоте представления,
безразлично, будем ли мы называть эту абстракцию
пространством, чистым созерцанием или чистым
мышлением; все это~-- то же самое, чт\'о индус называет
\emph{брамой}, когда он, оставаясь внешне неподвижным и не
побуждаемым никакими ощущениями, представлениями,
фантазиями, вожделениями и т.\,д., годами смотрит лишь
на кончик своего носа и лишь говорит внутренне, в себе,
<<ом, ом, ом>>, или вообще ничего не говорит. Это заглушённое,
пустое сознание, понимаемое как сознание, есть
\emph{бытие}.

В этой пустоте, говорит далее Якоби, с ним происходит
противоположное тому, чт\'о должно было бы произойти
с ним согласно уверению Канта; он ощущает себя не
каким-то \emph{множественным} и \emph{многообразным}, а, наоборот,
единым без всякой множественности, без всякого многообразия;
более того: <<Я сама \emph{невозможность}, \emph{уничтожение}
всякого многообразного и множественного\dots Исходя из
своей чистой, совершенно простой и неизменной сущности,
я \emph{не в состоянии} хоть что-нибудь \emph{восстановить} или
вызвать в себе как призрак\dots Таким образом, в этой чистоте
все внеположное и рядоположное, всякое покоящееся
на нем многообразие и множественность обнаруживаются
как \emph{чистая невозможность}>> (стр.\,149).

Эта невозможность есть не что иное, как тавтология,
она означает, что я упорно держусь абстрактного единства
и исключаю всякую множественность и всякое многообразие,
пребываю в том, чт\'о лишено различий и неопределенно,
и отвращаю свой взор от всего различенного
и определенного. В такую же абстракцию Якоби превращает
кантовский априорный синтез самосознания, т.\,е. деятельность
этого единства, состоящую в том, что оно расщепляет
себя и в этом расщеплении сохраняет само себя.
Этот <<синтез \emph{в себе}>>, <<\emph{первоначальное суждение}>> он односторонне
превращает (стр.\,125) в <<\emph{связку в себе}~-- в
[словечко] <<\emph{есть}>>, <<\emph{есть}>>, <<\emph{есть}>>, без начала и конца и
без <<что>>, <<кто>> и <<какие>>. Это продолжающееся до бесконечности
повторение повторения~-- единственное занятие,
функция и произведение наичистейшего синтеза; сам
синтез есть само голое, чистое, абсолютное повторение>>.
Или, в самом деле, так как в нем нет никакого перерыва
(Absatz), т.\,е. никакого отрицания, различения, то он не
повторение, а только неразличенное простое бытие.~-- Но
есть ли это еще синтез, если Якоби опускает как раз то,
благодаря чему единство есть синтетическое единство?

Если Якоби так укрепился в абсолютном, т.\,е. абстрактном
пространстве, времени, а также сознании, то
прежде всего следует сказать, что он таким образом обитает
и удерживается в чем-то \emph{эмпирически} ложном. \emph{Нет},
т.\,е. эмпирически не существует, такого пространства и
времени, которые были бы чем-то неограниченно пространственным
и временн\'ым, которые не были бы в своей
непрерывности наполнены многообразно ограниченным
наличным бытием и изменением, так что эти границы и
изменения нераздельно и неотделимо принадлежат пространственности
и временности. И точно так же сознание
наполнено определенными чувствами, представлениями,
желаниями и т.\,д.; оно существует нераздельно от какого
бы то ни было особого содержания.~-- Эмпирический
\emph{переход} и без того понятен сам собой; сознание может,
правда, сделать своим предметом и содержанием пустое
пространство, пустое время и само пустое сознание, или
чистое бытие, но оно на этом не останавливается и не
только выходит, но вырывается из такой пустоты, устремляясь
к лучшему, т.\,е. к каким-то образом более конкретному
содержанию, и, как бы плохо ни было в остальном
то или иное содержание, оно постольку лучше и истиннее;
именно такого рода содержание есть синтетическое
содержание вообще, синтетическое в более всеобщем
смысле. Так, Пармениду приходится иметь дело с видимостью
и мнением~-- с противоположностью бытия и
истины; так же Спинозе~-- с атрибутами, модусами, протяжением,
движением, рассудком, волей и т.\,д. Синтез
содержит и показывает неистинность указанных выше
абстракций; в нем они находятся в единстве со своим
иным, следовательно, даны не как сами по себе существующие,
не как абсолютные, а всецело как относительные.

Но речь идет не о показывании эмпирической ничтожности
пустого пространства и т.\,д. Сознание может, конечно,
путем абстрагирования наполнить себя и таким неопределенным
[содержанием], и фиксированные абстракции~--
это \emph{мысли} о чистом пространстве, чистом времени, чистом
сознании, чистом бытии. Должна быть показана
ничтожность мысли о чистом пространстве и т.\,д., т, е.
ничтожность чистого пространства \emph{самого по себе} и т.\,д.,
т.\,е. должно быть показано, что оно, как таковое, уже
есть своя противоположность, что в него самого уже проникла
его противоположность, что оно уже само по себе
есть совершившийся выход (das Herausgegangensein) из
самого себя~-- определенность.

Но это происходит непосредственно в них же. Они,
как подробно описывает Якоби, суть результаты абстракции,
ясно определены как \emph{неопределенное}, которое~-- если
обратиться к его простейшей форме~-- есть бытие. Но
именно эта \emph{неопределенность} и есть то, чт\'о составляет
его определенность; ибо неопределенность противоположна
определенности; она, стало быть, как противоположное,
сама есть определенное, или отрицательное, и
притом чистое, совершенно абстрактное отрицательное.
Эта-то неопределенность или абстрактное отрицание, которое
бытие имеет таким образом в самом себе, и есть то,
чт\'о высказывает и внешняя, и внутренняя рефлексия,
приравнивая бытие к ничто, объявляя его пустым порождением
мысли, ничем.~-- Или можно это выразить
иначе: так как бытие есть то, чт\'о лишено определений, то
оно не (утвердительная) определенность, которая оно
есть, не бытие, а ничто.

В чистой рефлексии начала, каковым в этой логике
является \emph{бытие}, как таковое, переход еще скрыт. Так как
\emph{бытие} положено лишь как непосредственное, то \emph{ничто}
выступает в нем наружу лишь непосредственно. Но все
последующие определения, как, например, \emph{наличное бытие},
более конкретны; в последнем уже \emph{положено} то, чт\'о
содержит и порождает противоречие указанных выше
абстракций, а потому и их переход. Напоминание о том,
что бытие как указанное простое, непосредственное есть
результат полной абстракции и, стало быть, уже потому
абстрактная отрицательность, ничто,~-- это напоминание
оставлено за порогом науки, которая в своих пределах,
особенно в разделе о \emph{сущности}, изобразит эту одностороннюю
\emph{непосредственность} как нечто опосредствованное,
где \emph{положено} бытие как \emph{существование}, а также основание~--
то, что опосредствует это бытие.

С помощью этого напоминания можно представить
или даже, как это называют, \emph{объяснить}\endnotemark{} и \emph{сделать постижимым}
переход бытия в ничто как нечто даже легкое
и тривиальное: бытие, сделанное [нами] началом науки,
есть, разумеется, ничто, ибо абстрагироваться можно от
всего, а когда мы от всего абстрагировались, остается ничто.
Но, можно продолжить, тем самым начало [здесь] не
нечто утвердительное, не бытие, а как раз ничто, и ничто
оказывается в таком случае и \emph{концом}; оно оказывается
этим концом в такой же мере, как непосредственное бытие,
и даже в еще большей мере, чем последнее. Проще
всего дать такому резонерству полную волю и посмотреть,
каковы результаты, которыми оно кичится. То обстоятельство,
что согласно этому ничто оказалось бы результатом
этого резонерства и что теперь следует начинать (как в
китайской философии) с ничто~-- ради этого не стоило бы
и пальцем шевельнуть, ибо раньше, чем мы шевельнули
бы им, это ничто точно так же превратилось бы в бытие
(см. выше: В. Ничто). Но, далее, если бы предполагали
такое абстрагирование от \emph{всего}, а ведь это все есть \emph{сущее},
то следует отнестись к нему более серьезно; результат
абстрагирования от всего сущего~-- это прежде всего абстрактное
бытие, \emph{бытие} вообще; так, в космологическом
доказательстве бытия бога из случайного бытия мира
(в этом доказательстве возвышаются над таким бытием)
бытие поднимается нами выше и приобретает определение
\emph{бесконечного бытия}. Но, разумеется, \emph{можно} абстрагироваться
и от этого чистого бытия, присоединить и бытие
ко всему, от чего уже абстрагировались; тогда остается
ничто. Затем, если решить забыть \emph{мышление} об этом ничто,
т.\,е. о его переходе в бытие, или если бы ничего не
знали об этом, \emph{можно} продолжать в стиле этой возможности;
а именно можно (слава богу!) абстрагироваться также
и от этого ничто (сотворение мира и в самом деле есть
абстрагирование от ничто), и тогда остается не ничто,
ибо как раз от него абстрагируются, а снова приходят к
бытию.~-- Эта \emph{возможность} дает внешнюю игру абстрагирования,
причем само абстрагирование есть лишь одностороннее
действование отрицательного. Сама эта возможность
состоит прежде всего в том, что для нее бытие так
же безразлично, как и ничто, и что в какой мере каждое
из них исчезает, в такой же мере и возникает; но столь
же безразлично, отправляться ли от действования ничто
или от ничто; действование ничто, т.\,е. одно лишь абстрагирование,
есть нечто истинное не больше и не меньше,
чем чистое ничто.

\endnotetext{В <<Феноменологии духа>> Гегель называет объяснение тавтологическим
  движением рассудка, <<которое не только ничего не
  объясняет, но отличается такой ясностью, что, собираясь сказать
  что-нибудь отличное от уже сказанного, оно скорее ничего не
  высказывает, а лишь повторяет то же самое>> (стр.\,85).}

Ту диалектику, в соответствии с которой Платон трактует
в <<Пармениде>> единое, также следует признать больше
диалектикой внешней рефлексии. Бытие и единое
суть оба элеатские формы, представляющие собой одно и
то же. Но их следует также отличать друг от друга. Такими
и берет их Платой в упомянутом диалоге. Удалив
из единого разнообразные определения целого и частей,
бытия в себе и бытия в ином и т.\,д., определения фигуры,
времени и т.\,д., он приходит к выводу, что единому не
присуще бытие, ибо бытие присуще некоторому нечто не
иначе, как в соответствии с одним из указанных способов
(р.\,141, е, Vol. Ill, ed. Steph.)\endnotemark{}. Затем Платон рассматривает
положение: \emph{единое есть}; и у него можно проследить,
каким образом, согласно этому положению, совершается
переход к \emph{небытию} единого: это происходит путем
\emph{сравнения} обоих определений предположенного положения:
\emph{единое есть}. В этом положении содержится единое
\emph{и} бытие, и <<единое есть>> содержит нечто большее, чем
если бы мы сказали лишь: <<единое>>. В том, что они \emph{различны},
раскрывается содержащийся в положении момент
отрицания. Ясно, что этот путь имеет некое предположение
и есть некоторая внешняя рефлексия.

\endnotetext{См. \emph{Платон}. Сочинения в трех томах, т.\,2. М., 1970, стр.\,428.}

Так же как единое приведено здесь в связь с бытием,
так и бытие, которое должно быть фиксировано абстрактно,
\emph{особо}, самым простым образом, не пускаясь в
мышление, раскрывается в связи, содержащей противоположное
тому, чт\'о должно утверждаться. Бытие, взятое
так, как оно есть непосредственно, принадлежит некоторому
\emph{субъекту}, есть нечто высказанное, обладает вообще
некоторым эмпирическим \emph{наличным бытием} и потому
стоит на почве предельного и отрицательного. В каких
бы терминах или оборотах ни выражал себя рассудок,
когда он отвергает единство бытия и ничто и ссылается
на то, чт\'о, дескать, непосредственно наличествует, он как
раз в этом опыте не найдет ничего другого, кроме \emph{определенного}
бытия, бытия с некоторым пределом или отрицанием,~--
не найдет ничего другого, кроме того единства,
которого не признает. Утверждение о непосредственном
бытии сводится таким образом к [утверждению] о
некотором эмпирическом существовании, от \emph{раскрытия}
которого оно не может отказаться, так как оно ведь желает
держаться именно непосредственности, существующей
вне мышления.

Точно так же обстоит дело и с \emph{ничто}, только противоположным
образом, и эта рефлексия известна и довольно
часто применялась к нему. Ничто, взятое в своей непосредственности,
оказывается \emph{сущим}, ибо по своей природе
оно то же самое, что и бытие. Мы мыслим ничто, представляем
его себе, говорим о нем; стало быть, оно \emph{есть};
ничто имеет свое бытие в мышлении, представлении,
речи и т.\,д. Но, кроме того, это бытие также и отлично
от него; поэтому, хотя и говорят, что ничто есть в мышлении,
представлении, но это означает, что не \emph{оно есть}, не
ему, как таковому, присуще бытие, а лишь мышление или
представление есть это бытие. При таком различении
нельзя также отрицать, что ничто находится в \emph{соотношении}
с некоторым бытием; но в этом соотношении, хотя оно
и содержит также различие, имеется единство с бытием.
Как бы ни высказывались о ничто или показывали его,
оно оказывается связанным или, если угодно, соприкасающимся
с некоторым бытием, оказывается неотделимым
от некоторого бытия, а именно находящимся в некотором
\emph{наличном бытии}.

Однако при таком показывании ничто в некотором
наличном бытии обычно все еще предстает следующее
отличие его от бытия: наличное бытие ничто (des Nichts)
вовсе-де не присуще ему самому, оно, само по себе взятое,
не имеет в себе бытия, оно не \emph{есть} бытие, как таковое;
ничто есть-де лишь отсутствие бытия; так, тьма~-- это
лишь \emph{отсутствие} света, холод~-- отсутствие тепла и т.\,д.
Тьма имеет-де значение лишь в отношении к глазу, во
внешнем сравнении с положительным, со светом, и точно
так же холод есть нечто лишь в нашем ощущении; свет
же, тепло, как и бытие, суть сами по себе объективное,
реальное, действенное, обладают совершенно другим качеством
и достоинством, чем указанные отрицательные
[моменты], чем ничто. Часто приводят как очень важное
соображение и значительное знание утверждение, что
тьма есть \emph{лишь отсутствие} света, холод~-- \emph{лишь отсутствие}
тепла. Относительно этого остроумного соображения
можно, оставаясь в этой области эмпирических предметов,
с эмпирической точки зрения заметить, что в
самом деле тьма оказывается действенным в свете, обусловливая
то, что свет становится цветом\endnotemark{} и лишь благодаря
этому сообщая ему зримость, ибо, как мы сказали
раньше, в чистом свете так же ничего не видно, как и в
чистой тьме. А зримость есть такая действенность в глазу,
в которой указанное отрицательное принимает такое
же участие, как и свет, считающийся реальным, положительным;
и точно так же холод дает себя достаточно почувствовать
воде, нашему ощущению и т.\,д., и если мы
ему отказываем в так называемой объективной реальности,
то от этого в нем ничего не убывает. Но, далее, достойно
порицания то, что здесь, как и выше, говорят о
чем-то отрицательном, обладающем определенным содержанием,
идут дальше самого ничто, по сравнению с которым
у бытия не больше и не меньше пустой абстрактности.~--
Однако следует тотчас же брать холод, тьму
и тому подобные определенные отрицания сами по себе и
посмотреть, чт\'о этим положено в отношении их всеобщего
определения, с которым мы теперь имеем дело. Они
должны быть не ничто вообще, а ничто света, тепла и т.\,д.,
ничто чего-то определенного, какого-то содержания; они,
таким образом, если можно так выразиться, определенные,
содержательные ничто. Но определенность, как мы
это еще увидим дальше, сама есть отрицание; таким образом,
они отрицательные ничто; но отрицательное ничто
есть нечто утвердительное. Превращение ничто через его
определенность (которая раньше представала как некое
\emph{наличное бытие} в субъекте или в чем бы то ни было другом)
в некоторое утвердительное представляется сознанию,
застревающему в рассудочной абстракции, как верх
парадоксальности; как ни прост взгляд, что отрицание
отрицания есть положительное, он, быть может, именно
из-за самой этой его простоты представляется чем-то
тривиальным, с которым гордому рассудку нет поэтому
надобности считаться, хотя это имеет свое основание,~--
а между тем оно не только имеет свое основание, но благодаря
всеобщности таких определений обладает бесконечным
распространением и всеобщим применением, так
что все же следовало бы с ним считаться.

\endnotetext{Гегель излагает теорию цвета Гёте.}

Относительно определения перехода бытия и ничто
друг в друга можно еще заметить, что этот переход следует
постигать, не прибегая к дальнейшим определениям
рефлексии. Он непосредствен и всецело абстрактен из-за
абстрактности переходящих моментов, т.\,е. вследствие
того, что в этих моментах еще не положена определенность
другого, посредством чего они переходили бы друг
в друга; ничто еще не \emph{положено} в бытии, хотя бытие есть
\emph{по своему существу} ничто, и наоборот. Поэтому недопустимо
применять здесь дальнейшие определенные опосредствования
и понимать бытие и ничто находящимися
в каком-то отношении,~-- этот переход еще не отношение.
Недозволительно, стало быть, говорить: ничто~-- \emph{основание}
бытия или бытие~-- \emph{основание} ничто; ничто~-- \emph{причина}
бытия и т.д.; или сказать: переход в ничто возможен
лишь \emph{при условии}, что нечто \emph{есть}, или: переход в бытие
возможен лишь \emph{при условии}, что есть небытие. Род соотношения
не может получить дальнейшее определение,
если бы не были в то же время далее определены соотносящиеся
\emph{стороны}. Связь основания и следствия и т.\,д.
имеет своими сторонами, которые она связывает, уже не
просто бытие и просто ничто, а непременно такое бытие,
которое есть основание, и нечто такое, что, хотя и есть
лишь нечто положенное, несамостоятельное, однако не
есть абстрактное ничто.

%%% Local Variables:
%%% mode: latex
%%% TeX-master: "../../../main"
%%% End:


\subsubsection{Примечание 4. [Непостижимость начала]}

Из предшествующего ясно видно, как обстоит дело с
диалектикой, отрицающей \emph{начало мира} и [возможность]
его гибели, с диалектикой, которая должна была доказать
\emph{вечность} материи, т.\,е. с диалектикой, отрицающей \emph{становление},
возникновение или прехождение вообще.~--
Кантовскую антиномию конечности или бесконечности
мира в пространстве и времени мы подробнее рассмотрим
ниже, когда будем рассматривать понятие количественной
бесконечности.~-- Указанная простая, тривиальная диалектика
основана на отстаивании противоположности
между бытием и ничто. Невозможность начала мира или
чего бы то ни было доказывается следующим образом.

Нет ничего такого, что могло бы иметь начало, ни поскольку
нечто есть, ни поскольку его нет; в самом деле,
поскольку оно есть, оно уже не начинается, а поскольку
его нет, оно также не начинается.~-- Если бы мир или нечто
имели начало, то он имел бы начало в ничто, но в
ничто нет начала или, иначе говоря, ничто не есть начало,
ведь начало заключает в себе некое бытие, а ничто не
содержит никакого бытия. Ничто~-- это только ничто.
В основании, причине и т.\,д.~-- если так определяют ничто,~--
содержится некое утверждение, бытие. На этом же
основании нечто не может и прекратиться. Ибо в таком
случае бытие должно было бы содержать ничто, но бытие~--
это только бытие, а не своя противоположность.

Ясно, что против становления или начала и прекращения,
против этого \emph{единства} бытия и ничто здесь не
приводится никакого доказательства, а его лишь ассерторически
отрицают и приписывают истинность бытию и
ничто в их отдельности друг от друга.~-- Однако эта диалектика
по крайней мере последовательнее рефлектирующего
представления. Последнее считает полной истиной,
что бытие и ничто существуют лишь раздельно, а, с
другой стороны, признает начало и прекращение столь
же истинными определениями; но, признавая их, оно
фактически принимает нераздельность бытия и ничто.

Разумеется, при предположении абсолютной раздельности
бытия и ничто начало или становление есть~--
это можно весьма часто слышать~-- нечто \emph{непостижимое}.
Ведь те, кто делает это предположение, упраздняют начало
или становление, которое, однако, они \emph{снова} допускают,
и это противоречие, которое они сами же создают
и разрешение которого они делают невозможным, они называют
\emph{непостижимостью}.

Изложенное выше и есть та же диалектика, какой
пользуется рассудок против даваемого высшим анализом
понятия \emph{бесконечно малых величин}. Это понятие будет
подробнее рассмотрено ниже.~-- Величины эти определены
как величины, \emph{существующие в своем исчезновении}~--
не \emph{до} своего исчезновения, ибо в таком случае они конечные
величины, и не \emph{после} своего исчезновения, ибо в таком
случае они ничто. Против этого чистого понятия
было выдвинуто постоянно повторявшееся возражение,
что такие величины суть \emph{либо} нечто, \emph{либо} ничто и что
нет \emph{промежуточного состояния} (<<состояние>> здесь неподходящее,
варварское выражение) между бытием и небытием.~--
При этом опять-таки признают абсолютную раздельность
бытия и ничто. Но против этого было показано,
что бытие и ничто суть на самом деле одно и то же или,
говоря языком выдвигающих это возражение, \emph{нет} ничего
такого, что не было бы \emph{промежуточным состоянием между
бытием и ничто}. Математика обязана своими самыми блестящими
успехами тому, что она приняла то определение,
которого не признает рассудок.

Приведенное рассуждение, делающее ложное предположение
об абсолютной раздельности бытия и небытия и
не идущее дальше этого предположения, следует называть
не \emph{диалектикой}, а \emph{софистикой}. В самом деле, софистика
есть резонерство, исходящее из необоснованного предположения,
истинность которого признается без критики и
необдуманно. Диалектикой же мы называем высшее разумное
движение, в котором такие кажущиеся безусловно
раздельными [моменты] переходят друг в друга благодаря
самим себе, благодаря тому, чт\'о они суть, и предположение
[об их раздельности] снимается. Диалектическая, имманентная
природа самого бытия и ничто в том и состоит,
что они свое единство~-- становление~-- обнаруживают
как свою истину.


%%% Local Variables:
%%% mode: latex
%%% TeX-master: "../../../main"
%%% End:


\subsection{Моменты становления: возникновение и прехождение}

Становление есть нераздельность бытия и ничто~-- не
единство, абстрагирующееся от бытия и ничто; как единство
\emph{бытия} и \emph{ничто} оно есть это \emph{определенное} единство,
или, иначе говоря, такое единство, в котором \emph{есть} и бытие,
и ничто. Но так как каждое из них, и бытие, и ничто,
нераздельно от своего иного, то \emph{их нет}. Они, следовательно,
\emph{суть} в этом единстве, но как исчезающие, лишь
как \emph{снятые}. Теряя свою \emph{самостоятельность}, которая, как
первоначально представлялось, была им присуща, они
низводятся до \emph{моментов, еще различимых}, но в то же
время снятых.

Взятые со стороны этой своей различимости, каждый
из них есть \emph{в этой различимости} единство с \emph{иным}. Становление
содержит, следовательно, бытие и ничто как \emph{два
таких единства}, каждое из которых само есть единство
бытия и ничто. Одно из них есть бытие как непосредственное
бытие и как соотношение с ничто; другое есть ничто
как непосредственное ничто и как соотношение с бытием.
Определения обладают в этих единствах неодинаковой
ценностью.

Становление дано, таким образом, в двояком определении;
в одном определении ничто есть непосредственное,
т.\,е. определение начинает с ничто, соотносящегося
с бытием, т.\,е. переходящего в него; в другом бытие дано
как непосредственное, т.\,е. определение начинает с бытия,
переходящего в ничто,~-- \emph{возникновение} и \emph{прехождение}.

Оба суть одно и то же, становление, и даже как эти
направления, различенные таким образом, они друг друга
проникают и парализуют. Одно есть \emph{прехождение}; бытие
переходит в ничто; но ничто есть точно так же и своя
противоположность, переход в бытие, возникновение. Это
возникновение есть другое направление; ничто переходит
в бытие, но бытие точно так же и снимает само себя и
есть скорее переход в ничто, есть прехождение.~-- Они не
снимают друг друга, одно внешне не снимает другое,
каждое из них снимает себя в себе самом (an sich selbst)
и есть в самом себе (an ihm selbst) своя противоположность.


%%% Local Variables:
%%% mode: latex
%%% TeX-master: "../../../main"
%%% End:


\subsection{Снятие становления}

Равновесие, в которое приводят себя возникновение
и прехождение,~-- это прежде всего само становление. Но
становление точно так же сходится (geht zusammen) в
\emph{спокойное единство}. Бытие и ничто находятся в становлении
лишь как исчезающие; становление же, как таковое,
имеется лишь благодаря их разности. Их исчезание
есть поэтому исчезание становления, иначе говоря, исчезание
самого исчезания. Становление есть неустойчивое
беспокойство, которое оседает, переходя в некоторый
спокойный результат.

Это можно было бы выразить и так: становление есть
исчезание бытия в ничто и ничто~-- в бытие, и исчезание
бытия и ничто вообще; но в то же время оно основывается
на различии последних. Оно, следовательно, противоречит
себе внутри самого себя, так как соединяет
в себе нечто противоположное себе; но такое соединение
разрушает себя.

Этот результат есть исчезновение (Verschwundensein),
но не как \emph{ничто}; в последнем случае он был бы лишь возвратом
к одному из уже снятых определений, а не результатом
ничто и \emph{бытия}. Этот результат есть ставшее
спокойной простотой единство бытия и ничто. Но спокойная
простота есть \emph{бытие}, однако бытие уже более не для
себя, а бытие как определение целого.

Становление как переход в такое единство бытия и
ничто, которое дано как \emph{сущее} или, иначе говоря, имеет
вид одностороннего \emph{непосредственного} единства этих моментов,
есть \emph{наличное бытие}.


%%% Local Variables:
%%% mode: latex
%%% TeX-master: "../../../main"
%%% End:


\subsubsection{Примечание . [Выражение <<снятие>>]}

\emph{Снятие} (Aufheben) и \emph{снятое} (\emph{идеальное}~"--- ideelle)~"---
одно из важнейших понятий философии, одно из главных
определений, которое встречается решительно всюду и
смысл которого следует точно понять и в особенности отличать
от ничто.~"--- Оттого, что нечто снимает себя, оно
не превращается в ничто. Ничто есть \emph{непосредственное};
снятое же есть нечто \emph{опосредствованное}: оно не-сущее, но
как \emph{результат}, имевший своим исходным пунктом некоторое
бытие, поэтому оно \emph{еще} имеет \emph{в себе определенность,
от которой оно происходит}.

Aufheben имеет в немецком языке двоякий смысл: оно
означает сохранить, \emph{удержать} и в то же время прекратить,
\emph{положить конец}. Само сохранение уже заключает в
себе отрицательное в том смысле, что для того, чтобы
удержать нечто, его лишают непосредственности и тем самым
наличного бытия, открытого для внешних воздействий.
Таким образом, снятое есть в то же время и сохраненное,
которое лишь потеряло свою непосредственность,
но от этого не уничтожено.~"--- Указанные два определения
\emph{снятия} можно лексически привести как два \emph{значения}
этого слова, но должно представляться странным, что в
языке одно и то же слово обозначает два противоположных
определения. Для спекулятивного мышления отрадно
находить в языке слова, имеющие в самих себе спекулятивное
значение; в немецком языке много таких слов.
Двоякий смысл латинского слова tollere (ставший знаменитым
благодаря остроумному выражению Цицерона:
tollendum esse Octavium)
\endnote{Латинское слово tollere многозначно. Среди его значений
  имеются и такие противоположные, как \emph{возносить}, \emph{возвеличивать}
  и \emph{убирать}, \emph{устранять}. Поэтому приведенное Гегелем выражение
  Цицерона двусмысленно: оно может быть понято и как
  <<д\'олжно вознести Октавия>> и как <<д\'олжно убрать Октавия>>.}
не идет так далеко: утвердительное
определение доходит лишь до [понятия] возвышения.
Нечто снято лишь постольку, поскольку оно вступило
в единство со своей противоположностью; для него,
взятого в этом более точном определении как нечто рефлектированное,
подходит название \emph{момента}. \emph{Вес} и \emph{расстояние}
от некоторой точки называются в рычаге его механическими
\emph{моментами} из-за \emph{тождественности} оказываемого
ими действия при всем прочем различии менаду
такой реальностью, как вес, и такой идеальностью, как чисто
пространственное определение, линия (см. <<Энциклопедию
философских наук>>, изд. 3-е, \S\,261, примечание).~"---
Еще чаще придется обращать внимание на то, что в
философской терминологии рефлектированные определения
обозначены латинскими терминами
\endnote{Гегель имеет в виду прежде всего латинское слово momentum
  (момент), которое он употребляет в значении, близком к Aufgehobenes (снятое).}
либо потому,
что в родном языке для них нет терминов, либо же, если,
как в данном случае, в нем имеются такие термины, потому,
что термин, которым располагает родной язык,
больше напоминает о непосредственном, а иностранный
термин~"--- больше о рефлектированном.

Более точный смысл и выражение, которые бытие и
ничто получают, поскольку они стали теперь \emph{моментами},
должны выявиться при рассмотрении наличного бытия,
как единства, в котором они сохранены. Бытие есть бытие
и ничто есть ничто лишь в их отличии друг от друга;
но в их истине, в их единстве, они исчезли как эти
определения и суть теперь иное. Бытие и ничто суть одно
и то же; \emph{именно потому, что они одно и то же, они
уже не бытие и ничто} и имеют различное определение:
в становлении они были возникновением и прехождением;
в наличном бытии как по-иному определенном единстве
они опять-таки по-иному определенные моменты.
Это единство остается отныне их основой, которую они
уже больше не покинут, чтобы не возвращаться к абстрактному
значению бытия и ничто.


%%% Local Variables:
%%% mode: latex
%%% TeX-master: "../../../main"
%%% End:


\chapter{Наличное бытие}

Наличное бытие есть \emph{определенное} бытие; его определенность
есть \emph{сущая} определенность, \emph{качество}. Своим
качеством \emph{нечто} противостоит иному, оно изменчиво\endnotemark{}
и \emph{конечно}, определено всецело отрицательно не только в
отношении иного, но и в самом себе. Это его отрицание
прежде всего по отношению к конечному нечто есть \emph{бесконечное};
абстрактная противоположность, в которой выступают
эти определения, разрешается в лишенную противоположности
бесконечность, в \emph{для-себя-бытие}.

\endnotetext{Ein Anderes~"--- иное; veränderlich (изменчиво)~"--- буквально:
  способно стать иным.}

Таким образом, исследование наличного бытия распадается
на следующие три раздела:

\begin{enumerate}[label=\Alph*)]
\item \emph{Наличное бытие, как таковое},
\item \emph{Нечто и иное, конечность},
\item \emph{Качественная бесконечность}.
\end{enumerate}


%%% Local Variables:
%%% mode: latex
%%% TeX-master: "../../../main"
%%% End:


\section{Наличное бытие, как таковое}

В наличном бытии

\begin{enumerate}[label=\alph*)]
\item \emph{как таковом} следует прежде всего различать его
определенность
\item как \emph{качество}. Качество же следует брать и в одном,
и в другом определении наличного бытия: как \emph{реальность}
и как \emph{отрицание}. Но в этих определенностях наличное бытие
также и рефлектировано в себя, и положенное как
таковое оно есть
\item \emph{нечто}, наличио сущее.
\end{enumerate}


%%% Local Variables:
%%% mode: latex
%%% TeX-master: "../../../../main"
%%% End:


\subsection{Наличное бытие вообще}

Из становления возникает наличное бытие. Наличное
бытие есть простое единство (Einssein) бытия и ничто.
Из-за этой простоты оно имеет форму чего-то \emph{непосредственного}.
Его опосредствование, становление, находится
позади него; это опосредствование сняло себя, и наличное
бытие предстает поэтому как некое первое, из которого
исходят. Оно прежде всего в одностороннем определении
\emph{бытия}; другое содержащееся в нем определение,
\emph{ничто}, равным образом проявится в нем как противостоящее
первому.

Оно не просто бытие, а \emph{наличное бытие}; взятое этимологически,
Dasein означает бытие в каком-то \emph{месте}; но
представление о пространстве здесь не приложимо. Наличное
бытие есть вообще по своему становлению \emph{бытие}
с некоторым \emph{небытием}, так что это небытие принято в
простое единство с бытием. \emph{Небытие}, принятое в бытие
таким образом, что конкретное целое имеет форму бытия,
непосредственности, составляет \emph{определенность}, как таковую.

\emph{Целое} также имеет форму, т.\,е. \emph{определенность} бытия,
так как и бытие обнаружило себя в становлении только
как снятый, отрицательно определенный момент\endnotemark{}; но
таково оно \emph{для нас в нашей рефлексии}; оно еще не \emph{положено}
в самом себе. Определенность же наличного бытия,
как таковая, есть положенная определенность, на что
указывает и термин <<\emph{наличное} бытие>>.~"--- Следует всегда
строго различать между тем, чт\'о есть для нас, и тем,
чт\'о положено; лишь то, чт\'о \emph{положено} в каком-то понятии,
входит в рассмотрение, развивающее это понятие,
входит в его содержание. Определенность же, еще не положенная
в нем самом~"--- все равно, касается ли она
природы самого понятия или она есть внешнее сравнение,~"---
принадлежит нашей рефлексии; обращая внимание
на определенность этого рода, можно лишь уяснить
или предварительно наметить путь, который обнаруживается
в самом развитии [понятия]. Что целое, единство
бытия и ничто, имеет \emph{одностороннюю определенность
бытия},~"--- это внешняя рефлексия. В отрицании же, в
нечто и \emph{ином} и т.\,д., это единство дойдет до того, что
окажется \emph{положенным}.~"--- Следовало здесь обратить внимание
на это различие; но давать себе отчет обо всем,
чт\'о рефлексия может позволить себе заметить,~"--- излишне;
это привело бы к слишком пространному изложению,
к предвосхищению того, чт\'о должно вытекать из самого
предмета (Sache). Хотя такого рода рефлексии и могут
облегчить обзор целого и тем самым и понимание, однако
они невыгодны тем, что выглядят неоправданными
утверждениями, основаниями и основами последующего.
Не надо поэтому придавать им большее значение, чем то,
которое они должны иметь, и надлежит отличать их от
того, чт\'о составляет момент в развитии самого предмета.

\endnotetext{Эта фраза в издании 1833\,г. (воспроизведенном Глокнером
  в 1928\,г.) дана с несколько необычными знаками препинания, что
  дало повод Лассону изменить пунктуацию, прибавив тире перед
  словами <<denn Sein hat\dots>>. Тем самым глагол <<ist>> приобрел
  в главном предложении значение связки, тогда как согласно
  пунктуации, даваемой в издании 1833\,г., его следует понимать
  в смысле самостоятельного глагола (<<имеется в форме>> или <<имеет
  форму>>). Если принять пунктуацию Лассона, то всю эту фразу
  надо перевести так: <<Это \emph{целое} также в форме, т.\,е. \emph{определенности}
  бытия (так как и бытие обнаружило себя в становлении
  имеющим характер всего лишь момента) есть нечто снятое, отрицательно
  определенное>>. Сопоставление этого места с серединой
  данного абзаца (<<что целое, единство бытия и ничто, имеет одностороннюю
  определенность бытия,~"--- это внешняя рефлексия>>)
  заставляет предпочесть интерпретацию Б.\,Г.~Столпнера, которая
  и принята в настоящем томе.}

Наличное бытие соответствует \emph{бытию} предшествующей
сферы; однако бытие есть неопределенное, поэтому
в нем не получается никаких определений. Наличное же
бытие есть определенное бытие, \emph{конкретное}; поэтому в
нем сразу же выявляется несколько определений, различенные
отношения его моментов.


%%% Local Variables:
%%% mode: latex
%%% TeX-master: "../../../../main"
%%% End:


\subsection{Качество}

Ввиду непосредственности, в которой бытие и ничто
едины в наличном бытии, они не выходят за пределы
друг друга; насколько наличное бытие есть сущее, настолько
же оно есть небытие, определено. Бытие не есть
\emph{всеобщее}, определенность не есть \emph{особенное}. Определенность
еще \emph{не отделилась от бытия}; правда, она уже не
будет отделяться от него, ибо лежащее отныне в основе
истинное есть единство небытия с бытием; на этом единстве
как на основе зиждутся все дальнейшие определения.
Но соотношение здесь определенности с бытием
есть непосредственное единство обоих, так что еще не
положено никакого различения их.

Определенность как изолированная сама по себе, как
\emph{сущая} определенность, есть \emph{качество}~-- нечто совершенно
простое, непосредственное. \emph{Определенность} вообще есть
более общее, которое точно так же может быть и количественным,
и далее определенным. Ввиду этой простоты
нечего более сказать о качестве, как таковом.

Но наличное бытие, в котором содержатся и ничто, и
бытие, само служит масштабом для односторонности качества
как лишь \emph{непосредственной} или \emph{сущей} определенности.
Качество должно быть положено и в определении
ничто, благодаря чему непосредственная или \emph{сущая} определенность
полагается как некая различенная, рефлектированная
определенность и, таким образом, ничто как то,
чт\'о определенно в некоторой определенности, есть также
нечто рефлектированное, некое \emph{отрицание}. Качество, взятое
таким образом, чтобы оно, будучи различенным, считалось
\emph{сущим}, есть \emph{реальность}; оно же, обремененное некоторым
отрицанием, есть \emph{отрицание} вообще, а также
некоторое качество, считающееся, однако, недостатком и
определяющееся в дальнейшем как граница, предел.

Оба суть наличное бытие; но в \emph{реальности} как качестве,
в котором акцентируется то, что оно \emph{сущее}, скрыто
то обстоятельство, что оно содержит определенность, следовательно,
и отрицание; реальность считается поэтому
лишь чем-то положительным, из которого исключены
отрицание, ограниченность, недостаток. Отрицание только
как недостаток было бы то же, что и ничто; но оно
наличное бытие, качество, только определяемое посредством
небытия.


%%% Local Variables:
%%% mode: latex
%%% TeX-master: "../../../../main"
%%% End:


\subsubsection{Примечание . [Реальность и отрицание]}

<<Реальность>> может показаться многозначным словом,
так как оно употребляется для обозначения разных
и даже противоположных определений. В философском
смысле говорят, например, о \emph{чисто эмпирической} реальности
как о лишенном ценности наличном бытии. Но
когда говорят о мыслях, понятиях, теориях, что они \emph{лишены
реальности}, то это означает, что у них нет \emph{действительности},
хотя \emph{в себе}, или в понятии, идея, например
платоновской республики, может, дескать, быть истинной.
Здесь не отрицается за идеей ее ценность, и \emph{наряду}
с реальностью допускают и ее. Но сравнительно с так называемыми
\emph{голыми} идеями, с \emph{голыми} понятиями реальное
считается единственно истинным.~"--- Смысл, в котором
внешнему наличному бытию приписывается решение
вопроса об истинности того или иного содержания,
столь же односторонен, как односторонне представление,
будто для идеи, сущности или даже внутреннего чувства
безразлично внешнее наличное бытие, и еще в большей
мере односторонне мнение о том, что они тем превосходнее,
чем более они отдалены от реальности.

Рассматривая термин <<реальность>>, следует коснуться
прежнего метафизического \emph{понятия бога}, из которого
исходило прежде всего так называемое онтологическое
доказательство бытия бога. Бога определяли как \emph{совокупность
всех реальностей}, и об этой совокупности говорилось,
что она не заключает в себе никакого противоречия,
что ни одна из реальностей не снимает другую; ибо
реальность следует, мол, понимать лишь как некоторое
совершенство, как нечто \emph{утвердительное}, не содержащее
никакого отрицания. Реальности, стало быть, не противоположны
и не противоречат друг другу.

При таком понимании реальности предполагают, что
она остается и тогда, когда мысленно устраняют всякое
отрицание; однако этим снимается всякая определенность
реальности. Реальность есть качество, наличное
бытие; тем самым она содержит момент отрицательности,
и лишь благодаря этому она есть то определенное, которое
она есть. В так называемом \emph{эминентном смысле}\endnotemark{}
или как \emph{бесконечная}~"--- в обычном значении этого слова,
т.\,е. в том смысле, в котором ее будто бы следует понимать,~"---
она становится неопределенной и теряет свое
значение. Божественная благость, утверждали, есть благость
не в обычном смысле, а в эминентном; она не отлична
от справедливости, а \emph{умеряется} (\emph{лейбницевское}
примиряющее выражение) ею, как и, наоборот, справедливость
умеряется благостью; таким образом, благость
уже перестает быть благостью и справедливость~"--- справедливостью.
Могущество [бога], говорят, умеряется [его]
мудростью, но в таком случае оно уже не могущество,
как таковое, ибо оно было бы подчинено мудрости; мудрость
[бога], утверждают, расширяется до могущества, но
в таком случае она исчезает как мудрость, определяющая
цель и меру. Истинное понятие бесконечного и его
\emph{абсолютное} единство~"--- понятие, к которому мы придем
позднее,~"--- нельзя понимать как \emph{умерение, взаимное ограничение}
или \emph{смешение}; это~"--- поверхностное, окутанное
неопределенным туманом соотношение, которым может
удовлетворяться лишь чуждое понятия представление,~"---
Реальность, как ее берут в указанной выше дефиниции
бога, т.\,е. реальность как определенное качество, выведенное
за пределы своей определенности, перестает быть реальностью;
оно превращается в абстрактное бытие; бог как
\emph{чисто} реальное во всем реальном или как \emph{совокупность}
всех реальностей так же лишен определения и содержания,
как и пустое абсолютное, в котором все есть одно.

\endnotetext{Термин \emph{эминентный} (eminenter) заимствован из средневековой
  философии и обозначает высшую степень какого-нибудь
  качества или свойства.}

Если же, напротив, брать реальность в ее определенности,
то ввиду того, что она содержит как нечто сущностное
момент отрицательности, совокупность всех реальностей
становится также совокупностью всех отрицаний,
совокупностью всех противоречий, прежде всего
абсолютным \emph{могуществом}, в котором все определенное
поглощается; но так как само это могущество имеется
лишь постольку, поскольку оно имеет против себя нечто,
еще не снятое им, то, когда его мыслят как могущество,
ставшее осуществленным, беспредельным, оно превращается
в абстрактное ничто. То реальное во всяком реальном,
\emph{бытие} во всяком \emph{наличном бытии}, которое будто
бы выражает понятие бога, есть не что иное, как абстрактное
бытие, то же, что и ничто.

Определенность есть отрицание, положенное как утвердительное,
это~"--- положение Спинозы: omnis determinatio
est negatio\endnotemark{}. Это чрезвычайно важное положение;
только надо сказать, что отрицание, как таковое, есть
бесформенная абстракция. Но не следует обвинять спекулятивную
философию в том, что для нее отрицание
или ничто есть нечто последнее; оно не есть для нее
последнее, как и реальность не есть для нее истинное.

\endnotetext{Выражение <<omnis determinate est negatio>> у Спинозы
  нигде нет. У него встречается <<determinate est negatio>>~"--- в
  письме Яриху Иеллесу от 2 июня 1674\,г. (см.~Б.~Спиноза. Избранные
  произведения в двух томах, т.\,2. М., 1957, стр.\,568). Из текста
  письма явствует, что determinatio означает в данном случае не
  <<определение>>, а <<ограничение>>.}

Необходимым выводом из положения о том, что определенность
есть отрицание, является \emph{единство спинозовской
субстанции} или то, что существует лишь одна субстанция.
\emph{Мышление} и \emph{бытие}, или протяжение, эти два определения,
рассматриваемые Спинозой, должны были быть
сведены им в одно в этом единстве, ибо как определенные
реальности они отрицания, бесконечность которых
есть их единство; согласно спинозовской дефиниции, о
которой будет сказано ниже, бесконечность [всякого] нечто
есть его утверждение. Он понимал поэтому оба определения
как атрибуты, т.\,е. как такие, которые не имеют
отдельного существования, в-себе-и-для-себя-бытия, а даны
лишь как снятые, как моменты; или, правильнее
сказать, они для него даже и не моменты, ибо субстанция
совершенно лишена определений в самой себе, а
атрибуты, равно как и модусы, суть различения, делаемые
внешним рассудком.~"--- Точно так же несовместима с этим
положением субстанциальность индивидов. Индивид есть
соотношение с собой в силу того, что он ставит границы
всему иному; но тем самым эти границы суть также
и границы его самого, суть соотношения с иным; он
не имеет своего наличного бытия в самом себе. Индивид,
правда, есть нечто \emph{большее}, чем только во всех отношениях
ограниченное, но это <<большее>> относится к другой
сфере~"--- понятия; в метафизике бытия он всецело определен;
и против того, чтобы индивид, чтобы конечное, как
таковое, существовало в себе и для себя, выступает определенность
в своем существе как отрицание и увлекает
конечное в то же отрицательное движение рассудка, которое
заставляет все исчезать в абстрактном единстве,
в субстанции.

Отрицание непосредственно противостоит реальности;
в дальнейшем, в сфере собственно рефлектированных
определений, оно противопоставляется \emph{положительному},
которое есть рефлектирующая в отрицание реальность,~"---
реальность, в которой \emph{светится} то отрицательное, которое
еще скрыто в реальности, как таковой.

Качество есть \emph{свойство} прежде всего лишь в том
смысле, что оно в некотором \emph{внешнем соотношении} показывает
себя \emph{имманентным определением}. Под свойствами,
например трав, понимают определения, которые
не только вообще \emph{свойственны} тому или иному нечто,
а свойственны ему постольку, поскольку благодаря им
оно присущим ему образом \emph{сохраняет} себя в соотношении
с иным, не дает воли внутри себя посторонним
положенным в нем воздействиям, а само \emph{показывает} в
ином \emph{силу} своих собственных определений, хотя и не
отстраняет от себя этого иного. Напротив, более спокойные
определенности, как, например, фигура, внешний вид,
не называют свойствами, как, впрочем, и не качествами,
поскольку их представляют себе изменчивыми, не тождественными
с \emph{бытием}.

Qualierung или Inqualierung~"--- термин философии
Якоба Бёме, проникающей вглубь, но в смутную глубь,~"---
означает движение того или иного качества (кислого,
терпкого, горячего и т.\,д.) в самом себе, поскольку оно
в своей отрицательной природе (в своей Qual\endnotemark{}) выделяется
из другого и укрепляется, поскольку оно вообще
есть свое собственное беспокойство в самом себе, сообразно
которому оно порождает и сохраняет себя лишь
в борьбе.

\endnotetext{Я.~Бёме этимологически связывал немецкое слово Qual
  (м\'ука) с латинским словом qualitas (качество). Искусственно
  образованные Бёме слова Qualierung и Inqualierung можно было
  бы перевести буквально <<качествование>> и <<вкачествование>>.}


%%% Local Variables:
%%% mode: latex
%%% TeX-master: "../../../../main"
%%% End:


\subsection{Нечто}

В наличном бытии мы различили его определенность
как качество; в качестве как налично сущем \emph{есть} различие~--
различие реальности и отрицания. Насколько
эти различия имеются в наличном бытии, настолько же
они ничтожны и сняты. Сама реальность содержит отрицание,
есть наличное, а не неопределенное, абстрактное
бытие. И точно так же отрицание есть наличное бытие;
оно не абстрактное, как считают, ничто, оно здесь положено
так, как оно есть в себе, как сущее, принадлежащее
к наличному бытию. Таким образом, качество вообще не
отделено от наличного бытия, которое есть лишь определенное,
качественное бытие.

Это снятие различения есть больше, чем только отказ
от него и еще одно внешнее отбрасывание его или простой
возврат к простому началу, к наличному бытию,
как таковому. Различие не может быть отброшено, ибо
оно \emph{есть}. Фактическое, стало быть, то, что имеется, есть
наличное бытие вообще, различие в нем и снятие этого
различия; не наличное бытие, лишенное различий, как
вначале, а наличное бытие как \emph{снова} равное самому себе
\emph{благодаря снятию различия}, как простота наличного
бытия, \emph{опосредствованная} этим снятием. Эта снятость
различия есть отличительная определенность наличного
бытия. Таким образом, оно есть \emph{внутри-себя-бытие}; наличное
бытие есть \emph{налично сущее}, \emph{нечто}.

Нечто есть \emph{первое отрицание отрицания} как простое
сущее соотношение с собой. Наличное бытие, жизнь,
мышление и т.\,д. в своей сущности определяют себя как
\emph{налично сущее, живое, мыслящее} (<<Я>>) и т.\,д. Это определение
в высшей степени важно, если хотят идти дальше
наличного бытия, жизни, мышления и т.\,д., а также
божественности (вместо бога) как всеобщностей. Представление
справедливо считает \emph{нечто реальным}. Однако
\emph{нечто} есть еще очень поверхностное определение, подобно
тому как \emph{реальность} и \emph{отрицание}, наличное бытие и
его определенность, хотя они уже не пустые бытие и
ничто, однако суть совершенно абстрактные определения.
Поэтому они и самые ходячие выражения, и философски
необразованная рефлексия чаще всего пользуется
ими, втискивает в них свои различения и мнит, будто
имеет в них что-то вполне добротное и строго определенное.~--
Отрицание отрицания как \emph{нечто} есть лишь начало
субъекта,~-- внутри-себя-бытие, еще совершенно неопределенное.
В дальнейшем оно определяет себя прежде
всего как сущее для себя, продолжая определять себя до
тех пор, пока оно не получит лишь в понятии конкретную
напряженность субъекта. В основе всех этих определений
лежит отрицательное единство с собой. Но при этом
следует различать между отрицанием как \emph{первым}, как
отрицанием \emph{вообще}, и вторым, отрицанием отрицания,
которое есть конкретная, \emph{абсолютная} отрицательность,
так Же как первое отрицание есть, напротив, лишь
\emph{абстрактная} отрицательность.

\emph{Нечто} есть \emph{сущее} как отрицание отрицания; ибо последнее~--
это восстановление простого соотношения с собой;
но тем самым нечто есть также и \emph{опосредствование
себя} с \emph{самим собой}. Уже в простоте [всякого] нечто, а
затем еще определеннее в для-себя-бытии, субъекте
и т.\,д. имеется опосредствование себя с самим собой; оно
имеется уже и в становлении, но в нем оно лишь совершенно
абстрактное опосредствование. В нечто опосредствование
с собой \emph{положено}, поскольку нечто определено
как простое \emph{тождественное}.~-- Можно обратить внимание
на то, что вообще имеется опосредствование, в противовес
принципу утверждаемой чистой непосредственности
знания, из которой опосредствование будто бы исключено;
но в дальнейшем нет нужды обращать особое внимание
на момент опосредствования, ибо он находится везде
и всюду, в каждом понятии.

Это опосредствование с собой, которое нечто есть \emph{в
себе}, взятое лишь как отрицание отрицания, своими сторонами
не имеет каких-либо конкретных определений;
так оно сводится в простое единство, которое есть \emph{бытие}.
Нечто \emph{есть}, и оно ведь \emph{есть} также налично сущее; далее,
оно есть \emph{в себе} также и \emph{становление}, которое, однако,
уже не имеет своими моментами только бытие и ничто.
Один из них~-- бытие~-- есть теперь наличное бытие, я,
далее, налично сущее; второй есть также нечто \emph{налично
сущее}, но определенное как отрицательность, присущая
нечто (Negatives des Etwas),~-- \emph{иное}. Нечто как становление
есть переход, моменты которого сами суть нечто и
который поэтому есть \emph{изменение},~-- становление, ставшее
уже \emph{конкретным}.~-- Но нечто изменяется сначала лишь
в своем понятии; оно, таким образом, еще не \emph{положено}
как опосредствующее и опосредствованное; вначале оно
положено как просто сохраняющее себя в своем соотношении
с собой, а его отрицательность~-- как некоторое
такое же качественное, как только \emph{иное} вообще.


%%% Local Variables:
%%% mode: latex
%%% TeX-master: "../../../../main"
%%% End:


\section{Конечность}

a) Нечто \emph{и иное}; они ближайшим образом безразличны
друг к другу; иное также есть непосредственно налично
сущее, нечто; отрицание, таким образом, имеет место
вне их обоих. Нечто есть \emph{в себе}, противостоящее своему
\emph{бытию-для-иного}. Но определенность принадлежит также
к его <<в-себе>> и есть

b) его \emph{определение}, переходящее также в \emph{свойство}
(Beschaffenheit), которое, будучи тождественным с первым,
составляет имманентное и в то же время подвергшееся
отрицанию бытие-для-иного, \emph{границу} [всякого]
нечто, которая

c) есть имманентное определение самого нечто, а нечто,
следовательно, есть \emph{конечное}.

В начале главы, где мы рассматривали \emph{наличное
бытие} вообще, последнее как взятое первоначально имело
определение \emph{сущего}. Моменты его развития, качество
и нечто, имеют поэтому также утвердительную определенность.
Напротив, в начале этого раздела развивается
заключающееся в наличном бытии отрицательное определение,
которое там еще было только отрицанием вообще,
\emph{первым} отрицанием, а теперь определено как \emph{внутри-себя-бытие}
[всякого] нечто, как отрицание отрицания.



%%% Local Variables:
%%% mode: latex
%%% TeX-master: "../../../../main"
%%% End:


\subsection{Нечто и иное}

1. \emph{Во-первых}, нечто и иное суть налично сущие, или
\emph{нечто}.

\emph{Во-вторых}, каждое из них есть также \emph{иное}. Безразлично,
которое из них мы называем сначала и лишь потому
именуем \emph{нечто} (по-латыни, когда они встречаются
в одном предложении, оба называются aliud, или <<один
другого>>~-- alius alium, а когда речь идет об отношении
взаимности, аналогичным выражением служит alter alterum).
Если мы одно наличное бытие называем $A$, а другое
$B$, то $B$ определено ближайшим образом как иное.
Но точно так же $A$ есть иное этого $B$. Оба одинаково суть
\emph{иные}. Для фиксирования различия и того нечто, которое
следует брать как утвердительное, служит [слово] <<это>>\endnotemark{}.
Но <<это>> как раз и выражает, что такое различение и выделение
одного нечто есть субъективное обозначение,
имеющее место вне самого нечто. В этом внешнем показывании
и заключается вся определенность; даже выражение
<<это>> не содержит никакого различия; всякое и каждое
нечто есть столь же <<\emph{это}>>, сколь и иное. \emph{Считается}, что
словом <<это>> выражают нечто совершенно определенное;
но при этом упускают из виду, что язык как произведение
рассудка выражает лишь всеобщее; исключение составляет
только \emph{имя} единичного предмета, но индивидуальное
имя есть нечто бессмысленное в том смысле, что оно
не выражает всеобщего, и по этой же причине оно представляется
чем-то лишь положенным, произвольным, как
и на самом деле собственные имена могут быть произвольно
приняты, даны или также изменены.

\endnotetext{Ср. раздел <<Чувственная достоверность или <<это>> и мнение>>
  в <<Феноменологии духа>> (стр.\,51--59).}

Итак, инобытие представляется определением, чуждым
определенному таким образом наличному бытию,
или, иначе говоря, иное выступает \emph{вне} данного наличного
бытия; отчасти так, что наличное бытие определяет
себя как иное только через \emph{сравнение}, производимое некоторым
третьим, отчасти так, что это наличное бытие
определяет себя как другое только из-за иного, находящегося
вне его, но само по себе оно не таково. В то же
время, как мы уже отметили, каждое наличное бытие
определяет себя и для представления в равной мере как
другое наличное бытие, так что не остается ни одного
наличного бытия, которое было бы определено лишь как
наличное бытие и не было бы вне некоторого наличного
бытия, следовательно, само не было бы некоторым иным.

Оба определены и как \emph{нечто} и как \emph{иное}, они, значит,
\emph{одно и то же}, и между ними еще нет никакого различия.
Но эта \emph{тождественность} определений также имеет мест
только во внешней рефлексии, в \emph{сравнении} их друг с другом;
но в том виде, в каком вначале положено \emph{иное}, оно
само по себе, правда, соотносится с нечто, однако оно
также и \emph{само по себе находится вне последнего}.

\emph{В-третьих}, следует поэтому брать \emph{иное} как изолированное,
в соотношении с самим собой, брать \emph{абстрактно}
как иное, как τό έτερον Платона, который противопоставляет
его \emph{единому} как один из моментов целокупности
и таким образом приписывает \emph{иному} свойственную ему
\emph{природу}. Таким образом, \emph{иное}, понимаемое лишь как таковое,
есть не иное некоторого нечто, а иное в самом
себе, т.\,е. иное самого себя.~-- \emph{Физическая природа} есть
по своему определению такое иное; она есть \emph{иное духа}.
Это ее определение есть, таким образом, вначале одна
лишь относительность, которая выражает не какое-то
качество самой природы, а лишь внешнее ей соотношение.
Но так как дух есть истинное нечто, а природа
поэтому есть в самой себе лишь то, что она есть по отношению
к духу, то, поскольку она берется сама по себе,
ее качество состоит именно в том, что она в самой себе
есть иное, \emph{вовне-себя-сущее} (в определениях пространства,
времени, материи).

Иное само по себе есть иное по отношению к самому
себе (an ihm selbst) и, следовательно, иное самого себя,
таким образом, иное иного,~-- следовательно, всецело неравное
внутри себя, отрицающее себя, \emph{изменяющееся}.
Но точно так же оно остается тождественным с собой,
ибо то, во что оно изменилось, есть \emph{иное}, которое помимо
этого не имеет никаких других определений. А то, чт\'о
изменяется, определено быть иным не каким-нибудь
другим образом, а тем же самым; оно поэтому \emph{соединяется}
в том ином \emph{лишь с самим собой}. Таким образом, оно
положено как рефлектированное в себя со снятием инобытия;
оно есть \emph{тождественное} с собой нечто, по отношению
к которому, следовательно, инобытие, составляющее
в то же время его момент, есть нечто отличное от него,
не принадлежащее ему самому как такому нечто.

2. Нечто \emph{сохраняется} в отсутствии своего наличного
бытия (Nichtdasein), оно по своему существу \emph{едино} с ним
и по своему существу \emph{не едино} с ним. Оно, следовательно,
\emph{соотносится} со своим инобытием; оно не есть только
свое инобытие. Инобытие в одно и то же время и содержится
в нем, и еще \emph{отделено} от него. Оно \emph{бытие-для-иного}.

Наличное бытие, как таковое, есть непосредственное,
безотносительное; иначе говоря, оно имеется в определении
\emph{бытия}. Но наличное бытие как включающее в себя
небытие есть \emph{определенное} бытие, подвергшееся внутри
себя отрицанию, а затем ближайшим образом~-- иное; но
так как оно в то же время и сохраняется, подвергнув себя
отрицанию, то оно есть лишь \emph{бытие-для-иного}.

Оно сохраняется в отсутствии своего наличного бытия
и есть бытие; но не бытие вообще, а как соотношение
с собой в \emph{противоположность} своему соотношению с иным,
как равенство с собой в противоположность своему неравенству.
Такое бытие есть \emph{в-себе-бытие}.

Бытие-для-иного и в-себе-бытие составляют \emph{оба момента}
[всякого] нечто. Здесь имеются \emph{две пары} определений:
1) \emph{нечто} и \emph{иное}; 2) \emph{бытие-для-иного} и \emph{в-себе-бытие}.
В первых имеется безотносительность их определенности:
нечто и иное расходятся. Но их истина~-- это соотношение
между ними; бытие-для-иного и в-себе-бытие
суть поэтому указанные определения, положенные как \emph{моменты}
одного и того же, как определения, которые суть
соотношения и остаются в своем единстве, в единстве наличного
бытия. Каждое из них, следовательно, в то же время
содержит в себе и свой отличный от себя момент.

Бытие и ничто в том их единстве, которое есть наличное
бытие, уже более не бытие и ничто: таковы они только
вне своего единства. Таким образом, в их беспокойном
единстве, в становлении, они суть возникновение и
прехождение.~-- Бытие во [всяком] нечто есть \emph{в-себе-бытие}.
Бытие, соотношение с собой, равенство с собой,
теперь уже не непосредственное, оно соотношение с собой
лишь как небытие инобытия (как рефлектированное
в себя наличное бытие).~-- И точно так же небытие как
момент [всякого] нечто в этом единстве бытия и небытия
есть не отсутствие наличного бытия вообще, а иное,
и, говоря определеннее, по \emph{различению} его и бытия оно
есть в то же время \emph{соотношение} с отсутствием своего
наличного бытия, бытие-для-иного.

Тем самым \emph{в-себе-бытие} есть, во-первых, отрицательное
соотношение с отсутствием наличного бытия, оно
имеет инобытие вовне себя и противоположно ему;
поскольку нечто есть в \emph{себе}, оно лишено инобытия и бытия
для иного. Но, во-вторых, оно имеет небытие и в самом
себе, ибо оно само есть \emph{не-бытие} бытия-для-иного.

Но \emph{бытие-для-иного} есть, во-первых, отрицание простого
соотношения бытия с собой, соотношения, которым
ближайшим образом должно быть наличное бытие и нечто;
поскольку нечто есть в ином или для иного, оно лишено
собственного бытия. Но, во-вторых, оно не отсутствие
наличного бытия как чистое ничто. Оно отсутствие
наличного бытия, указывающее на в-себе-бытие как на
свое рефлектированное в себя бытие, как и наоборот,
в-себе-бытие указывает на бытие-для-иного.

3. Оба момента суть определения одного и того же, а
именно определения [всякого] нечто. Нечто есть \emph{в себе},
поскольку оно ушло из бытия-для-иного, возвратилось
в себя. Но нечто имеет также определение или обстоятельство
\emph{в себе} (an sich) (здесь ударение падает на <<в>>)
или \emph{в самом себе} (an ihm), поскольку это обстоятельство
есть \emph{в нем} (an ihm) внешним образом, есть бытие-для-иного.

Это ведет к некоторому дальнейшему определению.
\emph{В-себе-бытие} и бытие-для-иного прежде всего различны,
но то, что нечто имеет \emph{то же самое}, чт\'о \emph{оно есть в себе}
(an sich), также и \emph{в самом себе} (an ihm), и,наоборот,то,
что оно есть как бытие-для-иного, оно есть и в себе~--
в этом состоит тождество в-себе-бытия и бытия-для-иного,
согласно определению, что само нечто есть тождество
обоих моментов и что они, следовательно, в нем нераздельны.~--
Формально это тождество получается уже в
сфере наличного бытия, но более определенное выражение
оно получит при рассмотрении сущности и затем
при рассмотрении отношения \emph{внутреннего} (Innerlichkeit)
и \emph{внешнего} (Äusserlichkeit), а определеннее всего~-- при
рассмотрении идеи как единства понятия и действительности.~--
Полагают, что словами <<\emph{в себе}>> и <<\emph{внутреннее}>>
высказывают нечто возвышенное; однако то, чт\'о нечто
есть \emph{только в себе}, есть также \emph{только в нем}; <<в себе>>
есть лишь абстрактное и, следовательно, внешнее определение.
Выражения <<\emph{в нем} ничего нет>>, <<\emph{в этом} что-то
есть>> имеют, хотя и смутно, тот смысл, что то, чт\'о в чем-то
есть, принадлежит также и к его \emph{в-себе-бытию}, к его
внутренней, истинной ценности.

Можно отметить, что здесь уясняется смысл \emph{вещи-в-себе},
которая есть очень простая абстракция, но в продолжение
некоторого времени слыла очень важным определением,
как бы чем-то изысканным, так же как положение
о том, что мы не знаем, каковы вещи в себе,
признавалось большой мудростью.~-- Вещи называются
вещами-в-себе, поскольку мы абстрагируемся от всякого
бытия-для-иного, т.\,е. вообще поскольку мы их мыслим
без всякого определения, как ничто. В этом смысле нельзя,
разумеется, знать, \emph{что такое} вещь-\emph{в-себе}. Ибо вопрос:
\emph{что такое?}~-- требует, чтобы были указаны \emph{определения};
но так как те вещи, определения которых следовало бы
указать, должны быть в то же время \emph{вещами-в-себе}, т.\,е.
как раз без всякого определения, то в вопрос необдуманно
включена невозможность ответить на него или же
дают только нелепый ответ на него.~-- Вещь-в-себе есть
то же самое, что то абсолютное, о котором знают только
то, что все в нем едино. Мы поэтому знаем очень хорошо,
чт\'о представляют собой эти вещи-в-себе; они, как
таковые, не что иное, как лишенные истинности, пустые
абстракции. Но что такое поистине вещь-в-себе, что поистине
есть в себе,~-- изложением этого служит логика,
причем, однако, под <<в-себе>> понимается нечто лучшее,
чем абстракция, а именно то, чт\'о нечто есть в своем
понятии; но понятие конкретно внутри себя постижимо
как понятие вообще и внутри себя познаваемо как определенное
и как связь своих определений.

В-себе-бытие имеет своим противостоящим моментом
прежде всего бытие-для-иного; но в-себе-бытию противопоставляется
также и \emph{положенностъ} (Gesetztsein). Это
выражение, правда, подразумевает также и бытие-для-иного,
но оно определенно разумеет уже происшедший
поворот\endnotemark{} от того, чт\'о не есть в себе, к тому, чт\'о есть
его в-себе-бытие, в чем оно \emph{положительно}. \emph{В-себе-бытие}
следует обычно понимать как абстрактный способ выражения
понятия; \emph{полагание}, собственно говоря, относится
уже к сфере сущности, объективной рефлексии; основание
\emph{полагает} то, чт\'о им обосновывается; причина, больше того,
\emph{производит} действие, наличное бытие, самостоятельность
которого \emph{непосредственно} отрицается и смысл которого
заключается в том, что оно имеет свою \emph{суть} (Sache),
свое бытие в ином. В сфере бытия наличное бытие \emph{происходит}
только из становления, иначе говоря, вместе с
нечто положено иное, вместе с конечным~-- бесконечное,
но конечное не производит бесконечного, не \emph{полагает} его.
В сфере бытия \emph{самоопределение} (\emph{Sichbestimmen}) понятия
само есть лишь \emph{в-себе}~-- и в таком случае оно называется
переходом. Рефлектирующие определения бытия,
как, например, нечто и иное или конечное и бесконечное,
хотя по своему существу и указывают друг на друга,
или даны как бытие-для-иного, также считаются как
\emph{качественные} существующими сами по себе; иное есть,
конечное считается точно так же \emph{непосредственно сущим}
и пребывающим само по себе, как и бесконечное; их
смысл представляется завершенным также и без их иного.
Напротив, положительное и отрицательное, причина
и действие, хотя они также берутся как изолированно
сущие, все же не имеют никакого смысла друг без друга;
они \emph{сами} светятся друг в друге, каждое из них светится
в своем ином.~-- В разных сферах определения и
в особенности в развитии изложения, или, точнее, в движении
понятия к своему изложению существенно всегда
надлежащим образом различать между тем, чт\'о еще есть
\emph{в себе}, и тем, чт\'о \emph{положено}, например определения, как
они суть в понятии и каковы они, будучи положенными
или сущими-для-иного. Это~-- различение, относящееся
только к диалектическому развитию, различение, которого
не знает метафизическое философствование, в том
числе и критическая философия; дефиниции метафизики,
равно как и ее предпосылки, различения и выводы, имеют
целью утверждать и выявлять лишь \emph{сущее} и притом
\emph{в-себе-сущее}.

\endnotetext{Zurückbengung (поворот) имеет то же значение, что и
  рефлексия.}

В единстве [всякого] нечто с собой \emph{бытие-для-иного}
тождественно со своим <<\emph{в себе}>>; \emph{в этом случае} бытие-для-иного
есть \emph{в} [самом] нечто. Рефлектированная таким
образом в себя определенность тем самым есть вновь
\emph{простое сущее}, есть, следовательно, вновь качество~--
\emph{определение}.


%%% Local Variables:
%%% mode: latex
%%% TeX-master: "../../../../main"
%%% End:


\subsection{Определение, свойство и граница}

<<\emph{В себе}>>, в которое нечто рефлектировано внутри себя
из своего бытия-для-иного, уже не есть абстрактное
<<в себе>>, а как отрицание своего бытия-для-иного опосредствовано
последним, которое составляет, таким образом,
его момент. <<В себе>> есть не только непосредственное
тождество [самого] нечто с собой, но и такое тождество,
благодаря которому нечто есть и \emph{в самом себе} то,
чт\'о оно есть \emph{в себе}; бытие-для-иного есть \emph{в нем}, потому
что <<\emph{в себе}>> есть его снятие, есть [выхождение] \emph{из него}
в себя, но уже и потому, что оно абстрактно, следовательно,
в своем существе обременено отрицанием, бытием-для-иного.
Здесь имеется не только качество и реальность,
сущая определенность, но и \emph{в-себе-сущая} определенность,
и ее развитие состоит в \emph{полагании} ее как этой
рефлектированной в себя определенности.

1. Качество, которое есть <<в себе>> в простом нечто
и сущностно находится в единстве с другим моментом
этого нечто, с \emph{в-нем-бытием}, можно назвать его \emph{определением},
поскольку это слово в более точном его значении
отличают от \emph{определенности} вообще. Определение есть
утвердительная определенность как в-себе-бытие, которому
нечто в своем наличном бытии, противодействуя своей
переплетенности с иным, которым оно было бы определено,
остается адекватным, сохраняясь в своем равенстве
с собой и проявляя это равенство в своем бытии-для-иного.
Нечто \emph{осуществляет} (erfüllt) свое определение\endnotemark{}, поскольку
дальнейшая определенность, многообразно вырастающая
прежде всего благодаря его отношению к иному,
становится его полнотой (Fülle) в соответствии с его
в-себе-бытием. Определение подразумевает, что то, чт\'о
нечто есть \emph{в себе}, есть также и \emph{в нем}.

\endnotetext{Bestimmung имеет двоякое значение — определение и предназначение.}

Мыслящий разум~"--- вот \emph{определение человека}; мышление
вообще есть его простая \emph{определенность}, ею человек
отличается от животного; он есть мышление \emph{в себе},
поскольку мышление отличается и от его бытия-для-иного,
от его собственной природности и чувственности,
которыми он непосредственно связан с иным. Но мышление
есть и \emph{в нем}: сам человек есть мышление, он \emph{налично
сущ} как мыслящий, оно его существование и действительность;
и далее, так как мышление имеется в его наличном
бытии, а его наличное бытие~"--- в мышлении, то
оно \emph{конкретно}, его следует брать имеющим содержание
и наполненным, оно мыслящий разум и таким образом
оно \emph{определение} человека. Но даже это определение
опять-таки дано лишь \emph{в себе} как долженствование, т.\,е.
оно вместе с включенным в его <<в-себе>> наполнением
дано в форме <<в-себе>> вообще, \emph{в противоположность} не
включенному в него наличному бытию, которое в то же
время еще есть внешне противостоящая [ему] чувственность
и природа.

2. Наполнение в-себе-бытия определенностью также
отлично от той определенности, которая есть лишь бытие-для-иного
и остается вне определения. В самом деле, в
области [категорий] качества различия сохраняют даже
в своей снятости непосредственное, качественное бытие в
отношении друг друга. То, чт\'о нечто имеет \emph{в самом себе},
разделяется таким именно образом, и с этой стороны
есть внешнее наличное бытие этого нечто, каковое наличное
бытие есть также \emph{его} наличное бытие, но не принадлежит
его в-себе-бытию. Определенность, таким образом,
есть \emph{свойство}.

Обладая тем или иным свойством, нечто подвергается
воздействию внешних влияний и обстоятельств. Это внешнее
соотношение, от которого зависит свойство, и определяемость
иным представляется чем-то случайным. Но
качество [всякого] нечто в том-то и состоит, чтобы быть
предоставленным этой внешности и обладать некоторым
\emph{свойством}.

Поскольку нечто изменяется, изменение относится
к свойству, которое есть \emph{в} нечто то, чт\'о становится иным.
Само нечто сохраняет себя в изменении, которое затрагивает
только эту непрочную поверхность его инобытия,
а не его определение.

Определение и свойство, таким образом, отличны
друг от друга; со стороны своего определения нечто безразлично
к своему свойству. Но то, чт\'о нечто имеет
\emph{в самом себе}, есть связующий их средний термин этого
силлогизма. Но \emph{бытие-в-нечто} (das Am-Etwas-Sein) оказалось,
напротив, распадающимся на указанные два крайних
термина. Простой средний термин есть \emph{определенность},
как таковая; к ее тождеству принадлежит и определение,
и свойство. Но определение переходит само по
себе в свойство и свойство [само по себе]~"--- в определение.
Это вытекает из предыдущего; связь, говоря более
точно, такова: поскольку то, чт\'о нечто \emph{есть в себе}, есть
также и \emph{в нем}, оно обременено бытием-для-иного; определение,
как таковое, открыто, следовательно, отношению
к иному. Определенность есть в то же время момент, но
вместе с тем содержит качественное различие~"--- она
отличается от в-себе-бытия, есть отрицание [данного] нечто,
другое наличное бытие. Определенность, охватывающая
таким образом иное, соединенная с в-себе-бытием,
вводит инобытие во в-себе-бытие или, иначе говоря, в
определение, которое в силу этого низводится до свойства.~"---
Наоборот, бытие-для-иного, изолированное и положенное
само по себе как свойство, есть в нем то же, что
иное, как таковое, иное в самом себе, т.\,е. иное самого
себя; но в таком случае оно есть \emph{соотносящееся с собой}
наличное бытие, есть, таким образом, в-себе-бытие с некоторой
определенностью, стало быть, \emph{определение}.~"---
Следовательно, поскольку оба должны быть сохранены
друг вне друга, свойство, представляющееся основанным
в некотором внешнем, в ином вообще, \emph{зависит} также и
от определения, и идущий от чуждого процесс определения
в то же время определен собственным, имманентным
определением [данного] нечто. Но, кроме того, свойство
принадлежит к тому, чт\'о нечто есть в себе; вместе со
своим свойством изменяется и нечто.

Это изменение [данного] нечто уже не первое его
изменение исключительно со стороны его бытия-для-иного;
первое изменение было только в себе сущим изменением,
принадлежащим внутреннему понятию; теперь
же изменение есть и положенное в нечто.~"--- Само нечто
определено далее, и отрицание положено как имманентное
ему, как его развитое \emph{внутри-себя-бытие}.

Переход определения и свойства друг в друга~"--- это
прежде всего снятие их различия; тем самым положено
наличное бытие или нечто вообще, а так как оно результат
указанного различия, заключающего в себе также и
качественное инобытие, то имеются два нечто, но не
только вообще иные по отношению друг к другу~"--- в таком
случае это отрицание оказалось бы еще абстрактным
и относилось бы лишь к сравниванию их [между собой]~"---
теперь это отрицание имеется как \emph{имманентное} этим
нечто. Как \emph{налично сущие} они безразличны друг к другу.
Но это утверждение их уже не есть непосредственное,
каждое из них соотносится с самим собой \emph{через посредство}
снятия того инобытия, которое в определении рефлектировано
во в-себе-бытие.

Таким образом, нечто относится к иному \emph{из самого
себя}, ибо инобытие положено в нем как его собственный
момент; его внутри-себя-бытие заключает в себе отрицание,
через посредство которого оно теперь вообще обладает
своим утвердительным наличным бытием. Но от
последнего иное отлично также качественно и, стало быть,
положено вне нечто. Отрицание своего иного есть лишь
качество [данного] нечто, ибо оно нечто именно как это
снятие своего иного. Только этим, собственно говоря,
иное само противопоставляет себя наличному бытию;
иное противопоставляется первому нечто лишь внешне,
иначе говоря, так как они на самом деле находятся во
взаимной связи безусловно, т.\,е. по своему понятию, то
их связь заключается в том, что наличное бытие \emph{перешло}
в инобытие, нечто~"--- в иное и что нечто в той же мере,
что и иное, есть иное. Поскольку же внутри-себя-бытие
есть небытие инобытия, которое в нем содержится, но в
то же время как сущее отлично от него, постольку само
нечто есть отрицание, \emph{прекращение в нем иного}: оно положено
как относящееся к нему отрицательно и тем самым
сохраняющее себя;~"--- это иное, внутри-себя-бытие
[данного] нечто как отрицание отрицания есть его \emph{в-себе-бытие},
и в то же время это снятие дано \emph{в нем} как простое
отрицание, а именно как отрицание им внешнего
ему другого нечто. Именно \emph{одна} их определенность, с
одной стороны, тождественна с внутри-себя-бытием этих
нечто как отрицание отрицания, а с другой, поскольку
эти отрицания противостоят одно другому как другие
нечто, она, исходя из них же самих, смыкает их и точно
так же отделяет их друг от друга, так как каждое из них
отрицает иное; это \emph{граница}.

3. \emph{Бытие-для-иного} есть неопределенная, утвердительная
общность [всякого] нечто со своим иным; в границе
выдвигается \emph{небытие}-для-иного, качественное отрицание
иного, недопускаемого вследствие этого к рефлектированному
в себя нечто. Следует присмотреться к развитию
(Entwicklung) этого понятия, каковое развитие, впрочем,
скорее оказывается запутанностью (Verwicklung) и противоречием.
Противоречие сразу же имеется в том, что граница
как рефлектированное в себя отрицание [данного] нечто
содержит в себе \emph{идеально} моменты нечто и иного, и в то же
время они как различенные моменты положены в сфере
наличного бытия как \emph{реально, качественно различные}.

$\alpha$) Нечто, следовательно, есть непосредственное соотносящееся
с собой наличное бытие и имеет границу прежде
всего как границу в отношении иного; она небытие
иного, а не самого нечто; последнее ограничивает в ней
свое иное.~"--- Но иное само есть некоторое нечто вообще;
стало быть, граница, которую нечто имеет в отношении
иного, есть также граница иного как нечто, граница этого
нечто, посредством которой оно не допускает к себе
первое нечто как \emph{свое} иное, или, иначе говоря, она есть
\emph{небытие этого первого нечто}; таким образом, она есть не
только небытие иного, но и небытие как одного, так и
другого нечто и, значит, небытие [всякого] \emph{нечто} вообще.

Но по своей сущности граница есть также и небытие
иного; таким образом, нечто в то же время \emph{есть} благодаря
своей границе. Будучи ограничивающим, нечто,
правда, низводится до того, что само оно оказывается
ограничиваемым, однако его граница как прекращение
иного в нем в то же время сама есть лишь бытие этого
нечто: \emph{благодаря ей нечто есть то, чт\'о оно есть, имеет в
ней свое качество}.~"--- Это отношение есть внешнее проявление
того, что граница есть простое или \emph{первое} отрицание,
иное же есть в то же время отрицание отрицания,
внутри-себя-бытие [данного] нечто.

Нечто как непосредственное наличное бытие есть, следовательно,
граница в отношении другого нечто, но оно
имеет ее \emph{в самом себе} и есть нечто через ее опосредствование,
которое в той же мере есть его небытие. Граница~"---
это опосредствование, через которое нечто и иное
\emph{и есть и не есть}.

$\beta$) Поскольку же нечто и \emph{есть и не есть} в своей границе
и эти моменты суть некоторое непосредственное, качественное
различие, постольку отсутствие наличного
бытия [данного] нечто и его наличное бытие оказываются
друг вне друга. Нечто имеет свое наличное бытие \emph{вне}
(или, как это также представляют себе, \emph{внутри}) своей
границы; точно так же и иное, так как оно есть нечто,
находится вне ее. Она \emph{середина между ними}, в которой
они прекращаются. Они имеют свое \emph{наличное бытие по
ту сторону} друг друга и \emph{их границы}; граница как небытие
каждого из них есть иное обоих.

В соответствии с таким различием между нечто и его
границей \emph{линия} представляется линией лишь вне своей
границы, точки; \emph{плоскость} представляется плоскостью
вне линии; \emph{тело} представляется телом лишь вне ограничивающей
его плоскости.~"--- Именно с этой стороны граница
схватывается прежде всего представлением, этим
вовне-себя-бытием понятия, и с этой же стороны она
берется преимущественно в пространственных предметах.

$\gamma$) Но, кроме того, нечто, как оно есть вне границы,
есть неограниченное нечто, лишь наличное бытие вообще.
Так оно не отличается от своего иного; оно лишь
наличное бытие, имеет, следовательно, одно и то же определение
со своим иным; каждое из них есть лишь нечто
вообще или, иначе говоря, каждое есть иное; оба суть,
таким образом, \emph{одно и то же}. Но это их сначала лишь
непосредственное наличное бытие теперь положено с
определенностью как границей, в которой оба суть то,
чт\'о они суть, отличные друг от друга. Но она точно так
же, как и наличное бытие, есть \emph{общее} им обоим различие,
их единство и различие. Это двоякое тождество
обоих~"--- наличное бытие и граница~"--- подразумевает, что
нечто имеет свое наличное бытие только в границе и
что, так как и граница и непосредственное наличное бытие
в то же время отрицают друг друга, то нечто, которое
есть только в своей границе, в такой же мере отделяет
себя от самого себя и по ту сторону себя указывает
на свое небытие и выражает свое небытие как свое бытие,
переходя, таким образом, в это бытие. Чтобы применить
это к предыдущему примеру, следует сказать, что по
одному определению нечто есть то, чт\'о оно есть, только
в своей границе; в таком случае \emph{точка} есть граница
\emph{линии} не только в том смысле, что линия лишь прекращается
в точке и как наличное бытие находится вне точки;
\emph{линия} есть граница \emph{плоскости} не только в том смысле,
что плоскость лишь прекращается в линии (это точно
так же применимо к \emph{плоскости} как к границе \emph{тела}),
но и в том смысле, что в точке линия также и \emph{начинается};
точка есть абсолютное начало линии. Даже и в том
случае, когда линию представляют себе продолженной
в обе ее стороны безгранично, или, как обычно выражаются,
бесконечно, точка составляет ее \emph{элемент}, подобно
тому как линия составляет элемент плоскости, а плоскость~"---
элемент тела. Эти \emph{границы} суть \emph{принцип} того,
чт\'о они ограничивают, подобно тому как единица, например
как сотая, есть граница, но также и элемент целой
сотни.

Другое определение~"--- беспокойство, присущее [всякому]
нечто и состоящее в том, что в своей границе, в
которой оно имманентно, нечто есть \emph{противоречие}, заставляющее
его выходить за свои пределы. Так, диалектика
самой точки~"--- это стать линией; диалектика линии~"---
стать плоскостью, диалектика плоскости~"--- стать целокупным
пространством. Вторая дефиниция линии, плоскости
и всего пространства гласит поэтому, что через
\emph{движение} точки возникает линия, через движение линии
возникает плоскость и т.\,д. Но на это \emph{движение} точки,
линии и т.\,д. смотрят как на нечто случайное или как
на нечто такое, чт\'о мы только представляем себе. Однако
такой взгляд опровергается, собственно говоря, уже тем,
что определения, из которых, согласно этой дефиниции,
возникают линии и т.\,д., суть их \emph{элементы} и \emph{принципы},
а последние в то же время суть не что иное, как и их границы;
возникновение, таким образом, рассматривается
не как случайное или лишь представляемое. Что точка,
линия, поверхность сами по себе, противореча себе, суть
начала, которые сами отталкиваются от себя, и что точка,
следовательно, из себя самой, через свое понятие,
переходит в линию, \emph{движется в себе} и заставляет возникнуть
линию и т.\,д.,~"--- это заключено в понятии границы,
имманентной [данному] нечто. Однако само применение
следует рассматривать там, где будем трактовать
о пространстве; чтобы здесь бегло указать на это применение,
скажем, что точка есть совершенно абстрактная
граница, но \emph{в некотором наличном бытии}; последнее берется
здесь еще совершенно неопределенно; оно есть так
называемое абсолютное, т.\,е. абстрактное \emph{пространство},
совершенно непрерывное вне-друг-друга-бытие. Тем, что
граница не абстрактное отрицание, а отрицание в \emph{этом
наличном бытии}, тем, что она \emph{пространственная} определенность,~"---
точка пространственна, представляет собой
противоречие между абстрактным отрицанием и непрерывностью
и, стало быть, совершающийся и совершившийся
переход в линию и т.\,д., ибо на самом деле \emph{нет}
ни точки, ни линии, ни поверхности.

Нечто вместе со своей имманентной границей, полагаемое
как противоречие самому себе, в силу которого
оно выводится и гонится дальше себя, есть \emph{конечное}.


%%% Local Variables:
%%% mode: latex
%%% TeX-master: "../../../../main"
%%% End:


\subsection{Конечность}

Наличное бытие определенно; нечто имеет некоторое
качество, и в нем оно не только определенно, но и ограниченно;
его качество есть его граница; обремененное
границей, нечто сначала остается утвердительным,
спокойным наличным бытием. Но это отрицание, когда
оно развито так, что противоположность между наличным
бытием данного нечто и отрицанием как имманентной ему
границей сама есть его внутри-себя-бытие и данное нечто,
таким образом, есть лишь становление в самом
себе,~-- это отрицание составляет в таком случае его конечность.

Когда мы говорим о вещах, что \emph{они конечны}, то разумеем
под этим, что они не только имеют некоторую
определенность, что качество дано не только как реальность
и в-себе-сущее определение, что они не только
ограничены,~-- в этом случае они еще обладают наличным
бытием вне своей границы,~-- но что скорее небытие составляет
их природу, их бытие. Конечные вещи \emph{суть}, но
их соотношение с самими собой состоит в том, что они
соотносятся с самими собой как \emph{отрицательные}, что они
именно в этом соотношении с самими собой гонят себя
дальше себя, дальше своего бытия. Они \emph{суть}, но истиной
этого бытия служит их \emph{конец}. Конечное не только изменяется,
как нечто вообще, а \emph{преходит}; и не только возможно,
что оно преходит, так что оно могло бы быть,
не преходя, но бытие конечных вещей, как таковое, состоит
в том, что они содержат зародыш прехождения как
свое внутри-себя-бытие, что час их рождения есть час
их смерти.


%%% Local Variables:
%%% mode: latex
%%% TeX-master: "../../../../main"
%%% End:


\paragraph{Непосредственность конечности}

Мысль о конечности вещей влечет за собой эту скорбь
по той причине, что конечность эта есть доведенное до
крайности качественное отрицание и что в простоте такого
определения им уже не оставлено никакого утвердительного
бытия, \emph{отличного} от их определения к гибели.
Ввиду этой качественной простоты отрицания, возвратившегося
к абстрактной противоположности между
ничто и прехождением, с одной стороны, и бытием~-- с
другой, конечность есть наиболее упрямая категория рассудка;
отрицание вообще, свойство, граница уживаются
со своим иным~-- с наличным бытием; даже от абстрактного
ничто самого по себе как от абстракции отказываются;
но конечность есть \emph{фиксированное в себе} отрицание
и поэтому резко противостоит своему утвердительному.
Конечное, правда, позволяет привести себя в движение,
оно само и состоит в том, что оно определено к
своему концу, но только к своему концу; оно скорее есть
отказ от того, чтобы его утвердительно приводили к его
утвердительному, к бесконечному, чтобы его приводили
в связь с последним. Оно, стало быть, положено нераздельным
со своим ничто, и этим отрезан путь к какому
бы то ни было его примирению со своим иным, с утвердительным.
Определение конечных вещей не простирается
далее их \emph{конца}. Рассудок никак не хочет отказаться
от этой скорби о конечности, делая небытие определением
вещей и вместе с тем \emph{непреходящим} и \emph{абсолютным}.
Их преходящность (Vergänglichkeit) могла бы прейти
лишь в ином, в утвердительном; тогда их конечность
отделилась бы от них; но она есть их неизменное качество,
т.\,е. не переходящее в свое иное, т.\,е. в свое утвердительное;
\emph{таким образом она вечна}.

Это весьма важное наблюдение; но что конечное
абсолютно~-- это такая точка зрения, которую, разумеется,
вряд ли какое-либо философское учение или какое-либо
воззрение или рассудок позволят навязать себе;
скорее в утверждении о конечном определенно содержится
противоположный взгляд: конечное есть ограниченное,
преходящее; конечное есть \emph{только} конечное, а не непреходящее;
это заключается непосредственно в его определении
и выражении. Но важно знать, настаивает ли это
воззрение на том, чтобы мы не шли дальше \emph{бытия конечности}
и рассматривали \emph{преходящность} как сохраняющуюся,
или же [на том, что] \emph{преходящность} и \emph{прехождение
преходят}? Что это не имеет места, фактически утверждается
как раз тем воззрением на конечное, которое
делает \emph{прехождение последним} [моментом] в конечном.
Оно определенно утверждает, что конечное не уживается
и несоединимо с бесконечным, что конечное полностью
противоположно бесконечному. Бесконечному приписывается
бытие, абсолютное бытие; конечное, таким
образом, остается по отношению к нему фиксированным
как его отрицательное; несоединимое с бесконечным, оно
остается абсолютно у себя; оно могло бы получить утвердительность
от утвердительного, от бесконечного и таким
образом оно прешло бы; но как раз соединение с последним
объявляется невозможным. Если верно, что оно по
отношению к бесконечному не остается неизменным, а
преходит, то, как мы сказали раньше, последний [момент]
в нем есть именно его прехождение, а не утвердительное,
которым могло бы быть лишь прехождение прехождения.
Если же конечное преходит не в утвердительном, а его конец
понимается как \emph{ничто}, то мы снова оказались бы у того
первого, абстрактного ничто, которое само давно прешло.

Однако у этого ничто, которое должно быть \emph{только}
ничто и которому в то же время приписывают некоторое
существование, а именно существование в мышлении,
представлении или речи, мы встречаем то же самое противоречие,
которое только что было указано у конечного,
с той лишь разницей, что в абстрактном ничто это противоречие
только \emph{встречается}, а в конечности оно \emph{решительно
выражено}. Там оно представляется субъективным,
здесь же утверждают, что конечное противостоит бесконечному
\emph{вечно}, \emph{есть} в себе ничтожное и дано \emph{как} в себе
ничтожное. Это нужно осознать; и развертывание конечного
показывает, что оно в самом себе как это противоречие
рушится внутри себя, но при этом действительно
разрешает указанное противоречие, [обнаруживая], что
оно не только преходяще и преходит, но что прехождение,
ничто не есть последний момент, а само преходит.


%%% Local Variables:
%%% mode: latex
%%% TeX-master: "../../../../main"
%%% End:



\printendnotes

\end{document}

%%% Local Variables:
%%% mode: latex
%%% TeX-master: t
%%% TeX-engine: xetex
%%% End:
